\documentclass[letterpaper,oneside,english,11pt,openany]{memoir}

% \usepackage{etoolbox} %gives a lot of useful tools, e.g. \newbool{...}, \setbool{...}{...}, ...

%% BEGIN : SETUP OF LANGUAGE AND FONTS %%

\usepackage[utf8]{inputenc} %use Unicode
\usepackage[T1]{fontenc} %European fonts

\usepackage[english=american]{csquotes} %set up \enquote to do American quoting style

\usepackage[english]{babel} %use English as the main language

%% END : SETUP OF LANGUAGE AND FONTS %%

%% BEGIN : STANDARD PACKAGES %%

\usepackage{%
  amsmath,       %some math tools
  amssymb,       %math symbols
  graphicx,      %enhanced graphics options
  mathtools,     %extension of amsmath
  microtype,     %small typographic effects
  bm,            %bold math symbols
  % todonotes,   %adds the option \todo{...} (use fixme instead)
  stmaryrd,      %some more math symbolswork
  nicematrix,    %nicer matrix controls
  mathrsfs,      %more math fonts
}

\usepackage[shortlabels]{enumitem} %better lists, e.g. enumerate

\usepackage[citestyle=alphabetic,bibstyle=alphabetic,uniquename=init,autolang=hyphen,backend=biber,sortcites=true]{biblatex} %used to make bibliography
\renewcommand*{\labelalphaothers}{}
\addbibresource{ref.bib}
\DeclareLabelalphaTemplate{
  \labelelement{
    \field[final]{shorthand}
    \field{label}
    \field[strwidth=3,strside=left,ifnames=1]{labelname}
    \field[strwidth=1,strside=left]{labelname}
  }
}
% \nobibintoc %removes the bibliography from the table of contents
% \nocite{*} %include all bibliography entries in the printed bibliography

\usepackage[thmmarks,amsmath]{ntheorem} %package for theorems, definitions, ...

\usepackage{varioref}

\usepackage[colorlinks=false]{hyperref} %make links to references

\usepackage[nameinlink]{cleveref} %clever referencing (uses varioref and hyperref above)
% make some aliases for later
\crefalias{theorembreak}{theorem}
\crefalias{definitionbreak}{definition}
\crefalias{examplebreak}{example}
\crefalias{examplesbreak}{examples}
\crefalias{corollarybreak}{corollary}
\crefalias{propositionbreak}{proposition}
\crefalias{lemmabreak}{lemma}
\crefalias{remarkbreak}{remark}
\crefalias{proofbreak}{proof}
%

\usepackage[layout=margin,draft]{fixme}
%\fxsetup{mode=multiuser}
\FXRegisterAuthor{dk}{andk}{DK} %register a user DK, and provide \dknote, \dkwarning, \dkerror and \dkfatal

\usepackage{calc} %used for calculating (useful when using tikz)

\usepackage{tikz} %used for drawing and making commutative diagrams
\usetikzlibrary{calc,shapes.geometric,decorations.markings,decorations.pathmorphing,arrows,cd,quotes,babel}
\usepackage{pgfplots} %for making graphs and plotting functions
\pgfplotsset{compat=1.18} %small fix to some compatibility issue

%\usepackage[left]{showlabels} %show labels in left margin while editing
\usepackage[notref,notcite]{showkeys}

%% END : STANDARD PACKAGES %%

%% BEGIN : SETUP INDEX AND NOMENCLATURE %%

%make index using memoir
\makeindex %create the default index

\usepackage[refpage,intoc]{nomencl} %used for nomenclature
\makenomenclature
\renewcommand*{\nomname}{List of Symbols}%name the nomenclature
\renewcommand*{\nomlabel}[1]{\hfil #1\hfil}%center symbol
\renewcommand*{\pagedeclaration}[1]{\dotfill\hyperpage{#1}} %pagenumber in nomenclature + link
\renewcommand{\nomgroup}[1]{%divide nomenclature into groups (can only use capital letters)
\item[\bfseries%
  \ifstrequal{#1}{A}{Cohomology of Unipotent groups}{%
    \ifstrequal{#1}{B}{Cohomology of pro-$p$ Iwahori subgroups}{%
      \ifstrequal{#1}{C}{List-Decodable Mean Estimation and Clustering}{}}}%
  ]}
\newcommand*{\nomuni}[1][]{\nomenclature[A#1]} %nomenclature for Cohomology of Unipotent groups
\newcommand*{\nomiwa}[1][]{\nomenclature[B#1]} %nomenclature for Cohomology of pro-p Iwahori subgroups
\newcommand*{\nomlis}[1][]{\nomenclature[C#1]} %nomenclature for List-Decodable Mean Estimation and Clustering

%% END : SETUP INDEX AND NOMENCLATURE %%

%% BEGIN : THEOREMS %%

\theoremseparator{.}

\theoremstyle{plain}
\theorembodyfont{\normalfont}
\theoremsymbol{\ensuremath{\clubsuit}}
\newtheorem{theorem}{Theorem}[chapter]
\newtheorem{corollary}[theorem]{Corollary}
\newtheorem{proposition}[theorem]{Proposition}
\newtheorem{lemma}[theorem]{Lemma}

\theoremsymbol{\ensuremath{\spadesuit}}
\newtheorem{definition}[theorem]{Definition}

\theoremsymbol{\ensuremath{\scriptstyle\bigcirc}}
\newtheorem{example}[theorem]{Example}
\newtheorem{examples}[theorem]{Examples}

\theoremheaderfont{\itshape}
\theoremsymbol{\ensuremath{\triangle}}
\newtheorem{remark}[theorem]{Remark}

\theoremstyle{nonumberplain}
\theoremsymbol{\ensuremath{\square}}
\newtheorem{proof}{Proof}

%corresponding break versions:

\theoremsymbol{}

\theoremheaderfont{\normalfont\bfseries}
\theorembodyfont{\normalfont}
\theoremstyle{break}
\theoremsymbol{\ensuremath{\clubsuit}}
\newtheorem{theorembreak}[theorem]{Theorem}
\newtheorem{corollarybreak}[theorem]{Corollary}
\newtheorem{propositionbreak}[theorem]{Proposition}
\newtheorem{lemmabreak}[theorem]{Lemma}

\theoremsymbol{\ensuremath{\spadesuit}}
\newtheorem{definitionbreak}[theorem]{Definition}

\theoremsymbol{\ensuremath{\scriptstyle\bigcirc}}
\newtheorem{examplebreak}{Example}
\newtheorem{examplesbreak}{Examples}

\theoremheaderfont{\itshape}
\theoremsymbol{\ensuremath{\triangle}}
\newtheorem{remarkbreak}{Remark}

\theoremstyle{nonumberbreak}
\theoremsymbol{\ensuremath{\square}}
\newtheorem{proofbreak}{Proof}

% no number versions:

\theoremsymbol{}
\theoremstyle{nonumberplain}
\theoremheaderfont{\normalfont\bfseries}
\theorembodyfont{\normalfont}
\newtheorem{notheorem}{Theorem}
\newtheorem{nocorollary}{Corollary}
\newtheorem{noproposition}{Proposition}
\newtheorem{nolemma}{Lemma}

\newtheorem{nodefinition}{Definition}

\theoremsymbol{\ensuremath{\scriptstyle\bigcirc}}
\newtheorem{noexample}{Example}
\newtheorem{noexamples}{Examples}

\theoremheaderfont{\itshape}
\theoremsymbol{\ensuremath{\triangle}}
\newtheorem{noremark}{Remark}

%corresponding no number break versions:

\theoremsymbol{}
\theoremheaderfont{\normalfont\bfseries}
\theorembodyfont{\normalfont}
\theoremstyle{nonumberbreak}
\newtheorem{notheorembreak}{Theorem}
\newtheorem{nocorollarybreak}{Corollary}
\newtheorem{nopropositionbreak}{Proposition}
\newtheorem{nolemmabreak}{Lemma}

\newtheorem{nodefinitionbreak}{Definition}

\theoremsymbol{\ensuremath{\scriptstyle\bigcirc}}
\newtheorem{noexamplebreak}{Example}
\newtheorem{noexamplesbreak}{Examples}

\theoremheaderfont{\itshape}
\theoremsymbol{\ensuremath{\triangle}}
\newtheorem{noremarkbreak}{Remark}

% special theorems:

\theoremseparator{.}
\theoremstyle{plain}
\theoremheaderfont{\normalfont\bfseries}
\theorembodyfont{\itshape}
\theoremsymbol{}
\newtheorem{fact}[theorem]{Fact}

\theoremseparator{:}
\theoremstyle{nonumberplain}
\theoremheaderfont{\normalfont\bfseries}
\theorembodyfont{\itshape}
\theoremsymbol{}
\newtheorem{univprop}{Universal Property}

%% END : THEOREMS %%

%% BEGIN : USEFUL SETTINGS FOR THIS DOCUMENT %%

\allowdisplaybreaks[1] %allows page change in align (but tries to avoid it)

\newcommand*{\yesnumber}{\refstepcounter{equation}\tag{\theequation}} %makes it possible to add a number in an equation

%% END : USEFUL SETTINGS FOR THIS DOCUMENT %%


\newcommand*\N{\mathbb{N}}
\newcommand*\Z{\mathbb{Z}}
\newcommand*\Q{\mathbb{Q}}
\newcommand*\R{\mathbb{R}}
\newcommand*\C{\mathbb{C}}
\newcommand*\F{\mathbb{F}}
\newcommand*\A{\mathbb{A}}

%% BEGIN : USER DEFINED COMMANDS %%

% define absolute value \abs{...}:
\DeclarePairedDelimiterX\abs[1]\lvert\rvert{%
  \ifblank{#1}{\:\cdot\:}{#1}
}
% define norm \norm{...}:
\DeclarePairedDelimiterX\norm[1]\lVert\rVert{%
  \ifblank{#1}{\:\cdot\:}{#1}
}
% define inner product \inner{...}{...}:
\DeclarePairedDelimiterX{\inner}[2]{\langle}{\rangle}{%
  \ifblank{#1}{\:\cdot\:}{#1},\ifblank{#2}{\:\cdot\:}{#2}
}

% define \set{...} to write sets and \given to write \set{... \given ...} for {...|...}
\newcommand*\setSymbol[1][]{
  \nonscript\:#1\vert\allowbreak\nonscript\:\mathopen{}
}
\providecommand\given{}
\DeclarePairedDelimiterX\set[1]{\lbrace}{\rbrace}{
  \renewcommand*\given{\setSymbol[\delimsize]}
  #1
}

%free group geneated by ... \free{...} or \free{... \given ...}
\DeclarePairedDelimiterX\free[1]{\langle}{\rangle}{
  \renewcommand\given{\nonscript\:\delimsize\vert\nonscript\:
    \mathopen{}}
  #1}

%define \lopen{...}{...}, \ropen{...}{...}, \open{...}{...}, \closed{...}{...} for intervals
\DeclarePairedDelimiterX\open[2](){#1,#2}
\DeclarePairedDelimiterX\lopen[2](]{#1,#2}
\DeclarePairedDelimiterX\ropen[2][){#1,#2}
\DeclarePairedDelimiterX\closed[2][]{#1,#2}

%define \Span{...} to write span of vectors
\DeclareMathOperator\Span{span}

%define \Mat to write for set of matrices
\newcommand*\Mat{\textup{Mat}}

%define \gen{...} to write <...>
\DeclarePairedDelimiterX\gen[1]\langle\rangle{
  \ifblank{#1}{\:\cdot\:}{#1}
}

\DeclareMathOperator{\Ker}{Ker} %Kernel
\DeclareMathOperator{\kernel}{ker} %kernel
\DeclareMathOperator{\image}{im}
\DeclareMathOperator{\coker}{coker}
\DeclareMathOperator*{\supp}{supp} %support
\DeclareMathOperator{\id}{id} %identity map
\DeclareMathOperator{\Id}{Id} %Identity map
\DeclareMathOperator{\ord}{ord} %order of a group
\DeclareMathOperator{\Syl}{Syl} %Sylow
\DeclareMathOperator{\GL}{GL} %GL
\DeclareMathOperator{\SL}{SL} %SL
\DeclareMathOperator{\Ad}{Ad} %Adjoint representation
\DeclareMathOperator{\ad}{ad} %adjoint representation
\DeclareMathOperator{\Char}{char} %characteristic (cannot write with small c since that is used in LaTeX already)
\DeclareMathOperator{\diag}{diag} %diagonal matrix
\DeclareMathOperator{\Tr}{Tr} %Trace
\DeclareMathOperator{\tr}{tr} %trace
\DeclareMathOperator{\rank}{rank} %rank
\DeclareMathOperator{\rk}{rk} %rank
\DeclareMathOperator{\Hom}{Hom} %Hom
\DeclareMathOperator{\homo}{hom} %hom
\DeclareMathOperator{\spec}{Spec} %spec
\DeclareMathOperator{\End}{End} %endomorphisms
\DeclareMathOperator{\Aut}{Aut} %automorphisms
\DeclareMathOperator{\Der}{Der} %derivations
\DeclareMathOperator{\Gal}{Gal} %Galois group
\DeclareMathOperator{\Frac}{Frac} %fraction field
\DeclareMathOperator{\Fr}{Fr} %Frobenius
\DeclareMathOperator{\Frob}{Frob} %Frobenius
\DeclareMathOperator{\Nm}{Nm} %norm
\DeclareMathOperator{\Mor}{Mor} %morphisms
\DeclareMathOperator{\Lie}{Lie} %Lie algebra of group scheme/Lie group
\DeclareMathOperator{\Ht}{ht} %height
\DeclareMathOperator{\uHom}{\mkern1mu\underline{\mkern-1mu Hom\mkern-1mu}\mkern1mu}
\DeclareMathOperator{\gr}{gr}
\DeclareMathOperator{\Ext}{Ext}
\DeclareMathOperator{\Tor}{Tor}
\DeclareMathOperator{\pr}{pr}
\DeclareMathOperator{\amp}{amp}
\DeclareMathOperator{\SNF}{SNF}
\DeclareMathOperator{\Fil}{Fil}
\DeclareMathOperator{\red}{red}


\newcommand*\op{^{\textup{op}}} %oposite ring/category/...
\renewcommand*\Re{\operatorname{Re}} %real part
\renewcommand*\Im{\operatorname{Im}} %imaginary part
\newcommand*{\liegp}[1]{\operatorname{#1}} %for writing Lie groups, e.g. \Lie{O}(n)
\newcommand*{\lie}[1]{\mathfrak{#1}} %for writing Lie algebras, e.g. \lie{g}
\newcommand*{\cat}[1]{\textup{\textbf{#1}}} %for writing categories, e.g. \cat{Sets}, \cat{$R$-mod},...
\newcommand*{\cats}[1]{\set{\:#1\:}} %for writing categories, e.g. \cats{$k$-algebras}, \cats{$R$-modules},...
\newcommand*\rad[1]{\sqrt{#1}} %radical
\newcommand*\cl[1]{\overline{#1}} %closure of a set
\newcommand*\rint[1]{\mathcal{O}_{#1}} %ring of integers
\newcommand*\act{\,.\,} %action of a group/Lie algebra/...
\newcommand*\rest[1]{_{\vert{#1}}} %restriction
\newcommand*\edot{\:\cdot\:} %dot for when function input is empty
\newcommand*\Sp{\textup{Sp}} %spectrum
\newcommand*\iso{\cong} %isomorphism
\newcommand*\std{_{\textup{std}}}

\newcommand*\gs[1]{\mathcal{#1}}
% \newcommand*\edot{{-}}

% \NiceMatrixOptions{cell-space-limits = 1pt}
\newcommand*\bmat[1]{\begin{bNiceMatrix} #1 \end{bNiceMatrix}}
\newcommand*\pmat[1]{\begin{pNiceMatrix} #1 \end{pNiceMatrix}}

\newcommand*\G{\mathbb{G}}
\newcommand*\T{\mathcal{T}}

\newcommand*\catF{\mathcal{F}}
\newcommand*\catG{\mathcal{G}}
\newcommand*\catP{\mathcal{P}}
\newcommand*\catD{\mathcal{D}}

\newcommand*\ZpG{\Z_p\llbracket G \rrbracket}
\newcommand*\ZpN{\Z_p\llbracket N \rrbracket}

\newcommand*\sO{\mathcal{O}}
\newcommand*\idm{\mathfrak{m}}
\newcommand*\sL{\mathcal{L}}

\newcommand\defeq{\coloneqq}

\renewcommand*\projlim{\varprojlim}

\newcommand\snfsim{\:\: \overset{\makebox[0pt]{\mbox{\normalfont\tiny\sffamily SNF}}}{\sim} \:\:}
\newcommand\Hd{H_{\mathrm{dsc}}}
\newcommand\Hc{H_{\mathrm{cts}}}
\newcommand\Cont{\mathcal{C}}

%% END : USER DEFINED COMMANDS %%


\title{On the Mod \texorpdfstring{$p$}{p} Cohomology of Pro-\texorpdfstring{$p$}{p} Iwahori Subgroups}
% \date{\today}
\author{Daniel Kongsgaard}

\makeevenhead{ruled}{\scshape\leftmark}{}{\rightmark}
\makeoddhead{ruled}{\scshape\leftmark}{}{\rightmark}
\makeevenfoot{ruled}{}{\thepage}{}
\makeoddfoot{ruled}{}{\thepage}{}
\renewcommand{\footnotesize}{\fontsize{10}{12}\selectfont}
\setlength{\parindent}{0.5in}

\setulmarginsandblock{1.6in}{*}{0.8}
\setlrmarginsandblock{1.5in}{*}{1}
\setlength{\footskip}{\lowermargin}
\addtolength{\footskip}{-0.5in}
\checkandfixthelayout

\setsecnumdepth{subsection}
\settocdepth{subsection}

%%%%%%%%%%%%%%%%%%%%%%%%%%%%%%%%% BEGIN : DOCUMENT %%%%%%%%%%%%%%%%%%%%%%%%%%%%%%%%%

\begin{document}

%\OnehalfSpacing
\DoubleSpacing

% \maketitle

\hypersetup{pageanchor=false}
\pagestyle{empty}

\hypersetup{pageanchor=false}
\begin{titlingpage}
  {
    \centering
    {\Large UNIVERSITY OF CALIFORNIA SAN DIEGO} \\[2em]
    {\Large On the mod \texorpdfstring{$p$}{p} cohomology of pro-\texorpdfstring{$p$}{p} Iwahori subgroups} \\[2em]
    {\Large A dissertation submitted in partial satisfaction of the requirements for the degree Doctor of Philosphy} \\[2em]
    {\Large in} \\[2em]
    {\Large Mathematics} \\[2em]
    {\Large by} \\[2em]
    {\Large Daniel Kongsgaard} \\[2.5em]
  }
  \noindent{\Large Committee in charge:

    Professor Claus Sørensen, Chair

    Professor Alireza Salehi Golsefidy

    Professor Kiran S.\ Kedleya

    Professor Young-Han Kim

    Professor Hans Wenzl}

  \vfill

  \begin{center}
    \Large 2022
  \end{center}
\end{titlingpage}
\hypersetup{pageanchor=true}

%%% Local Variables:
%%% mode: latex
%%% TeX-master: "main"
%%% End:


\clearpage
\null
\clearpage

\hypersetup{pageanchor=true}
\frontmatter

\pagestyle{plain}

\setcounter{page}{3}

\addcontentsline{toc}{chapter}{Dissertation Approval Page}

\noindent {\large The dissertation of Daniel Kongsgaard is approved, and it is acceptable in quality and form for publication on microfilm and electronically.}

\vfill

\begin{center}
  {\large University of California San Diego} \\[1em]
  {\large 2002}
\end{center}

\clearpage

\renewcommand{\contentsname}{Table of Contents}
\renewcommand{\cftchaptername}{\chaptername~}
\renewcommand{\cftappendixname}{\appendixname~}
\tableofcontents

\clearpage

\printnomenclature[4cm]
% To update call: makeindex main.nlo -s nomencl.ist -o main.nls

\clearpage

\listoftables

\clearpage

\addcontentsline{toc}{chapter}{Acknowledgements}

\begin{center}
  \Large\scshape Acknowledgements
\end{center}

I would like to acknowledge Professor Claus Sørensen for giving me interesting number theory research topics, and for his input and ideas during my research. I also thank him for his feedback on early drafts of this dissertation.

I would also like to acknowledge Daniel M.\ Kane for introducing me to research problems in robust statistics and his input and later collaboration on this research. In addition I would like to thank my other collaborators Ilias Diakonikolas, Jerry Li, and Kevin Tian.

\clearpage

\addcontentsline{toc}{chapter}{Vita}


\begin{center}
  \Large\scshape Vita
\end{center}
\begin{description}
  \item[2016--2017] Teaching Assistant, Aarhus University
  \item[2017] Bachelor of Science in Mathematics, Aarhus University
  \item[2021] Candidate in Philosophy in Mathematics, University of California San Diego
  \item[2017--2022] Teaching Assistant, University of California San Diego
  \item[2022] Doctor of Philosophy in Mathematics, University of California San Diego
\end{description}

\begin{center}
  \Large\scshape Publications (all joint)
\end{center}
\begin{enumerate}[$\bullet$]
  \item Ilias Diakonikolas, Daniel Kane, and Daniel Kongsgaard. 2020. List-Decodable Mean Estimation via Iterative Multi-Filtering. In \emph{Advances in Neural Information Processing Systems}, H. Larochelle, M. Ranzato, R. Hadsell, M.F. Balcan and H. Lin (Eds.), Vol. 33. Curran Associates, Inc., 9312--9323.

  \item Ilias Diakonikolas, Daniel Kane, Daniel Kongsgaard, Jerry Li, and Kevin Tian. 2021. List-Decodable Mean Estimation in Nearly-PCA Time. In \emph{Advances in Neural Information Processing Systems}, M. Ranzato, A. Beygelzimer, Y. Dauphin, P.S. Liang, and J. Wortman Vaughan (Eds.), Vol. 34. Curran Associates, Inc., 10195--10208.

  \item Ilias Diakonikolas, Daniel Kane, Daniel Kongsgaard, Jerry Li, and Kevin Tian. 2021. Clustering Mixture Models in Almost-Linear Time via List-Decodable Mean Estimation. (Accepted for ACM Symposium on Theory of Computing (STOC 2022).)
\end{enumerate}

% \begin{center}
%   \Large\scshape Fields of study
% \end{center}

% \begin{vplace}[0.7]
%   Some extra stuff. Acknowledgements, vita, abstact.
% \end{vplace}

\clearpage

\addcontentsline{toc}{chapter}{Abstract of the Dissertation}
% \renewcommand{\abstractnamefont}{\normalfont\Large\scshape}
% \renewcommand{\abstractname}{Abstract of the Dissertation}
% \begin{abstract}
%   Let $\gs{G}$ be a split and connected reductive $\Z_{p}$-group and let $\gs{N}$ be the unipotent radical of a Borel subgroup. In the first chapter of this dissertation we study the cohomology with trivial $\F_{p}$-coefficients of the unipotent pro-$p$ group $N = \gs{N}(\Z_{p})$ and the Lie algebra $\lie{n} = \Lie(\gs{N}_{\F_{p}})$. We proceed by arguing that $N$ is a $p$-valued group using ideas of Schneider and Zábrádi, which by a result of Sørensen gives us a spectral sequence $E_{1}^{s,t} = H^{s,t}(\lie{g},\F_{p}) \Longrightarrow H^{s+t}(N,\F_{p})$, where $\lie{g} = \F_{p} \otimes_{\F_{p}[\pi]} \gr N$ is the graded $\F_{p}$-Lie algebra attached to $N$ as in Lazards work. We then argue that $\lie{g} \iso \lie{n}$ by looking at the Chevalley constants, and, using results of Polo and Tilouine and ideas from Große-Klönne, we show that the dimensions of the $\F_{p}$-cohomology of $\lie{n}$ and $N$ agree, which allows us to conclude that the spectral sequence collapses on the first page.

%   In the second chapter we study the mod $p$ cohomology of the pro-$p$ Iwahori subgroups $I$ of $\SL_{n}$ and $\GL_{n}$ over $\Q_{p}$ for $n=2,3,4$ and over a quadratic extension $F/\Q_{p}$ for $n=2$. Here we again use the spectral sequence $E_{1}^{s,t} = H^{s,t}(\lie{g},\F_{p}) \Longrightarrow H^{s+t}(I,\F_{p})$ due to Sørensen, but in this chapter we do explicit calculations with an ordered basis of $I$, which gives us a basis of $\lie{g} = \F_{p} \otimes_{\F_{p}[\pi]} I$ that we use to calculate $H^{s,t}(\lie{g},\F_{p})$. We note that the spectral sequence $E_{1}^{s,t} = H^{s,t}(\lie{g},\F_{p})$ collapses on the first page by noticing that all maps on each page are necessarily trivial. Finally we note some connections to cohomology of quaternion algebras over $\Q_{p}$ and point out some future research directions.
% \end{abstract}

\begin{center}
  {\Large\scshape Abstract of the Dissertation}\\[1.5em]
  {\large On the mod \texorpdfstring{$p$}{p} cohomology of pro-\texorpdfstring{$p$}{p} Iwahori subgroups}\\[1.5em]
  by \\[1.5em]
  Daniel Kongsgaard \\[1em]
  Doctor of Philosophy in Mathematics \\[1em]
  University of California San Diego \\[1.5em]
  Professor Claus Sørensen, Chair \\[2em]
\end{center}

Let $\gs{G}$ be a split and connected reductive $\Z_{p}$-group and let $\gs{N}$ be the unipotent radical of a Borel subgroup. In the first chapter of this dissertation we study the cohomology with trivial $\F_{p}$-coefficients of the unipotent pro-$p$ group $N = \gs{N}(\Z_{p})$ and the Lie algebra $\lie{n} = \Lie(\gs{N}_{\F_{p}})$. We proceed by arguing that $N$ is a $p$-valued group using ideas of Schneider and Zábrádi, which by a result of Sørensen gives us a spectral sequence $E_{1}^{s,t} = H^{s,t}(\lie{g},\F_{p}) \Longrightarrow H^{s+t}(N,\F_{p})$, where $\lie{g} = \F_{p} \otimes_{\F_{p}[\pi]} \gr N$ is the graded $\F_{p}$-Lie algebra attached to $N$ as in Lazards work. We then argue that $\lie{g} \iso \lie{n}$ by looking at the Chevalley constants, and, using results of Polo and Tilouine and ideas from Große-Klönne, we show that the dimensions of the $\F_{p}$-cohomology of $\lie{n}$ and $N$ agree, which allows us to conclude that the spectral sequence collapses on the first page.

In the second chapter we study the mod $p$ cohomology of the pro-$p$ Iwahori subgroups $I$ of $\SL_{n}$ and $\GL_{n}$ over $\Q_{p}$ for $n=2,3,4$ and over a quadratic extension $F/\Q_{p}$ for $n=2$. Here we again use the spectral sequence $E_{1}^{s,t} = H^{s,t}(\lie{g},\F_{p}) \Longrightarrow H^{s+t}(I,\F_{p})$ due to Sørensen, but in this chapter we do explicit calculations with an ordered basis of $I$, which gives us a basis of $\lie{g} = \F_{p} \otimes_{\F_{p}[\pi]} I$ that we use to calculate $H^{s,t}(\lie{g},\F_{p})$. We note that the spectral sequence $E_{1}^{s,t} = H^{s,t}(\lie{g},\F_{p})$ collapses on the first page by noticing that all maps on each page are necessarily trivial. Finally we note some connections to cohomology of quaternion algebras over $\Q_{p}$ and point out some future research directions.

\clearpage

\mainmatter

\pagestyle{ruled}

\chapter{Introduction}%
\label{cha:intro}

The cohomology of Lie groups has a long history. In the late forties Chevalley and Eilenberg found that $H^{*}(G,\R) \iso H^{*}(\lie{g},\R)$ for a connected compact Lie group $G$ with Lie algebra $\lie{g}$ (cf.\ \cite{Chev}), and since then there has been much research into different types of Lie group cohomology.  In particular, the mod $p$ cohomology of a connected compact real Lie group has been well understood by Kac since the eighties (cf.\ \cite{Kac}), and the continuous mod $p$ cohomology $H^*(G,\F_p)$ of an equi-$p$-valued compact $p$-adic Lie group $G$ was already described by Lazard in the sixties (cf.\ \cite{Laz}). We note here that (except for Lazard's work) $H^{*}(G,\R)$ and $H^{*}(G,\F_{p})$ indicate the cohomology of $G$ as a topological space, and not continuous group cohomology, which can be thought of as the cohomology of the classifying space $BG$.

This dissertation's main interest is the continuous mod $p$ cohomology $H^{*}(G,\F_{p})$ of compact $p$-adic Lie groups $G$ for specific cases of $G$. Since $p$-adic Lie groups are totally disconnected, working with them requires very different methods than what Chevalley and Eilenberg or Kac used for real Lie groups, and we have to follow the ideas of Lazard (see \cite{Laz}) and Serre. In particular we need a $p$-valuation on $G$ (and on the completed group algebras associated with $G$), and we work with the graded \enquote{Lazard} Lie algebra $\lie{g} = \F_{p} \otimes_{\F_{p}[\pi]} \gr G$ attached to $G$. We will repeatedly use that Sørensen (in \cite{Sor}) showed that $H^{*}(\lie{g},\F_{p})$ determines $H^{*}(G,\F_{p})$ via a multiplicative spectral sequence \[ E_{1}^{s,t} = H^{s,t}(\lie{g},\F_{p}) \Longrightarrow H^{s+t}(G,\F_{p}). \] When $G$ is equi-$p$-valuable, we get that $\lie{g}$ is concentrated in a single degree, and Lazard showed that $H^{*}(G,\F_{p}) \iso \bigwedge H^{1}(\lie{g},\F_{p})$, while Sørensen showed that this also follows from the above spectral sequence. We are interested in cases where $G$ is \emph{not} equi-$p$-valuable, and we note that the spectral sequence of Sørensen allows us to work purely with $G$ and $\lie{g}$ without having to worry about the completed group algebras $\Lambda(G) = \Z_{p}\llbracket G \rrbracket$ and $\Omega(G) = \F_{p}\llbracket G \rrbracket$.

Before describing our particular results in the following paragraph, we emphasize the following remark of Sørensen from \cite{Sor}: It is known (due to Lazard) that any compact $p$-adic Lie group contains an open equi-$p$-valuable subgroup (see \cite[Chap.~V~2.2.7.1]{Laz}), which gives the impression that the distinction between $p$-valued and equi-$p$-valued groups is somewhat nuanced, which is true for some questions. But there are many examples of naturally occurring $p$-valuable groups $G$ which are not equi-$p$-valuable, where detailed information about $H^{*}(G,\F_{p})$ is important. For example unipotent groups (i.e., the $\Z_{p}$-points of the unipotent radical of a Borel in a split reductive group), Serre's standard groups with $e>1$ as in \cite[Lem.~2.2.2]{Laz-complements}, pro-$p$ Iwahori subgroups for large enough $p$, and $1 + \idm_{D}$ where $D$ is the quaternion division algebra over $\Q_{p}$ for $p > 3$ (or more generally a central division algebra over $\Q_{p}$). Sørensen explicitly calculates $H^{*}\bigl( (1+\idm_{D})^{\Nrd = 1} , \F_{p} \bigr)$ for $p>3$ and uses it to describe $H^{*}(1+\idm_{D},\F_{p})$, and he notes that $1+\idm_{D}$ plays an important role both in number theory (in the Jacquet-Langlands correspondence for instance, see \cite{JL}) and algebraic topology, where $1+\idm_{D}$ is known as the (strict) Morava stabilizer in stable homotopy theory, and $H^{*}(1+\idm_{D},\F_{p})$ somehow controls certain localization functors with respect to Morava $K$-theory (see e.g.\ \cite{Henn}).

Our work in \Cref{cha:cohunigps} will build on ideas of Lazard and Serre from their more general (but not yet finished) description of the case when $G$ is not equi-$p$-valued, and especially the refinement of these ideas as described by Sørensen and Schneider in \cite{Sor} and \cite{Sch-notes}. We will focus on unipotent groups $N$ originating from split and connected reductive $\Z_p$-groups, which is similar to recent work in the case of $\Z_p$ coefficients by Ronchetti (cf.\ \cite{Ron}). We note that this work can be considered a slight refinement of \cite{GK} since we retain information about the cup product on $H^{*}(N,\F_{p})$.

In \Cref{cha:cohiwagps} we focus on the case of pro-$p$ Iwahori subgroups of $\SL_{n}$ and $\GL_{n}$ over $\Q_{p}$ for $n=2,3,4$ or over quadratic extensions $F/\Q_{p}$ for $n=2$. We explicitly calculate the algebra structure of $H^{*}(I,\F_{p})$ for the pro-$p$ Iwahori subgroups $I_{\SL_{2}(\Q_{p})} \subseteq \SL_{2}(\Z_{p})$ and $I_{\GL_{2}(\Q_{p})} \subseteq \GL_{2}(\Z_{p})$, and we note that these are isomorphic as algebras to $H^{*}\bigl( (1+\idm_{D})^{\Nrd = 1},\F_{p} \bigr)$ and $H^{*}(1+\idm_{D},\F_{p})$ respectively. We finish the chapter by mentioning some future research directions and a conjecture on the connection between the mod $p$ cohomology of $(1+\idm_{D})^{\Nrd = 1}$ (resp.\ $1+\idm_{D}$) for central division algebras and $I_{\SL_{n}(\Q_{p})}$ (resp.\ $I_{\GL_{n}(\Q_{p})}$).

Finally, the appendix will end with a very brief description of other research (all joint) that I have participated in.


%%% Local Variables:
%%% mode: latex
%%% TeX-master: "../main"
%%% End:


\chapter{Cohomology of Unipotent Groups}%
\label{cha:cohunigps}

\section{Introduction}%
\label{sec:cohunigps-intro}

In this chapter we show that the cohomology of certain unipotent groups can be found via a simpler cohomology calculation for related Lie algebras. This is done using a spectral sequence due to \cite{Sor}.

\subsection{Background and motivation}

% This chapter's main interest is a special case of the continuous mod $p$ cohomology $H^{*}(G,\F_{p})$ of a compact $p$-adic Lie group $G$. Our work will build on several ideas of Lazard and Serre in their more general (but not yet finished) description of the case when $G$ is not equi-$p$-valued, but we will focus only on unipotent groups originating from split and connected reductive $\Z_p$-groups, which is similar to recent work in the case of $\Z_p$ coefficients by Ronchetti (cf.\ \cite{Ron}).

As mentioned in \Cref{cha:intro}, we will focus on describing the continuous mod $p$ cohomology of unipotent groups $N$ originating from split and connected reductive $\Z_p$-groups in this chapter. To be precise, let $\gs{N}$ be the unipotent radical of a Borel in a split and connected reductive $\Z_{p}$-group, and let $N = \gs{N}(\Z_{p})$ be the $\Z_{p}$-points of $\gs{N}$. In this chapter we will show that $N$ is a $p$-valuable group (with a nice $p$-valuation), which will allow us to show that the Lazard Lie algebra $\lie{g} = \F_{p} \otimes_{\F_{p}[\pi]} \gr N$ is isomorphic to $\lie{n} = \Lie(\gs{N}_{\F_{p}})$. Using an idea of Große-Klönne (cf.\ \cite[Sect.~7]{GK}) we get that discrete mod $p$ cohomology of $\gs{N}_{\Z}(\Z)$ is isomorphic to the continuous mod $p$ cohomology of $N = \gs{N}(\Z_{p})$, and results of Polo and Tilouine (see \cite{PT}) allow us to compare the discrete cohomology of $\lie{n}_{\Z} = \Lie(\gs{N}_{\Z})$ and $\gs{N}_{\Z}(\Z)$, which by a short argument allows us to compare the dimensions of the continuous mod $p$ cohomology of $N$ and the mod $p$ cohomology of $\lie{g} \iso \lie{n}$. This will let us conclude that the multiplicative spectral sequence \[ E_{1}^{s,t} = H^{s,t}(\lie{g},\F_{p}) \Longrightarrow H^{s+t}(N,\F_{p}) \] due to Sørensen collapses at the first page, which gives us a description of $H^{*}(N,\F_{p})$.

It is worth noting that this work started out as an attempt to better understand the proof of \cite[Theorem~7.1]{GK}, in particular the part using the result of Grünenfelder (which by a remark of Polo and Tilouine might have a problem), but the work has since develop in a different direction, where the coefficients are more restricted, but we obtain a more precise (or indeed any) description of the cup product.

\subsection{Notation and setup}

Let $p$ be an odd prime (that will be further restricted later).

\paragraph{Algebraic groups.} We will work with schemes using the functorial approach and notation described in \cite{Jan}. In particular, given an integral domain $R$, we note that a \emph{$R$-group functor}\index{R-group@$R$-group!functor} is a functor from the category of all $R$-algebras to the category of groups, a \emph{$R$-group scheme}\index{R-group!scheme} is a $R$-group functor that is an affine scheme over $R$ when considered as a $R$-functor, and an \emph{algebraic $R$-group}\index{R-group!algebraic}\index{algebraic R-group@algebraic $R$-group} is a $R$-group scheme that is algebraic as an affine scheme. For more in depth introduction to these concepts, we refer to \cite{Con-book} and \cite{Jan}.

\paragraph{Base change.} If $R'$ is a $R$-algebra, then any $R'$-algebra $A$ is in a natural way a $R$-algebra by combining the structural homomorphisms $R \to R'$ and $R' \to A$. We can therefore associate to each $R$-functor $X$ a $R'$-functor $X_{R'}$ by $X_{R'}(A) = X(A)$ for any $R'$-algebra $A$. For any morphism $f \colon X \to X'$ of $R$-functors, we get a morphism $f_{R'} \colon X_{R'} \to X'_{R'}$ of $k'$-functors by $f_{R'}(A) = f(A)$ for any $R'$-algebra $A$. In this way we get a functor $X \mapsto X_{R'}$, $f \mapsto f_{R'}$ from the category of $R$-functors to the category of $R'$-functors, which we call the \emph{base change}\index{base change} from $R$ to $R'$.

\paragraph{Fixed $\Z_{p}$-groups and roots.} We fix a split and connected reductive algebraic $\Z_{p}$-group $\gs{G}$\nomuni[G]{$\gs{G}$}{a (fixed) split and connected reductive algebraic $\Z_{p}$-group}\index{G@$\gs{G}$} as well as a split maximal torus $\gs{T} \subseteq \gs{G}$.\nomuni[T]{$\gs{T}$}{a (fixed) split maximal torus of $\gs{G}$} Let $\Phi = \Phi(\gs{G},\gs{T})$\nomuni[Phi]{$\Phi$}{${} = \Phi(\gs{G},\gs{T})$, the root system of $\gs{G}$ with respect to $\gs{T}$} be the \emph{root system}\index{root!system} of $\gs{G}$ with respect to $\gs{T}$. For any $\alpha \in \Phi$ we have the root subgroup\index{root!subgroup} $\gs{N}_{\alpha} \subseteq \gs{G}$ with Lie algebra $\Lie \gs{N}_{\alpha} =  (\Lie \gs{G})_{\alpha}$. We fix a $\Z_{p}$-basis $(X_{\alpha})_{\alpha \in \Phi}$ of $\Lie \gs{N}_{\alpha}$, and note that this choice gives rise to unique isomorphisms of group schemes $x_{\alpha} \colon \G_{a} \xrightarrow{\iso} \gs{N}_{\alpha}$ such that $(dx_\alpha)(1) = X_\alpha$. We furthermore fix a basis $\Delta \subseteq \Phi$\nomuni[Delta]{$\Delta$}{a (fixed) basis of the root system $\Phi$} of the root system,\index{root system} so we get a decomposition $\Phi = \Phi^+ \cup \Phi^-$\nomuni[Phi+]{$\Phi^{+}$ / $\Phi^{-}$}{the positive/negative roots in $\Phi$ with respect to $\Delta$} into positive and negative roots. Let $\gs{B} = \gs{T}\gs{N}$\nomuni[B]{$\gs{B}$ / $\gs{B}^{+}$}{(${} = \gs{T}\gs{N}$ / ${} = \gs{T}\gs{N}^{+}$) the Borel subgroups of $\gs{G}$ corresponding to $\Phi^{-}$ / $\Phi^{+}$} and $\gs{B}^+ = \gs{T}\gs{N}^+$ denote the Borel subgroups of $\gs{G}$ corresponding to $\Phi^-$ and $\Phi^+$, respectively, with unipotent radicals $\gs{N}$ and $\gs{N}^+$.\nomuni[N]{$\gs{N}$ / $\gs{N}^{+}$}{the unipotent radical of $\gs{B}$ / $\gs{B}^{+}$}\index{N@$\gs{N}$} Finally let $N = \gs{N}(\Z_{p})$ and let $\lie{n} = \Lie(\gs{N}_{\F_p})$ be the Lie algebra of $\gs{N}_{\F_p}$ over $\F_p$.

\paragraph{$\Z$-models.} Let $\gs{G}_\Z$ be the Chevalley group\index{Chevalley group} over $\Z$ corresponding to $\gs{G}$ (cf.\ \cite[§1]{Con}), and consider the subgroups $\gs{T}_{\Z},\gs{B}_{\Z},\gs{N}_{\Z}$ corresponding to $\gs{T},\gs{B},\gs{N}$. Let furthermore $\lie{n}_\Z = \Lie(\gs{N}_\Z)$ be the Lie algebra of $\gs{N}_\Z$ over $\Z$, and note that $N = \gs{N}_{\Z}(\Z_{p})$ and $\lie{n} =  \lie{n}_{\Z} \otimes \F_{p}$. (Note also that $(\gs{G}_{\Z})_{\Z_{p}} = \gs{G}$, so although we abuse notation a bit here, it wont be a problem.)

\paragraph{Total ordering of $\Phi^{-}$.} For any total ordering\index{total ordering} of $\Phi^-$ the multiplication induces an isomorphism of schemes $\prod_{\alpha \in \Phi^-} \gs{N}_\alpha \xrightarrow{\iso} \gs{N}$. For convenience we fix a total ordering which has the additional property that $\alpha_1 \geq \alpha_2$ if $\Ht(\alpha_1) \leq \Ht(\alpha_2)$. All products indexed by $\Phi^-$ are meant to be taken according to this ordering. Here we have the height function $\Ht \colon \Z[\Delta] \to \Z$ given by $\sum_{\alpha \in \Delta} m_\alpha \alpha \mapsto \sum_{\alpha \in \Delta} m_\alpha$. In particular, since $\Phi \subseteq \Z[\Delta]$ the height $\Ht(\beta)$ of any root $\beta \in \Phi$ is defined.

\paragraph{Coxeter number and $p$.} Let $h$\nomuni[h]{$h$}{the Coxeter number of $\gs{G}$} be the Coxeter number of $\gs{G}$ and assume from now on that $p \geq h-1$.\nomuni[p]{$p$}{a prime, $p \geq h-1$, where $h$ is the Coxeter number of $\gs{G}$}

\paragraph{Weyl group and module.} Let $\Phi^{\vee}$\nomuni[Phiv]{$\Phi^{\vee}$}{the dual root system of $\Phi$} be the dual root system\index{root system!dual} of $\Phi$ and let $W$\nomuni[W]{$W$}{the Weyl group corresponding to $\Phi$ and $\Phi^{\vee}$} be the corresponding Weyl group\index{Weyl!group} with length function\index{length} $\ell$ on $W$. Let furthermore $X = X(\gs{T}) \iso X(\gs{T}_\Z)$\nomuni[X]{$X$}{$=X(\gs{T}) \iso X(\gs{T}_{\Z})$, the character group of $\gs{T}$} be the character group\index{character group} of $\gs{T}$, and set
\begin{equation*}
  X^{+} = \set{\lambda \in X \given \inner{\lambda}{\alpha^\vee} \geq 0 \text{ for all } \alpha \in \Phi^+}.
\end{equation*}%
\nomuni[X+]{$X^{+}$}{${} = \set{\lambda \in X \given \inner{\lambda}{\alpha^\vee} \geq 0 \text{ for all } \alpha \in \Phi^+}$}%
For any $\lambda \in X^+$, let $V_{\Z}(\lambda)$\nomuni[VZlambda]{$V_{\Z}(\lambda)$}{the Weyl module for $\gs{G}_{\Z}$ over $\Z$ with highest weight $\lambda$} be the Weyl module\index{Weyl!module} for $\gs{G}_\Z$ over $\Z$ with highest weight $\lambda$, and let $V_{\F_{p}}(\lambda) = V_{\Z}(\lambda) \otimes_\Z \F_{p}$.\nomuni[VFplambda]{$V_{\F_{p}}(\lambda)$}{${} = V_{\Z}(\lambda) \otimes_{\Z} \F_{p}$}

\paragraph{Lazard theory.} We will introduce concepts from Lazard theory in next subsection, but we note now that we will let $\lie{g} = \F_p \otimes_{\F_p[\pi]} \gr N$\nomuni[g]{$\lie{g}$}{${} = \F_{p} \otimes_{\F_{p}[\pi]} \gr N$, the Lazard Lie algebra corresponding to $N$} be the Lazard Lie algebra corresponding to $N$.

\paragraph{Cohomology.} For any ring $R$, we denote (using the Chevalley-Eilenberg complex) the Lie algebra cohomology\index{cohomology!Lie algebra} of any $R$-Lie algebra $\lie{g}$ by $H^{\bullet}(\lie{g}, \edot)$,\nomuni[H]{$H^{\bullet}(\lie{g},\edot)$}{the cohomology of the Lie algebra $\lie{g}$} while we write $\Hd^{\bullet}(G, \edot)$\nomuni[Hdsc]{$\Hd^{\bullet}(G, \edot)$}{the discrete group cohomology of a topological group $H$} and $\Hc^{\bullet}(H, \edot)$\nomuni[Hcts]{$\Hc^{\bullet}(G, \edot)$}{the continuous group cohomology of a topological group $G$} for the discrete (resp.\ continuous) group cohomology of a topological group $G$. Later we will introduce filtrations and then gradings on the cohomology, in which case we always use the notation $H^{s,t} = \gr^{s}H^{s+t}$\nomuni[Hst]{$H^{s,t}$}{${} = \gr^{s}H^{s+t}$ for some cohomology $H$} for any type of cohomology $H$.

\paragraph{Spectral sequences.} Given a ring $R$, a cohomological spectral sequence\index{spectral sequence} is a choice of $r_0 \in \N$ and a collection of
\begin{enumerate}[$\bullet$]
  \item $R$-modules $E_r^{s,t}$ for each $s,t \in \Z$ and all integers $r \geq r_0$
  \item differentials $d_r^{s,t} \colon E_r^{s,t} \to E_r^{s+r,t+1-r}$ such that $d_r^2 = 0$ and $E_{r+1}$ is isomorphic to the homology of $(E_r,d_r)$, i.e.,
  \[
    E_{r+1}^{s,t} = \frac{\kernel(d_r^{s,t} \colon E_r^{s,t} \to E_r^{s+r,t+1-r})}{\image(d_r^{s-r,t+r-1} \colon E_r^{s-r,t+r-1} \to E_r^{s,t})}.
  \]
\end{enumerate}
For a given $r$, the collection $(E_r^{s,t},d_r^{s,t})_{s,t\in\Z}$ is called the $r$-th page. A spectral sequence \emph{converges}\index{spectral sequence!convergent} if $d_r$ vanishes on $E_r^{s,t}$ for any $s,t$ when $r\gg0$. In this case $E_r^{s,t}$ is independent of $r$ for sufficiently large $r$, we denote it by $E_{\infty}^{s,t}$ and write
  \[
    E_{r}^{s,t} \Longrightarrow E_\infty^{s+t}.
  \]
Also, we say that the spectral sequence collapses at the $r'$-th page if $E_{r} = E_{\infty}$ for all $r \geq r'$, but not for $r < r'$. Finally, when we have terms $E_\infty^{n}$  with a natural filtration $F^\bullet E_\infty^n$ (but no natural double grading), we set $E_\infty^{s,t} = \gr^{s} E_\infty^{s+t}= F^{s}E_\infty^{s+t}/F^{s+1}E_\infty^{s+t}$.



% Let furthermore $\rho$ be the half-sum of the elements of $\Phi^+$, let $X = X(\gs{T}) \iso X(\gs{T}_\Z)$ be the character group of $\gs{T}$, let
% \begin{equation*}
%   X^+ = \set{\lambda \in X \given \inner{\lambda}{\alpha^\vee} \geq 0 \text{ for all } \alpha \in \Phi^+}.
% \end{equation*}
% and let $h$ be the Coxeter number of $\gs{G}$ and assume from now on that $p \geq h-1$.
% and let
% \begin{equation*}
%   \cl{C}_p = \set{\lambda \in X \given 0 \leq \inner{\lambda + \rho}{\beta^\vee} \leq p \text{ for all } \beta\in\Phi^+}.
% \end{equation*}
% For any $\lambda \in X^+$, let $V_\Z(\lambda)$ be the Weyl module for $\gs{G}_\Z$ over $\Z$ with highest weight $\lambda$, and let $V_k(\lambda) = V_\Z(\lambda) \otimes_\Z k$.

% Let $\Phi^\vee$ be the dual root system of $\Phi$ and let $W$ be the corresponding Weyl group with length function $\ell$ on $W$. Let $\lie{n}_\Z = \Lie(\gs{N}_\Z)$ be the Lie algebra of $\gs{N}_\Z$ over $\Z$ and $\lie{n} = \lie{n}_{\F_p} = \Lie(\gs{N}_{\F_p}) = \lie{n}_\Z \otimes \F_p$ be the Lie algebra of $\gs{N}_{\F_p}$ over $\F_p$.

% Finally let $N = \gs{N}(\Z_p) = \gs{N}_\Z(\Z_p)$ and let $\lie{g} = \F_p \otimes_{\F_p[\pi]} \gr G$.

\subsection{Lazard theory}%
\label{subsec:Laz-theory}

In this subsection we will briefly introduce elements of Lazard theory as presented in \cite{Sch}.

Let $G$ be any abstract group and let the commutator be normalized as $[g,h]=ghg^{-1}h^{-1}$.

\begin{definition}
  A \emph{$p$-valuation}\index{p-valuation@$p$-valuation} $\omega$ on $G$ is a real valued function
  \begin{equation*}
    \omega \colon G \setminus \set{1} \to \open{0}{\infty}
  \end{equation*}%
  \nomuni[omega]{$\omega \colon G \setminus \set{1} \to \open{0}{\infty}$}{a $p$-valuation on $G$}%
  which, with the convention that $\omega(1)=\infty$, satisfies
  \begin{enumerate}[(a)]
    \item $\omega(g) > \tfrac{1}{p-1}$,
    \item $\omega(g^{-1}h) \geq \min(\omega(g),\omega(h))$,
    \item $\omega([g,h]) \geq \omega(g) + \omega(h)$,
    \item $\omega(g^p) = \omega(g)+1$
  \end{enumerate}
  for any $g,h\in G$.
\end{definition}

For the rest of this subsection, let $(G,\omega)$ be a $p$-valued group\index{p-valued group@$p$-valued group}, i.e., a group with a $p$-valuation.

For any real number $\nu>0$ put
\begin{equation*}
  G_{\nu} \coloneqq \set{g\in G : \omega(g)\geq \nu} \quad \text{ and } \quad G_{\nu+} \coloneqq \set{g\in G : \omega(g)>\nu},
\end{equation*}%
\nomuni[Gnu]{$G_{\nu}$}{${}\coloneqq \set{g\in G : \omega(g)\geq \nu}$}%
\nomuni[Gnuplus]{$G_{\nu+}$}{${}\coloneqq \set{g\in G : \omega(g)>\nu}$}%
and note that these are normal subgroups, cf.\ \cite[Sect.~23]{Sch}.

The subgroups $G_{\nu}$ form a decreasing exhaustive and separated filtration of $G$ with the additional properties
\begin{equation*}
  G_{\nu} = \bigcap_{\nu'<\nu}G_{\nu'} \quad \text{ and } \quad [G_\nu,G_{\nu'}] \subseteq G_{\nu+\nu'}.
\end{equation*}
There is a unique (Hausdorff) topological group structure on $G$ for which the $G_\nu$ form a fundamental system of open neighborhoods of the identity element. It will be called the \emph{topology defined by $\omega$}.\index{topology defined by omega@topology!defined by $\omega$} We will assume that $G$ is profinte in the topology defined by $\omega$. Hence $G = \projlim_{\nu > 0} G/G_{\nu}$ as topological groups, and thus $G$ must be a pro-$p$-group\index{pro-p group@pro-$p$ group} since $\omega(g^{p}) = \omega(g) + 1$ implies that $G/G_{\nu}$ is a $p$-group (finite since $G_{\nu}$ is open).

We now form, for each $\nu>0$, the subquotient group
\begin{equation*}
  \gr_\nu G \coloneqq G_\nu/G_{\nu+}.
\end{equation*}%
\nomuni[grnuG]{$\gr_\nu G$}{${}\coloneqq G_\nu/G_{\nu+}$}%
It is commutative by (c) and therefore will be denoted additively. We now consider the graded abelian group
\begin{equation*}
  \gr G \coloneqq \bigoplus_{\nu>0} \gr_\nu G.
\end{equation*}%
\nomuni[grG]{$\gr G$}{${} \coloneqq \bigoplus_{\nu>0} \gr_\nu G$ (a graded Lie algebra over $\F_{p}[\pi]$)}%
An element $\xi \in \gr G$ is called, as usual, homogeneous\index{homogeneous} (of degree $\nu$) if it lies in $\gr_\nu G$. Furthermore, in this case any $g\in G_\nu$ such that $\xi = gG_{\nu+}$ is called a representative of $\xi$.

Note that $p\xi = 0$ for any homogeneous element $\xi \in \gr G$ since $\omega(g^{p}) = \omega(g) + 1$. Hence $\gr G$ in fact is an $\F_p$-vector space. Furthermore, by bilinear extension of the map
\begin{align*}
  \gr_\nu G \times \gr_{\nu'} G &\to \gr_{\nu+\nu'} G \\
  (\xi,\eta) &\mapsto [\xi,\eta]\coloneqq [g,h]G_{\nu+\nu'}+,
\end{align*}
for $\nu,\nu'>0$, we obtain a graded $\F_p$-bilinear map
\begin{equation*}
  [\edot,\edot] \colon \gr G \times \gr G \to \gr G
\end{equation*}
which satisfies
\begin{equation*}
  [\xi,\xi]=0 \qquad \text{for any }\xi\in\gr G.
\end{equation*}
One can check that $[\edot,\edot]$ satisfies the Jacobi identity, and thus $\gr G$ is a graded Lie algebra over $\F_p$, cf.\ \cite[Sect.~23]{Sch}.

Now, noticing that the map
\begin{align*}
  \gr_{\nu} G &\to \gr_{\nu+1} G \\
  gG_{\nu+} &\mapsto g^pG_{(\nu+1)+}
\end{align*}
is well defined and $\F_p$-linear, by considering for varying $\nu$ the direct sum of these maps, we can introduce an $\F_p$-linear map of degree one
\begin{equation*}
  \pi \colon \gr G \to \gr G.
\end{equation*}%
\nomuni[pi]{$\pi \colon \gr G \to \gr G$}{the direct sum of the maps $gG_{\nu+} \mapsto g^pG_{(\nu+1)+}$}%
We can and will therefore view $\gr G$ as a graded module over the polynomial ring $\F_p[\pi]$ in one variable over $\F_p$. Furthermore the Lie bracket on $\gr G$ is bilinear for the $\F_p[\pi]$-module structure, i.e., $\gr G$ is a Lie algebra over the ring $\F_p[\pi]$. For more details, we refer to \cite[Sect.~25]{Sch}.
\begin{definition}
  The pair $(G,\omega)$ is called of finite rank\index{p-valued group!finite rank} if $\gr G$ is finitely generated as an $\F_p[\pi]$-module.
\end{definition}
Note that $G$ being of finite rank does not depend on the choice of the $p$-valuation, and assume from now on that $(G,\omega)$ is of finite rank. Note that $\gr G$ is finitely generated and torsionfree over the principal ideal domain $\F_{p}[\pi]$, and thus by the elementary divisor theorem  $\gr G$ is free. We call
\begin{equation*}
  \rank(G,\omega) \coloneqq \rank_{\F_p[P]} \gr G
\end{equation*}%
\nomuni[rankGomega]{$\rank(G,\omega)$}{${}\coloneqq \rank_{\F_p[P]} \gr G$ the rank of the pair $(G,\omega)$}%
the \emph{rank}\index{p-valued group!rank} of the pair $(G,\omega)$.

For any $g\in G$ note that we then have a group homomorphism
\begin{align*}
  c\colon \Z &\to G\\
  m &\mapsto g^m.
\end{align*}
Since $G/N$, for any $N \triangleleft G$, is a $p$-group, we obtain $c^{-1}(N) = p^{a_N}\Z$ for some $a_N\geq0$. It follows that $c$ extends uniquely to a continuous group homomorphism
\begin{equation*}
  \tilde{c} \colon \Z_p \to \projlim_{N \triangleleft G} \Z/p^{a_N}\Z \overset{c}{\longrightarrow} \projlim_N G/N = G
\end{equation*}
which we always will write as $g^x \coloneqq \tilde{c}(x)$. More generally, for any finitely many elements $g_1,\dotsc,g_r\in G$, we have the continuous map
\begin{equation}\label{eq:ZprtoG}
  \begin{aligned}
    \Z_p^r &\to G \\
    (x_1,\dotsc,x_r) &\mapsto g_1^{x_1}\dotsb g_r^{x_r}
  \end{aligned}
\end{equation}
which depends on the order of the $g_i$ and therefore is not a group homomorphism. However we introduce the following notation, where $v_{p}$ denotes the usual $p$-adic valuation on $\Q_p$.
\begin{definition}
  The sequence of elements $(g_1,\dotsc,g_r)$ in $G$ is called an \emph{ordered basis} of $(G,\omega)$\index{p-valued group!ordered basis} if the map \eqref{eq:ZprtoG} is a bijection (and hence, by compactness, a homeomorphism) and
  \begin{equation*}
    \omega(g_1^{x_1}\dotsb g_r^{x_r}) = \min_{1 \leq i \leq r}(\omega(g_i)+v(x_i)) \qquad \text{for any } x_1,\dotsc,x_r\in\Z_p.
  \end{equation*}
\end{definition}

\begin{definition}
  For any $g \in G \setminus \set{1}$, we put $\sigma(g) \coloneqq gG_{\omega(g)+} \in \gr G$.
\end{definition}

By \cite[Remark~26.3]{Sch}, we note that for $g \in G \setminus \set{1}$ and $x \in \Z_{p} \setminus \set{0}$
\begin{equation}
  \label{eq:sigma-gx}
  \omega(g^{x}) = \omega(g) + v_{p}(x) \quad \text{ and } \quad \sigma(g^{x}) = \bar{x}\pi^{v_{p}(x)} \act \sigma(g),
\end{equation}
where $\bar{x}$ is the image of $p^{-v_{p}(x)}x$ in $\F_{p}^{\times}$ (i.e., the first non-zero coefficient of $x = \sum_{k=0}^{\infty} a_{k}p^{k}$). We note that an ordered basis $(g_{1},\dotsc,g_{d})$ of $(G,\omega)$ corresponds to an ordered $\F_{p}[\pi]$-basis $\bigl( \sigma(g_{1}), \dotsc, \sigma(g_{d}) \bigr)$ of $\gr G$, cf.\ \cite[Prop.~26.5]{Sch}.

Finally we let $\lie{g} = \F_{p} \otimes_{\F_{p}[\pi]} \gr G = \F_{p} \otimes_{\F_{p}} \gr G/\pi\gr G$, and note that this is a Lie algebra over $\F_{p}$ with an $\F_{p}$-basis of vectors $\xi_{i} = 1 \otimes \sigma(g_{i})$. % Furthermore, for later use, we note that any $p$-valuable group $G$ admits a $p$-valuation $\omega$ with values in $\frac{1}{e}\Z$ for some $e \in \N$, cf.\ \cite[Cor.~33.3]{Sch}. Suppose that $\omega$ has this property and note that we can introduce a $\Z$-filtration
% \begin{equation*}
%   \Fil^{i} G = G_{\frac{i}{e}}, \qquad i=0,1,2,\dotsc
% \end{equation*}
% with corresponding grading
% \begin{equation*}
%   \gr^{i} G = \gr_{\frac{i}{e}} G = \Fil^{i} G / \Fil^{i+1} G,
% \end{equation*}
% which gives the same $\gr G$ as above.

\subsection{Cohomology theories and the spectral sequence}%
\label{subsec:coh-and-spec-seq}

One of the main results we use in this chapter is the spectral sequence introduced in \cite[§6.1]{Sor}, so in this subsection we aim to introduce the concepts needed to use this spectral sequence. We also mention a translation between continuous and discrete group cohomology for the groups we work with.

Let $R$ be a ring and let $\lie{g}$ be a $R$-Lie algebra with $R$ a trivial (left) $\lie{g}$-module. Then we use the cochain complex $C^\bullet(\lie{g},R) = \Hom_{R}(\bigwedge^{\bullet}\lie{g}, R)$ to define Lie algebra cohomology, i.e., the cochain complex
\[
  \begin{tikzcd}
    0 \ar[r] & R \ar[r,"\partial_{1}"] & \Hom_{R}(\lie{g},R) \ar[r,"\partial_{2}"] & \Hom_{R}\Bigl(\bigwedge^2\lie{g}, R\Bigr) \ar[r,"\partial_{3}"] &  \cdots,
  \end{tikzcd}
\]
where the coboundary map $\partial_{n}$ is given by
\begin{equation*}
  \partial_{n}(f)(x_1,\dotsc,x_{n}) = \sum_{i<j}(-1)^{i+j}f([x_i,x_j],x_1,\dotsc,\widehat{x}_i,\dotsc,\widehat{x}_j,\dotsc,x_{n}),
\end{equation*}
where $\widehat{x}_{i}$ means excluding $x_{i}$. For more details we refer to \cite[Thm.~7.1]{CartanHomAlg} or \cite[Chap.~1~§3]{Fuks}, and note that we are considering the trivial action on $R$, which simplifies the formula slightly (cf.\ \cite[Chap.~1~§3.2]{Fuks}).

Now consider $R=\F_{p}$ in the following and suppose that $\lie{g} = \lie{g}^0 \oplus \lie{g}^1 \oplus \dotsb$ is a graded Lie algebra. Then $\bigwedge^n \lie{g}$ is also graded by letting
\[
  \gr^j\Bigl( \bigwedge^n \lie{g} \Bigr) = \bigoplus_{j_1+\dotsb+j_n = j} \lie{g}^{j_1} \wedge \dotsb \wedge \lie{g}^{j_n}.
\]
Letting $\F_{p}$ be a $\Z$-graded (concentrated in degree $0$) $\lie{g}$-module, we get a grading
\[
  \Hom_{\F_{p}}\Bigl( \bigwedge^n \lie{g}, \F_{p} \Bigr) = \bigoplus_{s \in \Z} \Hom_{\F_{p}}^s\Bigl( \bigwedge^n\lie{g}, \F_{p} \Bigr)
\]
where $\Hom_{\F_{p}}^s$ denotes the homogeneous $\F_{p}$-linear maps of degree $s$, cf.\ \cite[Lem.~4.2]{Fossum}. One can check that this passes to bigrading of Lie algebra cohomology
\begin{equation*}
  H^{s,t}(\lie{g}, \F_{p}) = H^{s+t}\Bigl( \gr^s \Hom_{\F_{p}}\bigl(\bigwedge^{\bullet} \lie{g}, \F_{p} \bigr) \Bigr).
\end{equation*}%
\nomuni[HstgR]{$H^{s,t}(\lie{g}, \F_{p})$}{${} = H^{s+t}\bigl( \gr^{s} \Hom_{\F_{p}}(\bigwedge^{\bullet}\lie{g}, \F_{p}) \bigr)$}%
See \cite[Chap.~1~§3]{Fuks} for more details.

In the spectral sequence described in \cite[§6.1]{Sor}, we take $r_{0} = 1$ (i.e., the spectral sequence starts from the first page) and $E_{1}^{s,t} = H^{s,t}(\lie{g}, \F_{p})$, where $\lie{g} = \F_{p} \otimes \gr G$ indeed is (positively) $\Z$-graded.

Let now $G$ be a topological group and $\F_{p}$ a $G$-module. Then we will define two types of group cohommology: continuous and discrete.

Continuous group cohomology $\Hc^{n}(G,\F_{p})$ is the cohomology of the complex $C^{\bullet}(G,\F_{p}) = \Cont(G^{\bullet},\F_{p})$ of continuous maps $G \times G \times \dotsb \times G \to \F_{p}$, i.e.,
\[
  \begin{tikzcd}
    0 \ar[r] & \F_{p} \ar[r,"\partial_{1}"] & \Cont(G, \F_{p}) \ar[r,"\partial_{2}"] & \Cont(G^{2}, \F_{p}) \ar[r,"\partial_{3}"] &  \Cont(G^{3},\F_{p}) \ar[r,"\partial_{4}"] & \cdots,
  \end{tikzcd}
\]
where the coboundary map $\partial_{n}$ is given by
\begin{equation}\label{eq:group-coh-d}
  \partial_{n}(f)(g_{1},\dotsc,g_{n}) = f(g_{2},\dotsc,g_{n}) + \sum_{i=0}^{n}(-1)^{i}f(g_{1},\dotsc,g_{i}g_{i+1},\dotsc,g_{n}),
\end{equation}
where $n$-th term is interpreted as $(-1)^{n}f(g_{1},\dotsc,g_{n-1})$, cf.\ \cite[§3]{Sor} or \cite[§2]{GalCoh}. Note again that our formula is slightly simpler since we only consider the trivial action on $\F_{p}$.

Discrete group cohomology $\Hd^{n}(G,\F_{p})$ is the cohomology of the complex $C^{\bullet}(G,\F_{p}) = \Hom_{G}(\Z[G^{\bullet}],\F_{p})$, i.e.,
% as follows. One can check that
% \[
%   \begin{tikzcd}
%     \cdots \ar[r,"d_{4}"] & \Z[G^3] \ar[r,"d_{3}"] & \Z[G^2] \ar[r,"d_{2}"] & \Z[G] \ar[r,"d_{1}"] & \Z \ar[r] & 0
%   \end{tikzcd}
% \]
% with boundary maps
% \[
%   d_{n} \colon (g_0,g_1,\dotsc,g_n) \mapsto \sum_{i=0}^{n} (-1)^{i}(g_0,\dotsc,\widehat{g}_i,\dotsc,g_n)
% \]
% is a chain complex, and thus we get a cochain complex $C^\bullet(G,\F_p) = \Hom_G(C_\bullet,\F_p)$,
\[
  \begin{tikzcd}
    0 \ar[r] & \F_{p} \ar[r,"\partial_{1}"] & \Hom_G(\Z[G],\F_p) \ar[r,"\partial_{2}"] & \Hom_G(\Z[G^2],\F_p) \ar[r,"\partial_{3}"] & \cdots,
  \end{tikzcd}
\]
where the coboundary map $\delta_{n}$ is given by \eqref{eq:group-coh-d}, see e.g. \cite[Chap.~VII]{Ser}. Note that this discrete cohomology can be viewed as continuous cohomology if we equip $G$ with the discrete topology.

% Now suppose that $G$ has a $\Z$-filtration and note that we can introduce a natural $\Z$-filtration on $C^{\bullet}(G,\F_{p}) = \Cont(G^{\bullet},\F_{p})$ by defining
% \begin{equation*}
%   \Fil^{s} C^{n} \coloneqq \set{ f : f(\Fil^{i} G^{n}) \subset \Fil^{i+s} \F_{p} \text{ for all } i\geq0 }.
% \end{equation*}

Note that \cite{Sor} gets the spectral sequence we are interested in by using an isomorphism to translate $\Hc^{\bullet}(G,\F_{p})$ to $HH^{\bullet}(\Omega(G),\F_{p})$ (essentially what is known as Mac Lane isomorphism) and introducing a $\Z$-filtration and grading on $HH^{\bullet}(\Omega(G),\F_{p})$, which is used in the spectral sequence. Here $\Omega(G) = \F_{p}\llbracket G \rrbracket$ is the completed group algebra. We will skip the full details of this translation and just note that we get a $\Z$-filtration and grading on $H^{\bullet}(G,\F_{p})$, which with $k=\F_{p}$ gives us the following, cf.\ \cite[Thm.~5.5--§6.1]{Sor}.

\begin{theorem}\label{thm:spec-seq}
  Let $(G,\omega)$ be a $p$-valuable group and $\lie{g} = \F_{p} \otimes_{\F_{p}[\pi]} \gr G$ its Lazard Lie algebra. Then there is a convergent multiplicative spectral sequence collapsing at a finite stage,
  \[
    E_1^{s,t} = H^{s,t}(\lie{g},\F_{p}) \Longrightarrow \Hc^{s+t}(G,\F_{p}).
  \]

  This means that each sheet $E_{r}$ has a multiplication $E_{r} \otimes E_{r} \to E_{r}$ compatible with the $(s,t)$-bigrading and satisfying Leibniz formula. Furthermore $H^{*}(E_{r}) \iso E_{r+1}$ as algebras. I.e., the multiplication on $E_\infty$ is compatible with the cup product on $H^{*}(G,\F_{p})$ in the sense that the following diagram commutes.
  \[
    \begin{tikzcd}
      E_\infty^{s,n-s} \otimes E_\infty^{s',n'-s'} \ar[r] \ar[d,swap,"\iso"] & E_\infty^{s+s',n+n'-s-s'} \ar[d,"\iso"] \\
      \gr^s \Hc^n(G,\F_{p}) \otimes \gr^{s'} \Hc^{n'}(G,\F_{p}) \ar[r] & \gr^{s+s'}\Hc^{n+n'}(G,\F_{p})
    \end{tikzcd}
  \]
\end{theorem}

\begin{remark}
  We note that \cite[Thm.~2.10]{CohComp} implies that $\Hc^{n}(N,\F_{p}) \iso \Hd^{n}(N,\F_{p})$ for all $n$ (with $N = \gs{N}(\Z_{p})$ as above), if we can show that $N$ is a pro-$p$ group which is poly-$\Z_{p}$ by finite.
  \begin{definition}
    A group $G$ is poly-$\Z_{p}$\index{poly-Zp@poly-$\Z_{p}$} if it has a normal series
    \begin{equation*}
      G = G_{1} \supseteq G_{2} \supseteq \cdots  \supseteq G_{n} = 1
    \end{equation*}
    such that each factor group $G_{i}/G_{i+1}$ is isomorphic to $\Z_{p}$.

    A group is poly-$\Z_{p}$ by finite\index{poly-Zp!by finite} (virtually poly-$\Z_{p}$) if it contains a poly-$\Z_{p}$ subgroup of finite index.
  \end{definition}
  Note that \cite[Prop.~5.1.16(2) and Cor.~5.2.5]{Con-book} (as seen in the proof of \cite[Cor.~5.2.13]{Con-book} or \cite[Thm.~5.4.3]{Con-book}) gives us a composition series of $\gs{N}$ such that the successive quotients are $\G_{a}$, which implies that $N = \gs{N}(\Z_{p})$ is poly-$\Z_{p}$ by finite since $\G_{a}(\Z_{p}) = \Z_{p}$. Thus, assuming that $\gs{N}(\Z_{p})$ is a pro-$p$ group, we get that
  \begin{equation}
    \label{eq:coh-comp}
    \Hc^{n}(N,\F_{p}) \iso \Hd^{n}(N,\F_{p}) \qquad \text{ for all } n.
  \end{equation}
\end{remark}

\subsection{Main result}%
\label{subsec:main-res}

We show first that $N$ is $p$-valuable, which implies by \cite[§6.1]{Sor} that we get a convergent multiplicative spectral sequence
\begin{equation}
  \label{eq:spec-seq}
  E_1^{s,t} = H^{s,t}(\lie{g},\F_p) \Longrightarrow \Hc^{s+t}(N,\F_p).
\end{equation}
We note that $\lie{g} \iso \lie{n}$ and then use ideas of \cite[§7]{GK} to transfer results from \cite{PT} about (the dimension of) $H^n(\lie{n}_\Z,\F_p)$ and $\Hd^n(\gs{N}_\Z(\Z),\F_p)$ to $H^n(\lie{n},\F_p)$ and $\Hc^n(\gs{N}(\Z_p),\F_p)$, giving us that \[\sum_{s+t=n} \dim_{\F_p} H^{s,t}(\lie{g},\F_p) = \dim_{\F_p} H^n(\lie{n},\F_p) = \dim_{\F_p} \Hc^n(N,\F_p).\] This implies that \eqref{eq:spec-seq} collapses on the first page, and thus $H^{s,n-s}(\lie{n},\F_p) \iso \gr^s \Hc^n(N,\F_p)$. Noting that $E_\infty^{s,t} = E_1^{s,t}$, we get that the cup product on $E_1^{s,t} = H^{s,t}(\lie{n},\F_p)$ (from $H^*(\lie{n},\F_p)$) is compatible with the cup product on $\Hc^*(N,\F_p)$ in the sense that the following diagram commutes.
\[
  \begin{tikzcd}
    H^{s,n-s}(\lie{n},\F_p) \otimes H^{s',n'-s'}(\lie{n},\F_p) \ar[r] \ar[d,swap,"\iso"] & H^{s+s',n+n'-s-s'}(\lie{n},\F_p) \ar[d,"\iso"] \\
    \gr^s \Hc^n(N,\F_p) \otimes \gr^{s'} H^{n'}(N,\F_p) \ar[r] & \gr^{s+s'}\Hc^{n+n'}(N,\F_p)
  \end{tikzcd}
\]

\section{The \texorpdfstring{$p$}{p}-valuation}\label{sec:pval}

In this section we will prove that $N$ is $p$-valuable group, which we will need in multiple arguments later. It should be noted that this section, with the exception of \Cref{prop:N-p-val}, is a slightly rewritten version of \cite{Sch-notes} which expands on some of the arguments. Also, the proof of \Cref{prop:N-p-val} is based on \cite[Lem.~1]{Zab}.

Note that as a set $N$ is the direct product $N = \prod_{\alpha \in \Phi^{-}} x_\alpha(\Z_p)$, which allows us to introduce the function
\begin{equation}\label{eq:p-val}
  \begin{split}
    \omega \colon N \setminus \set{1} &\to \N \\
    \prod_{\alpha \in \Phi^{-}} x_\alpha(a_\alpha) &\mapsto \min_{\alpha \in \Phi^{-}} \bigl( v_p(a_\alpha) - \Ht(\alpha) \bigr),
  \end{split}
\end{equation}
where $v_p$ denotes the usual $p$-adic valuation on $\Z_p$. Here it is important to note that we write any $g \in N$ uniquely as product
\begin{equation*}
  g = \prod_{\alpha \in \Phi^{-}} x_\alpha(a_\alpha)
\end{equation*}
by taking the product following the total ordering $\geq$ of $\Phi^{-}$ defined above. Now, with the convention that $\omega(1) \coloneqq \infty$, we define the descending sequence of subsets
\begin{equation*}
  N_{m} \coloneqq \set{g \in N \given \omega(g) \geq m}
\end{equation*}
in $N$ for $m\geq0$, following the notation used for $p$-valuable groups. The goal of this section is to show that this $\omega$ is a $p$-valuation by a careful analysis of the sequence of subsets given by $N_m$.

\begin{remark}
  If we are willing to restrict from $p+1 \geq h$ to $p-1 > h$, then we can restrict the $p$-valuation of the pro-$p$ Iwahori subgroup of $\gs{G}$ introduced in \Cref{sec:cohiwagps-intro} to a $p$-valuation on $N$. We prefer the above $p$-valuation because it will introduce a grading on $\lie{g}$ that will directly correspond to the grading (by height) on $\lie{n}$, whereas the restricted $p$-valuation is a scalar multiple of this $p$-valuation on a basis.
\end{remark}

We first note that clearly $N_1 = N$, $\bigcap_m N_m = \set{1}$, and
\begin{equation}
  \label{eq:N_m}
  \begin{split}
    N_m &= \prod_{\alpha \in \Phi^{-}} x_\alpha(p^{\max(0,m+\Ht(\alpha))} \Z_p) \\
    &= \prod_{\substack{\alpha\in \Phi^{-} \\ \Ht(\alpha) = -1}} x_\alpha(p^{m-1} \Z_p) \dotsb \prod_{\substack{\alpha\in \Phi^{-} \\ \Ht(\alpha) = -(m-1)}} x_\alpha(p\Z_p) \prod_{\substack{\alpha\in \Phi^{-} \\ \Ht(\alpha) \leq -m}} x_\alpha(\Z_p).
  \end{split}
\end{equation}

In our analysis of this sequence it will be helpful to introduce the following two other filtrations of $N$. Firstly we will consider the filtration by congruence subgroups
\begin{equation}
  \label{eq:N-par-m}
  N(m) \coloneqq \ker\bigl( \gs{N}(\Z_p) \to \gs{N}(\Z/p^m\Z) \bigr) = \prod_{\alpha \in \Phi^{-}} x_\alpha(p^m\Z_p)
\end{equation}
for $m \geq 0$. Secondly, using the descending central series of the group $\gs{G}(\Q_p)$ defined by $C^1\gs{G}(\Q_p) \coloneqq \gs{G}(\Q_p)$ and $C^{m+1} \gs{G}(\Q_p) \coloneqq [C^m \gs{G}(\Q_p),\gs{G}(\Q_p)]$, we consider the filtration given by
\begin{equation*}
    N_{(m)} \coloneqq N \cap C^m \gs{G}(\Q_p)
\end{equation*}
for $m \geq 1$. By \cite[Prop.~4.7(iii)]{BT} we have that
\begin{equation}
  \label{eq:N_par-m}
  N_{(m)} = \prod_{\substack{\alpha \in \Phi^{-} \\ \Ht(\alpha) \leq -m}} x_\alpha(\Z_p),
\end{equation}
and we note that the natural map
\begin{equation*}
  \prod_{\substack{\alpha\in \Phi^{-} \\ \Ht(\alpha) = -m}} x_\alpha(\Z_p) \to N_{(m)}/N_{(m+1)}
\end{equation*}
is an isomorphism of abelian groups, and that all the subgroups $N(m)$ and $N_{(m)}$ are normal in $N$.

We are now ready to prove the following lemma, which will help us when showing that $\omega$ is a $p$-valuation.

\begin{lemmabreak}\label{lem:N_m}
  \begin{enumerate}[(i)]
  \item $N_m = \prod_{1 \leq i \leq m} N(m-i) \cap N_{(i)}$, for any $m \geq 1$, is a normal subgroup of $N$ which is independent of the choices made.\label{item:N_m}

  \item $[N_\ell,N_m] \subseteq N_{\ell + m}$ for any $\ell,m \geq 1$.\label{item:N_mcom}

  \item $N_m/N_{m+1}$, for any $m \geq 1$, is an $\F_p$-vector space of dimension equal to $\abs{\set{\alpha \in \Phi^{-} \given \Ht(\alpha) \geq -m}}$.

  \item Let $g \in N_m$ for some $m \geq 1$. If $g^p \in N_{m+2}$, then $g \in N_{m+1}$.\label{item:g^p}
  \end{enumerate}
\end{lemmabreak}
\begin{proof}
  \begin{enumerate}[(i),wide]
  \item Using \eqref{eq:N-par-m} and \eqref{eq:N_par-m} we note that
    \begin{equation*}
      \prod_{\substack{\alpha\in \Phi^{-} \\ \Ht(\alpha) = -i}} x_\alpha(p^{m-i} \Z_p) \subseteq N(m-i) \cap N_{(i)} \quad \text{and} \quad \prod_{\substack{\alpha\in \Phi^{-} \\ \Ht(\alpha) \leq -m}} x_\alpha(\Z_p) = N(0) \cap N_{(m)}
    \end{equation*}
    for $1 \leq i < m$, so by \eqref{eq:N_m} it is clear that $N_m \subseteq \prod_{1 \leq i \leq m} N(m-i) \cap N_{(i)}$. We also note, by \eqref{eq:N-par-m} and \eqref{eq:N_par-m}, that
    \begin{align*}
      &\bigl( N(m-i) \cap N_{(i)} \bigr)\bigl( N(m-i-1) \cap N_{(i+1)} \bigr) \\
      &\subseteq \Bigl( \prod_{\substack{\alpha\in \Phi^{-} \\ \Ht(\alpha) = -i}} x_\alpha(p^{m-i} \Z_p) \Bigr)\bigl( N(m-i-1) \cap N_{(i+1)} \bigr)
    \end{align*}
    for any $1 \leq i < m$, so
    \begin{align*}
      &\prod_{1 \leq i \leq m} N(m-i) \cap N_{(i)} \\
      &\subseteq \prod_{\substack{\alpha\in \Phi^{-} \\ \Ht(\alpha) = -1}} x_\alpha(p^{m-1} \Z_p) \dotsb \prod_{\substack{\alpha\in \Phi^{-} \\ \Ht(\alpha) = -(m-1)}} x_\alpha(p \Z_p) \bigl( N(0) \cap N_{(m)} \bigr) \\
      &= N_m
    \end{align*}
    by induction, \eqref{eq:N_m} and \eqref{eq:N_par-m}. This shows the equality and that $N_m$ is normal clearly follows.

  \item We first recall the following formulas for commutators
    \begin{equation}\label{eq:comformulas}
      [gh,k] = g[h,k]g^{-1}[g,k] \quad \text{ and } \quad [g,hk] = [g,h]h[g,k]h^{-1}.
    \end{equation}
    Now, using \eqref{eq:comformulas}, \ref{item:N_m} and the fact that all the involved subgroups are normal, it is enough to show that
    \begin{equation*}
      [N(\ell) \cap N_{(i)}, N(m) \cap N_{(j)}] \subseteq N(\ell+m) \cap N_{(i+j)}.
    \end{equation*}
    This further reduces to showing that
    \begin{equation*}
      [N(\ell),N(m)] \subseteq N(\ell+m) \quad \text{ and } \quad [N_{(i)},N_{(j)}] \subseteq N_{(i+j)}.
    \end{equation*}
    The right inclusion is a well known property of the descending central series, so it follows from our definition of $N_{(m)}$. For the left inclusion it suffices, by \eqref{eq:N-par-m} and \eqref{eq:comformulas}, to show that
    \begin{equation*}
      [x_\alpha(p^\ell \Z_p), x_\beta(p^m \Z_p)] \subseteq N(\ell + m)
    \end{equation*}
    for any $\alpha,\beta \in \Phi^{-}$. To show this inclusion we recall Chevalley's commutator formula, cf.\ \cite[Prop.~5.1.14]{Con-book},
    \begin{equation*}
      [x_\alpha(a),x_\beta(b)] \in x_{\alpha+\beta}(c_{\alpha,\beta,1,1}ab\Z_p) \prod_{\substack{i,j \geq 1 \\ i+j > 2}} x_{i\alpha + j\beta}(c_{\alpha,\beta,i,j}a^{i}b^{j}\Z_p),
    \end{equation*}
    where $c_{\alpha,\beta,i,j} \in \Z_{p}$ and on the right hand side we use the convention is that $x_{i\alpha + j\beta} \equiv 1$ if $i\alpha + j\beta \notin \Phi$. From \eqref{eq:N-par-m} and Chevalley's commutator formula the inclusion follows.

  \item We note that
    \begin{equation*}
      N(m-i) \cap N_{(i)} = \prod_{\substack{\alpha \in \Phi^{-} \\ \Ht(\alpha) \leq -i}} x_\alpha(p^{m-i} \Z_p)
    \end{equation*}
    for $1 \leq i \leq m$, so the statement follows from \ref{item:N_m} and \ref{item:N_mcom}.

  \item For any $1 \leq \ell \leq m$ we consider the chain of normal subgroups
    \begin{equation*}
      N_{m+2}(N_m \cap N_{(\ell+1)}) \subseteq N_{m+1}(N_m \cap N_{(\ell+1)}) \subseteq N_{m+1}(N_m \cap N_{(\ell)})
    \end{equation*}
    between $N_{m+2}$ and $N_m$. By \eqref{eq:comformulas} and an argument like in \ref{item:N_mcom}, we get that
    \begin{equation*}
      [N_{m+1}(N_m \cap N_{(\ell)}),N_{m+1}(N_m \cap N_{(\ell)})] \subseteq N_{m+2}(N_m \cap N_{(\ell+1)}),
    \end{equation*}
    so the quotient group
    \begin{equation*}
      N_{m+1}(N_m \cap N_{(\ell)}) / N_{m+2}(N_m \cap N_{(\ell+1)})
    \end{equation*}
    is abelian. Now looking carefully at the groups as sets, we see that
    \begin{equation*}
      N_{m} \cap N_{(\ell)} = \prod_{\substack{\alpha \in \Phi^{-} \\ \Ht(\alpha) \leq -\ell}} x_\alpha(p^{\max(0,m+\Ht(\alpha))} \Z_p)
    \end{equation*}
    and thus (using Chevalley's commutator formula and the fact that $\Ht(i\alpha+j\beta) \leq \Ht(\alpha+\beta) < \Ht(\alpha), \Ht(\beta)$ to move the products for the $\Ht(\alpha) = -\ell$ term)
    \begin{align*}
      N_{m+1}(N_{m} \cap N_{(\ell)}) &= \prod_{\substack{\alpha \in \Phi^{-} \\ \Ht(\alpha) > -\ell}} x_\alpha(p^{\max(0,m+1+\Ht(\alpha))} \Z_p) \\
      &\phantom{{}={}} \cdot \prod_{\substack{\alpha \in \Phi^{-} \\ \Ht(\alpha) = -\ell}} x_\alpha(p^{m-\ell} \Z_p) \\
      &\phantom{{}={}} \cdot \prod_{\substack{\alpha \in \Phi^{-} \\ \Ht(\alpha) < -\ell}} x_\alpha(p^{\max(0,m+\Ht(\alpha))} \Z_p).
    \end{align*}
    Similarly
    \begin{align*}
      N_{m+2}(N_{m} \cap N_{(\ell+1)}) &= \prod_{\substack{\alpha \in \Phi^{-} \\ \Ht(\alpha) > -\ell}} x_\alpha(p^{\max(0,m+2+\Ht(\alpha))} \Z_p) \\
      &\phantom{{}={}} \cdot \prod_{\substack{\alpha \in \Phi^{-} \\ \Ht(\alpha) = -\ell}} x_\alpha(p^{m+2-\ell} \Z_p) \\
      &\phantom{{}={}} \cdot \prod_{\substack{\alpha \in \Phi^{-} \\ \Ht(\alpha) \leq -(\ell+1)}} x_\alpha(p^{\max(0,m+\Ht(\alpha))} \Z_p),
    \end{align*}
    and since the quotient group
    \begin{equation*}
      N_{m+1}(N_m \cap N_{(\ell)}) / N_{m+2}(N_m \cap N_{(\ell+1)})
    \end{equation*}
    is abelian, we see that it is isomorphic to
    \begin{equation*}
      \prod_{\substack{\alpha \in \Phi^{-} \\ \Ht(\alpha) > -\ell}} \frac{x_\alpha(p^{\max(0,m+1+\Ht(\alpha))} \Z_p)}{x_\alpha(p^{\max(m+2+\Ht(\alpha))} \Z_p)} \times \prod_{\substack{\alpha \in \Phi^{-} \\ \Ht(\alpha) = -\ell}} \frac{x_\alpha(p^{m-\ell} \Z_p)}{x_\alpha(p^{m+2-\ell} \Z_p)}.
    \end{equation*}
    Here the subgroup
    \begin{equation*}
      N_{m+1}(N_m \cap N_{(\ell+1)}) / N_{m+2}(N_m \cap N_{(\ell+1)})
    \end{equation*}
    corresponds to
    \begin{equation*}
      \prod_{\substack{\alpha \in \Phi^{-} \\ \Ht(\alpha) > -\ell}} \frac{x_\alpha(p^{\max(0,m+1+\Ht(\alpha))} \Z_p)}{x_\alpha(p^{\max(0,m+2+\Ht(\alpha))} \Z_p)} \times \prod_{\substack{\alpha \in \Phi^{-} \\ \Ht(\alpha) = -\ell}} \frac{x_\alpha(p^{m+1-\ell} \Z_p)}{x_\alpha(p^{m+2-\ell} \Z_p)}.
    \end{equation*}
    It follows that $N_{m+1}(N_m \cap N_{(\ell+1)}) / N_{m+2}(N_m \cap N_{(\ell+1)})$ is the $p$-torsion subgroup of $N_{m+1}(N_m \cap N_{(\ell)}) / N_{m+2}(N_m \cap N_{(\ell+1)})$.

    Now let $g\in N_m$ for some $m\geq1$. For $\ell = 1$ we have $g \in N_m = N_{m+1}(N_m \cap N_{(1)})$, since $N_{(1)} = N$, and clearly $g^p \in N_{m+2}(N_m \cap N_{(2)})$ because $g^{p} \in N_{(2)}$ by Chevalley's commutator formula and \eqref{eq:N_par-m}. Since $N_{m+1}(N_m \cap N_{(2)}) / N_{m+2}(N_m \cap N_{(2)})$ is the $p$-torsion subgroup of $N_{m+1}(N_m \cap N_{(1)}) / N_{m+2}(N_m \cap N_{(2)})$, it follows that $g \in N_{m+1}(N_m \cap N_{(2)})$ and thus $g^p \in N_{m+2}(N_m \cap N_{(3)})$ by Chevalley's commutator formula and \eqref{eq:N_par-m}. By induction on $\ell$, we thus get that $g \in N_{m+1}(N_m \cap N_{(m+1)}) = N_{m+1}$. Here the last equality follows from the fact that $N_{(m+1)} \subseteq N_{m+1}$ by \eqref{eq:N_m} and \eqref{eq:N_par-m}.
  \end{enumerate}
\end{proof}

With this lemma, we are now ready to prove that $\omega$ is a $p$-valuation on $N$.

\begin{proposition}\label{prop:N-p-val}
  The function $\omega$ from \eqref{eq:p-val} is a $p$-valuation on $N$, i.e., it satisfies for any $g,h \in N$:
  \begin{enumerate}[(a)]
  \item $\omega(g) > \frac{1}{p-1}$,
  \item $\omega(g^{-1}h) \geq \min(\omega(g),\omega(h))$,
  \item $\omega([g,h]) \geq \omega(g) + \omega(h)$,
  \item $\omega(g^p) = \omega(g) + 1$.
  \end{enumerate}
\end{proposition}
\begin{proof}
  We note that (a) is obvious by our definition of $\omega$, (c) follows from \Cref{lem:N_m}~\ref{item:N_mcom} and (d) follows from \Cref{lem:N_m}~\ref{item:g^p}.

  It only remains to show (b), which we will do by following the proof idea of \cite[Lem.~1]{Zab}, i.e., we are going to use triple induction. Here we note that all products $\prod_{\alpha \in \Phi^{-}} x_\alpha(a_\alpha)$ are in ascending order in $\Phi^{-}$ (so descending in height). For ease of notation, we prove equivalently that $\omega(gh^{-1}) \geq \min(\omega(g),\omega(h))$ for $g,h \in N$.

  At first by induction on the number of non-zero coordinates among $(a_{\beta})_{\beta \in \Phi^{-}}$ in $\prod_{\beta \in \Phi^{-}} x_{\beta}(a_{\beta})$ we are reduced to the case where $h$ is of the form $h = x_{\beta}(a_{\beta})$ for some $\beta \in \Phi^{-}$ and $a_{\beta} \in \Z_p$. To see this let $h \in N \setminus \set{1}$ and write $h = \prod_{\beta \in \Phi^{-}} x_{\beta}(a_{\beta})$ in our unique way (according to the ordering of $\Phi^{-}$), and let $\alpha$ be the smallest element of $\Phi^{-}$ for which $a_{\alpha} \neq 0$ so that $h = x_{\alpha}(a_{\alpha}) \cdot h'$. Then $gh^{-1} = g(h')^{-1} \cdot x_{\alpha}(a_{\alpha})^{-1}$ and thus strong induction will imply that
\begin{align*}
  \omega(gh^{-1}) &\geq \min\bigl( \omega(g(h')^{-1}), v(a_{\alpha})-\Ht(\alpha)\bigr) \\
  &\geq \min\bigl( \omega(g),\omega(h'), v(a_{\alpha})-\Ht(\alpha) \bigr) = \min\bigl( \omega(g),\omega(h) \bigr).
\end{align*}

Fix $h = x_{\beta}(a_{\beta})$ and let now $g$ be of the form $g = \prod_{k=1}^r x_{\alpha_k}(a_{\alpha_k})$ with $\alpha_1 < \alpha_2 < \dotsb < \alpha_r$ in $\Phi^{-}$. If $\beta > \alpha_{r}$, then $gh^{-1} = \prod_{k=1}^{r-1} x_{\alpha_k}(a_{\alpha_k}) \cdot x_{\alpha_{r}}(a_{\alpha_{r}}) x_{\beta}(-a_{\beta})$, so (b) is clearly true if $\beta > \alpha_1$ (by the definition of $\omega$), and if $\beta = \alpha_{r}$, then $x_{\alpha_{r}}(a_{\alpha_{r}})x_{\beta}(-a_{\beta}) = x_{\beta}(a_{\alpha_{r}} - a_{\beta})$ and (b) follows from $v_p(a-b) \geq \min(v_p(a),v_p(b))$ for $a,b \in \Z_p$.

On the other hand, if $\beta < \alpha_{r}$, then we write
\begin{align*}
  gh^{-1} &= \prod_{k=1}^{r} x_{\alpha_{k}}(a_{\alpha_{k}}) \cdot x_{\beta}(-a_{\beta}) \\
  &= \prod_{k=1}^{r-1} x_{\alpha_{k}}(a_{\alpha_{k}}) \cdot x_{\beta}(-a_{\beta}) \cdot x_{\alpha_{r}}(a_{\alpha_{r}}) \cdot [x_{\alpha_{r}}(-a_{\alpha_{r}}), x_{\beta}(a_{\beta})].
\end{align*}

Now we use descending induction on $\beta$ in the chosen ordering of $\Phi^{-}$ and suppose that the statement (b) is true for any $g$ and any $h'$ of the form $h' = x_{\beta'}(a_{\beta'})$ with $\beta' > \beta$. Note that the base case is trivial and recall that $\Phi^{-}$ is finite and totally ordered. Note furthermore that Chevalley's commutator formula gives us
\begin{equation}\label{eq:Chevalley}
  [x_{\alpha'}(a_{\alpha'}),x_{\beta'}(a_{\beta'})] = \prod_{\substack{i\alpha' + j\beta' \in \Phi^{-} \\ i,j>0}} x_{i\alpha'+ j\beta'}(c_{\alpha',\beta',i,j}a_{\alpha'}^{i}a_{\beta'}^{j})
\end{equation}
for any $\alpha',\beta' \in \Phi^{-}$, where $c_{\alpha',\beta',i,j} \in \Z_p$. Also, we have $\Ht(i\alpha'+j\beta') \leq \Ht(\alpha'+\beta') < \Ht(\alpha'),\Ht(\beta')$, so we can apply the induction hypothesis for $x_{\alpha_{r}}(a_{\alpha_{r}})$ and each $x_{i\alpha_{r}+j\beta}(c_{\alpha_{r},\beta,i,j}(-a_{\alpha_{r}})^{i}a_{\beta}^{j})$ in $[x_{\alpha_{r}}(-a_{\alpha_{r}},x_{\beta}(a_{\beta}))]$, since $\alpha_{r} > \beta$ and all terms on the right side of \eqref{eq:Chevalley} are larger than $\beta$ (and $\alpha_{r}$) in the ordering of $\Phi^{-}$. We thus obtain
\begin{equation}\label{eq:omega-par-ginvh}
  \begin{aligned}
    \omega(gh^{-1}) &\geq \min\biggl(\min_{\substack{i\alpha_{r}+j\beta \in \Phi^{-} \\ i,j>0}} \omega(x_{i\alpha_{r} + j\beta}(c_{\alpha_{r},\beta,i,j}(-a_{\alpha_{r}})^{i}a_{\beta}^{j})), \\*
    &\phantom{{} \geq \min\biggl( {}} \omega(x_{\alpha_{r}}(a_{\alpha_{r}})), \omega\Bigl( \prod_{k=1}^{r-1} x_{\alpha_k}(a_{\alpha_k}) \cdot x_{\beta}(-a_{\beta}) \Bigr) \biggr).
  \end{aligned}
\end{equation}

Now, for $i,j>0$ with $i\alpha'+j\beta' \in \Phi^{-}$,
\begin{equation}\label{eq:omega-par-Chev}
  \begin{aligned}
    \omega(x_{i\alpha'+j\beta'}(c_{\alpha',\beta',i,j}a_{\alpha'}^{i}a_{\beta'}^{j})) &= v_p(c_{\alpha',\beta',i,j}a_{\alpha'}^{i}a_{\beta'}^{j}) - \Ht(i\alpha'+j\beta') \\*
    &\geq v_p(c_{\alpha',\beta',i,j}) + v_p(a_{\alpha'}^{i}) + v_p(a_{\beta'}^{j}) - \Ht(\alpha'+\beta') \\*
    &\geq v_p(a_{\alpha'}) - \Ht(\alpha') + v_p(a_{\beta'}) - \Ht(\beta') \\*
    &= \omega(x_{\alpha'}(a_{\alpha'})) + \omega(x_{\beta'}(a_{\beta'})) \\*
    &\geq \min\bigl( \omega(x_{\alpha'}(a_{\alpha'})), \omega(x_{\beta'}(a_{\beta'})) \bigr).
  \end{aligned}
\end{equation}
So taking $\alpha' = \alpha_{r}$ and $\beta' = \beta$ and using \eqref{eq:omega-par-Chev} in \eqref{eq:omega-par-ginvh}, we get that
\begin{equation}\label{eq:omega-par-ginvh-2}
  \omega(gh^{-1}) \geq \min\biggl( \omega(x_{\alpha_{r}}(a_{\alpha_{r}})), \omega(x_{\beta}(a_{\beta})), \omega\Bigl( \prod_{k=1}^{r-1} x_{\alpha_{k}}(a_{\alpha_{k}}) \cdot x_{\beta}(-a_{\beta}) \Bigr) \biggr).
\end{equation}

Finally induction on $r$ will imply that
\begin{align*}
  \omega\Bigl( \prod_{k=1}^{r-1} x_{\alpha_{k}}(a_{\alpha_{k}}) \cdot x_{\beta}(-a_{\beta}) \Bigr) &\geq \min\biggl( \omega\Bigl( \prod_{k=1}^{r-1} x_{\alpha_{k}}(a_{\alpha_{k}})\Bigr) , \omega(x_{\beta}(a_{\beta})) \biggr) \\
                                                     &= \min\bigl( \min_{1 \leq k \leq r-1} \omega(x_{\alpha_{k}}(a_{\alpha_{k}})), \omega(x_{\beta}(a_{\beta})) \bigr),
\end{align*}
which by \eqref{eq:omega-par-ginvh-2} implies that
\begin{align*}
  \omega(gh^{-1}) &\geq \min\bigl( \min_{1 \leq k \leq r} \omega(x_{\alpha_{k}}(a_{\alpha_{k}})), \omega(x_{\beta}(a_{\beta})) \bigr) \\
             &= \min\bigl( \omega(g), \omega(h) \bigr),
\end{align*}
thus finishing the proof.
\end{proof}

We have now shown that $N = \gs{N}(\Z_{p})$ is a $p$-valuable group with the $p$-valuation $\omega$ introduced in \eqref{eq:p-val}, which is the main result of this section. Before continuing, we will clarify what this means based on Lazard theory as described in \Cref{sec:cohunigps-intro}.

We note that
\begin{equation*}
  \gr N \coloneqq \bigoplus_{m\geq1} N_{m}/N_{m+1}
\end{equation*}
is a graded $\F_{p}$-vector space, and recall the following well known result, cf.\ \cite{Laz} or \cite[Sect.~25]{Sch}.

\begin{proposition}
  $\gr N$ is a Lie algebra over the polynomial ring $\F_{p}[\pi]$ in one variable $\pi$ where
  \begin{equation*}
    [gN_{\ell+1},hN_{m+1}] \coloneqq [g,h]N_{\ell+m+1} \quad \text{ and } \quad \pi(gN_{m+1}) \coloneqq g^{p}N_{m+2},
  \end{equation*}
  and as an $\F_{p}[\pi]$-module $\gr N$ is free of rank $\abs{\Phi^{-}}$.
\end{proposition}


\section[Spectral sequence]{Spectral sequence and cohomology}\label{sec:specsec}

Recall that $N = \gs{N}(\Z_{p})$, $\lie{g} = \F_p \otimes_{\F_p[\pi]} \gr N$ and $\lie{n} = \Lie(\gs{N}_{\F_{p}})$. In this section we will first look at the spectral sequence from \cite{Sor} (cf.\ \Cref{thm:spec-seq}), i.e.,
\begin{equation*}
  E_{1}^{s,t} = H^{s,t}(\lie{g}, \F_{p}) \Longrightarrow \Hc^{s+t}(N,\F_{p}),
\end{equation*}
and note that we can work with the left side using that $H^{s,t}(\lie{g},\F_{p}) \iso H^{s,t}(\lie{n},\F_{p})$. % and for the right side $\Hc^{s+t}(N,\F_{p}) \iso \Hd^{s+t}(N,\F_{p})$.
Afterwards, we will use results from \cite{PT} to argue that the spectral sequence collapses on the first page.

We will start by showing that $\lie{g} \iso \lie{n}$, for which we will need the following lemma, which is again from \cite{Sch-notes}.

\begin{lemma}
  $\gr N \iso \F_{p}[\pi] \otimes_{\F_{p}} \lie{n}$ as graded Lie algebras (where $\pi$ has degree $1$).
\end{lemma}
\begin{proof}
  We first note that the elements $X_{\alpha}$, where $X_{\alpha}$ is our fixed $\Z_{p}$-basis of $\Lie \gs{N}_{\alpha}$, reduce modulo $p$ to an $\F_{p}$-basis $\set{\overline{X}_{\alpha}}_{\alpha \in \Phi^{-}}$ of $\lie{n}$. On the other hand all
  \begin{equation*}
    \sigma\bigl(x_{\alpha}(1)\bigr) \in \gr_{-\Ht(\alpha)} N,
  \end{equation*}
  with $x_{\alpha}(1) \in N_{-\Ht(\alpha)}$, form an $\F_{p}[\pi]$-basis of $\gr N$, cf.\ \cite{Sch} Proposition~26.5. Hence the map
  \begin{align*}
    \F_{p}[\pi] \otimes_{\F_{p}} \lie{n} &\to \gr N \\
    f \otimes \overline{X}_{\alpha} &\mapsto f \act \sigma\bigl(x_{\alpha}(1)\bigr)
  \end{align*}
  is an isomorphism of graded modules. Chevalley's commutator formula (cf.\ \cite[Prop.~5.1.14]{Con-book}) says that there are $p$-adic integers $c_{\alpha,\beta} = c_{\alpha,\beta,1,1}$ such that $[X_{\alpha},X_{\beta}] = c_{\alpha,\beta}X_{\alpha+\beta}$ and
  \begin{equation*}
    [x_{\alpha}(1),x_{\beta}(1)] \in x_{\alpha+\beta}(c_{\alpha,\beta})N_{-\Ht(\alpha)-\Ht(\beta)+1} = x_{\alpha+\beta}(1)^{c_{\alpha,\beta}}N_{-\Ht(\alpha)-\Ht(\beta)+1},
  \end{equation*}
  where $X_{\alpha+\beta} = 0$ and $x_{\alpha+\beta} \equiv 1$ if $\alpha+\beta \notin \Phi$. This implies that the image of the above map is a Lie subalgebra, and thus that the map is an isomorphism of Lie algebras. We note that $\lie{n}$ is graded by the height function, which corresponds to the grading on $\gr N$ by the definition of $\omega$ in \eqref{eq:p-val}.
\end{proof}

Now $\gr N \iso \F_p[\pi] \otimes_{\F_p} \lie{n}$ implies that $\lie{g} \iso \F_p \otimes_{\F_p[\pi]} \F_p[\pi] \otimes_{\F_p} \lie{n} \iso \lie{n}$, where both $\lie{g}$ and $\lie{n}$ is graded by the height function. From this it clearly follows that $H^{s,t}(\lie{g},\F_{p}) \iso H^{s,t}(\lie{n},\F_{p})$. (Note that this can also be seen directly by looking at the Chevalley constants.)
% Finally, since we proved in the previous section that $N$ is a pro-$p$ group, we get (as noted in \eqref{eq:coh-comp}) that $\Hc^{n}(N,\F_{p}) \iso \Hd^{n}(N,\F_{p})$ for all $n$.

By \cite[§2.10]{PT} (using that $p \geq h-1$) and the Universal Coefficient Theorem (as used in \cite[§3.8]{PT}), we get an $\F_{p}$-vector space isomorphism
\begin{equation*}
  H^{n}(\lie{n}_{\Z},\F_{p}) = H^{n}(\lie{n}_\Z,V_{\F_p}(0)) \iso \bigoplus_{\substack{w \in W \\ \ell(w) = n}} V_{\F_p}(w \cdot 0),
\end{equation*}
where $V_{\F_{p}}(0) = \F_{p}$ with the trivial action (concentrated in degree $0$). Similarly, by the corollary in \cite[§3.8]{PT}, we have an $\F_{p}$-vector space isomorphism
\begin{equation*}
  \gr \Hd^{n}(\gs{N}_{\Z}(\Z),\F_{p}) = \gr \Hd^{n}(\gs{N}_\Z(\Z),V_{\F_p}(0)) \iso \bigoplus_{\substack{w \in W \\ \ell(w) = n}} V_{\F_p}(w \cdot 0).
\end{equation*}
Here the grading on cohomology will not be important, since we just need that
\begin{equation}
  \label{eq:PT-dims}
  \dim_{\F_{p}} H^{n}(\lie{n}_{\Z},\F_{p}) = \dim_{\F_{p}} \Hd^{n}(\gs{N}_{\Z}(\Z),\F_{p}).
\end{equation}

We now equip $\gs{N}_{\Z}(\Z)$ with the discrete topology and claim that
\begin{equation*}
  \Hd^{n}(\gs{N}_{\Z}(\Z),\F_{p}) = \Hc^{n}(\gs{N}_\Z(\Z),\F_{p}) \iso \Hc^{n}(\gs{N}(\Z_p),\F_{p}).
\end{equation*}
Here the first equality is clear since $\gs{N}_{\Z}(\Z)$ is equipped with the discrete topology. To see the isomorphism, first note that $\Z$ is a discrete group, $\Z_p$ is a profinite group, and the homomorphism $\Z \to \Z_p$ has dense image in $\Z_p$. So we have homomorphisms
\begin{equation*}
  \Hc^n(\Z_p,\F_p) \to \Hc^n(\Z,\F_p)
\end{equation*}
for all $n\geq0$ from \cite[Sect.~I~§2.6]{GalCoh}. Now both $\Hc^0(\Z,\edot)$ and $\Hc^0(\Z_p,\edot)$ are the functor of taking invariant, both $\Hc^1(\Z,\edot)$ and $\Hc^1(\Z_p,\edot)$ are what \cite{GK} calls the functor of taking ``coinvariants'' (giving the group of continuous crossed-homomorphisms of $G$ into $\edot$, cf.\ \cite[I.~§2]{GalCoh}), and all $\Hc^n(\Z,\edot)$ and $\Hc^n(\Z_p,\edot)$ vanish for $n\geq2$, so $\Z$ is \enquote{good} in the sense of \cite[Section~I~§2.6 Exercise~2]{GalCoh}. Thus \cite[Section~I~§2.6 Exercise~2(d)]{GalCoh} implies that the homomorphisms
\begin{equation*}
  \Hc^n(\gs{N}(\Z_p),\F_p) \to \Hc^n(\gs{N}(\Z),\F_p) \qquad n\geq0,
\end{equation*}
induced by the homomorphism $\gs{N}(\Z) \to \gs{N}(\Z_p)$, are all isomorphisms. To see this one can consider a filtration of $\gs{N}(\Z)$ with subquotients isomorphic with $\Z$, and its parallel filtration of $\gs{N}(\Z_p)$ with subquotients isomorphic with $\Z_p$ as in \cite[Sect.~7]{GK}, which will make it follow directly from \cite[Section~I~§2.6 Exercise~2(d)]{GalCoh}.

Hence
\begin{equation*}
  \dim_{\F_p} H^{n}(\lie{n}_\Z,\F_p) = \dim_{\F_p} \Hd^n(\gs{N}_\Z(\Z),\F_p)= \dim_{\F_p} \Hc^{n}(\gs{N}(\Z_p),\F_p).
\end{equation*}

Now $\lie{n} = \lie{n}_\Z \otimes \F_p$, and $H^n(\lie{g},\F_p) \iso H^{n}(\lie{n},\F_p)$ (since $\lie{g} \iso \lie{n}$) is the cohomology of the complex
\begin{equation*}
  C^\bullet(\lie{n},\F_p) = \Hom_{\F_p}\Bigl( \bigwedge^\bullet \lie{n},\F_p \Bigr)
\end{equation*}
while $H^{n}(\lie{n}_\Z,\F_p)$ is the homology of the complex
\begin{equation*}
  C^\bullet(\lie{n}_\Z,\F_p) = \Hom_{\F_p}\Bigl( \bigwedge^\bullet \lie{n}_\Z,\F_p \Bigr).
\end{equation*}
Here $\bigwedge^\bullet \lie{n}_\Z$ is a free $\Z$-module and $(\bigwedge^\bullet \lie{n}_\Z) \otimes \F_p \iso \bigwedge^\bullet (\lie{n}_\Z \otimes \F_p) \iso \bigwedge^\bullet \lie{n}$, so we have natural isomorphisms
\begin{equation*}
  \Hom_{\F_p}\Bigl( \bigwedge^\bullet \lie{n}_\Z,\F_p \Bigr) \iso \Hom_{\F_p}\Bigl( \Bigl( \bigwedge^\bullet \lie{n}_\Z \Bigr) \otimes \F_p, \F_p \Bigr) \iso \Hom_{\F_p}\Bigl( \bigwedge^\bullet \lie{n},\F_p \Bigr).
\end{equation*}
These isomorphisms are clearly compatible with the differentials, so $C^\bullet(\lie{n},\F_p) \iso C^\bullet(\lie{n}_\Z,\F_p)$, and thus $H^n(\lie{n},\F_p) \iso H^n(\lie{n}_\Z,\F_p)$. Hence
\begin{equation*}
  \dim_{\F_p} H^n(\lie{n},\F_p) = \dim_{\F_p} H^n(\lie{n}_\Z,\F_p) = \dim_{\F_p} H^{n}(\gs{N}(\Z_p),\F_p).
\end{equation*}

Now $\dim_{\F_{p}} H^{n}(\lie{n},\F_{p}) = \dim_{\F_{p}}^{n}(\lie{g},\F_{p})$ and $N = \gs{N}(\Z_{p})$ implies that
\begin{equation*}
  \sum_{s+t = n} \dim_{\F_p} H^{s,t}(\lie{g},\F_p) = \dim_{\F_p} H^n(\lie{g},\F_p) = \dim_{\F_p} H^n(N,\F_p),
\end{equation*}
so the multiplicative spectral sequence
\begin{equation*}
  E_{1}^{s,t} = H^{s,t}(\lie{g},\F_p) \Longrightarrow H^{s+t}(N,\F_p)
\end{equation*}
collapses on the first page, since the dimension of $E_{r}^{s,t}$ is non-increasing as $r$ increases. Since the spectral sequence collapses on the first page, we get that $E_{1}^{s,t} = E_{\infty}^{s,t}$, so
\begin{equation*}
  \gr^{s} H^{n}(N,\F_p) \iso H^{s,t}(\lie{g},\F_p) \iso H^{s,t}(\lie{n},\F_p),
\end{equation*}
giving us a good description of $H^n(\gs{N}(\Z_p),\F_p)$. Furthermore, we can describe the cup product, by calculating it in $H^{*}(\lie{g},\F_{p})$ or $H^{*}(\lie{n},\F_{p})$, cf.\ \Cref{thm:spec-seq} for the details. I.e., we have shown:

\begin{theorem}
  Let $N = \gs{N}(\Z_{p})$ be the $\Z_{p}$ points of $\gs{N}$, where $\gs{N}$ is the unipotent radical of a Borel in a split and connected reductive $\Z_{p}$-group, and let and $\lie{n} = \Lie{\gs{N}_{\F_{p}}}$. Then $\omega$ from \eqref{eq:p-val} gives a $p$-valuation on $N$, and if we let $\lie{g} = \F_{p} \otimes_{\F_{p}[\pi]} \gr N$ be the Lazard Lie algebra of $N$, then $\lie{g} \iso \lie{n}$.

  Furthermore, there is a convergent multiplicative spectral sequence
  \[
    E_1^{s,t} = H^{s,t}(\lie{g},\F_{p}) \Longrightarrow \Hc^{s+t}(G,\F_{p})
  \]
  collapsing at the first page, so $\gr^{s} H^{n}(N,\F_{p}) \iso H^{s,t}(\lie{g},\F_{p}) \iso H^{s,t}(\lie{n},\F_{p})$, and the cup product on $H^{*}(\lie{g},\F_{p})$ is compatible with the cup product on $H^{*}(N,\F_{p})$ in the sense that the following diagram commutes.
  \[
    \begin{tikzcd}
      H^{s,n-s}(\lie{g}, \F_{p}) \otimes H^{s',n'-s'}(\lie{g},\F_{p}) \ar[r] \ar[d,swap,"\iso"] & H^{s+s',n+n'-s-s'}(\lie{g},\F_{p}) \ar[d,"\iso"] \\
      \gr^s \Hc^n(G,\F_{p}) \otimes \gr^{s'} \Hc^{n'}(G,\F_{p}) \ar[r] & \gr^{s+s'}\Hc^{n+n'}(G,\F_{p})
    \end{tikzcd}
  \]
\end{theorem}


\section{Example: \texorpdfstring{$N \subseteq \SL_{3}(\Z_{p})$}{N in SL3(Zp)}}%
\label{sec:ex-N-in-SL3}

In the case of $\gs{G} = \SL_3$ (in this case $h=3$, so $p\geq3$), we can take $\gs{T}$ to be the diagonal matrices in $\SL_3$ ($\det = 1$), $\gs{B}$ upper triangular matrices in $\SL_3$ and
\[
  \gs{N} = \set[\Bigg]{\pmat{1 & * & * \\ 0 & 1 & * \\ 0 & 0 & 1}} \subseteq \SL_n.
\]
Furthermore we can take $\Phi^{-} = \set{\alpha_1,\alpha_2,\alpha_3=\alpha_1+\alpha_2}$ with
\begin{align*}
  X_{\alpha_{1}} &= \pmat{1 & 1 & 0 \\ 0 & 1 & 0 \\ 0 & 0 & 1}, & x_{\alpha_{1}}(A)(a) &= \pmat{1 & a & 0 \\ 0 & 1 & 0 \\ 0 & 0 & 1}, \\
  X_{\alpha_{2}} &= \pmat{1 & 0 & 0 \\ 0 & 1 & 1 \\ 0 & 0 & 1}, & x_{\alpha_{2}}(A)(a) &= \pmat{1 & 0 & 0 \\ 0 & 1 & a \\ 0 & 0 & 1}, \\
  X_{\alpha_{3}} &= \pmat{1 & 0 & 1 \\ 0 & 1 & 0 \\ 0 & 0 & 1}, & x_{\alpha_{3}}(A)(a) &= \pmat{1 & 0 & a \\ 0 & 1 & 0 \\ 0 & 0 & 1},
\end{align*}
for any $\Z_p$-algebra $A$ and $a \in A$. Here $\Ht(\alpha_1) = \Ht(\alpha_2) = -1$ and $\Ht(\alpha_3) = -2$, and explicit calculations show that, in $N = \gs{N}(\Z_p)$, $g_1=x_{\alpha_1}(1), g_2=x_{\alpha_2}(1), g_3=x_{\alpha_3}(1)$ is an ordered basis of $(N,\omega)$. Thus (cf.\ \cite[Prop.~26.5]{Sch}) $\sigma(g_1),\sigma(g_2),\sigma(g_3)$ is a basis of the $\F_p[\pi]$-module $\gr N$, and $\xi_1,\xi_2,\xi_3$ is a basis of $\lie{g} = \F_p \otimes_{\F_p[\pi]} \gr N$, where $\xi_i = 1 \otimes \sigma(g_i)$. Furthermore $\lie{g} = \lie{g}^1 \oplus \lie{g}^2$, where $\lie{g}^1 = \Span(\xi_1,\xi_2)$ and $\lie{g}^2 = \Span(\xi_3)$.

The only non-trivial commutator among the $g_i$'s is $[g_1,g_2] = x_{\alpha_3}(-1) = g_{3}^{-1}$, which implies (cf.\ \cite[Rem.~26.3]{Sch}) that $\sigma([g_1,g_2]) = -\sigma(g_3)$ and thus $[\xi_1,\xi_2] = -\xi_3$. In particular $[\lie{g},\lie{g}] = \lie{g}^2$.

Now $H^1(\lie{g},\F_p) = \Hom_k(\lie{g}/[\lie{g},\lie{g}],\F_p) = H^{-1,2}(\lie{g},\F_p)$, and, since $\bigwedge^3 \lie{g} = \lie{g}^1 \wedge \lie{g}^1 \wedge \lie{g}^2$ is degree $4$, $H^3(\lie{g},\F_p) = H^{-4,7}(\lie{g},\F_p)$. A version of Poincaré duality (cf.\ \cite[Chap.~1~§3.6--7]{Fuks}) gives us that $H^1 \times H^2 \to H^3$ with $H^{-1,2} \times H^{s,t} \to H^{-4,7}$ is only non-trivial for $(s,t) = (-3,5)$, so $H^2(\lie{g},\F_p) = H^{-3,5}(\lie{g},\F_p)$. By considering the maps $d_{r}^{s,t} \colon E_{r}^{s,t} \to E_{1}^{s+r,t+1-r}$, we see that the spectral sequence collapses at the first page, so this gives us a description of $H^*(N,\F_p)$, and we note that the only non-trivial cup product is $H^1(N,\F_p) \times H^2(N,\F_p) \to H^3(N,\F_p)$.

We note that we skipped some of the details above since we will go through many examples of this kind of computation in \Cref{cha:cohiwagps}, so we refer to there for more of the details in this type of argument.


%%% Local Variables:
%%% mode: latex
%%% TeX-master: "../main"
%%% End:


\chapter{Cohomology of pro-\texorpdfstring{$p$}{p} Iwahori Subgroups}%
\label{cha:cohiwagps}

\section{Intoduction}%
\label{sec:cohiwagps-intro}

In this chapter we will calculate the cohomology over perfect fields $k$ of a collection of pro-$p$ Iwahori subgroups of $\SL_{n}$ and $\GL_{n}$ over $\Z_{p}$ or (low degree) extensions of $\Z_{p}$.\dknote{Maybe change perfect fields to just $\F_{p}$.}

\subsection{Background and motivation}%
\label{subsec:background-iwa}

Write later.

\subsection{Setup and notation}%
\label{subsec:setup-iwa}

Let $p$ be an odd prime (further restricted later) and let $k$ be a perfect field of characteristic $p$.\nomiwa[k]{$k$}{a perfect field of characteristic $p$}

\paragraph{Field extension of $\Q_{p}$.} We fix a finite extension of $F/\Q_{p}$\nomiwa[F]{$F$}{a finite extension of $\Q_{p}$} of degree $D$\nomiwa[D]{$D$}{${} = [F : \Q_{p}]$, the degree of the extension $F/\Q_{p}$} with valuation ring $\sO_{F}$\nomiwa[OF]{$\sO_{F}$}{the valuation ring of $F$} and maximal ideal $\idm_{F} = (\varpi_{F}) \subseteq \sO_{F}$.\nomiwa[mF]{$\idm_{F}$}{maximal ideal of the valuation ring $\sO_{F}$}\nomiwa[piF]{$\varpi_{F}$}{a uniformizer of $F$} Let $e = e(F/\Q_{p})$ be the \emph{ramification index}\index{ramification index} and $f = f(F/\Q_{p})$ the \emph{inertia degree}\index{inertia degree} of the extension $F/\Q_{p}$. Let furthermore $v$ be the valuation on $F$ for which $v(p) = 1$, and thus $v(\varpi_{F}) = \frac{1}{e}$.

\paragraph{$\exp$ and $\log$.} Given a $\idm$-adic number field $F$ with valuation ring $\sO_{F}$ and maximal ideal $\idm_{F}$ with $p\sO_{F} = \idm_{F}^{e}$, we get by \cite[Prop.~(5.5)]{Neukirch} (noting that we will ensure that $1 > \frac{e}{p-1}$ later) that the power series
\begin{equation*}
  \exp(x) = 1 + x + \frac{x^{2}}{2!} + \frac{x^{3}}{3!} + \dotsb \quad \text{ and } \quad \log(1+z) = z - \frac{z^{2}}{2} + \frac{z^{3}}{3} - \dotsb,
\end{equation*}
are two mutually inverse isomorphisms (and homeomorphisms)
\[
  \begin{tikzcd}
    \idm_{F} \ar[r, yshift=0.7ex, "\exp"] & U_{F}^{(1)}. \ar[l, yshift=-0.7ex, "\log"]
  \end{tikzcd}
\]
Note that this implies that a $\Z_{p}$-basis of $\idm$ translates to a $\Z_{p}$-basis of $U_{F}^{(1)} = 1+\idm_{F}$ via $\exp$.

\paragraph{Big-$O$ notation.} For elements of $\sO_{F}$ we write $x = y + O(p^{r})$ if and only if $x-y \in p^{r}\sO_{F}$.\index{big-O notation@big-$O$ notation}\index{Opr@$O(p^{r})$}\nomiwa[Opr]{$O(p^{r})$}{for elements of $\sO_{F}$ we write $x = y + O(p^{r})$ if and only if $x-y \in p^{r}\sO_{F}$}

\paragraph{Matrices.} Let $E_{ij}$\nomiwa[Eij]{$E_{ij}$}{the matrix with $1$ in the $(i,j)$ entry and zeroes in all other entries} denote the matrix with $1$ in the $(i,j)$ entry, and zeroes in all other entries, and write $1_{n}$ for the identity matrix in $M_{n}(F)$.\dknote{Use correct name for $E_{ij}$.} Let $A = (a_{ij})$. We write $A = \diag(a_{1},\dotsc,a_{n})$\nomiwa[diag]{$\diag(a_{1},\dotsc,a_{n})$}{(${} = (a_{ij})$) the diagonal matrix with entries $a_{ii} = a_{i}$} for the diagonal matrix in $M_{n}(F)$ with entries $a_{ii}=a_{i}$ in the diagonal, and $A = \diag_{i_{1},\dotsc,i_{k}}(a_{1},\dotsc,a_{k})$\nomiwa[diagi]{$\diag_{i_{1},\dotsc,i_{k}}(a_{1},\dotsc,)$}{(${} = (a_{ij})$) the matrix with entries $a_{i_{\ell}i_{\ell}} = a_{\ell}$ for $\ell=1,\dotsc,k$ and zeroes in all other entries} for the diagonal matrix in $M_{n}(F)$ with entries $a_{i_{\ell}i_{\ell}} = a_{\ell}$ for $\ell = 1,\dotsc,k$ and zeroes in all other entries.

\paragraph{Smith normal form.} Let $R$ be an integral domain and consider only non-zero matrices over $R$ in this paragraph. Given an $n \times m$ matrix $A$, there exist invertible $m \times m$ and $n \times n$ matrices $S$ and $T$ such that
\begin{equation*}
  SAT =
  \begin{pNiceMatrix}[nullify-dots]
    a_{1} & 0 & \Cdots \hspace*{3mm} & & & & & 0 \\
    0 & a_{2} & \Ddots & & & & & \Vdots \\
    \Vdots & \Ddots & \Ddots \\
    & & & a_{r} \\
    & & & & 0 & \Ddots \hspace*{3mm} \\
    & & & & & \Ddots \\
    & & & & & & & 0 \\
    0 & \Cdots & & & & & 0 & 0
  \end{pNiceMatrix}
\end{equation*}
and the diagonal entries $a_{i}$ satisfy $a_{i} \mid a_{i+1}$ for $i=1,\dotsc,r-1$. This matrix is called the Smith normal form of the matrix $A$. Given $n \times m$ matrices $A,B$, we write $A \snfsim B$ if $A$ and $B$ have the same Smith normal form. This notation will mainly be used when $B$ is already a matrix in Smith normal form. Finally we introduce the notation $A = \SNF^{n\times m}(a_{1},\dotsc,a_{r},0,\dotsc,0)$ for the $n \times m$ matrix with $a_{ii} = a_{i}$ for $i=1,\dotsc,r$ and zeroes in all other entries. In next subsection, we will note that the Smith normal form will be useful for our cohomology calculations.

\paragraph{Algebraic groups.} We will work with schemes using the functorial approach and notation described in \cite{Jan}. In particular, given an integral domain $R$, we note that a \emph{$R$-group functor}\index{R-group@$R$-group!functor} is a functor from the category of all $R$-algebras to the category of groups, a \emph{$R$-group scheme}\index{R-group!scheme} is a $R$-group functor that is an affine scheme over $R$ when considered as a $R$-functor, and an \emph{algebraic $R$-group}\index{R-group!algebraic}\index{algebraic R-group@algebraic $R$-group} is a $R$-group scheme that is algebraic as an affine scheme. For more in depth introduction to these concepts, we refer to \cite{Con-book} and \cite{Jan}.

\paragraph{Fixed groups and roots.} We fix a split and connected reductive algebraic $F$-group $\gs{G}$\nomiwa[G]{$\gs{G}$}{a (fixed) split and connected reductive algebraic $F$-group}\index{G}, and consider the locally profinite group $G = \gs{G}(F)$.\nomiwa[GF]{$G$}{${} = \gs{G}(F)$, a locally profinite group} We then fix split maximal torus $\gs{T} \subseteq \gs{G}$\nomiwa[T]{$\gs{T}$}{a (fixed) split maximal torus of $\gs{G}$} and let $T = \gs{T}{F}$.\nomiwa[TF]{$T$}{${} = \gs{T}(F)$} In $T$ we have a maximal compact subgroup $T^{0}$ and its Sylow pro-$p$ subgroup $T^{1}$.

Let $\Phi = \Phi(\gs{G},\gs{T})$\nomiwa[Phi]{$\Phi$}{${} = \Phi(\gs{G},\gs{T})$, the root system of $\gs{G}$ with respect to $\gs{T}$} be the \emph{root system}\index{root!system} of $\gs{G}$ with respect to $\gs{T}$, and let $(X^{*}(T),\Phi,X_{*}(T),\Phi^{\vee})$\nomiwa[XTPhi]{$(X^{*}(T),\Phi,X_{*}(T),\Phi^{\vee})$}{the root datum associated with $\Phi = \Phi(\gs(G),\gs{T})$} be the associated root datum.\index{root!datum} Fix a system of positive roots $\Phi^{+}$ and let $\Pi \subseteq \Phi^{+}$ be the simple roots.\index{root!simple}\dknote{Write $\Delta$ instead of $\Pi$?} For any $\alpha \in \Phi$ we have the root subgroup\index{root!subgroup} $\gs{U}_{\alpha} \subseteq \gs{G}$ with Lie algebra $\Lie \gs{U}_{\alpha} =  (\Lie \gs{G})_{\alpha}$. We let $U_{\alpha} = \gs{U}_{\alpha}(F)$ and choose an isomorphism $x_{\alpha} \colon F \xrightarrow{\iso} U_{\alpha}$\nomiwa[xa]{$x_{\alpha} \colon F \xrightarrow{\iso} U_{\alpha}$}{an isomorphism such that $tx_{\alpha}(x)t^{-1} = x_{\alpha}(\alpha(t)x)$ for $t \in T$ and $x \in F$} such that $tx_{\alpha}(x)t^{-1} = x_{\alpha}(\alpha(t)x)$ for $t \in T$ and $x \in F$. For $r \in \Z_{\geq 0}$ we let $U_{\alpha,r} = x_{\alpha}(\idm_{F}^{r})$.\nomiwa[Uar]{$U_{\alpha,r}$}{${} = x_{\alpha}(\idm_{F}^{r})$}


% We furthermore fix a basis $\Delta \subseteq \Phi$\nomiwa[Delta]{$\Delta$}{a (fixed) basis of the root system $\Phi$} of the root system,\index{root system} so we get a decomposition $\Phi = \Phi^+ \cup \Phi^-$\nomiwa[Phi+]{$\Phi^{+}$ / $\Phi^{-}$}{the positive/negative roots in $\Phi$ with respect to $\Delta$} into positive and negative roots. Let $\gs{B} = \gs{T}\gs{U}$\nomiwa[B]{$\gs{B}$ / $\gs{B}^{+}$}{(${} = \gs{T}\gs{U}$ / ${} = \gs{T}\gs{U}^{+}$) the Borel subgroups of $\gs{G}$ corresponding to $\Phi^{-}$ / $\Phi^{+}$} and $\gs{B}^+ = \gs{T}\gs{U}^+$ denote the Borel subgroups of $\gs{G}$ corresponding to $\Phi^-$ and $\Phi^+$, respectively, with unipotent radicals $\gs{U}$ and $\gs{U}^+$.\nomuni[U]{$\gs{U}$ / $\gs{U}^{+}$}{the unipotent radical of $\gs{B}$ / $\gs{B}^{+}$}\index{U@$\gs{U}$}

\paragraph{Pro-$p$ Iwahori subgroups.} We follow the definitions of \cite{SchOll-modular} with $\gs{G}, \gs{T}$ and $\gs(U)_{\alpha}$ as above. Let $I$ be the pro-$p$ Iwahori subgroup of $G$ (associated with a positive chamber as in \cite{SchOll-modular}, but we don't need the exact definition). We note by \cite[Lem.~2.1(i)]{SchOll-modular} and the proof of \cite[Lem.~2.3]{SchOll-modular} that $I$ has the following factorization: Multiplication defines a homeomorphism
\begin{equation*}
  \prod_{\alpha \in \Phi^{-}} U_{\alpha,1} \times T^{1} \times \prod_{\alpha \in \Phi^{+}} U_{\alpha,0} \xrightarrow{\iso} I,
\end{equation*}
where the products are ordered in an arbitrarily chosen way. For a more detailed introduction to these pro-$p$ groups we refer to \cite{SchOll-modular}.\dknote{Maybe add a more general reference too.}

\paragraph{Pro-$p$ Iwahori subgroups of $\GL_{n}$ and $\SL_{n}$.} In this chapter, we will only work with pro-$p$ Iwahori subgroups of $\GL_{n}(F)$ or $\SL_{n}(F)$, which simplifies the definitions. When $\gs{G} = \GL_{n}$ or $\gs{G} = \SL_{n}$, we can always take $\gs{T}$ the diagonal maximal torus, and we can take $I$ to be the subgroup of $\gs{G}(\sO_{F})$ which is upper triangular and unipotent modulo $\varpi$. In this case we have that $U_{\alpha,1}$ for $\alpha \in \Phi^{-}$ correspond to entries below the diagonal and $U_{\alpha,0}$ for $\alpha \in \Phi^{+}$ corresponds to the entries above the diagonal.

\paragraph{Coxeter number and $p$.} Let $h$\nomiwa[h]{$h$}{the Coxeter number of $\gs{G}$} be the Coxeter number of $\gs{G}$ and assume from now on that $p-1 > eh$.\nomiwa[p]{$p$}{a prime, $p-1 \geq eh$, where $h$ is the Coxeter number of $\gs{G}$}

\paragraph{$p$-valuation on $I$.} By a recent pre-print by Lahiri and Sørensen (not yet published),\dknote{Figure out exactly how to write this. Probably cite as (unpublished).} we know (since $p-1 > eh$) that $I$ admits a $p$-valuation $\omega$ with the property
\begin{equation}\label{eq:Iwa-p-val}
  \omega(x_{\alpha}(x)) = v(x) + \frac{\Ht(\alpha)}{eh} \qquad
  \begin{dcases*}
    x \in \idm_{F} & if $\alpha \in \Phi^{-}$, \\
    x \in \sO_{F} & if $\alpha \in \Phi^{+}$.
  \end{dcases*}
\end{equation}

\paragraph{Lazard theory.} For an introduction to Lazard theory see \Cref{sec:cohunigps-intro}, or \cite{Sch} for more details. In particular, note that the Lazard Lie algebra generalizes from $\F_{p}$ to general $k$ of characteristic $p$. We will let $\lie{g} = k \otimes_{\F_p[\pi]} \gr I$\nomiwa[g]{$\lie{g}$}{${} = k \otimes_{\F_{p}[\pi]} \gr I$, the Lazard Lie algebra corresponding to the pro-$p$ Iwahori subgroup $I$} be the Lazard Lie algebra corresponding to the pro-$p$ Iwahori subgroup $I$. Furthermore, recall that a sequence of elements $(g_1,\dotsc,g_r)$ in $G$ is called an \emph{ordered basis} of $(G,\omega)$\index{p-valued group!ordered basis} if the map $\Z_{p}^{r} \to G$ given by $(x_{1},\dotsc,x_{r}) \mapsto g_{1}^{x_{1}} \dotsb g_{r}^{x_{r}}$ is a bijection (and hence, by compactness, a homeomorphism) and
\begin{equation*}
  \omega(g_1^{x_1}\dotsb g_r^{x_r}) = \min_{1 \leq i \leq r}(\omega(g_i)+v(x_i)) \qquad \text{for any } x_1,\dotsc,x_r\in\Z_p.
\end{equation*}

\paragraph{Ordered basis of $I$.} Let $\set{b_{1},\dotsc,b_{D}}$ be a $\Z_{p}$-basis of $\sO_{F}$ and let $\set{u_{1},\dotsc,u_{D}}$ be a $\Z_{p}$-basis of $U_{F}^{(1)} = 1+\idm_{F}$, where $D = [F:\Q_{p}]$. Then $\bigl( x_{\alpha}(b_{1}), \dotsc, x_{\alpha}(b_{D}) \bigr)$ is an ordered basis for $U_{\alpha,0}$ when $\alpha \in \Phi^{+}$, and $\bigl( x_{\alpha}(\varpi_{F}b_{1}), \dotsc, x_{\alpha}(\varpi_{F}b_{D}) \bigr)$ is an ordered basis for $U_{\alpha,1}$ when $\alpha \in \Phi^{-}$. Furthermore, when $G$ is semisimple and simply connected, we have that the simple coroots\index{coroots} $\set{ \alpha^{\vee} : \alpha \in \Pi }$ form a $\Z$-basis of $X_{*}(T)$, and thus $\bigl( \alpha^{\vee}(u_{1}), \dotsc, \alpha^{\vee}(u_{D}) \bigr)_{\alpha \in \Pi}$ form an ordered basis of $T^{1}$. By \cite[Prop.~3.1]{IwaBasis}, given orderings of $\Phi^{+}$ and $\Phi^{-}$, and assuming that $G$ is semisimple and simply connected, we now get: the sequence of elements
\begin{enumerate}[$\bullet$]
  \item $\bigl( x_{\alpha}(\varpi_{F}b_{1}), \dotsc, x_{\alpha}(\varpi_{F}b_{D}) \bigr)_{\alpha \in \Phi^{-}}$,
  \item $\bigl( \alpha^{\vee}(u_{1}), \dotsc, \alpha^{\vee}(u_{D}) \bigr)_{\alpha \in \Pi}$,
  \item $\bigl( x_{\alpha}(b_{1}), \dotsc, x_{\alpha}(b_{D}) \bigr)_{\alpha \in \Phi^{+}}$
\end{enumerate}
forms an ordered basis of $(I,\omega)$ (with $\omega$ from \eqref{eq:Iwa-p-val}) which is a saturated $p$-valued group. Recalling from above that $\exp \colon \idm_{F} = (\varpi_{F}) \to U_{F}^{(1)} = 1 + \idm_{F}$ takes a basis to a basis, and noting that $\set{ \varpi_{F}b_{1}, \dotsc, \varpi_{F}b_{D} }$ is a $\Z_{p}$-basis of $\idm_{F} = \varpi_{F}\sO_{F}$, we see that we can take $u_{i} = \exp(\varpi_{F}b_{i})$ for $i=1,\dotsc,D$. When $\gs{G} = \SL_{n}$,\dknote{Reference for this being simply connected.} we have that $\Phi = \set{ \varepsilon_{i}-\varepsilon_{j} \given 1 \leq i,j \leq n, i\neq j }$ and can take
\begin{equation*}
  \Pi = \set{\alpha_{1}=\varepsilon_{1}-\varepsilon_{2}, \alpha_{2}=\varepsilon_{2}-\varepsilon_{3}, \dotsc, \alpha_{n-1}=\varepsilon_{n-1}-\varepsilon_{n}},
\end{equation*}
where $\varepsilon_{i}$ is the map that takes a diagonal matrix to its $i$-th diagonal entry.\dknote{Standard Lie algebra theory, add a reference.} In this case $\alpha_{i}^{\vee}(u) = \diag(0,\dotsc,0,u,-u,0,\dotsc,0) = \diag_{i,i+1}(u,-u)$, where the non-zero entries are the $i$-th and $(i+1)$-th entries. This together with the above gives us the following ordered basis (in the listed order and with a chosen ordering of $\set{ (i,j) : 1 \leq i,j \leq n }$) in the case $\gs{G} = \SL_{n}$:
\begin{enumerate}[$\bullet$]
  \item $\bigl( 1_{n}+\varpi_{F}b_{1}E_{ij}, \dotsc, 1_{n}+\varpi_{F}b_{D}E_{ij} \bigr)_{1 \leq j < i \leq n}$,
  \item $\bigl( \diag_{i,i+1}(\exp(\varpi_{F}b_{1})), \dotsc, \diag_{i,i+1}(\exp(\varpi_{F}b_{D})) \bigr)_{i=1,\dotsc,n-1}$,
  \item $\bigl( 1_{n}+b_{1}E_{ij}, \dotsc, 1_{n}+b_{D}E_{ij} \bigr)_{1 \leq i < j \leq n}$.
\end{enumerate}

Finally note that an ordered basis of $\GL_{n}$ can be obtained from an ordered basis of $\SL_{n}$ by adding a non-trivial element of the center, which in the above corresponds to adding $\bigl(  \exp(\varpi_{F}b_{1})1_{n}, \dotsc, \exp(\varpi_{F}b_{D})1_{n} \bigr)$ to the middle item above.

\paragraph{Cohomology.} We denote (using the Chevalley-Eilenberg complex) the Lie algebra cohomology\index{cohomology!Lie algebra} of any $k$-Lie algebra $\lie{g}$ by $H^{\bullet}(\lie{g}, \edot)$,\nomiwa[Hg]{$H^{\bullet}(\lie{g},\edot)$}{the cohomology of the Lie algebra $\lie{g}$} while we write $H^{\bullet}(G,\edot)$\nomiwa[HG]{$H^{\bullet}(G,\edot)$}{the continuous group cohomology of the topological group $G$} for the continuous group cohomology of a topological group $G$. Here we let the entries distinguish between different types of cohomology without any ambiguity. As in \Cref{sec:cohunigps-intro}, we introduce filtrations and then gradings on the cohomology and use the notation $H^{s,t} = \gr^{s}H^{s+t}$\nomiwa[Hst]{$H^{s,t}$}{${} = \gr^{s}H^{s+t}$ for some cohomology $H$} for any type of cohomology $H$.

\paragraph{Spectral sequences.} Given a ring $R$, a cohomological spectral sequence\index{spectral sequence}\index{spectral sequence!cohomological} is a choice of $r_0 \in \N$ and a collection of
\begin{enumerate}[$\bullet$]
  \item $R$-modules $E_r^{s,t}$ for each $s,t \in \Z$ and all integers $r \geq r_0$
  \item differentials $d_r^{s,t} \colon E_r^{s,t} \to E_r^{s+r,t+1-r}$ such that $d_r^2 = 0$ and $E_{r+1}$ is isomorphic to the homology of $(E_r,d_r)$, i.e.,
  \[
    E_{r+1}^{s,t} = \frac{\kernel(d_r^{s,t} \colon E_r^{s,t} \to E_r^{s+r,t+1-r})}{\image(d_r^{s-r,t+r-1} \colon E_r^{s-r,t+r-1} \to E_r^{s,t})}.
  \]
\end{enumerate}
For a given $r$, the collection $(E_r^{s,t},d_r^{s,t})_{s,t\in\Z}$ is called the $r$-th page. A spectral sequence \emph{converges}\index{spectral sequence!convergent} if $d_r$ vanishes on $E_r^{s,t}$ for any $s,t$ when $r\gg0$. In this case $E_r^{s,t}$ is independent of $r$ for sufficiently large $r$, we denote it by $E_{\infty}^{s,t}$ and write
  \[
    E_{r}^{s,t} \Longrightarrow E_\infty^{s+t}.
  \]
Also, we say that the spectral sequence collapses at the $r'$-th page if $E_{r} = E_{\infty}$ for all $r \geq r'$, but not for $r < r'$. Finally, when we have terms $E_\infty^{n}$  with a natural filtration $F^\bullet E_\infty^n$ (but no natural double grading), we set $E_\infty^{s,t} = \gr^{s} E_\infty^{s,t}= F^{s}E_\infty^{s+t}/F^{s+1}E_\infty^{s+t}$.

\subsection{Smith normal form and cohomology}%
\label{subsec:SNF-coh}

Note how SNF is useful.

\section{Techniques}%
\label{sec:tech-iwa}

Let $(G, \omega)$ be a $p$-valuable group and let $k$ be a perfect field of characteristic $p$. In this section we will describe how the spectral sequence
\begin{equation*}
  E_{1}^{s,t} = H^{s,t}(\lie{g}, k) \Longrightarrow H^{s+t}(G, k)
\end{equation*}
from \cite[§6.1]{Sor} can be used to calculate information about the dimensions of $H^{n}(G,k)$ for varying $n$ and information about the cup product on $H^{*}(G,k)$. After this, we will then briefly discuss how this applies to pro-$p$ Iwahori subgroups $I$ of $\GL_{n}$ or $\SL_{n}$.



\section{\texorpdfstring{$I \subseteq \SL_{2}(\Z_{p})$}{I in SL2(Zp)}}%
\label{sec:Iwa-SL2}

\begin{equation*}
  I = \pmat{1+p\Z_{p} & \Z_{p} \\ p\Z_{p} & 1+p\Z_{p}} \subseteq \SL_{2}(\Z_{p}).
\end{equation*}

Obvious try (using that $(1+p)^{\Z_{p}} = 1+p\Z_{p}$):
\begin{align*}
  g_{1}' &= \pmat{1 & 0 \\ p & 1}, & g_{2}' &= \pmat{1+p & 0 \\ 0 & (1+p)^{-1}}, & g_{3}' &= \pmat{1 & 1 \\ 0 & 1}.
\end{align*}

Better:
\begin{align}
  \label{eq:gis-SL2}
  g_{1} &= \pmat{1 & 0 \\ p & 1}, & g_{2} &= \pmat{\exp(p) & 0 \\ 0 & \exp(-p)}, & g_{3} &= \pmat{1 & 1 \\ 0 & 1}.
\end{align}

For $g = (a_{ij})$
\begin{equation*}
  \omega(g) \defeq \min\bigl( v_{p}(a_{11}-1), \tfrac{1}{2} + v_{p}(a_{12}), -\tfrac{1}{2} + v_{p}(a_{21}), v_{p}(a_{22}-1) \bigr).
\end{equation*}

\begin{equation}
  \label{eq:gixi-SL2}
  g_{1}^{x_{1}}g_{2}^{x_{2}}g_{3}^{x_{3}} = \pmat{\exp(px_{2}) & x_{3}\exp(px_{2}) \\ px_{1}\exp(px_{2}) & px_{1}x_{3}\exp(px_{2}) + \exp(-px_{2})} = \pmat{a_{11} & a_{12} \\ a_{21} & a_{22}}.
\end{equation}

$g_{ij} = [g_{i},g_{j}]$

In the following we use that $\frac{1}{p-1} = 1 + p + p^{2} + \dotsb$ and $\log(1-p) = -p - \frac{p^{2}}{2} - \frac{p^{3}}{3} - \dotsb$.

\begin{description}
  \item[$g_{12} = \pmat{ 1 & 0 \\ p\bigl( 1 - \exp(-2p) \bigr) }$:] Comparing $g_{12}$ with \eqref{eq:gixi-SL2}, we see that $x_{2} = x_{3} = 0$. This leaves $a_{21} = px_{1} = p\bigl( 1 - \exp(-2p) \bigr) = 2p^{2} + O(p^{3})$, which implies that $x_{1} = 2p + O(p^{2})$. Hence $\sigma(g_{12}) = 2\pi \act \sigma(g_{1})$, which implies that $\xi_{12} = 0$.

  \item[$g_{13} = \pmat{ 1-p & p \\ -p^{2} & 1+p+p^{2} }$:] Comparing $g_{13}$ with \eqref{eq:gixi-SL2}, we see that
        \begin{align*}
          a_{11} &= \exp(px_{2}) = 1-p, \\
          a_{12} &= x_{3}\exp(px_{2}) = x_{3}(1-p) = p, \\
          a_{21} &= px_{1}\exp(px_{2}) = px_{1}(1-p) = -p^{2},
        \end{align*}
        and thus
        \begin{align*}
          x_{2} &= \dfrac{1}{p}\log(1-p) = \dfrac{1}{p}\bigl( (-p) + O(p^{2}) \bigr) = -1 + O(p), \\
          x_{3} &= \dfrac{p}{1-p} = p + O(p^{2}), \\
          x_{1} &= \dfrac{-p^{2}}{p(1-p)} = -p + O(p^{2}).
        \end{align*}
        Hence $\sigma(g_{13}) = -\pi \act \sigma(g_{1}) - \sigma(g_{2}) - \pi \act \sigma(g_{3})$, which implies that $\xi_{13} = -\xi_{2}$.

  \item[$g_{23} = \pmat{ 1 & \exp(2p)-1 \\ 0 & 1 }$:] Comparing $g_{23}$ with \eqref{eq:gixi-SL2}, we see that $x_{1} = x_{2} = 0$. This leaves $a_{12} = x_{3} = \exp(2p)-1 = 2p + O(p^{2})$. Hence $\sigma(g_{23}) = 2\pi \act \sigma(g_{3})$, which implies that $\xi_{23} = 0$.
\end{description}


\begin{align*}
  \sigma(g_{12}) &= 2\pi \act \sigma(g_{1}), \\
  \sigma(g_{13}) &= \pi \act \sigma(g_{1}) + (p-1)\sigma(g_{2}) + \pi \act \sigma(g_{3}), \\
  \sigma(g_{23}) &= \pi \act \sigma(g_{3}).
\end{align*}


So with $\xi_{i} = 1 \otimes \sigma(g_{i})$:
\begin{align*}
  [\xi_{1},\xi_{2}] &= 0, & [\xi_{1},\xi_{3}] &= -\xi_{2}, & [\xi_{2},\xi_{3}] &= 0.
\end{align*}

\section{\texorpdfstring{$I \subseteq \GL_{2}(\Z_{p})$}{I in GL2(Zp)}}%
\label{sec:Iwa-GL2}

\begin{equation}
  \label{eq:gis-GL2}
  \begin{gathered}
    g_{1} = \pmat{1 & 0 \\ p & 1}, \qquad g_{2} = \pmat{\exp(p) & 0 \\ 0 & \exp(-p)}, \\
    g_{3} = \pmat{\exp(p) & 0 \\ 0 & \exp(p)}, \qquad g_{4} = \pmat{1 & 1 \\ 0 & 1}.
  \end{gathered}
\end{equation}

\begin{align}
  & g_{1}^{x_{1}}g_{2}^{x_{2}}g_{3}^{x_{3}}g_{4}^{x_{4}}  \notag \\
                &= \pmat{ \exp\bigl( p(x_{2}+x_{3}) \bigr) & \exp\bigl( p(x_{2}+x_{3}) \bigr)x_{4} \\ px_{1}\exp\bigl( p(x_{2}+x_{3}) \bigr) & \exp\bigl( p(x_{2}+x_{3}) \bigr)px_{1}x_{4} + \exp\bigl( p(x_{3}-x_{2}) \bigr) } \\
                &= \pmat{ a_{11} & a_{12} \\ a_{21} & a_{22} }. \notag
\end{align}

$g_{ij} = [g_{i},g_{j}]$
\begin{align*}
  \sigma(g_{12}) &= (p-2)\pi \act \sigma(g_{1}), \\
  \sigma(g_{14}) &= (p-1)\pi \act \sigma(g_{1}) + (p-1)\sigma(g_{2}) + \pi \act \sigma(g_{3}), \\
  \sigma(g_{24}) &= (p-2)\pi \act \sigma(g_{3}), \\
  \sigma(g_{13}) &= \sigma(g_{23}) = \sigma(g_{24}) = 0.
\end{align*}

So with $\xi_{i} = 1 \otimes \sigma(g_{i})$:
\begin{equation*}
  [\xi_{1},\xi_{4}] = -\xi_{2}
\end{equation*}
is the only non-zero commutator.

\section{\texorpdfstring{$I \subseteq \SL_{3}(\Z_{p})$}{I in SL3(Zp)}}%
\label{sec:Iwa-SL3}

To make the notation easier to read for the bigger matrices, we will write any zeros as blank space in matrices in this section.
\begin{equation}
  \label{eq:gis-SL3}
  \begin{gathered}
    g_{1} = \pmat{ 1 \\ & 1 \\ p && 1 }, \quad g_{2} = \pmat{ 1 \\ p & 1 \\ && 1 }, \quad g_{3} = \pmat{ 1 \\ & 1 \\ & p & 1 }, \\
    g_{4} = \pmat{ \exp(p) \\ & \exp(-p) \\ && 1 }, \quad g_{5} = \pmat{ 1 \\ & \exp(p) \\ && \exp(-p) }, \\
    g_{6} = \pmat{ 1 \\ & 1 & 1 \\ && 1 }, \quad g_{7} = \pmat{ 1 & 1 \\ & 1 \\ && 1 }, \quad g_{8} = \pmat{ 1 && 1 \\ & 1 \\ && 1 }.
  \end{gathered}
\end{equation}

\begin{equation*}
    g_{1}^{x_{1}}g_{2}^{x_{2}}g_{3}^{x_{3}}g_{4}^{x_{4}}g_{5}^{x_{5}}g_{6}^{x_{6}}g_{7}^{x_{7}}g_{8}^{x_{8}} = \pmat{ a_{11} & a_{12} & a_{13} \\ a_{21} & a_{22} & a_{23} \\ a_{31} & a_{32} & a_{33}}
\end{equation*}
where
\begin{equation}
  \label{eq:gixi-SL3}
  \begin{aligned}
    a_{11} &= \exp(px_{4}), \\
    a_{12} &= x_{7}\exp(px_{4}), \\
    a_{13} &= x_{8}\exp(px_{4}), \\
    a_{21} &= px_{2}\exp(px_{4}), \\
    a_{22} &= px_{2}x_{7}\exp(px_{4}) + \exp\bigl( p(x_{5}-x_{4}) \bigr), \\
    a_{23} &= px_{2}x_{8}\exp(px_{4}) + x_{6}\exp\bigl( p(x_{5}-x_{4}) \bigr), \\
    a_{31} &= px_{1}\exp(px_{4}), \\
    a_{32} &= px_{1}x_{7}\exp(px_{4}) + px_{3}\exp\bigl( p(x_{5}-x_{4}) \bigr), \\
    a_{33} &= px_{1}x_{8}\exp(px_{4}) + px_{3}x_{6}\exp\bigl( p(x_{5}-x_{4}) \bigr) + \exp(-px_{5}).
  \end{aligned}
\end{equation}


\subsection{\texorpdfstring{Non-identiy $[g_{i},g_{j}]$}{Non-identity [gi,gj]}}%
\label{subsec:non-id-gij-SL3}

$g_{ij} = [g_{i},g_{j}]$

Except in the first case, we will note that $x_{i} \in p\Z_{p}$ implies that the coefficient on $\xi_{k}$ in $\xi_{ij}$ is zero. \dknote{Introduce $O(p^{k})$ notation.}

Note that we repeatedly use that $-1 = (p-1) + (p-1)p + (p-1)p^{2} + \dotsb$ in $\Z_{p}$ and $-1 = p-1$ in $\F_{p}$.

\begin{description}
  \item[$g_{14} = \pmat{1 \\ & 1 \\ p\bigl( 1-\exp(-p) \bigr) && 1}$:] Comparing $g_{14}$ with \eqref{eq:gixi-SL3}, we see that $x_{2} = x_{4} = x_{7} = x_{8} = 0$, and thus also $x_{3} = x_{5} = x_{6} = 0$. This leaves $a_{31} = px_{1} = p\bigl( 1-\exp(-p) \bigr) = p^{2} + O(p^{3})$, which implies that $x_{1} = p + O(p^{2})$. Hence $\sigma(g_{14}) = \pi \act \sigma(g_{1})$, which implies that $\xi_{14} = 0$.

  \item[$g_{15} = \pmat{1 \\ & 1 \\ p\bigl( 1-\exp(-p) \bigr) && 1}$:] Since $g_{15} = g_{14}$, the above calculation shows that $\xi_{15} = 0$.

  \item[$g_{16} = \pmat{1 \\ -p & 1 \\ && 1}$:] Comparing $g_{16}$ with \eqref{eq:gixi-SL3}, we see that $x_{1} = x_{4} = x_{7} = x_{8} = 0$, and thus also $x_{3} = x_{5} = x_{6} = 0$. This leaves $a_{21} = px_{2} = -p$, which implies that $x_{2} = -1$. Hence $\sigma(g_{16}) = -\sigma(g_{2})$, which implies that $\xi_{16} = -\xi_{2}$.

  \item[$g_{17} = \pmat{1 \\ & 1 \\ & p & 1}$:] Comparing $g_{17}$ with \eqref{eq:gixi-SL3}, we see that $x_{1} = x_{2} = x_{4} = x_{7} = x_{8} = 0$, and thus also $x_{5} = x_{6} = 0$. This leaves $a_{32} = px_{3} = p$, which implies that $x_{3} = 1$. Hence $\sigma(g_{17}) = \sigma(g_{3})$, which implies that $\xi_{17} = \xi_{3}$.

  \item[$g_{18} = \pmat{1-p && p \\ & 1 \\ -p^{2} && 1+p+p^{2}}$:] Comparing $g_{18}$ with \eqref{eq:gixi-SL3}, we see that $x_{2} = x_{7} = 0$, and thus also $x_{3} = x_{6} = 0$ and $x_{4} = x_{5}$. Using
        \begin{align*}
          a_{11} &= \exp(px_{4}) = 1-p, \\
          a_{13} &= x_{8}\exp(px_{4}) = x_{8}(1-p) = p, \\
          a_{31} &= px_{1}\exp(px_{4}) = px_{1}(1-p) = -p^{2},
        \end{align*}
        we get that
        \begin{align*}
          x_{4} &= \dfrac{1}{p}\log(1-p) = \dfrac{1}{p}\bigl( (-p) + O(p^{2}) \bigr) = -1 + O(p), \\
          x_{8} &= \dfrac{p}{1-p} = p + O(p^{2}), \\
          x_{1} &= \dfrac{-p^{2}}{p(1-p)} = -p + O(p^{2}).
        \end{align*}
        Hence $\sigma(g_{18}) = -\pi \act \sigma(g_{1}) - \sigma(g_{4}) - \sigma(g_{5}) + \pi \act \sigma(g_{8})$, which implies that $\xi_{18} = -(\xi_{4}+\xi_{5})$.

  \item[$g_{23} = \pmat{1 \\ & 1 \\ -p^{2} && 1}$:] Comparing $g_{23}$ with \eqref{eq:gixi-SL3}, we see that $x_{2} = x_{4} = x_{7} = x_{8} = 0$, and thus also $x_{3} = x_{5} = x_{6} = 0$. This leaves $a_{31} = px_{1} = -p^{2}$, which implies that $x_{1} = -p$. Hence $\sigma(g_{23}) = -\pi \act \sigma(g_{1})$, which implies that $\xi_{23} = 0$.

  \item[$g_{24} = \pmat{1 \\ p\bigl( 1-\exp(-2p) \bigr) & 1 \\ && 1}$:] Comparing $g_{24}$ with \eqref{eq:gixi-SL3}, we see that $x_{1} = x_{4} = x_{7} = x_{8} = 0$, and thus also $x_{3} = x_{5} = x_{6} = 0$. This leaves $a_{21} = px_{2} = p\bigl( 1-\exp(-2p) \bigr) = p\bigl( 1-\bigl( 1+(-2p)+O(p^{2}) \bigr) \bigr) = 2p^{2} + O(p^{3})$, which implies that $x_{2} = 2p + O(p^{2})$. Hence $\sigma(g_{24}) = 2\pi \act \sigma(g_{1})$, which implies that $\xi_{24} = 0$.

  \item[$g_{25} = \pmat{1 \\ p\bigl( 1-\exp(p) \bigr) & 1 \\ && 1}$:] Except a factor $-2$ in the exponential, which clearly doesn't change the final result, we have the same calculation as for $g_{24}$. Thus $\xi_{25} = 0$.

  \item[$g_{27} = \pmat{ 1-p & p \\ -p^{2} & 1+p+p^{2} \\ && 1}$:] Comparing $g_{27}$ with \eqref{eq:gixi-SL3}, we see that $x_{1} = x_{8} = 0$, and thus also $x_{3} = x_{6} = 0$, so $x_{5} = 0$. Using
        \begin{align*}
          a_{11} &= \exp(px_{4}) = 1-p, \\
          a_{12} &= x_{7}\exp(px_{4}) = x_{8}(1-p) = p, \\
          a_{21} &= px_{2}\exp(px_{4}) = px_{2}(1-p) = -p^{2},
        \end{align*}
        we get that
        \begin{align*}
          x_{4} &= \dfrac{1}{p}\log(1-p) = \dfrac{1}{p}\bigl( (-p) + O(p^{2}) \bigr) = -1 + O(p), \\
          x_{7} &= \dfrac{p}{1-p} = p + O(p^{2}), \\
          x_{2} &= \dfrac{-p^{2}}{p(1-p)} = -p + O(p^{2}).
        \end{align*}
        Hence $\sigma(g_{27}) = -\pi \act \sigma(g_{2}) - \sigma(g_{4}) + \pi \act \sigma(g_{7})$, which implies that $\xi_{27} = -\xi_{4}$.

  \item[$g_{28} = \pmat{ 1 \\ & 1 & p \\ && 1}$:] Comparing $g_{28}$ with \eqref{eq:gixi-SL3}, we see that $x_{1} = x_{2} = x_{4} = x_{7} = x_{8} = 0$, and thus also $x_{3} = x_{5} = 0$. This leaves $a_{23} = x_{6} = p$. Hence $\sigma(g_{28}) = \pi \act \sigma(g_{6})$, which implies that $\xi_{28} = 0$.

  \item[$g_{34} = \pmat{ 1 \\ & 1 \\ & p\bigl( 1-\exp(p) \bigr) & 1}$:] Comparing $g_{34}$ with \eqref{eq:gixi-SL3}, we see that $x_{1} = x_{2} = x_{4} = x_{7} = x_{8} = 0$, and thus also $x_{5} = x_{6} = 0$. This leaves $a_{32} = px_{3} = p\bigl( 1-\exp(p) \bigr) = p\bigl( 1-\bigl( 1+p+O(p^{2}) \bigr) \bigr) = -p^{2} + O(p^{3})$, which implies that $x_{3} = -p + O(p^{2})$. Hence $\sigma(g_{34}) = -\pi \act \sigma(g_{3})$, which implies that $\xi_{34} = 0$.

  \item[$g_{35} = \pmat{ 1 \\ & 1 \\ & p\bigl( 1-\exp(-2p) \bigr) & 1}$:] Except a factor $-2$ in the exponential, which clearly doesn't change the final result, we have the same calculation as for $g_{34}$. Thus $\xi_{35} = 0$.

  \item[$g_{36} = \pmat{ 1 \\ & 1-p & p \\ & -p^{2} & 1+p+p^{2}}$:] Comparing $g_{36}$ with \eqref{eq:gixi-SL3}, we see that $x_{1} = x_{2} = x_{4} = x_{7} = x_{8} = 0$. Using
        \begin{align*}
          a_{22} &= \exp(px_{5}) = 1-p, \\
          a_{23} &= x_{6}\exp(px_{5}) = x_{6}(1-p) = p, \\
          a_{32} &= px_{3}\exp(px_{5}) = px_{3}(1-p) = -p^{2},
        \end{align*}
        we get that
        \begin{align*}
          x_{5} &= \dfrac{1}{p}\log(1-p) = \dfrac{1}{p}\bigl( (-p) + O(p^{2}) \bigr) = -1 + O(p), \\
          x_{6} &= \dfrac{p}{1-p} = p + O(p^{2}), \\
          x_{3} &= \dfrac{-p^{2}}{p(1-p)} = -p + O(p^{2}).
        \end{align*}
        Hence $\sigma(g_{36}) = -\pi \act \sigma(g_{3}) - \sigma(g_{5}) + \pi \act \sigma(g_{6})$, which implies that $\xi_{36} = -\xi_{5}$.

  \item[$g_{38} = \pmat{ 1 & -p \\ & 1 \\ && 1}$:] Comparing $g_{38}$ with \eqref{eq:gixi-SL3}, we see that $x_{1} = x_{2} = x_{4} = x_{8} = 0$, and thus also $x_{3} = x_{5} = x_{6} = 0$. This leaves $a_{12} = x_{7} = -p$. Hence $\sigma(g_{38}) = -\pi \act \sigma(g_{3})$, which implies that $\xi_{38} = 0$.

  \item[$g_{46} = \pmat{ 1 \\ & 1 & \exp(-p)-1 \\ && 1}$:] Comparing $g_{46}$ with \eqref{eq:gixi-SL3}, we see that $x_{1} = x_{2} = x_{4} = x_{7} = x_{8} = 0$, and thus also $x_{3} = x_{5} = 0$. This leaves $a_{23} = x_{6} = \exp(-p) - 1 = -p + O(p^{2})$. Hence $\sigma(g_{46}) = -\pi \act \sigma(g_{6})$, which implies that $\xi_{46} = 0$.

  \item[$g_{47} = \pmat{ 1 & \exp(2p)-1 \\ & 1 \\ && 1}$:] Comparing $g_{47}$ with \eqref{eq:gixi-SL3}, we see that $x_{1} = x_{2} = x_{4} = x_{8} = 0$, and thus also $x_{3} = x_{5} = x_{6} = 0$. This leaves $a_{12} = x_{7} = \exp(2p) - 1 = 2p + O(p^{2})$. Hence $\sigma(g_{47}) = 2\pi \act \sigma(g_{7})$, which implies that $\xi_{47} = 0$.

  \item[$g_{48} = \pmat{ 1 && \exp(p)-1 \\ & 1 \\ && 1}$:] Comparing $g_{48}$ with \eqref{eq:gixi-SL3}, we see that $x_{1} = x_{2} = x_{4} = x_{7} = 0$, and thus also $x_{3} = x_{5} = x_{6} = 0$. This leaves $a_{13} = x_{8} = \exp(p) - 1 = p + O(p^{2})$. Hence $\sigma(g_{48}) = \pi \act \sigma(g_{8})$, which implies that $\xi_{48} = 0$.

  \item[$g_{56} = \pmat{ 1 \\ & 1 & \exp(2p)-1 \\ && 1}$:] Except a factor $-2$ in the exponential, which clearly doesn't change the final result, we have the same calculation as for $g_{46}$. Thus $\xi_{56} = 0$.

  \item[$g_{57} = \pmat{ 1 & \exp(-p)-1 \\ & 1 \\ && 1}$:] Except a factor $-2$ in the exponential, which clearly doesn't change the final result, we have the same calculation as for $g_{47}$. Thus $\xi_{57} = 0$.

  \item[$g_{58} = \pmat{ 1 && \exp(p)-1 \\ & 1 \\ && 1}$:] Since $g_{58} = g_{48}$, the above calculation shows that $\xi_{58} = 0$.

  \item[$g_{67} = \pmat{ 1 && -1 \\ & 1 \\ && 1}$:] Comparing $g_{67}$ with \eqref{eq:gixi-SL3}, we see that $x_{1} = x_{2} = x_{4} = x_{7} = 0$, and thus also $x_{3} = x_{5} = x_{6} = 0$. This leaves $a_{13} = x_{8} = -1$. Hence $\sigma(g_{67}) = -\sigma(g_{8})$, which implies that $\xi_{67} = -\xi_{8}$.
\end{description}


The non-zero commutators are:
\begin{equation}
  \label{eq:xi_ij-SL3}
  \begin{aligned}
    [\xi_{1},\xi_{6}] &= -\xi_{2}, & [\xi_{1},\xi_{7}] &= \xi_{3}, & [\xi_{1},\xi_{8}] &= -(\xi_{4}+\xi_{5}), \\
    [\xi_{2},\xi_{7}] &= -\xi_{4}, & [\xi_{3},\xi_{6}] &= -\xi_{5}, & [\xi_{6},\xi_{7}] &= -\xi_{8}.
  \end{aligned}
\end{equation}

\begin{equation*}
  \lie{g} = k \otimes_{\F_{p}[\pi]} \gr I = \Span\set{\xi_{1},\dotsc,\xi_{8}} = \lie{g}_{\frac{1}{3}} \oplus \lie{g}_{\frac{2}{3}} \oplus \lie{g}_{1} = \lie{g}^{1} \oplus \lie{g}^{2} \oplus \lie{g}^{3}.
\end{equation*}

\begin{equation}
  \label{eq:5}
  [\lie{g}^{i},\lie{g}^{j}] =
  \begin{dcases*}
    \lie{g}^{2} & if $i=j=1$, \\
    \lie{g}^{3} & if $(i,j)\in \set{(1,2),(2,1)}$, \\
    0 & otherwise.
  \end{dcases*}
\end{equation}

\begin{equation*}
  \gr^{j}\Bigl( \bigwedge^{n}\lie{g} \Bigr) = \bigoplus_{j_{1} + \dotsb + j_{n} = j} \lie{g}^{j_{1}} \wedge \dotsb \wedge \lie{g}^{j_{n}}.
\end{equation*}

\begin{description}
  \item[$n\geq9:$]
        \begin{equation*}
          \gr^{j}\Bigl( \bigwedge^{n}\lie{g} \Bigr) = 0 \text{ for all } j.
        \end{equation*}
  \item[$n=8:$]
        \begin{equation*}
          \gr^{j}\Bigl( \bigwedge^{8}\lie{g} \Bigr) =
          \begin{dcases}
            \lie{g}^{1} \wedge \lie{g}^{1} \wedge \lie{g}^{1} \wedge \lie{g}^{2} \wedge \lie{g}^{2} \wedge \lie{g}^{2} \wedge \lie{g}^{3} \wedge \lie{g}^{3} & j=15, \\
            0                                                                                                         & \text{otherwise.}
          \end{dcases}
        \end{equation*}
  \item[$n=7:$]
        \begin{equation*}
          \gr^{j}\Bigl( \bigwedge^{7}\lie{g} \Bigr) =
          \begin{dcases}
            \lie{g}^{1} \wedge \lie{g}^{1} \wedge \lie{g}^{2} \wedge \lie{g}^{2} \wedge \lie{g}^{2} \wedge \lie{g}^{3} \wedge \lie{g}^{3}                    & j=14, \\
            \lie{g}^{1} \wedge \lie{g}^{1} \wedge \lie{g}^{1} \wedge \lie{g}^{2} \wedge \lie{g}^{2} \wedge \lie{g}^{3} \wedge \lie{g}^{3}                    & j=13, \\
            \lie{g}^{1} \wedge \lie{g}^{1} \wedge \lie{g}^{1} \wedge \lie{g}^{2} \wedge \lie{g}^{2} \wedge \lie{g}^{2} \wedge \lie{g}^{3}                    & j=12, \\
            0                                                                                                               & \text{otherwise.}
          \end{dcases}
        \end{equation*}

  \item[$n=6:$]
        \begin{equation*}
          \gr^{j}\Bigl( \bigwedge^{6}\lie{g} \Bigr) =
          \begin{dcases}
            \lie{g}^{1} \wedge \lie{g}^{2} \wedge \lie{g}^{2} \wedge \lie{g}^{2} \wedge \lie{g}^{3} \wedge \lie{g}^{3} & j=13,                                                                                                                   \\
            \lie{g}^{1} \wedge \lie{g}^{1} \wedge \lie{g}^{2} \wedge \lie{g}^{2} \wedge \lie{g}^{3} \wedge \lie{g}^{3}                                                                                                                   & j=12, \\
            \!\begin{aligned} & \lie{g}^{1} \wedge \lie{g}^{1} \wedge \lie{g}^{1} \wedge \lie{g}^{2} \wedge \lie{g}^{3} \wedge \lie{g}^{3} \\ & \oplus \lie{g}^{1} \wedge \lie{g}^{1} \wedge \lie{g}^{2} \wedge \lie{g}^{2} \wedge \lie{g}^{2} \wedge \lie{g}^{3} \end{aligned} & j=11, \\
            \lie{g}^{1} \wedge \lie{g}^{1} \wedge \lie{g}^{1} \wedge \lie{g}^{2} \wedge \lie{g}^{2} \wedge \lie{g}^{3}                                                                                                                   & j=10, \\
            \lie{g}^{1} \wedge \lie{g}^{1} \wedge \lie{g}^{1} \wedge \lie{g}^{2} \wedge \lie{g}^{2} \wedge \lie{g}^{2}                                                                                                                   & j=9,  \\
            0                                                                                                                                                                                                & \text{otherwise.}
          \end{dcases}
        \end{equation*}

  \item[$n=5:$]
        \begin{equation*}
          \gr^{j}\Bigl( \bigwedge^{5}\lie{g} \Bigr) =
          \begin{dcases}
            \lie{g}^{2} \wedge \lie{g}^{2} \wedge \lie{g}^{2} \wedge \lie{g}^{3} \wedge \lie{g}^{3} & j=12,                                                                                                                   \\
            \lie{g}^{1} \wedge \lie{g}^{2} \wedge \lie{g}^{2} \wedge \lie{g}^{3} \wedge \lie{g}^{3} & j=11, \\
            \!\begin{aligned} & \lie{g}^{1} \wedge \lie{g}^{1} \wedge \lie{g}^{2} \wedge \lie{g}^{3} \wedge \lie{g}^{3} \\ & \oplus \lie{g}^{1} \wedge \lie{g}^{2} \wedge \lie{g}^{2} \wedge \lie{g}^{2} \wedge \lie{g}^{3} \end{aligned} & j=10, \\
            \!\begin{aligned} & \lie{g}^{1} \wedge \lie{g}^{1} \wedge \lie{g}^{1} \wedge \lie{g}^{3} \wedge \lie{g}^{3} \\ & \oplus \lie{g}^{1} \wedge \lie{g}^{1} \wedge \lie{g}^{2} \wedge \lie{g}^{2} \wedge \lie{g}^{3} \end{aligned} & j=9, \\
            \!\begin{aligned} & \lie{g}^{1} \wedge \lie{g}^{1} \wedge \lie{g}^{1} \wedge \lie{g}^{2} \wedge \lie{g}^{3} \\ & \oplus \lie{g}^{1} \wedge \lie{g}^{1} \wedge \lie{g}^{2} \wedge \lie{g}^{2} \wedge \lie{g}^{2} \end{aligned} & j=8, \\
            \lie{g}^{1} \wedge \lie{g}^{1} \wedge \lie{g}^{1} \wedge \lie{g}^{2} \wedge \lie{g}^{2}                                                                                                                  & j=7, \\
            0                                                                                                                                                                                                      & \text{otherwise.}
          \end{dcases}
        \end{equation*}

  \item[$n=4:$]
        \begin{equation*}
          \gr^{j}\Bigl( \bigwedge^{4}\lie{g} \Bigr) =
          \begin{dcases}
            \lie{g}^{2} \wedge \lie{g}^{2} \wedge \lie{g}^{3} \wedge \lie{g}^{3} & j=10,                                                                                                                   \\
            \!\begin{aligned} & \lie{g}^{1} \wedge \lie{g}^{2} \wedge \lie{g}^{3} \wedge \lie{g}^{3} \\ & \oplus \lie{g}^{2} \wedge \lie{g}^{2} \wedge \lie{g}^{2} \wedge \lie{g}^{3} \end{aligned} & j=9, \\
            \!\begin{aligned} & \lie{g}^{1} \wedge \lie{g}^{1} \wedge \lie{g}^{3} \wedge \lie{g}^{3} \\ & \oplus \lie{g}^{1} \wedge \lie{g}^{2} \wedge \lie{g}^{2} \wedge \lie{g}^{3} \end{aligned} & j=8, \\
            \!\begin{aligned} & \lie{g}^{1} \wedge \lie{g}^{1} \wedge \lie{g}^{2} \wedge \lie{g}^{3} \\ & \oplus \lie{g}^{1} \wedge \lie{g}^{2} \wedge \lie{g}^{2} \wedge \lie{g}^{2} \end{aligned} & j=7, \\
            \!\begin{aligned} & \lie{g}^{1} \wedge \lie{g}^{1} \wedge \lie{g}^{1} \wedge \lie{g}^{3} \\ & \oplus \lie{g}^{1} \wedge \lie{g}^{1} \wedge \lie{g}^{2} \wedge \lie{g}^{2} \end{aligned} & j=6, \\
            \lie{g}^{1} \wedge \lie{g}^{1} \wedge \lie{g}^{1} \wedge \lie{g}^{2}                                                                                                                  & j=5, \\
            0                                                                                                                                                                                   & \text{otherwise.}
          \end{dcases}
        \end{equation*}

  \item[$n=3:$]
        \begin{equation*}
          \gr^{j}\Bigl( \bigwedge^{3}\lie{g} \Bigr) =
          \begin{dcases}
            \lie{g}^{2} \wedge \lie{g}^{3} \wedge \lie{g}^{3} & j=8,                                                                                                                   \\
            \!\begin{aligned} & \lie{g}^{1} \wedge \lie{g}^{3} \wedge \lie{g}^{3} \\ & \oplus \lie{g}^{2} \wedge \lie{g}^{2} \wedge \lie{g}^{3} \end{aligned} & j=7, \\
            \!\begin{aligned} & \lie{g}^{1} \wedge \lie{g}^{2} \wedge \lie{g}^{3} \\ & \oplus \lie{g}^{2} \wedge \lie{g}^{2} \wedge \lie{g}^{2} \end{aligned} & j=6, \\
            \!\begin{aligned} & \lie{g}^{1} \wedge \lie{g}^{1} \wedge \lie{g}^{3} \\ & \oplus \lie{g}^{1} \wedge \lie{g}^{2} \wedge \lie{g}^{2} \end{aligned} & j=5, \\
            \lie{g}^{1} \wedge \lie{g}^{1} \wedge \lie{g}^{2}                                                                                                & j=4, \\
            \lie{g}^{1} \wedge \lie{g}^{1} \wedge \lie{g}^{1}                                                                                                & j=3, \\
            0                                                                                                                                              & \text{otherwise.}
          \end{dcases}
        \end{equation*}

  \item[$n=2:$]
        \begin{equation*}
          \gr^{j}\Bigl( \bigwedge^{2}\lie{g} \Bigr) =
          \begin{dcases}
            \lie{g}^{3} \wedge \lie{g}^{3} & j=6,                                                                             \\
            \lie{g}^{2} \wedge \lie{g}^{3} & j=5,                                                                             \\
            \!\begin{aligned} & \lie{g}^{1} \wedge \lie{g}^{3} \\ & \oplus \lie{g}^{2} \wedge \lie{g}^{2} \end{aligned} & j=4, \\
            \lie{g}^{1} \wedge \lie{g}^{2}                                                                             & j=3, \\
            \lie{g}^{1} \wedge \lie{g}^{1}                                                                             & j=2, \\
            0                                                                                                         & \text{otherwise.}
          \end{dcases}
        \end{equation*}

  \item[$n=1:$]
        \begin{equation*}
          \gr^{j}(\lie{g}) =
          \begin{dcases}
            \lie{g}^{3} & j=3, \\
            \lie{g}^{2} & j=2, \\
            \lie{g}^{1} & j=1, \\
            0          & \text{otherwise.}
          \end{dcases}
        \end{equation*}

  \item[$n=0:$]
        \begin{equation*}
          \gr^{j}(k) =
          \begin{dcases}
            k & j=0, \\
            0 & \text{otherwise.}
          \end{dcases}
        \end{equation*}
\end{description}

\begin{table}[h]
  \centering
  $\begin{NiceArray}{*{17}{c}}[hvlines]
    \diagbox{n}{j} & 0 & 1 & 2 & 3 & 4 & 5 & 6 & 7 & 8 & 9 & 10 & 11 & 12 & 13 & 14 & 15 \\
    0 & 1 \\
    1 & & 3 & 3 & 2 \\
    2 & & & 3 & 9 & 9 & 6 & 1 \\
    3 & & & & 1 & 9 & 15 & 19 & 9 & 3 \\
    4 & & & & & & 3 & 11 & 21 & 21 & 11 & 3 \\
    5 & & & & & & & & 3 & 9 & 19 & 15 & 9 & 1 \\
    6 & & & & & & & & & & 1 & 6 & 9 & 9 & 3 \\
    7 & & & & & & & & & & & & & 2 & 3 & 3 \\
    8 & & & & & & & & & & & & & & & & 1
  \end{NiceArray}$
  \caption[Graded complex dimensions for $I \subseteq \SL_{3}(\Z_{p})$]{Dimensions of $\gr^{j}\bigl( \bigwedge^{n} \lie{g} \bigr)$.}
  \label{tab:graded-dims-SL3}
\end{table}

\begin{equation*}
  \Hom_{k}\Bigl( \bigwedge^{n}\lie{g}, k \Bigr) = \bigoplus_{s \in \Z} \Hom_{k}^{s}\Bigl( \bigwedge^{n}\lie{g}, k \Bigr)
\end{equation*}

With $j=-s$, we get the same table for dimensions of the graded hom-spaces.

Note that when finding cohomology, we only need to consider $H^{s,t} = H^{s,n-s}$ for the non-zero entries of \Cref{tab:graded-dims-SL3}.

We repeatedly use that, if
\begin{equation*}
  d \snfsim \SNF^{n,m}(a_{1},\dotsc,a_{r},0,\dotsc,0)
\end{equation*}
with $a_{1},\dotsc,a_{r}$ non-zero (in $\F_{p}$), then
\begin{align*}
  \dim \kernel(d) &= m-r, \\
  \dim \image(d) &= r, \\
  \dim \coker(d) &= n-r.
\end{align*}

$\gr^{0}:$
\[
  \begin{tikzcd}
    0 \ar[r] & k \ar[r] & 0
  \end{tikzcd}
\]



\[
  \begin{tikzcd}
    0 & \ar[l] \Hom_{k}^{0}(k,k) & \ar[l] 0
  \end{tikzcd}
\]

So $H^{0} = H^{0,0}$ with $\dim H^{0,0} = 1$.

$\gr^{1}:$
\[
  \begin{tikzcd}
    0 \ar[r] & \lie{g}^{1} \ar[r] & 0
  \end{tikzcd}
\]

\[
  \begin{tikzcd}
    0 & \ar[l] \Hom_{k}^{-1}(\lie{g},k) & \ar[l] 0
  \end{tikzcd}
\]

So $\dim H^{-1,2} = 3$ by \Cref{tab:graded-dims-SL3}.

$\gr^{2}:$
\[
  \begin{tikzcd}[ampersand replacement=\&]
    0 \ar[r] \& \lie{g}^{1} \wedge \lie{g}^{1} \ar[r, "{\begin{pmatrix} 1 & 0 & 0 \\ 0 & -1 & 0 \\ 0 & 0 & 1 \end{pmatrix}}" {yshift=7pt}] \& \lie{g}^{2} \ar[r] \& 0
  \end{tikzcd}
\]

\begin{align*}
  \lie{g}^{1} \wedge \lie{g}^{1} &\to \lie{g}^{2} \\
  \xi_{1} \wedge \xi_{6} &\mapsto -[\xi_{1},\xi_{6}] = \xi_{2} \\
  \xi_{1} \wedge \xi_{7} &\mapsto -[\xi_{1},\xi_{7}] = -\xi_{3} \\
  \xi_{6} \wedge \xi_{7} &\mapsto -[\xi_{6},\xi_{7}] = \xi_{8}.
\end{align*}

\[
  \begin{tikzcd}[ampersand replacement=\&]
    0 \& \ar[l] \Hom_{k}^{-2}\bigl( \bigwedge^{2} \lie{g}, k \bigr) \& \ar[l, "{\begin{pmatrix} 1 & 0 & 0 \\ 0 & -1 & 0 \\ 0 & 0 & 1 \end{pmatrix}}"' {yshift=7pt}] \Hom_{k}^{-2}(\lie{g},k) \& \ar[l] 0
  \end{tikzcd}
\]

\begin{equation*}
  d = \pmat{1&0&0 \\ 0&-1&0 \\ 0&0&1} \snfsim  \SNF^{3\times3}(1,-1,1).
\end{equation*}

So
\begin{align*}
  \dim H^{-2,3} &= \dim \kernel(d) = 0, \\
  \dim H^{-2,4} &= \dim \coker(d) = 0.
\end{align*}

$\gr^{3}:$
\[
  \begin{tikzcd}[ampersand replacement=\&, column sep=4em]
    0 \ar[r] \& \lie{g}^{1} \wedge \lie{g}^{1} \wedge \lie{g}^{1} \ar[r, "{\begin{pmatrix} 0 & 0 & -1 & 0 & -1 & 0 & -1 & 0 & 0 \end{pmatrix}^{\top}}" {yshift=7pt}] \& \lie{g}^{1} \wedge \lie{g}^{2} \ar[r, "{\begin{pmatrix} 0 & 0 & 1 & 0 & 0 & 0 & -1 & 0 & 0 \\ 0 & 0 & 1 & 0 & -1 & 0 & 0 & 0 & 0 \end{pmatrix}}"' {yshift=-7pt}] \& \lie{g}^{3} \ar[r] \& 0
  \end{tikzcd}
\]

\[
  \begin{tikzcd}[ampersand replacement=\&, column sep=1em]
    0 \& \ar[l] \Hom_{k}^{-3}\bigl( \bigwedge^{3}\lie{g}, k \bigr) \& \ar[l, "{\begin{pmatrix} 0 & 0 & -1 & 0 & -1 & 0 & -1 & 0 & 0 \end{pmatrix}}"' {yshift=7pt}] \Hom_{k}^{-3}\bigl( \bigwedge^{2}\lie{g}, k \bigr) \& \ar[l, "{\begin{pmatrix} 0 & 0 & 1 & 0 & 0 & 0 & -1 & 0 & 0 \\ 0 & 0 & 1 & 0 & -1 & 0 & 0 & 0 & 0 \end{pmatrix}^{\top}}" {yshift=-7pt}] \Hom_{k}^{-3}(\lie{g}, k) \& \ar[l] 0
  \end{tikzcd}
\]

\begin{align*}
  d_{1} = \pmat{0&0 \\ 0&0 \\ 1&1 \\ 0&0 \\ 0&-1 \\ 0&0 \\ 0&-1 \\ 0&0 \\ 0&0} &\snfsim \SNF^{9\times2}(1,-1), \\
  d_{2} = \pmat{0&0&-1&0&-1&0&-1&0&0} &\snfsim \SNF^{1\times9}(-1).
\end{align*}

So
\begin{align*}
  \dim H^{-3,4} &= \dim \kernel(d_{1}) = 2-2 = 0, \\
  \dim H^{-3,5} &= \dim \dfrac{\kernel(d_{2})}{\image(d_{1})} = (9-1) - 2 = 6, \\
  \dim H^{-3,6} &= \dim \coker(d_{2}) = 1 - 1 = 0.
\end{align*}

$\gr^{4}:$
\[
  \begin{tikzcd}[ampersand replacement=\&]
    0 \ar[r] \& \lie{g}^{1} \wedge \wedge \lie{g}^{1} \wedge \lie{g}^{2} \ar[r,"d^{\top}"] \& \begin{aligned} &\lie{g}^{1} \wedge \lie{g}^{3} \\ &\oplus \lie{g}^{2} \wedge \lie{g}^{2} \end{aligned} \ar[r] \& 0
  \end{tikzcd}
\]

\[
  \begin{tikzcd}
    0 & \ar[l] \Hom_{k}^{-4}\bigl( \bigwedge^{3}\lie{g}, k \bigr) & \ar[l,"d"'] \Hom_{k}^{-4}\bigl( \bigwedge^{2}\lie{g}, k \bigr) &  \ar[l] 0
  \end{tikzcd}
\]

\begin{equation*}
  d \snfsim \SNF^{9\times9}(1,1,1,-1,1,-1,0,0,0)
\end{equation*}

So
\begin{align*}
  \dim H^{-4,6} &= \dim \kernel(d) = 9-6 = 3, \\
  \dim H^{-4,7} &= \dim \coker(d) = 9-6 = 3.
\end{align*}

$\gr^{5}:$
\[
  \begin{tikzcd}[ampersand replacement=\&]
    0 \ar[r] \& \lie{g}^{1} \wedge \lie{g}^{1} \wedge \lie{g}^{1} \wedge \lie{g}^{2} \ar[r,"d_{2}^{\top}"] \& \begin{aligned} &\lie{g}^{1} \wedge \lie{g}^{1} \wedge \lie{g}^{3} \\ &\oplus \lie{g}^{1} \wedge \lie{g}^{2} \wedge \lie{g}^{2} \end{aligned} \ar[r,"d_{1}^{\top}"] \& \lie{g}^{2} \wedge \lie{g}^{3} \ar[r] \& 0
  \end{tikzcd}
\]

\[
  \begin{tikzcd}[column sep=1em]
    0 & \ar[l] \Hom_{k}^{-5}\bigl( \bigwedge^{4}\lie{g}, k \bigr) & \ar[l,"d_{2}"' {yshift=2pt}] \Hom_{k}^{-5}\bigl( \bigwedge^{3}\lie{g}, k \bigr) & \ar[l,"d_{1}"' {yshift=2pt}] \Hom_{k}^{-5}\bigl( \bigwedge^{2}\lie{g}, k \bigr) & \ar[l] 0
  \end{tikzcd}
\]

\begin{align*}
  d_{1} &\snfsim \SNF^{15\times6}(1,1,-1,-1,1,1), \\
  d_{2} &\snfsim \SNF^{3\times15}(-1,1,1).
\end{align*}

So
\begin{align*}
  \dim H^{-5,7} &= \dim \kernel(d_{1}) = 6-6 = 0, \\
  \dim H^{-5,8} &= \dim \dfrac{\kernel(d_{2})}{\image(d_{1})} = (15-3) - 6 = 6, \\
  \dim H^{-5,9} &= \dim \coker(d_{2}) = 3 - 3 = 0.
\end{align*}

$\gr^{6}:$
\[
  \begin{tikzcd}[ampersand replacement=\&]
    0 \ar[r] \& \begin{aligned} &\lie{g}^{1} \wedge \lie{g}^{1} \wedge \lie{g}^{1} \wedge \lie{g}^{3} \\ &\oplus \lie{g}^{1} \wedge \lie{g}^{1} \wedge \lie{g}^{2} \wedge \lie{g}^{2} \end{aligned} \ar[r,"d_{2}^{\top}"] \& \begin{aligned} &\lie{g}^{1} \wedge \lie{g}^{2} \wedge \lie{g}^{3} \\ &\oplus \lie{g}^{2} \wedge \lie{g}^{2} \wedge \lie{g}^{2} \end{aligned} \ar[r,"d_{1}^{\top}"] \& \lie{g}^{3} \wedge \lie{g}^{3} \ar[r] \& 0
  \end{tikzcd}
\]

\[
  \begin{tikzcd}[column sep=1em]
    0 & \ar[l] \Hom_{k}^{-6}\bigl( \bigwedge^{4}\lie{g}, k \bigr) & \ar[l,"d_{2}"' {yshift=2pt}] \Hom_{k}^{-6}\bigl( \bigwedge^{3}\lie{g}, k \bigr) & \ar[l,"d_{1}"' {yshift=2pt}] \Hom_{k}^{-6}\bigl( \bigwedge^{2}\lie{g}, k \bigr) & \ar[l] 0
  \end{tikzcd}
\]

\begin{align*}
  d_{1} &\snfsim \SNF^{19\times1}(-1), \\
  d_{2} &\snfsim \SNF^{11\times19}(-1,1,-1,1,-1,-1,-1,1,1,1,-2).
\end{align*}

So
\begin{align*}
  \dim H^{-6,8} &= \dim \kernel(d_{1}) = 1-1 = 0, \\
  \dim H^{-6,9} &= \dim \dfrac{\kernel(d_{2})}{\image(d_{1})} = (19-11) - 1 = 7, \\
  \dim H^{-6,10} &= \dim \coker(d_{2}) = 11 - 11 = 0.
\end{align*}

$\gr^{7}:$
\[
  \begin{tikzcd}[ampersand replacement=\&, column sep=1em]
    0 \ar[r] \& \lie{g}^{1} \wedge \lie{g}^{1} \wedge \lie{g}^{1} \wedge \lie{g}^{2} \wedge \lie{g}^{2} \ar[r,"d_{2}^{\top}" {yshift=2pt}] \& \begin{aligned} &\lie{g}^{1} \wedge \lie{g}^{1} \wedge \lie{g}^{2} \wedge \lie{g}^{3} \\ &\oplus \lie{g}^{1} \wedge \lie{g}^{2} \wedge \lie{g}^{2} \wedge \lie{g}^{2} \end{aligned} \ar[r,"d_{1}^{\top}" {yshift=2pt}] \& \begin{aligned} &\lie{g}^{1} \wedge \lie{g}^{3} \wedge \lie{g}^{3} \\ &\oplus \lie{g}^{2} \wedge \lie{g}^{2} \wedge \lie{g}^{3} \end{aligned}\ar[r] \& 0
  \end{tikzcd}
\]

\[
  \begin{tikzcd}[column sep=1em]
    0 & \ar[l] \Hom_{k}^{-7}\bigl( \bigwedge^{5}\lie{g}, k \bigr) & \ar[l,"d_{2}"' {yshift=2pt}] \Hom_{k}^{-7}\bigl( \bigwedge^{4}\lie{g}, k \bigr) & \ar[l,"d_{1}"' {yshift=2pt}] \Hom_{k}^{-7}\bigl( \bigwedge^{3}\lie{g}, k \bigr) & \ar[l] 0
  \end{tikzcd}
\]

\begin{align*}
  d_{1} &\snfsim \SNF^{21\times9}(-1,-1,-1,1,1,1,1,-1,1), \\
  d_{2} &\snfsim \SNF^{3\times21}(1,1,-1).
\end{align*}

So
\begin{align*}
  \dim H^{-7,10} &= \dim \kernel(d_{1}) = 9-9 = 0, \\
  \dim H^{-7,11} &= \dim \dfrac{\kernel(d_{2})}{\image(d_{1})} = (21-3) - 9 = 9, \\
  \dim H^{-7,12} &= \dim \coker(d_{2}) = 3 - 3 = 0.
\end{align*}

The following calculations are not necessary, since we can get the results using a version of Poincaré duality for Lie algebra cohomology, but we keep the sketch work to make it clear that nothing goes wrong.

$\gr^{8}:$

\begin{align*}
  d_{1} &\snfsim \SNF^{21\times3}(1,-1,1), \\
  d_{2} &\snfsim \SNF^{9\times21}(-1,-1,-1,1,1,-1,-1,1,-1).
\end{align*}

So
\begin{align*}
  \dim H^{-8,11} &= \dim \kernel(d_{1}) = 3-3 = 0, \\
  \dim H^{-8,12} &= \dim \dfrac{\kernel(d_{2})}{\image(d_{1})} = (21-9) - 3 = 9, \\
  \dim H^{-8,13} &= \dim \coker(d_{2}) = 9 - 9 = 0.
\end{align*}


$\gr^{9}:$

\begin{align*}
  d_{1} &\snfsim \SNF^{19\times11}(-1,-1,1,-1,1,-1,-1,-1,-1,1,-1), \\
  d_{2} &\snfsim \SNF^{1\times19}(-1).
\end{align*}

So
\begin{align*}
  \dim H^{-9,13} &= \dim \kernel(d_{1}) = 11-11 = 0, \\
  \dim H^{-9,14} &= \dim \dfrac{\kernel(d_{2})}{\image(d_{1})} = (19-1) - 11 = 7, \\
  \dim H^{-9,15} &= \dim \coker(d_{2}) = 1 - 1 = 0.
\end{align*}

$\gr^{10}:$

\begin{align*}
  d_{1} &\snfsim \SNF^{15\times3}(1,1,-1), \\
  d_{2} &\snfsim \SNF^{6\times15}(-1,1,1,-1,1,1).
\end{align*}

So
\begin{align*}
  \dim H^{-10,14} &= \dim \kernel(d_{1}) = 3-3 = 0, \\
  \dim H^{-10,15} &= \dim \dfrac{\kernel(d_{2})}{\image(d_{1})} = (15-6) - 3 = 6, \\
  \dim H^{-10,16} &= \dim \coker(d_{2}) = 6-6 = 0.
\end{align*}

$\gr^{11}:$

\begin{equation*}
  d \snfsim \SNF^{9\times9}(1,1,-1,-1,-1,-1,0,0,0).
\end{equation*}

So
\begin{align*}
  \dim H^{-11,16} &= \dim \kernel(d) = 9-6 = 3, \\
  \dim H^{-11,17} &= \dim \coker(d) = 9-6 = 3.
\end{align*}

$\gr^{12}:$

\begin{align*}
  d_{1} &\snfsim \SNF^{9\times1}(1), \\
  d_{2} &\snfsim \SNF^{2\times9}(1,-1).
\end{align*}

So
\begin{align*}
  \dim H^{-12,17} &= \dim \kernel(d_{1}) = 1-1 = 0, \\
  \dim H^{-12,18} &= \dim \dfrac{\kernel(d_{2})}{\image(d_{1})} = (9-2) - 1 = 6, \\
  \dim H^{-12,19} &= \dim \coker(d_{2}) = 2-2 = 0.
\end{align*}

$\gr^{13}:$

\begin{equation*}
  d \snfsim \SNF^{3\times3}(-1,1,-1).
\end{equation*}

So
\begin{align*}
  \dim H^{-13,19} &= \dim \kernel(d) = 3-3 = 0, \\
  \dim H^{-13,20} &= \dim \coker(d) = 3-3 = 0.
\end{align*}

$\gr^{14}:$
\[
  \begin{tikzcd}
    0 \ar[r] & \lie{g}^{1} \wedge \lie{g}^{1} \wedge \lie{g}^{2} \wedge \lie{g}^{2} \wedge \lie{g}^{2} \wedge \lie{g}^{3} \wedge \lie{g}^{3}  \ar[r] & 0
  \end{tikzcd}
\]

\[
  \begin{tikzcd}
    0 & \ar[l] \Hom_{k}^{-14}\bigl( \bigwedge^{7}\lie{g},k \bigr) & \ar[l] 0
  \end{tikzcd}
\]

So $\dim H^{-14,21} = 3$ by \Cref{tab:graded-dims-SL3}.

$\gr^{15}:$
\[
  \begin{tikzcd}
    0 \ar[r] & \lie{g}^{1} \wedge \lie{g}^{1} \wedge \lie{g}^{1} \wedge \lie{g}^{2} \wedge \lie{g}^{2} \wedge \lie{g}^{2} \wedge \lie{g}^{3} \wedge \lie{g}^{3}  \ar[r] & 0
  \end{tikzcd}
\]

\[
  \begin{tikzcd}
    0 & \ar[l] \Hom_{k}^{-15}\bigl( \bigwedge^{8}\lie{g},k \bigr) & \ar[l] 0
  \end{tikzcd}
\]

So $H^{8} = H^{-15,23}$ with $\dim H^{-15,23} = 1$ by \Cref{tab:graded-dims-SL3}.

Altogether:
\begin{align*}
  H^{0} &= H^{0,0}, \\
  H^{1} &= H^{-1,2}, \\
  H^{2} &= H^{-3,5} \oplus H^{-4,6}, \\
  H^{3} &= H^{-4,7} \oplus H^{-5,8} \oplus H^{-6,9}, \\
  H^{4} &= H^{-7,11} \oplus H^{-8,12}, \\
  H^{5} &= H^{-9,14} \oplus H^{-10,15} \oplus H^{-11,16}, \\
  H^{6} &= H^{-11,17} \oplus H^{-12,18}, \\
  H^{7} &= H^{-14,21}, \\
  H^{8} &= H^{-15,23}
\end{align*}
and we have the following table:
\begin{table}[ht]
  \centering
  \scalebox{0.8}{%
  $\begin{NiceArray}{*{17}{c}}[hvlines]
    \diagbox{t}{s} & 0 & -1 & -2 & -3 & -4 & -5 & -6 & -7 & -8 & -9 & -10 & -11 & -12 & -13 & -14 & -15 \\
    0 & 1\\
    1 \\
    2 && 3 \\
    3 \\
    4 \\
    5 &&&& 6\\
    6 &&&&& 3 \\
    7 &&&&& 3\\
    8 &&&&&& 6\\
    9 &&&&&&& 7\\
    10 \\
    11 &&&&&&&& 9 \\
    12 &&&&&&&&& 9 \\
    13 \\
    14 &&&&&&&&&& 7 \\
    15 &&&&&&&&&&& 6 \\
    16 &&&&&&&&&&&& 3 \\
    17 &&&&&&&&&&&& 3 \\
    18 &&&&&&&&&&&&& 6 \\
    19 \\
    20 \\
    21 &&&&&&&&&&&&&&& 3 \\
    22 \\
    23 &&&&&&&&&&&&&&&& 1
  \end{NiceArray}$%
  }
  \caption[Graded cohomology dimensions for $I \subseteq \SL_{3}(\Z_{p})$]{Dimensions of $E_{1}^{s,t} = H^{s,t} = \gr^{s} H^{s+t}(\lie{g},k)$.}
  \label{tab:graded-coh-dims-SL3}
\end{table}
Thus
\begin{equation*}
  \dim H^{i} =
  \begin{dcases}
    1 & i=0, \\
    3 & i=1, \\
    9 & i=2, \\
    16 & i=3, \\
    18 & i=4, \\
    16 & i=5, \\
    9 & i=6, \\
    3 & i=7, \\
    1 & i=8.
  \end{dcases}
\end{equation*}

\section{\texorpdfstring{$I \subseteq \GL_{3}(\Z_{p})$}{I in GL3(Zp)}}%
\label{sec:Iwa-GL3}

\begin{equation}
  \label{eq:gis-GL3}
  \begin{gathered}
    g_{1} = \pmat{ 1 \\ & 1 \\ p && 1 }, \quad g_{2} = \pmat{ 1 \\ p & 1 \\ && 1 }, \quad g_{3} = \pmat{ 1 \\ & 1 \\ & p & 1 }, \\
    g_{4} = \pmat{ \exp(p) \\ & \exp(-p) \\ && 1 }, \quad g_{5} = \pmat{ 1 \\ & \exp(p) \\ && \exp(-p) }, \\
    g_{6} = \pmat{ \exp(p) \\ & \exp(p) \\ && \exp(p) },  \\
    g_{7} = \pmat{ 1 \\ & 1 & 1 \\ && 1 }, \quad g_{8} = \pmat{ 1 & 1 \\ & 1 \\ && 1 }, \quad g_{9} = \pmat{ 1 && 1 \\ & 1 \\ && 1 }.
  \end{gathered}
\end{equation}


\begin{table}[ht]
  \centering
  \scalebox{0.7}{%
  $\begin{NiceArray}{*{20}{c}}[hvlines]
    \diagbox{t}{s} & 0 & -1 & -2 & -3 & -4 & -5 & -6 & -7 & -8 & -9 & -10 & -11 & -12 & -13 & -14 & -15 & -16 & -17 & -18 \\
    0 & 1\\
    1 \\
    2 && 3 \\
    3 \\
    4 &&&& 1 \\
    5 &&&& 6 \\
    6 &&&&& 6 \\
    7 &&&&& 3\\
    8 &&&&&& 6 \\
    9 &&&&&&& 13 \\
    10 &&&&&&&& 3 \\
    11 &&&&&&&& 12 \\
    12 &&&&&&&&& 15 \\
    13 &&&&&&&&&& 7 \\
    14 &&&&&&&&&& 7 \\
    15 &&&&&&&&&&& 15 \\
    16 &&&&&&&&&&&& 12 \\
    17 &&&&&&&&&&&& 3 \\
    18 &&&&&&&&&&&&& 13 \\
    19 &&&&&&&&&&&&&& 6 \\
    20 &&&&&&&&&&&&&&& 3 \\
    21 &&&&&&&&&&&&&&& 6 \\
    22 &&&&&&&&&&&&&&&& 6 \\
    23 &&&&&&&&&&&&&&&& 1 \\
    24 \\
    25 &&&&&&&&&&&&&&&&&& 3 \\
    26 \\
    27 &&&&&&&&&&&&&&&&&&& 1
  \end{NiceArray}$%
  }
  \caption[Graded cohomology dimensions for $I \subseteq \GL_{3}(\Z_{p})$]{Dimensions of $E_{1}^{s,t} = H^{s,t} = \gr^{s} H^{s+t}(\lie{g},k)$.}
  \label{tab:graded-coh-dims-GL3}
\end{table}

\section{Future work}%
\label{sec:future}

Mention spectral sequence talked about with Claus. Trying to generalize current work. Conjectures based on this chapter.

%%% Local Variables:
%%% mode: latex
%%% TeX-master: "../main"
%%% End:


\clearpage

\appendix

\chapter{Calculations}

\section{\texorpdfstring{$I \subseteq \SL_{4}(\Z_{p})$}{I in SL4(Zp)}}%
\label{sec:SL4-calc}


In this section we will describe the continuous group cohomology of the pro-$p$ Iwahori subgroup $I$ of $\SL_{4}(\Q_{p})$.

When $I$ is the pro-$p$ Iwahori subgroup in $\SL_{4}(\Q_{p})$, we know by \Cref{sec:cohiwagps-intro} that we can take it to be of the form
\begin{equation*}
  I = \pmat{1+p\Z_{p} & \Z_{p} & \Z_{p} & \Z_{p} \\ p\Z_{p} & 1+p\Z_{p} & \Z_{p} & \Z_{p} \\ p\Z_{p} & p\Z_{p} & 1+p\Z_{p} & \Z_{p} \\ p\Z_{p} & p\Z_{p} & p\Z_{p} & 1+p\Z_{p}}^{\!\!\det = 1} \subseteq \SL_{4}(\Z_{p}),
\end{equation*}
and, by \Cref{sec:cohiwagps-intro}, we have an ordered basis
\begin{equation}
  \label{eq:gis-SL3}
  \begin{gathered}
    g_{1} = \pmat{ 1 \\ & 1 \\ && 1 \\ p && 1 }, \quad g_{2} = \pmat{ 1 \\ & 1 \\ p && 1 \\ &&& 1 }, \quad g_{3} = \pmat{ 1 \\ p & 1 \\ && 1 \\ &&& 1 }, \\
    g_{4} = \pmat{ 1 \\ & 1 \\ && 1 \\ & 1 && 1 }, \quad g_{5} = \pmat{ 1 \\ & 1 \\ & p & 1 \\ &&& 1 }, \quad g_{6} = \pmat{ 1 \\ & 1 \\ && 1 \\ && p & 1 }, \\
    g_{7} = \pmat{ \exp(p) \\ & \exp(-p) \\ && 1 \\ &&& 1 }, \quad g_{8} = \pmat{ 1 \\ & \exp(p) \\ && \exp(-p) \\ &&& 1 }, \\
    g_{9} = \pmat{ 1 \\ & 1 \\ && \exp(p) \\ &&& \exp(-p) }, \\
    g_{10} = \pmat{ 1 &&& 1 \\ & 1 \\ && 1 \\ &&& 1 }, \quad g_{11} = \pmat{ 1 \\ & 1 && 1 \\ && 1 \\ &&& 1 }, \quad g_{12} = \pmat{ 1 \\ & 1 \\ && 1 & 1 \\ &&& 1 }, \\
    g_{13} = \pmat{1 && 1 \\ & 1 \\ && 1 \\ &&& 1 }, \quad g_{14} = \pmat{ 1 \\ & 1 & 1 \\ && 1 \\ &&& 1 }, \quad g_{15} = \pmat{ 1 & 1 \\ & 1 \\ && 1 \\ &&& 1 }.
  \end{gathered}
\end{equation}
Here we write any zeros as blank space in matrices, to make the notation easier to read for the bigger matrices.

\begin{remark}
  Note that the order is not going from the lower left corner to the upper right corner along ``diagonals'', which might be a more standard ordering to chose. The reason we choose this alternative order is to simplify some calculations. In particular, this order gives simpler $a_{ij}$ below.
\end{remark}

\subsection{Finding the commutators \texorpdfstring{$[\xi_{i},\xi_{j}]$}{[xi-i,xi-j]}}%
\label{subsec:non-id-xi_ij-SL4}

Now
\begin{equation*}
    g_{1}^{x_{1}}g_{2}^{x_{2}} \dotsb g_{15}^{x_{15}} = \pmat{ a_{11} & a_{12} & a_{13} & a_{14} \\ a_{21} & a_{22} & a_{23} & a_{24} \\ a_{31} & a_{32} & a_{33} & a_{34} \\ a_{41} & a_{42} & a_{43} & a_{44} },
\end{equation*}
where
\begin{equation}
  \label{eq:gixi-SL4}
  \begin{aligned}
    a_{11} &= \exp(px_{7}), \\
    a_{12} &= x_{15}\exp(px_{7}), \\
    a_{13} &= x_{13}\exp(px_{7}), \\
    a_{14} &= x_{10}\exp(px_{7}), \\
    a_{21} &= px_{3}\exp(px_{7}), \\
    a_{22} &= px_{15}x_{3}\exp(px_{7}) + \exp\bigl( p(x_{8}-x_{7}) \bigr), \\
    a_{23} &= px_{13}x_{3}\exp(px_{7}) + x_{14}\exp\bigl( p(x_{8}-x_{7}) \bigr), \\
    a_{24} &= px_{10}x_{3}\exp(px_{7}) + x_{11}\exp\bigl( p(x_{8}-x_{7}) \bigr), \\
    a_{31} &= px_{2}\exp(px_{7}), \\
    a_{32} &= px_{15}x_{2}\exp(px_{7}) + px_{5}\exp\bigl( p(x_{8}-x_{7}) \bigr), \\
    a_{33} &= px_{13}x_{2}\exp(px_{7}) + px_{14}x_{5}\exp\bigl( p(x_{8}-x_{7}) \bigr) + \exp\bigl( p(x_{9}-x_{8}) \bigr), \\
    a_{34} &= px_{10}x_{2}\exp(px_{7}) + px_{11}x_{5}\exp\bigl( p(x_{8}-x_{7}) \bigr) + x_{12}\exp\bigl( p(x_{9}-x_{8}) \bigr), \\
    a_{41} &= px_{1}\exp(px_{7}), \\
    a_{42} &= px_{1}x_{15}\exp(px_{7}) + px_{4}\exp\bigl( p(x_{8}-x_{7}) \bigr) \\
    a_{43} &= px_{1}x_{13}\exp(px_{7}) + px_{14}x_{4}\exp\bigl( p(x_{8}-x_{7}) \bigr) + px_{6}\exp\bigl( p(x_{9}-x_{8}) \bigr), \\
    a_{44} &= px_{1}x_{10}\exp(px_{7}) + px_{11}x_{4}\exp\bigl( p(x_{8}-x_{7}) \bigr) + px_{12}x_{6}\exp\bigl( p(x_{9}-x_{8}) \bigr) \\*
    &\phantom{{}={}} + \exp(-px_{9}).
  \end{aligned}
\end{equation}

Furthermore, write $g_{i,j} = [g_{i},g_{j}]$ and $\xi_{i,j} = [\xi_{i},\xi_{j}]$. Then we are ready to find $x_{1},\dotsc,x_{15}$ such that $g_{i,j} = g_{1}^{x_{1}} \dotsb g_{15}^{x_{15}}$ for different $i<j$. (In the following we use that $\frac{1}{p-1} = 1 + p + p^{2} + \dotsb$ and $\log(1-p) = -p - \frac{p^{2}}{2} - \frac{p^{3}}{3} - \dotsb$.) Also, except in the first case, we will note that $x_{k} \in p\Z_{p}$ implies that the coefficient on $\xi_{k}$ in $\xi_{i,j}$ is zero.

We now list all non-identity commutators $g_{i,j} = [g_{i},g_{j}]$ and find $\xi_{i,j} = [\xi_{i},\xi_{j}]$ based on these. (For $g_{i,j} = 1_{4}$ it's clear that $x_{1} = \cdots = x_{15} = 0$, and thus $\xi_{i,j} = 0$.)

\begin{description}
  \item[$g_{1,7} = \pmat{1 \\ & 1 \\ && 1 \\  p\bigl( 1-\exp(-p) \bigr) &&& 1 }$:] Comparing $g_{1,7}$ with \eqref{eq:gixi-SL4}, we see that $x_{2} = x_{3} = x_{7} = x_{10} = x_{13} = x_{15} = 0$, and thus also $x_{4} = x_{5} = x_{8} = x_{11} = x_{14} = 0$, which implies that $x_{6} = x_{9} = x_{12} = 0$. This leaves $a_{41} = px_{1} = p\bigl( 1-\exp(-p) \bigr) = p^{2} + O(p^{3})$, which implies that $x_{1} = p + O(p^{2})$. Hence $\sigma(g_{1,7}) = \pi \act \sigma(g_{1})$, which implies that $\xi_{1,7} = 0$.

  \item[$g_{1,9} = \pmat{1 \\ & 1 \\ && 1 \\  p\bigl( 1-\exp(-p) \bigr) &&& 1 }$:] Since $g_{1,9} = g_{1,7}$, the above calculation shows that $\xi_{1,8} = 0$.

  \item[$g_{1,10} = \pmat{1-p &&& p \\ & 1 \\ && 1 \\ -p^{2} &&& 1+p+p^{2}}$:] Comparing $g_{1,10}$ with \eqref{eq:gixi-SL4}, we see that all $x_{i}$ are in $p\Z_{p}$ except $x_{7} = x_{8} = x_{9}$, for which we have $a_{11} = \exp(px_{7}) = 1-p$, and thus $x_{7} = \frac{1}{p}\log(1-p) = -1 + O(p)$. Hence $\xi_{1,10} = -\xi_{7}-\xi_{8}-\xi_{9}$.

  \item[$g_{1,11} = \pmat{1 \\ -p & 1 \\ && 1 \\ &&& 1}$:] Comparing $g_{1,11}$ with \eqref{eq:gixi-SL4}, we see that all $x_{i}$ are in $p\Z_{p}$ except $x_{3}$, for which we have $a_{21} = px_{3} = -p$, and thus $x_{3} = -1$. Hence $\xi_{1,11} = -\xi_{3}$.

  \item[$g_{1,12} = \pmat{1 \\ & 1 \\ -p && 1 \\ &&& 1}$:] Comparing $g_{1,12}$ with \eqref{eq:gixi-SL4}, we see that all $x_{i}$ are in $p\Z_{p}$ except $x_{2}$, for which we have $a_{31} = px_{2} = -p$, and thus $x_{2} = -1$. Hence $\xi_{1,12} = -\xi_{2}$.

  \item[$g_{1,13} = \pmat{1 \\ & 1 \\ && 1 \\ && p & 1}$:] Comparing $g_{1,13}$ with \eqref{eq:gixi-SL4}, we see that all $x_{i}$ are in $p\Z_{p}$ except $x_{6}$, for which we have $a_{43} = px_{6} = p$, and thus $x_{6} = 1$. Hence $\xi_{1,13} = \xi_{6}$.

  \item[$g_{1,15} = \pmat{1 \\ & 1 \\ && 1 \\ & p && 1}$:] Comparing $g_{1,15}$ with \eqref{eq:gixi-SL4}, we see that all $x_{i}$ are in $p\Z_{p}$ except $x_{4}$, for which we have $a_{42} = px_{4} = p$, and thus $x_{4} = 1$. Hence $\xi_{1,15} = \xi_{4}$.

  \item[$g_{2,6} = \pmat{1 \\ & 1 \\ && 1 \\ -p^{2} &&& 1}$:] Comparing $g_{2,6}$ with \eqref{eq:gixi-SL4}, we see that all $x_{i}$ are in $p\Z_{p}$. Hence $\xi_{2,6} = 0$.

  \item[$g_{2,7} = \pmat{1 \\ & 1 \\ p\bigl( 1-\exp(-p) \bigr) && 1 \\ &&& 1}$:] Comparing $g_{2,7}$ with \eqref{eq:gixi-SL4}, we see that all $x_{i}$ are in $p\Z_{p}$. Hence $\xi_{2,7} = 0$.

  \item[$g_{2,8} = \pmat{1 \\ & 1 \\ p\bigl( 1-\exp(-p) \bigr) && 1 \\ &&& 1 }$:] Since $g_{2,8} = g_{2,7}$, the above shows that $\xi_{2,8} = 0$.

  \item[$g_{2,9} = \pmat{1 \\ & 1 \\ p\bigl( 1-\exp(p) \bigr) && 1 \\ &&& 1}$:] Comparing $g_{2,9}$ with \eqref{eq:gixi-SL4}, we see that all $x_{i}$ are in $p\Z_{p}$. Hence $\xi_{2,9} = 0$.

  \item[$g_{2,10} = \pmat{1 \\ & 1 \\ && 1 & p \\ &&& 1}$:] Comparing $g_{2,10}$ with \eqref{eq:gixi-SL4}, we see that all $x_{i}$ are in $p\Z_{p}$. Hence $\xi_{2,10} = 0$.

  \item[$g_{2,13} = \pmat{1-p && p \\ & 1 \\ -p^{2} && 1+p+p^{2} \\ &&& 1}$:] Comparing $g_{2,13}$ with \eqref{eq:gixi-SL4}, we see that all $x_{i}$ are in $p\Z_{p}$ except $x_{7} = x_{8}$, for which we have $a_{11} = \exp(px_{7}) = 1-p$, and thus $x_{7} = \frac{1}{p}\log(1-p) = -1 + O(p)$. Hence $\xi_{2,13} = -\xi_{7}-\xi_{8}$.

  \item[$g_{2,14} = \pmat{1 \\ -p & 1 \\ && 1 \\ &&& 1}$:] Comparing $g_{2,14}$ with \eqref{eq:gixi-SL4}, we see that all $x_{i}$ are in $p\Z_{p}$ except $x_{3}$, for which we have $a_{21} = px_{3} = -p$, and thus $x_{3} = -1$. Hence $\xi_{2,14} = -\xi_{3}$.

  \item[$g_{2,15} = \pmat{1 \\ & 1 \\ & p & 1 \\ &&& 1}$:] Comparing $g_{2,15}$ with \eqref{eq:gixi-SL4}, we see that all $x_{i}$ are in $p\Z_{p}$ except $x_{5}$, for which we have $a_{32} = px_{5} = p$, and thus $x_{5} = 1$. Hence $\xi_{2,15} = \xi_{5}$.

  \item[$g_{3,4} = \pmat{1 \\ & 1 \\ && 1 \\ -p^{2} &&& 1}$:] Comparing $g_{3,4}$ with \eqref{eq:gixi-SL4}, we see that all $x_{i}$ are in $p\Z_{p}$. Hence $\xi_{3,4} = 0$.

  \item[$g_{3,5} = \pmat{1 \\ & 1 \\ -p^{2} && 1 \\ &&& 1}$:] Comparing $g_{3,5}$ with \eqref{eq:gixi-SL4}, we see that all $x_{i}$ are in $p\Z_{p}$. Hence $\xi_{3,5} = 0$.

  \item[$g_{3,7} = \pmat{1 \\ p\bigl( 1-\exp(-2p) \bigr) & 1 \\ && 1 \\ &&& 1}$:] Comparing $g_{3,7}$ with \eqref{eq:gixi-SL4}, we see that all $x_{i}$ are in $p\Z_{p}$. Hence $\xi_{3,7} = 0$.

  \item[$g_{3,8} = \pmat{1 \\ p\bigl( 1-\exp(p) \bigr) & 1 \\ && 1 \\ &&& 1}$:] Comparing $g_{3,8}$ with \eqref{eq:gixi-SL4}, we see that all $x_{i}$ are in $p\Z_{p}$. Hence $\xi_{3,8} = 0$.

  \item[$g_{3,10} = \pmat{1 \\ & 1 && p \\ && 1 \\ &&& 1}$:] Comparing $g_{3,10}$ with \eqref{eq:gixi-SL4}, we see that all $x_{i}$ are in $p\Z_{p}$. Hence $\xi_{3,10} = 0$.

  \item[$g_{3,13} = \pmat{1 \\ & 1 & p \\ && 1 \\ &&& 1}$:] Comparing $g_{3,13}$ with \eqref{eq:gixi-SL4}, we see that all $x_{i}$ are in $p\Z_{p}$. Hence $\xi_{3,13} = 0$.

  \item[$g_{3,15} = \pmat{1-p & p \\ -p^{2} & 1+p+p^{2} \\ && 1 \\ &&& 1}$:] Comparing $g_{3,15}$ with \eqref{eq:gixi-SL4}, we see that all $x_{i}$ are in $p\Z_{p}$ except $x_{7}$, for which we have $a_{11} = \exp(px_{7}) = 1-p$, and thus $x_{7} = \frac{1}{p}\log(1-p) = -1 + O(p)$. Hence $\xi_{3,15} = -\xi_{7}$.

  \item[$g_{4,7} = \pmat{1 \\ & 1 \\ && 1 \\ & p\bigl( 1-\exp(p) \bigr) && 1}$:] Comparing $g_{4,7}$ with \eqref{eq:gixi-SL4}, we see that all $x_{i}$ are in $p\Z_{p}$. Hence $\xi_{4,7} = 0$.

  \item[$g_{4,8} = \pmat{1 \\ & 1 \\ && 1 \\ & p\bigl( 1-\exp(-p) \bigr) && 1}$:] Comparing $g_{4,8}$ with \eqref{eq:gixi-SL4}, we see that all $x_{i}$ are in $p\Z_{p}$. Hence $\xi_{4,8} = 0$.

  \item[$g_{4,9} = \pmat{1 \\ & 1 \\ && 1 \\ & p\bigl( 1-\exp(-p) \bigr) && 1 }$:] Since $g_{4,9} = g_{4,8}$, the above shows that $\xi_{4,9} = 0$.

  \item[$g_{4,10} = \pmat{1 & -p \\ & 1 \\ && 1 \\ &&& 1}$:] Comparing $g_{4,10}$ with \eqref{eq:gixi-SL4}, we see that all $x_{i}$ are in $p\Z_{p}$. Hence $\xi_{4,10} = 0$.

  \item[$g_{4,11} = \pmat{1 \\ & 1-p && p \\ && 1 \\ & -p^{2} && 1+p+p^{2}}$:] Comparing $g_{4,11}$ with \eqref{eq:gixi-SL4}, we see that all $x_{i}$ are in $p\Z_{p}$ except $x_{8}=x_{9}$, for which we have $a_{22} = \exp(px_{8}) = 1-p$, and thus $x_{8} = \frac{1}{p}\log(1-p) = -1 + O(p)$. Hence $\xi_{4,11} = -\xi_{8}-\xi_{9}$.

  \item[$g_{4,12} = \pmat{1 \\ & 1 \\ & -p & 1 \\ &&& 1}$:] Comparing $g_{4,12}$ with \eqref{eq:gixi-SL4}, we see that all $x_{i}$ are in $p\Z_{p}$ except $x_{5}$, for which we have $a_{32} = px_{5} = -p$, and thus $x_{5} = -1$. Hence $\xi_{4,12} = -\xi_{5}$.

  \item[$g_{4,14} = \pmat{1 \\ & 1 \\ && 1 \\ && p & 1}$:] Comparing $g_{4,14}$ with \eqref{eq:gixi-SL4}, we see that all $x_{i}$ are in $p\Z_{p}$ except $x_{6}$, for which we have $a_{43} = px_{6} = p$, and thus $x_{6} = 1$. Hence $\xi_{4,14} = \xi_{6}$.

  \item[$g_{5,6} = \pmat{1 \\ & 1 \\ && 1 \\ & -p^{2} && 1}$:] Comparing $g_{5,6}$ with \eqref{eq:gixi-SL4}, we see that all $x_{i}$ are in $p\Z_{p}$. Hence $\xi_{5,6} = 0$.

  \item[$g_{5,7} = \pmat{1 \\ & 1 \\ & p\bigl( 1-\exp(p) \bigr) & 1 \\ &&& 1}$:] Comparing $g_{5,7}$ with \eqref{eq:gixi-SL4}, we see that all $x_{i}$ are in $p\Z_{p}$. Hence $\xi_{5,7} = 0$.

  \item[$g_{5,8} = \pmat{1 \\ & 1 \\ & p\bigl( 1-\exp(-2p) \bigr) & 1 \\ &&& 1}$:] Comparing $g_{5,8}$ with \eqref{eq:gixi-SL4}, we see that all $x_{i}$ are in $p\Z_{p}$. Hence $\xi_{5,8} = 0$.

  \item[$g_{5,9} = \pmat{1 \\ & 1 \\ & p\bigl( 1-\exp(p) \bigr) & 1 \\ &&& 1}$:] Since $g_{5,9} = g_{5,7}$, the above shows that $\xi_{5,9} = 0$.

  \item[$g_{5,11} = \pmat{1 \\ & 1 \\ && 1 & p \\ &&& 1}$:] Comparing $g_{5,11}$ with \eqref{eq:gixi-SL4}, we see that all $x_{i}$ are in $p\Z_{p}$. Hence $\xi_{5,11} = 0$.

  \item[$g_{5,13} = \pmat{1 & -p \\ & 1 \\ && 1 \\ &&& 1}$:] Comparing $g_{5,13}$ with \eqref{eq:gixi-SL4}, we see that all $x_{i}$ are in $p\Z_{p}$. Hence $\xi_{5,13} = 0$.

  \item[$g_{5,14} = \pmat{1 \\ & 1-p & p \\ & -p^{2} & 1+p+p^{2} \\ &&& 1}$:] Comparing $g_{5,14}$ with \eqref{eq:gixi-SL4}, we see that all $x_{i}$ are in $p\Z_{p}$ except $x_{8}$, for which we have $a_{22} = \exp(px_{8}) = 1-p$, and thus $x_{8} = \frac{1}{p}\log(1-p) = -1 + O(p)$. Hence $\xi_{5,14} = -\xi_{8}$.

  \item[$g_{6,8} = \pmat{1 \\ & 1 \\ && 1 \\ && p\bigl( 1-\exp(p) \bigr) & 1}$:] Comparing $g_{6,8}$ with \eqref{eq:gixi-SL4}, we see that all $x_{i}$ are in $p\Z_{p}$. Hence $\xi_{6,8} = 0$.

  \item[$g_{6,9} = \pmat{1 \\ & 1 \\ && 1 \\ && p\bigl( 1-\exp(-2p) \bigr) & 1}$:] Comparing $g_{6,9}$ with \eqref{eq:gixi-SL4}, we see that all $x_{i}$ are in $p\Z_{p}$. Hence $\xi_{6,9} = 0$.

  \item[$g_{6,10} = \pmat{1 && -p \\ & 1 \\ && 1 \\ &&& 1}$:] Comparing $g_{6,10}$ with \eqref{eq:gixi-SL4}, we see that all $x_{i}$ are in $p\Z_{p}$. Hence $\xi_{6,10} = 0$.

  \item[$g_{6,11} = \pmat{1 \\ & 1 & -p \\ && 1 \\ &&& 1}$:] Comparing $g_{6,11}$ with \eqref{eq:gixi-SL4}, we see that all $x_{i}$ are in $p\Z_{p}$. Hence $\xi_{6,11} = 0$.

  \item[$g_{6,12} = \pmat{1 \\ & 1 \\ && 1-p & p \\ && -p^{2} & 1+p+p^{2}}$:] Comparing $g_{6,12}$ with \eqref{eq:gixi-SL4}, we see that all $x_{i}$ are in $p\Z_{p}$ except $x_{9}$, for which we have $a_{33} = \exp(px_{9}) = 1-p$, and thus $x_{9} = \frac{1}{p}\log(1-p) = -1 + O(p)$. Hence $\xi_{6,12} = -\xi_{9}$.

  \item[$g_{7,10} = \pmat{1 &&& \exp(p)-1 \\ & 1 \\ && 1 \\ &&& 1}$:] Comparing $g_{7,10}$ with \eqref{eq:gixi-SL4}, we see that all $x_{i}$ are in $p\Z_{p}$. Hence $\xi_{7,10} = 0$.

  \item[$g_{7,11} = \pmat{1 \\ & 1 && \exp(-p)-1 \\ && 1 \\ &&& 1}$:] Comparing $g_{7,11}$ with \eqref{eq:gixi-SL4}, we see that all $x_{i}$ are in $p\Z_{p}$. Hence $\xi_{7,11} = 0$.

  \item[$g_{7,13} = \pmat{1 && \exp(p)-1 \\ & 1 \\ && 1 \\ &&& 1}$:] Comparing $g_{7,13}$ with \eqref{eq:gixi-SL4}, we see that all $x_{i}$ are in $p\Z_{p}$. Hence $\xi_{7,13} = 0$.

  \item[$g_{7,14} = \pmat{1 \\ & 1 & \exp(-p)-1 \\ && 1 \\ &&& 1}$:] Comparing $g_{7,14}$ with \eqref{eq:gixi-SL4}, we see that all $x_{i}$ are in $p\Z_{p}$. Hence $\xi_{7,14} = 0$.

  \item[$g_{7,15} = \pmat{1 & \exp(2p)-1 \\ & 1 \\ && 1 \\ &&& 1}$:] Comparing $g_{7,15}$ with \eqref{eq:gixi-SL4}, we see that all $x_{i}$ are in $p\Z_{p}$. Hence $\xi_{7,15} = 0$.

  \item[$g_{8,11} = \pmat{1 \\ & 1 && \exp(p)-1 \\ && 1 \\ &&& 1}$:] Comparing $g_{8,11}$ with \eqref{eq:gixi-SL4}, we see that all $x_{i}$ are in $p\Z_{p}$. Hence $\xi_{8,11} = 0$.

  \item[$g_{8,12} = \pmat{1 \\ & 1 \\ && 1 & \exp(-p)-1 \\ &&& 1}$:] Comparing $g_{8,12}$ with \eqref{eq:gixi-SL4}, we see that all $x_{i}$ are in $p\Z_{p}$. Hence $\xi_{8,12} = 0$.

  \item[$g_{8,13} = \pmat{1 && \exp(p)-1 \\ & 1 \\ && 1 \\ &&& 1}$:] Since $g_{8,13} = g_{7,13}$, the above shows that $\xi_{8,13} = 0$.

  \item[$g_{8,14} = \pmat{1 \\ & 1 & \exp(2p)-1 \\ && 1 \\ &&& 1}$:] Comparing $g_{8,14}$ with \eqref{eq:gixi-SL4}, we see that all $x_{i}$ are in $p\Z_{p}$. Hence $\xi_{8,14} = 0$.

  \item[$g_{8,15} = \pmat{1 & \exp(-p)-1 \\ & 1 \\ && 1 \\ &&& 1}$:] Comparing $g_{8,15}$ with \eqref{eq:gixi-SL4}, we see that all $x_{i}$ are in $p\Z_{p}$. Hence $\xi_{8,15} = 0$.

  \item[$g_{9,10} = \pmat{1 &&& \exp(p)-1 \\ & 1\\ && 1 \\ &&& 1}$:] Since $g_{9,10} = g_{7,10}$, the above shows that $\xi_{8,15} = 0$.

  \item[$g_{9,11} = \pmat{1 \\ & 1 && \exp(p)-1 \\ && 1 \\ &&& 1}$:] Since $g_{9,11} = g_{8,11}$, the above shows that $\xi_{9,11} = 0$.

  \item[$g_{9,12} = \pmat{1 \\ & 1 \\ && 1 & \exp(2p)-1 \\ &&& 1}$:] Comparing $g_{9,12}$ with \eqref{eq:gixi-SL4}, we see that all $x_{i}$ are in $p\Z_{p}$. Hence $\xi_{9,12} = 0$.

  \item[$g_{9,13} = \pmat{1 && \exp(-p)-1 \\ & 1 \\ && 1 \\ &&& 1}$:] Comparing $g_{9,13}$ with \eqref{eq:gixi-SL4}, we see that all $x_{i}$ are in $p\Z_{p}$. Hence $\xi_{9,13} = 0$.

  \item[$g_{9,14} = \pmat{1 \\ & 1 & \exp(-p)-1 \\ && 1 \\ &&& 1}$:] Comparing $g_{9,14}$ with \eqref{eq:gixi-SL4}, we see that all $x_{i}$ are in $p\Z_{p}$. Hence $\xi_{9,14} = 0$.

  \item[$g_{11,15} = \pmat{1 &&& -1 \\ & 1 \\ && 1 \\ &&& 1}$:] Comparing $g_{11,15}$ with \eqref{eq:gixi-SL4}, we see that all $x_{i}$ are in $p\Z_{p}$ except $x_{10}$, for which we have $a_{14} = x_{10} = -1$. Hence $\xi_{11,15} = -\xi_{10}$.

  \item[$g_{12,13} = \pmat{1 &&& -1 \\ & 1 \\ && 1 \\ &&& 1}$:] Since $g_{12,13} = g_{11,15}$, the above shows that $\xi_{12,13} = -\xi_{10}$.

  \item[$g_{12,14} = \pmat{1 \\ & 1 && -1 \\ && 1 \\ &&& 1}$:] Comparing $g_{12,14}$ with \eqref{eq:gixi-SL4}, we see that all $x_{i}$ are in $p\Z_{p}$ except $x_{11}$, for which we have $a_{24} = x_{11} = -1$. Hence $\xi_{12,14} = -\xi_{11}$.

  \item[$g_{14,15} = \pmat{1 && -1 \\ & 1 \\ && 1 \\ &&& 1}$:] Comparing $g_{14,15}$ with \eqref{eq:gixi-SL4}, we see that all $x_{i}$ are in $p\Z_{p}$ except $x_{13}$, for which we have $a_{13} = x_{13} = -1$. Hence $\xi_{14,15} = -\xi_{13}$.
\end{description}

Thus the non-zero commutators $[\xi_{i},\xi_{j}]$ with $i<j$ are:
\begin{equation}
  \label{eq:xi_ij-SL4}
  \begin{aligned}
    [\xi_{1},\xi_{10}] &= -(\xi_{7}+\xi_{8}+\xi_{9}), & [\xi_{1},\xi_{11}] &= -\xi_{3}, & [\xi_{1},\xi_{12}] &= -\xi_{2}, \\
    [\xi_{1},\xi_{13}] &= \xi_{6}, & [\xi_{1},\xi_{15}] &= \xi_{4}, & [\xi_{2},\xi_{13}] &= -(\xi_{7}+\xi_{8}), \\
    [\xi_{2},\xi_{14}] &= -\xi_{3}, & [\xi_{2},\xi_{15}] &= \xi_{5}, & [\xi_{3},\xi_{15}] &= -\xi_{7}, \\
    [\xi_{4},\xi_{11}] &= -(\xi_{8}+\xi_{9}), & [\xi_{4},\xi_{12}] &= -\xi_{5}, & [\xi_{4},\xi_{14}] &= \xi_{6}, \\
    [\xi_{5},\xi_{14}] &= -\xi_{8}, & [\xi_{6},\xi_{12}] &= -\xi_{9}, & [\xi_{11},\xi_{15}] &= -\xi_{10}, \\
    [\xi_{12},\xi_{13}] &= -\xi_{10}, & [\xi_{12},\xi_{14}] &= -\xi_{11}, & [\xi_{14},\xi_{15}] &= -\xi_{13}.
  \end{aligned}
\end{equation}

\subsection{Describing the graded chain complex, \texorpdfstring{$\gr^{j}\bigl(\bigwedge^{n}\lie{g}\bigr)$}{grj(wedge-n g)}}%
\label{subsec:graded-complex-SL4}

Looking at \eqref{eq:Iwa-p-val-basis-SLn} (with $e=1$ and $h=3$), we see that
\begin{align*}
  \omega(g_{1}) &= 1-\frac{2}{3} = \frac{1}{3}, \\
  \omega(g_{2}) &= 1-\frac{1}{3} = \frac{2}{3}, \\
  \omega(g_{3}) &= 1-\frac{1}{3} = \frac{2}{3}, \\
  \omega(g_{4}) &= 1, \\
  \omega(g_{5}) &= 1, \\
  \omega(g_{6}) &= \frac{1}{3}, \\
  \omega(g_{7}) &= \frac{1}{3}, \\
  \omega(g_{8}) &= \frac{2}{3}.
\end{align*}
Hence
\begin{equation*}
  \lie{g} = k \otimes_{\F_{p}[\pi]} \gr I = \Span_{k}(\xi_{1},\dotsc,\xi_{8}) = \lie{g}^{1} \oplus \lie{g}^{2} \oplus \lie{g}^{3},
\end{equation*}
where
\begin{align*}
  \lie{g}^{1} &= \lie{g}_{\frac{1}{3}} = \Span_{k}(\xi_{1},\xi_{6},\xi_{7}), \\
  \lie{g}^{2} &= \lie{g}_{\frac{2}{3}} = \Span_{k}(\xi_{2},\xi_{3},\xi_{8}), \\
  \lie{g}^{3} &= \lie{g}_{1} = \Span_{k}(\xi_{4},\xi_{5}).
\end{align*}
See \Cref{rem:g-Z-grading} for more details.




\section{\texorpdfstring{$I \subseteq \GL_{4}(\Z_{p})$}{I in GL4(Zp)}}%
\label{sec:GL4-calc}



%%% Local Variables:
%%% mode: latex
%%% TeX-master: "../main"
%%% End:


\chapter{Other research}%
\label{cha:robstat}

\section{Introduction}%
\label{sec:robstat-intro}

%%% Local Variables:
%%% mode: latex
%%% TeX-master: "../main"
%%% End:


\clearpage

\backmatter

\pagestyle{plain}

\printbibliography

\clearpage

\printindex

\end{document}

%%%%%%%%%%%%%%%%%%%%%%%%%%%%%%%%% END : DOCUMENT %%%%%%%%%%%%%%%%%%%%%%%%%%%%%%%%%

%%% Local Variables:
%%% mode: latex
%%% TeX-master: t
%%% End:
