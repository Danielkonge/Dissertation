\newcommand*\N{\mathbb{N}}
\newcommand*\Z{\mathbb{Z}}
\newcommand*\Q{\mathbb{Q}}
\newcommand*\R{\mathbb{R}}
\newcommand*\C{\mathbb{C}}
\newcommand*\F{\mathbb{F}}
\newcommand*\A{\mathbb{A}}

%% BEGIN : USER DEFINED COMMANDS %%

% define absolute value \abs{...}:
\DeclarePairedDelimiterX\abs[1]\lvert\rvert{%
  \ifblank{#1}{\:\cdot\:}{#1}
}
% define norm \norm{...}:
\DeclarePairedDelimiterX\norm[1]\lVert\rVert{%
  \ifblank{#1}{\:\cdot\:}{#1}
}
% define inner product \inner{...}{...}:
\DeclarePairedDelimiterX{\inner}[2]{\langle}{\rangle}{%
  \ifblank{#1}{\:\cdot\:}{#1},\ifblank{#2}{\:\cdot\:}{#2}
}

% define \set{...} to write sets and \given to write \set{... \given ...} for {...|...}
\newcommand*\setSymbol[1][]{
  \nonscript\:#1\vert\allowbreak\nonscript\:\mathopen{}
}
\providecommand\given{}
\DeclarePairedDelimiterX\set[1]{\lbrace}{\rbrace}{
  \renewcommand*\given{\setSymbol[\delimsize]}
  #1
}

%free group geneated by ... \free{...} or \free{... \given ...}
\DeclarePairedDelimiterX\free[1]{\langle}{\rangle}{
  \renewcommand\given{\nonscript\:\delimsize\vert\nonscript\:
    \mathopen{}}
  #1}

%define \lopen{...}{...}, \ropen{...}{...}, \open{...}{...}, \closed{...}{...} for intervals
\DeclarePairedDelimiterX\open[2](){#1,#2}
\DeclarePairedDelimiterX\lopen[2](]{#1,#2}
\DeclarePairedDelimiterX\ropen[2][){#1,#2}
\DeclarePairedDelimiterX\closed[2][]{#1,#2}

%define \Span{...} to write span of vectors
\DeclareMathOperator\Span{span}

%define \Mat to write for set of matrices
\newcommand*\Mat{\textup{Mat}}

%define \gen{...} to write <...>
\DeclarePairedDelimiterX\gen[1]\langle\rangle{
  \ifblank{#1}{\:\cdot\:}{#1}
}

\DeclareMathOperator{\Ker}{Ker} %Kernel
\DeclareMathOperator{\kernel}{ker} %kernel
\DeclareMathOperator{\image}{im}
\DeclareMathOperator{\coker}{coker}
\DeclareMathOperator*{\supp}{supp} %support
\DeclareMathOperator{\id}{id} %identity map
\DeclareMathOperator{\Id}{Id} %Identity map
\DeclareMathOperator{\ord}{ord} %order of a group
\DeclareMathOperator{\Syl}{Syl} %Sylow
\DeclareMathOperator{\GL}{GL} %GL
\DeclareMathOperator{\SL}{SL} %SL
\DeclareMathOperator{\Ad}{Ad} %Adjoint representation
\DeclareMathOperator{\ad}{ad} %adjoint representation
\DeclareMathOperator{\Char}{char} %characteristic (cannot write with small c since that is used in LaTeX already)
\DeclareMathOperator{\diag}{diag} %diagonal matrix
\DeclareMathOperator{\Tr}{Tr} %Trace
\DeclareMathOperator{\tr}{tr} %trace
\DeclareMathOperator{\rank}{rank} %rank
\DeclareMathOperator{\rk}{rk} %rank
\DeclareMathOperator{\Hom}{Hom} %Hom
\DeclareMathOperator{\homo}{hom} %hom
\DeclareMathOperator{\spec}{Spec} %spec
\DeclareMathOperator{\End}{End} %endomorphisms
\DeclareMathOperator{\Aut}{Aut} %automorphisms
\DeclareMathOperator{\Der}{Der} %derivations
\DeclareMathOperator{\Gal}{Gal} %Galois group
\DeclareMathOperator{\Frac}{Frac} %fraction field
\DeclareMathOperator{\Fr}{Fr} %Frobenius
\DeclareMathOperator{\Frob}{Frob} %Frobenius
\DeclareMathOperator{\Nm}{Nm} %norm
\DeclareMathOperator{\Mor}{Mor} %morphisms
\DeclareMathOperator{\Lie}{Lie} %Lie algebra of group scheme/Lie group
\DeclareMathOperator{\Ht}{ht} %height
\DeclareMathOperator{\uHom}{\mkern1mu\underline{\mkern-1mu Hom\mkern-1mu}\mkern1mu}
\DeclareMathOperator{\gr}{gr}
\DeclareMathOperator{\Ext}{Ext}
\DeclareMathOperator{\Tor}{Tor}
\DeclareMathOperator{\pr}{pr}
\DeclareMathOperator{\amp}{amp}
\DeclareMathOperator{\SNF}{SNF}
\DeclareMathOperator{\Fil}{Fil}
\DeclareMathOperator{\red}{red}


\newcommand*\op{^{\textup{op}}} %oposite ring/category/...
\renewcommand*\Re{\operatorname{Re}} %real part
\renewcommand*\Im{\operatorname{Im}} %imaginary part
\newcommand*{\liegp}[1]{\operatorname{#1}} %for writing Lie groups, e.g. \Lie{O}(n)
\newcommand*{\lie}[1]{\mathfrak{#1}} %for writing Lie algebras, e.g. \lie{g}
\newcommand*{\cat}[1]{\textup{\textbf{#1}}} %for writing categories, e.g. \cat{Sets}, \cat{$R$-mod},...
\newcommand*{\cats}[1]{\set{\:#1\:}} %for writing categories, e.g. \cats{$k$-algebras}, \cats{$R$-modules},...
\newcommand*\rad[1]{\sqrt{#1}} %radical
\newcommand*\cl[1]{\overline{#1}} %closure of a set
\newcommand*\rint[1]{\mathcal{O}_{#1}} %ring of integers
\newcommand*\act{\,.\,} %action of a group/Lie algebra/...
\newcommand*\rest[1]{_{\vert{#1}}} %restriction
\newcommand*\edot{\:\cdot\:} %dot for when function input is empty
\newcommand*\Sp{\textup{Sp}} %spectrum
\newcommand*\iso{\cong} %isomorphism
\newcommand*\std{_{\textup{std}}}

\newcommand*\gs[1]{\mathcal{#1}}
% \newcommand*\edot{{-}}

% \NiceMatrixOptions{cell-space-limits = 1pt}
\newcommand*\bmat[1]{\begin{bNiceMatrix} #1 \end{bNiceMatrix}}
\newcommand*\pmat[1]{\begin{pNiceMatrix} #1 \end{pNiceMatrix}}

\newcommand*\G{\mathbb{G}}
\newcommand*\T{\mathcal{T}}

\newcommand*\catF{\mathcal{F}}
\newcommand*\catG{\mathcal{G}}
\newcommand*\catP{\mathcal{P}}
\newcommand*\catD{\mathcal{D}}

\newcommand*\ZpG{\Z_p\llbracket G \rrbracket}
\newcommand*\ZpN{\Z_p\llbracket N \rrbracket}

\newcommand*\sO{\mathcal{O}}
\newcommand*\idm{\mathfrak{m}}
\newcommand*\idp{\mathfrak{p}}
\newcommand*\fo{\mathfrak{o}}
\newcommand*\sL{\mathcal{L}}

\newcommand*\defeq{\coloneqq}

\renewcommand*\projlim{\varprojlim}

\newcommand\snfsim{\:\: \overset{\makebox[0pt]{\mbox{\normalfont\tiny\sffamily SNF}}}{\sim} \:\:}
\newcommand\Hd{H_{\mathrm{dsc}}}
\newcommand\Hc{H_{\mathrm{cts}}}
\newcommand\Cont{\mathcal{C}}

%% END : USER DEFINED COMMANDS %%
