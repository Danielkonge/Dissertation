\section{\texorpdfstring{$I \subseteq \SL_{4}(\Z_{p})$}{I in SL4(Zp)}}%
\label{sec:SL4-calc}


In this section we will describe the work need to find the continuous group cohomology of the pro-$p$ Iwahori subgroup $I$ of $\SL_{4}(\Q_{p})$.

When $I$ is the pro-$p$ Iwahori subgroup in $\SL_{4}(\Q_{p})$, we know by \Cref{sec:cohiwagps-intro} that we can take it to be of the form
\begin{equation*}
  I = \pmat{1+p\Z_{p} & \Z_{p} & \Z_{p} & \Z_{p} \\ p\Z_{p} & 1+p\Z_{p} & \Z_{p} & \Z_{p} \\ p\Z_{p} & p\Z_{p} & 1+p\Z_{p} & \Z_{p} \\ p\Z_{p} & p\Z_{p} & p\Z_{p} & 1+p\Z_{p}}^{\!\!\det = 1} \subseteq \SL_{4}(\Z_{p}),
\end{equation*}
and, by \Cref{sec:cohiwagps-intro}, we have an ordered basis
\begin{equation}
  \label{eq:gis-SL4}
  \begin{gathered}
    g_{1} = \pmat{ 1 \\ & 1 \\ && 1 \\ p &&& 1 }, \quad g_{2} = \pmat{ 1 \\ & 1 \\ p && 1 \\ &&& 1 }, \quad g_{3} = \pmat{ 1 \\ p & 1 \\ && 1 \\ &&& 1 }, \\
    g_{4} = \pmat{ 1 \\ & 1 \\ && 1 \\ & p && 1 }, \quad g_{5} = \pmat{ 1 \\ & 1 \\ & p & 1 \\ &&& 1 }, \quad g_{6} = \pmat{ 1 \\ & 1 \\ && 1 \\ && p & 1 }, \\
    g_{7} = \pmat{ \exp(p) \\ & \exp(-p) \\ && 1 \\ &&& 1 }, \quad g_{8} = \pmat{ 1 \\ & \exp(p) \\ && \exp(-p) \\ &&& 1 }, \\
    g_{9} = \pmat{ 1 \\ & 1 \\ && \exp(p) \\ &&& \exp(-p) }, \\
    g_{10} = \pmat{ 1 &&& 1 \\ & 1 \\ && 1 \\ &&& 1 }, \quad g_{11} = \pmat{ 1 \\ & 1 && 1 \\ && 1 \\ &&& 1 }, \quad g_{12} = \pmat{ 1 \\ & 1 \\ && 1 & 1 \\ &&& 1 }, \\
    g_{13} = \pmat{1 && 1 \\ & 1 \\ && 1 \\ &&& 1 }, \quad g_{14} = \pmat{ 1 \\ & 1 & 1 \\ && 1 \\ &&& 1 }, \quad g_{15} = \pmat{ 1 & 1 \\ & 1 \\ && 1 \\ &&& 1 }.
  \end{gathered}
\end{equation}
Here we write any zeros as blank space in matrices, to make the notation easier to read for the bigger matrices.

\begin{remark}
  Note that the order is not going from the lower left corner to the upper right corner along ``diagonals'', which might be a more standard ordering to chose. The reason we choose this alternative order is to simplify some calculations. In particular, this order gives simpler $a_{ij}$ below.
\end{remark}

\subsection{Finding the commutators \texorpdfstring{$[\xi_{i},\xi_{j}]$}{[xi-i,xi-j]}}%
\label{subsec:non-id-xi_ij-SL4}

Now
\begin{equation*}
    g_{1}^{x_{1}}g_{2}^{x_{2}} \dotsb g_{15}^{x_{15}} = \pmat{ a_{11} & a_{12} & a_{13} & a_{14} \\ a_{21} & a_{22} & a_{23} & a_{24} \\ a_{31} & a_{32} & a_{33} & a_{34} \\ a_{41} & a_{42} & a_{43} & a_{44} },
\end{equation*}
where
\begin{equation}
  \label{eq:gixi-SL4}
  \begin{aligned}
    a_{11} &= \exp(px_{7}), \\
    a_{12} &= x_{15}\exp(px_{7}), \\
    a_{13} &= x_{13}\exp(px_{7}), \\
    a_{14} &= x_{10}\exp(px_{7}), \\
    a_{21} &= px_{3}\exp(px_{7}), \\
    a_{22} &= px_{15}x_{3}\exp(px_{7}) + \exp\bigl( p(x_{8}-x_{7}) \bigr), \\
    a_{23} &= px_{13}x_{3}\exp(px_{7}) + x_{14}\exp\bigl( p(x_{8}-x_{7}) \bigr), \\
    a_{24} &= px_{10}x_{3}\exp(px_{7}) + x_{11}\exp\bigl( p(x_{8}-x_{7}) \bigr), \\
    a_{31} &= px_{2}\exp(px_{7}), \\
    a_{32} &= px_{15}x_{2}\exp(px_{7}) + px_{5}\exp\bigl( p(x_{8}-x_{7}) \bigr), \\
    a_{33} &= px_{13}x_{2}\exp(px_{7}) + px_{14}x_{5}\exp\bigl( p(x_{8}-x_{7}) \bigr) + \exp\bigl( p(x_{9}-x_{8}) \bigr), \\
    a_{34} &= px_{10}x_{2}\exp(px_{7}) + px_{11}x_{5}\exp\bigl( p(x_{8}-x_{7}) \bigr) + x_{12}\exp\bigl( p(x_{9}-x_{8}) \bigr), \\
    a_{41} &= px_{1}\exp(px_{7}), \\
    a_{42} &= px_{1}x_{15}\exp(px_{7}) + px_{4}\exp\bigl( p(x_{8}-x_{7}) \bigr) \\
    a_{43} &= px_{1}x_{13}\exp(px_{7}) + px_{14}x_{4}\exp\bigl( p(x_{8}-x_{7}) \bigr) + px_{6}\exp\bigl( p(x_{9}-x_{8}) \bigr), \\
    a_{44} &= px_{1}x_{10}\exp(px_{7}) + px_{11}x_{4}\exp\bigl( p(x_{8}-x_{7}) \bigr) + px_{12}x_{6}\exp\bigl( p(x_{9}-x_{8}) \bigr) \\*
    &\phantom{{}={}} + \exp(-px_{9}).
  \end{aligned}
\end{equation}

Furthermore, write $g_{i,j} = [g_{i},g_{j}]$ and $\xi_{i,j} = [\xi_{i},\xi_{j}]$. Then we are ready to find $x_{1},\dotsc,x_{15}$ such that $g_{i,j} = g_{1}^{x_{1}} \dotsb g_{15}^{x_{15}}$ for different $i<j$. (In the following we use that $\frac{1}{p-1} = 1 + p + p^{2} + \dotsb$ and $\log(1-p) = -p - \frac{p^{2}}{2} - \frac{p^{3}}{3} - \dotsb$.) Also, except in the first case, we will note that $x_{k} \in p\Z_{p}$ implies that the coefficient on $\xi_{k}$ in $\xi_{i,j}$ is zero.

We now list all non-identity commutators $g_{i,j} = [g_{i},g_{j}]$ and find $\xi_{i,j} = [\xi_{i},\xi_{j}]$ based on these. (For $g_{i,j} = 1_{4}$ it's clear that $x_{1} = \cdots = x_{15} = 0$, and thus $\xi_{i,j} = 0$.)

\begin{description}
  \item[$g_{1,7} = \pmat{1 \\ & 1 \\ && 1 \\  p\bigl( 1-\exp(-p) \bigr) &&& 1 }$:] Comparing $g_{1,7}$ with \eqref{eq:gixi-SL4}, we see that $x_{2} = x_{3} = x_{7} = x_{10} = x_{13} = x_{15} = 0$, and thus also $x_{4} = x_{5} = x_{8} = x_{11} = x_{14} = 0$, which implies that $x_{6} = x_{9} = x_{12} = 0$. This leaves $a_{41} = px_{1} = p\bigl( 1-\exp(-p) \bigr) = p^{2} + O(p^{3})$, which implies that $x_{1} = p + O(p^{2})$. Hence $\sigma(g_{1,7}) = \pi \act \sigma(g_{1})$, which implies that $\xi_{1,7} = 0$.

  \item[$g_{1,9} = \pmat{1 \\ & 1 \\ && 1 \\  p\bigl( 1-\exp(-p) \bigr) &&& 1 }$:] Since $g_{1,9} = g_{1,7}$, the above calculation shows that $\xi_{1,8} = 0$.

  \item[$g_{1,10} = \pmat{1-p &&& p \\ & 1 \\ && 1 \\ -p^{2} &&& 1+p+p^{2}}$:] Comparing $g_{1,10}$ with \eqref{eq:gixi-SL4}, we see that all $x_{i}$ are in $p\Z_{p}$ except $x_{7} = x_{8} = x_{9}$, for which we have $a_{11} = \exp(px_{7}) = 1-p$, and thus $x_{7} = \frac{1}{p}\log(1-p) = -1 + O(p)$. Hence $\xi_{1,10} = -\xi_{7}-\xi_{8}-\xi_{9}$.

  \item[$g_{1,11} = \pmat{1 \\ -p & 1 \\ && 1 \\ &&& 1}$:] Comparing $g_{1,11}$ with \eqref{eq:gixi-SL4}, we see that all $x_{i}$ are in $p\Z_{p}$ except $x_{3}$, for which we have $a_{21} = px_{3} = -p$, and thus $x_{3} = -1$. Hence $\xi_{1,11} = -\xi_{3}$.

  \item[$g_{1,12} = \pmat{1 \\ & 1 \\ -p && 1 \\ &&& 1}$:] Comparing $g_{1,12}$ with \eqref{eq:gixi-SL4}, we see that all $x_{i}$ are in $p\Z_{p}$ except $x_{2}$, for which we have $a_{31} = px_{2} = -p$, and thus $x_{2} = -1$. Hence $\xi_{1,12} = -\xi_{2}$.

  \item[$g_{1,13} = \pmat{1 \\ & 1 \\ && 1 \\ && p & 1}$:] Comparing $g_{1,13}$ with \eqref{eq:gixi-SL4}, we see that all $x_{i}$ are in $p\Z_{p}$ except $x_{6}$, for which we have $a_{43} = px_{6} = p$, and thus $x_{6} = 1$. Hence $\xi_{1,13} = \xi_{6}$.

  \item[$g_{1,15} = \pmat{1 \\ & 1 \\ && 1 \\ & p && 1}$:] Comparing $g_{1,15}$ with \eqref{eq:gixi-SL4}, we see that all $x_{i}$ are in $p\Z_{p}$ except $x_{4}$, for which we have $a_{42} = px_{4} = p$, and thus $x_{4} = 1$. Hence $\xi_{1,15} = \xi_{4}$.

  \item[$g_{2,6} = \pmat{1 \\ & 1 \\ && 1 \\ -p^{2} &&& 1}$:] Comparing $g_{2,6}$ with \eqref{eq:gixi-SL4}, we see that all $x_{i}$ are in $p\Z_{p}$. Hence $\xi_{2,6} = 0$.

  \item[$g_{2,7} = \pmat{1 \\ & 1 \\ p\bigl( 1-\exp(-p) \bigr) && 1 \\ &&& 1}$:] Comparing $g_{2,7}$ with \eqref{eq:gixi-SL4}, we see that all $x_{i}$ are in $p\Z_{p}$. Hence $\xi_{2,7} = 0$.

  \item[$g_{2,8} = \pmat{1 \\ & 1 \\ p\bigl( 1-\exp(-p) \bigr) && 1 \\ &&& 1 }$:] Since $g_{2,8} = g_{2,7}$, the above shows that $\xi_{2,8} = 0$.

  \item[$g_{2,9} = \pmat{1 \\ & 1 \\ p\bigl( 1-\exp(p) \bigr) && 1 \\ &&& 1}$:] Comparing $g_{2,9}$ with \eqref{eq:gixi-SL4}, we see that all $x_{i}$ are in $p\Z_{p}$. Hence $\xi_{2,9} = 0$.

  \item[$g_{2,10} = \pmat{1 \\ & 1 \\ && 1 & p \\ &&& 1}$:] Comparing $g_{2,10}$ with \eqref{eq:gixi-SL4}, we see that all $x_{i}$ are in $p\Z_{p}$. Hence $\xi_{2,10} = 0$.

  \item[$g_{2,13} = \pmat{1-p && p \\ & 1 \\ -p^{2} && 1+p+p^{2} \\ &&& 1}$:] Comparing $g_{2,13}$ with \eqref{eq:gixi-SL4}, we see that all $x_{i}$ are in $p\Z_{p}$ except $x_{7} = x_{8}$, for which we have $a_{11} = \exp(px_{7}) = 1-p$, and thus $x_{7} = \frac{1}{p}\log(1-p) = -1 + O(p)$. Hence $\xi_{2,13} = -\xi_{7}-\xi_{8}$.

  \item[$g_{2,14} = \pmat{1 \\ -p & 1 \\ && 1 \\ &&& 1}$:] Comparing $g_{2,14}$ with \eqref{eq:gixi-SL4}, we see that all $x_{i}$ are in $p\Z_{p}$ except $x_{3}$, for which we have $a_{21} = px_{3} = -p$, and thus $x_{3} = -1$. Hence $\xi_{2,14} = -\xi_{3}$.

  \item[$g_{2,15} = \pmat{1 \\ & 1 \\ & p & 1 \\ &&& 1}$:] Comparing $g_{2,15}$ with \eqref{eq:gixi-SL4}, we see that all $x_{i}$ are in $p\Z_{p}$ except $x_{5}$, for which we have $a_{32} = px_{5} = p$, and thus $x_{5} = 1$. Hence $\xi_{2,15} = \xi_{5}$.

  \item[$g_{3,4} = \pmat{1 \\ & 1 \\ && 1 \\ -p^{2} &&& 1}$:] Comparing $g_{3,4}$ with \eqref{eq:gixi-SL4}, we see that all $x_{i}$ are in $p\Z_{p}$. Hence $\xi_{3,4} = 0$.

  \item[$g_{3,5} = \pmat{1 \\ & 1 \\ -p^{2} && 1 \\ &&& 1}$:] Comparing $g_{3,5}$ with \eqref{eq:gixi-SL4}, we see that all $x_{i}$ are in $p\Z_{p}$. Hence $\xi_{3,5} = 0$.

  \item[$g_{3,7} = \pmat{1 \\ p\bigl( 1-\exp(-2p) \bigr) & 1 \\ && 1 \\ &&& 1}$:] Comparing $g_{3,7}$ with \eqref{eq:gixi-SL4}, we see that all $x_{i}$ are in $p\Z_{p}$. Hence $\xi_{3,7} = 0$.

  \item[$g_{3,8} = \pmat{1 \\ p\bigl( 1-\exp(p) \bigr) & 1 \\ && 1 \\ &&& 1}$:] Comparing $g_{3,8}$ with \eqref{eq:gixi-SL4}, we see that all $x_{i}$ are in $p\Z_{p}$. Hence $\xi_{3,8} = 0$.

  \item[$g_{3,10} = \pmat{1 \\ & 1 && p \\ && 1 \\ &&& 1}$:] Comparing $g_{3,10}$ with \eqref{eq:gixi-SL4}, we see that all $x_{i}$ are in $p\Z_{p}$. Hence $\xi_{3,10} = 0$.

  \item[$g_{3,13} = \pmat{1 \\ & 1 & p \\ && 1 \\ &&& 1}$:] Comparing $g_{3,13}$ with \eqref{eq:gixi-SL4}, we see that all $x_{i}$ are in $p\Z_{p}$. Hence $\xi_{3,13} = 0$.

  \item[$g_{3,15} = \pmat{1-p & p \\ -p^{2} & 1+p+p^{2} \\ && 1 \\ &&& 1}$:] Comparing $g_{3,15}$ with \eqref{eq:gixi-SL4}, we see that all $x_{i}$ are in $p\Z_{p}$ except $x_{7}$, for which we have $a_{11} = \exp(px_{7}) = 1-p$, and thus $x_{7} = \frac{1}{p}\log(1-p) = -1 + O(p)$. Hence $\xi_{3,15} = -\xi_{7}$.

  \item[$g_{4,7} = \pmat{1 \\ & 1 \\ && 1 \\ & p\bigl( 1-\exp(p) \bigr) && 1}$:] Comparing $g_{4,7}$ with \eqref{eq:gixi-SL4}, we see that all $x_{i}$ are in $p\Z_{p}$. Hence $\xi_{4,7} = 0$.

  \item[$g_{4,8} = \pmat{1 \\ & 1 \\ && 1 \\ & p\bigl( 1-\exp(-p) \bigr) && 1}$:] Comparing $g_{4,8}$ with \eqref{eq:gixi-SL4}, we see that all $x_{i}$ are in $p\Z_{p}$. Hence $\xi_{4,8} = 0$.

  \item[$g_{4,9} = \pmat{1 \\ & 1 \\ && 1 \\ & p\bigl( 1-\exp(-p) \bigr) && 1 }$:] Since $g_{4,9} = g_{4,8}$, the above shows that $\xi_{4,9} = 0$.

  \item[$g_{4,10} = \pmat{1 & -p \\ & 1 \\ && 1 \\ &&& 1}$:] Comparing $g_{4,10}$ with \eqref{eq:gixi-SL4}, we see that all $x_{i}$ are in $p\Z_{p}$. Hence $\xi_{4,10} = 0$.

  \item[$g_{4,11} = \pmat{1 \\ & 1-p && p \\ && 1 \\ & -p^{2} && 1+p+p^{2}}$:] Comparing $g_{4,11}$ with \eqref{eq:gixi-SL4}, we see that all $x_{i}$ are in $p\Z_{p}$ except $x_{8}=x_{9}$, for which we have $a_{22} = \exp(px_{8}) = 1-p$, and thus $x_{8} = \frac{1}{p}\log(1-p) = -1 + O(p)$. Hence $\xi_{4,11} = -\xi_{8}-\xi_{9}$.

  \item[$g_{4,12} = \pmat{1 \\ & 1 \\ & -p & 1 \\ &&& 1}$:] Comparing $g_{4,12}$ with \eqref{eq:gixi-SL4}, we see that all $x_{i}$ are in $p\Z_{p}$ except $x_{5}$, for which we have $a_{32} = px_{5} = -p$, and thus $x_{5} = -1$. Hence $\xi_{4,12} = -\xi_{5}$.

  \item[$g_{4,14} = \pmat{1 \\ & 1 \\ && 1 \\ && p & 1}$:] Comparing $g_{4,14}$ with \eqref{eq:gixi-SL4}, we see that all $x_{i}$ are in $p\Z_{p}$ except $x_{6}$, for which we have $a_{43} = px_{6} = p$, and thus $x_{6} = 1$. Hence $\xi_{4,14} = \xi_{6}$.

  \item[$g_{5,6} = \pmat{1 \\ & 1 \\ && 1 \\ & -p^{2} && 1}$:] Comparing $g_{5,6}$ with \eqref{eq:gixi-SL4}, we see that all $x_{i}$ are in $p\Z_{p}$. Hence $\xi_{5,6} = 0$.

  \item[$g_{5,7} = \pmat{1 \\ & 1 \\ & p\bigl( 1-\exp(p) \bigr) & 1 \\ &&& 1}$:] Comparing $g_{5,7}$ with \eqref{eq:gixi-SL4}, we see that all $x_{i}$ are in $p\Z_{p}$. Hence $\xi_{5,7} = 0$.

  \item[$g_{5,8} = \pmat{1 \\ & 1 \\ & p\bigl( 1-\exp(-2p) \bigr) & 1 \\ &&& 1}$:] Comparing $g_{5,8}$ with \eqref{eq:gixi-SL4}, we see that all $x_{i}$ are in $p\Z_{p}$. Hence $\xi_{5,8} = 0$.

  \item[$g_{5,9} = \pmat{1 \\ & 1 \\ & p\bigl( 1-\exp(p) \bigr) & 1 \\ &&& 1}$:] Since $g_{5,9} = g_{5,7}$, the above shows that $\xi_{5,9} = 0$.

  \item[$g_{5,11} = \pmat{1 \\ & 1 \\ && 1 & p \\ &&& 1}$:] Comparing $g_{5,11}$ with \eqref{eq:gixi-SL4}, we see that all $x_{i}$ are in $p\Z_{p}$. Hence $\xi_{5,11} = 0$.

  \item[$g_{5,13} = \pmat{1 & -p \\ & 1 \\ && 1 \\ &&& 1}$:] Comparing $g_{5,13}$ with \eqref{eq:gixi-SL4}, we see that all $x_{i}$ are in $p\Z_{p}$. Hence $\xi_{5,13} = 0$.

  \item[$g_{5,14} = \pmat{1 \\ & 1-p & p \\ & -p^{2} & 1+p+p^{2} \\ &&& 1}$:] Comparing $g_{5,14}$ with \eqref{eq:gixi-SL4}, we see that all $x_{i}$ are in $p\Z_{p}$ except $x_{8}$, for which we have $a_{22} = \exp(px_{8}) = 1-p$, and thus $x_{8} = \frac{1}{p}\log(1-p) = -1 + O(p)$. Hence $\xi_{5,14} = -\xi_{8}$.

  \item[$g_{6,8} = \pmat{1 \\ & 1 \\ && 1 \\ && p\bigl( 1-\exp(p) \bigr) & 1}$:] Comparing $g_{6,8}$ with \eqref{eq:gixi-SL4}, we see that all $x_{i}$ are in $p\Z_{p}$. Hence $\xi_{6,8} = 0$.

  \item[$g_{6,9} = \pmat{1 \\ & 1 \\ && 1 \\ && p\bigl( 1-\exp(-2p) \bigr) & 1}$:] Comparing $g_{6,9}$ with \eqref{eq:gixi-SL4}, we see that all $x_{i}$ are in $p\Z_{p}$. Hence $\xi_{6,9} = 0$.

  \item[$g_{6,10} = \pmat{1 && -p \\ & 1 \\ && 1 \\ &&& 1}$:] Comparing $g_{6,10}$ with \eqref{eq:gixi-SL4}, we see that all $x_{i}$ are in $p\Z_{p}$. Hence $\xi_{6,10} = 0$.

  \item[$g_{6,11} = \pmat{1 \\ & 1 & -p \\ && 1 \\ &&& 1}$:] Comparing $g_{6,11}$ with \eqref{eq:gixi-SL4}, we see that all $x_{i}$ are in $p\Z_{p}$. Hence $\xi_{6,11} = 0$.

  \item[$g_{6,12} = \pmat{1 \\ & 1 \\ && 1-p & p \\ && -p^{2} & 1+p+p^{2}}$:] Comparing $g_{6,12}$ with \eqref{eq:gixi-SL4}, we see that all $x_{i}$ are in $p\Z_{p}$ except $x_{9}$, for which we have $a_{33} = \exp(px_{9}) = 1-p$, and thus $x_{9} = \frac{1}{p}\log(1-p) = -1 + O(p)$. Hence $\xi_{6,12} = -\xi_{9}$.

  \item[$g_{7,10} = \pmat{1 &&& \exp(p)-1 \\ & 1 \\ && 1 \\ &&& 1}$:] Comparing $g_{7,10}$ with \eqref{eq:gixi-SL4}, we see that all $x_{i}$ are in $p\Z_{p}$. Hence $\xi_{7,10} = 0$.

  \item[$g_{7,11} = \pmat{1 \\ & 1 && \exp(-p)-1 \\ && 1 \\ &&& 1}$:] Comparing $g_{7,11}$ with \eqref{eq:gixi-SL4}, we see that all $x_{i}$ are in $p\Z_{p}$. Hence $\xi_{7,11} = 0$.

  \item[$g_{7,13} = \pmat{1 && \exp(p)-1 \\ & 1 \\ && 1 \\ &&& 1}$:] Comparing $g_{7,13}$ with \eqref{eq:gixi-SL4}, we see that all $x_{i}$ are in $p\Z_{p}$. Hence $\xi_{7,13} = 0$.

  \item[$g_{7,14} = \pmat{1 \\ & 1 & \exp(-p)-1 \\ && 1 \\ &&& 1}$:] Comparing $g_{7,14}$ with \eqref{eq:gixi-SL4}, we see that all $x_{i}$ are in $p\Z_{p}$. Hence $\xi_{7,14} = 0$.

  \item[$g_{7,15} = \pmat{1 & \exp(2p)-1 \\ & 1 \\ && 1 \\ &&& 1}$:] Comparing $g_{7,15}$ with \eqref{eq:gixi-SL4}, we see that all $x_{i}$ are in $p\Z_{p}$. Hence $\xi_{7,15} = 0$.

  \item[$g_{8,11} = \pmat{1 \\ & 1 && \exp(p)-1 \\ && 1 \\ &&& 1}$:] Comparing $g_{8,11}$ with \eqref{eq:gixi-SL4}, we see that all $x_{i}$ are in $p\Z_{p}$. Hence $\xi_{8,11} = 0$.

  \item[$g_{8,12} = \pmat{1 \\ & 1 \\ && 1 & \exp(-p)-1 \\ &&& 1}$:] Comparing $g_{8,12}$ with \eqref{eq:gixi-SL4}, we see that all $x_{i}$ are in $p\Z_{p}$. Hence $\xi_{8,12} = 0$.

  \item[$g_{8,13} = \pmat{1 && \exp(p)-1 \\ & 1 \\ && 1 \\ &&& 1}$:] Since $g_{8,13} = g_{7,13}$, the above shows that $\xi_{8,13} = 0$.

  \item[$g_{8,14} = \pmat{1 \\ & 1 & \exp(2p)-1 \\ && 1 \\ &&& 1}$:] Comparing $g_{8,14}$ with \eqref{eq:gixi-SL4}, we see that all $x_{i}$ are in $p\Z_{p}$. Hence $\xi_{8,14} = 0$.

  \item[$g_{8,15} = \pmat{1 & \exp(-p)-1 \\ & 1 \\ && 1 \\ &&& 1}$:] Comparing $g_{8,15}$ with \eqref{eq:gixi-SL4}, we see that all $x_{i}$ are in $p\Z_{p}$. Hence $\xi_{8,15} = 0$.

  \item[$g_{9,10} = \pmat{1 &&& \exp(p)-1 \\ & 1\\ && 1 \\ &&& 1}$:] Since $g_{9,10} = g_{7,10}$, the above shows that $\xi_{8,15} = 0$.

  \item[$g_{9,11} = \pmat{1 \\ & 1 && \exp(p)-1 \\ && 1 \\ &&& 1}$:] Since $g_{9,11} = g_{8,11}$, the above shows that $\xi_{9,11} = 0$.

  \item[$g_{9,12} = \pmat{1 \\ & 1 \\ && 1 & \exp(2p)-1 \\ &&& 1}$:] Comparing $g_{9,12}$ with \eqref{eq:gixi-SL4}, we see that all $x_{i}$ are in $p\Z_{p}$. Hence $\xi_{9,12} = 0$.

  \item[$g_{9,13} = \pmat{1 && \exp(-p)-1 \\ & 1 \\ && 1 \\ &&& 1}$:] Comparing $g_{9,13}$ with \eqref{eq:gixi-SL4}, we see that all $x_{i}$ are in $p\Z_{p}$. Hence $\xi_{9,13} = 0$.

  \item[$g_{9,14} = \pmat{1 \\ & 1 & \exp(-p)-1 \\ && 1 \\ &&& 1}$:] Comparing $g_{9,14}$ with \eqref{eq:gixi-SL4}, we see that all $x_{i}$ are in $p\Z_{p}$. Hence $\xi_{9,14} = 0$.

  \item[$g_{11,15} = \pmat{1 &&& -1 \\ & 1 \\ && 1 \\ &&& 1}$:] Comparing $g_{11,15}$ with \eqref{eq:gixi-SL4}, we see that all $x_{i}$ are in $p\Z_{p}$ except $x_{10}$, for which we have $a_{14} = x_{10} = -1$. Hence $\xi_{11,15} = -\xi_{10}$.

  \item[$g_{12,13} = \pmat{1 &&& -1 \\ & 1 \\ && 1 \\ &&& 1}$:] Since $g_{12,13} = g_{11,15}$, the above shows that $\xi_{12,13} = -\xi_{10}$.

  \item[$g_{12,14} = \pmat{1 \\ & 1 && -1 \\ && 1 \\ &&& 1}$:] Comparing $g_{12,14}$ with \eqref{eq:gixi-SL4}, we see that all $x_{i}$ are in $p\Z_{p}$ except $x_{11}$, for which we have $a_{24} = x_{11} = -1$. Hence $\xi_{12,14} = -\xi_{11}$.

  \item[$g_{14,15} = \pmat{1 && -1 \\ & 1 \\ && 1 \\ &&& 1}$:] Comparing $g_{14,15}$ with \eqref{eq:gixi-SL4}, we see that all $x_{i}$ are in $p\Z_{p}$ except $x_{13}$, for which we have $a_{13} = x_{13} = -1$. Hence $\xi_{14,15} = -\xi_{13}$.
\end{description}

Thus the non-zero commutators $[\xi_{i},\xi_{j}]$ with $i<j$ are:
\begin{equation}
  \label{eq:xi_ij-SL4}
  \begin{aligned}
    [\xi_{1},\xi_{10}] &= -(\xi_{7}+\xi_{8}+\xi_{9}), & [\xi_{1},\xi_{11}] &= -\xi_{3}, & [\xi_{1},\xi_{12}] &= -\xi_{2}, \\
    [\xi_{1},\xi_{13}] &= \xi_{6}, & [\xi_{1},\xi_{15}] &= \xi_{4}, & [\xi_{2},\xi_{13}] &= -(\xi_{7}+\xi_{8}), \\
    [\xi_{2},\xi_{14}] &= -\xi_{3}, & [\xi_{2},\xi_{15}] &= \xi_{5}, & [\xi_{3},\xi_{15}] &= -\xi_{7}, \\
    [\xi_{4},\xi_{11}] &= -(\xi_{8}+\xi_{9}), & [\xi_{4},\xi_{12}] &= -\xi_{5}, & [\xi_{4},\xi_{14}] &= \xi_{6}, \\
    [\xi_{5},\xi_{14}] &= -\xi_{8}, & [\xi_{6},\xi_{12}] &= -\xi_{9}, & [\xi_{11},\xi_{15}] &= -\xi_{10}, \\
    [\xi_{12},\xi_{13}] &= -\xi_{10}, & [\xi_{12},\xi_{14}] &= -\xi_{11}, & [\xi_{14},\xi_{15}] &= -\xi_{13}.
  \end{aligned}
\end{equation}

\subsection{Describing the graded chain complex, \texorpdfstring{$\gr^{j}\bigl(\bigwedge^{n}\lie{g}\bigr)$}{grj(wedge-n g)}}%
\label{subsec:graded-complex-SL4}

Looking at \eqref{eq:Iwa-p-val-basis-SLn} (with $e=1$ and $h=4$), we see that
\begin{align*}
  \omega(g_{1}) &= 1-\frac{3}{4} = \frac{1}{4}, & \omega(g_{2}) &= 1-\frac{2}{4} = \frac{1}{2}, & \omega(g_{3}) &= 1-\frac{1}{4} = \frac{3}{4}, \\
  \omega(g_{4}) &= 1-\frac{2}{4} = \frac{1}{2}, & \omega(g_{5}) &= 1-\frac{1}{4} = \frac{3}{4}, & \omega(g_{6}) &= 1-\frac{1}{4} = \frac{3}{4},
  \omega(g_{7}) &= 1, & \omega(g_{8}) &= 1, & \omega(g_{8}) &= 1, & \omega(g_{9}) &= 1, \\
  \omega(g_{10}) &= \frac{3}{4}, &  \omega(g_{11}) &= \frac{2}{4} = \frac{1}{2}, & \omega(g_{12}) &= \frac{1}{4}, \\
  \omega(g_{13}) &= \frac{2}{4} = \frac{1}{2}, & \omega(g_{14}) &= \frac{1}{4}, & \omega(g_{15}) &= \frac{1}{4}.
\end{align*}
Hence
\begin{equation*}
  \lie{g} = k \otimes_{\F_{p}[\pi]} \gr I = \Span_{k}(\xi_{1},\dotsc,\xi_{15}) = \lie{g}^{1} \oplus \lie{g}^{2} \oplus \lie{g}^{3} \oplus \lie{g}^{4},
\end{equation*}
where
\begin{align*}
  \lie{g}^{1} &= \lie{g}_{\frac{1}{4}} = \Span_{k}(\xi_{1},\xi_{12},\xi_{14},\xi_{15}), \\
  \lie{g}^{2} &= \lie{g}_{\frac{1}{2}} = \Span_{k}(\xi_{2},\xi_{4},\xi_{11},\xi_{13}), \\
  \lie{g}^{3} &= \lie{g}_{\frac{3}{4}} = \Span_{k}(\xi_{3},\xi_{5},\xi_{6},\xi_{10}), \\
  \lie{g}^{4} &= \lie{g}_{1} = \Span_{k}(\xi_{7},\xi_{8},\xi_{9}).
\end{align*}
See \Cref{rem:g-Z-grading} for more details.

This is enough to calculate the graded mod $p$ cohomology of $\lie{g}$, see \cite{code} for the details. We write the result in \Cref{tab:graded-coh-dims-SL4}.

\section{\texorpdfstring{$I \subseteq \GL_{4}(\Z_{p})$}{I in GL4(Zp)}}%
\label{sec:GL4-calc}

In this section we will briefly describe the work needed to find continuous group cohomology of the pro-$p$ Iwahori subgroup $I$ of $\GL_{4}(\Q_{p})$.

When $I$ is the pro-$p$ Iwahori subgroup in $\GL_{4}(\Q_{p})$, we know by \Cref{sec:cohiwagps-intro} that we can take it to be of the form
\begin{equation*}
  I = \pmat{1+p\Z_{p} & \Z_{p} & \Z_{p} & \Z_{p} \\ p\Z_{p} & 1+p\Z_{p} & \Z_{p} & \Z_{p} \\ p\Z_{p} & p\Z_{p} & 1+p\Z_{p} & \Z_{p} \\ p\Z_{p} & p\Z_{p} & p\Z_{p} & 1+p\Z_{p}} \subseteq \GL_{4}(\Z_{p}),
\end{equation*}
and, by \Cref{sec:cohiwagps-intro}, we have an ordered basis
\begin{equation}
  \label{eq:gis-GL4}
  \begin{gathered}
    g_{1} = \pmat{ 1 \\ & 1 \\ && 1 \\ p && 1 }, \quad g_{2} = \pmat{ 1 \\ & 1 \\ p && 1 \\ &&& 1 }, \quad g_{3} = \pmat{ 1 \\ p & 1 \\ && 1 \\ &&& 1 }, \\
    g_{4} = \pmat{ 1 \\ & 1 \\ && 1 \\ & 1 && 1 }, \quad g_{5} = \pmat{ 1 \\ & 1 \\ & p & 1 \\ &&& 1 }, \quad g_{6} = \pmat{ 1 \\ & 1 \\ && 1 \\ && p & 1 }, \\
    g_{7} = \pmat{ \exp(p) \\ & \exp(-p) \\ && 1 \\ &&& 1 }, \quad g_{8} = \pmat{ 1 \\ & \exp(p) \\ && \exp(-p) \\ &&& 1 }, \\
    g_{9} = \pmat{ 1 \\ & 1 \\ && \exp(p) \\ &&& \exp(-p) }, \quad g_{10} = \pmat{ \exp(p) \\ & \exp(p) \\ && \exp(p) \\ &&& \exp(p) }, \\
    g_{11} = \pmat{ 1 &&& 1 \\ & 1 \\ && 1 \\ &&& 1 }, \quad g_{12} = \pmat{ 1 \\ & 1 && 1 \\ && 1 \\ &&& 1 }, \quad g_{13} = \pmat{ 1 \\ & 1 \\ && 1 & 1 \\ &&& 1 }, \\
    g_{14} = \pmat{1 && 1 \\ & 1 \\ && 1 \\ &&& 1 }, \quad g_{15} = \pmat{ 1 \\ & 1 & 1 \\ && 1 \\ &&& 1 }, \quad g_{16} = \pmat{ 1 & 1 \\ & 1 \\ && 1 \\ &&& 1 }.
  \end{gathered}
\end{equation}
Since we just renamed some elements and added an element of the center of $\GL_{4}(\Z_{p})$ when comparing to the ordered basis of $I \subseteq \SL_{4}(\Z_{p})$ from \Cref{sec:SL4-calc}, it's clear from \Cref{eq:xi_ij-SL4} that the only non-zero commutators $[\xi_{i},\xi_{j}]$ with $i<j$ are:
\begin{equation}
  \label{eq:xi_ij-GL4}
  \begin{aligned}
    [\xi_{1},\xi_{11}] &= -(\xi_{7}+\xi_{8}+\xi_{9}), & [\xi_{1},\xi_{12}] &= -\xi_{3}, & [\xi_{1},\xi_{12}] &= -\xi_{2}, \\
    [\xi_{1},\xi_{14}] &= \xi_{6}, & [\xi_{1},\xi_{16}] &= \xi_{4}, & [\xi_{2},\xi_{14}] &= -(\xi_{7}+\xi_{8}), \\
    [\xi_{2},\xi_{15}] &= -\xi_{3}, & [\xi_{2},\xi_{16}] &= \xi_{5}, & [\xi_{3},\xi_{16}] &= -\xi_{7}, \\
    [\xi_{4},\xi_{12}] &= -(\xi_{8}+\xi_{9}), & [\xi_{4},\xi_{13}] &= -\xi_{5}, & [\xi_{4},\xi_{15}] &= \xi_{6}, \\
    [\xi_{5},\xi_{15}] &= -\xi_{8}, & [\xi_{6},\xi_{13}] &= -\xi_{9}, & [\xi_{12},\xi_{16}] &= -\xi_{11}, \\
    [\xi_{12},\xi_{14}] &= -\xi_{11}, & [\xi_{12},\xi_{15}] &= -\xi_{12}, & [\xi_{15},\xi_{16}] &= -\xi_{14}.
  \end{aligned}
\end{equation}

Looking at \Cref{sec:SL4-calc}, we easily see that
\begin{align*}
  \lie{g}^{1} &= \lie{g}_{\frac{1}{4}} = \Span_{k}(\xi_{1},\xi_{13},\xi_{15},\xi_{16}), \\
  \lie{g}^{2} &= \lie{g}_{\frac{1}{2}} = \Span_{k}(\xi_{2},\xi_{4},\xi_{12},\xi_{14}), \\
  \lie{g}^{3} &= \lie{g}_{\frac{3}{4}} = \Span_{k}(\xi_{3},\xi_{5},\xi_{6},\xi_{11}), \\
  \lie{g}^{4} &= \lie{g}_{1} = \Span_{k}(\xi_{7},\xi_{8},\xi_{9},\xi_{10}).
\end{align*}

This is enough to calculate the graded mod $p$ cohomology of $\lie{g}$, see \cite{code} for the details. We write the result in \Cref{tab:graded-coh-dims-GL4}.

%%% Local Variables:
%%% mode: latex
%%% TeX-master: "../main"
%%% End:
