\chapter{Introduction}%
\label{cha:intro}

The cohomology of Lie groups has a long history. In the late forties Chevalley and Eilenberg found that $H^{*}(G,\R) \iso H^{*}(\lie{g},\R)$ for a connected compact Lie group $G$ with Lie algebra $\lie{g}$ (cf.\ \cite{Chev}), and since then there has been much research into different types of Lie group cohomology.  In particular, the mod $p$ cohomology of a connected compact real Lie group has been well understood by Kac since the eighties (cf.\ \cite{Kac}), and the continuous mod $p$ cohomology $H^*(G,\F_p)$ of an equi-$p$-valued compact $p$-adic Lie group $G$ was already described by Lazard in the sixties (cf.\ \cite{Laz}). We note here that (except for Lazard's work) $H^{*}(G,\R)$ and $H^{*}(G,\F_{p})$ indicate the cohomology of $G$ as a topological space, and not continuous group cohomology, which can be thought of as the cohomology of the classifying space $BG$.

This dissertation's main interest is the continuous mod $p$ cohomology $H^{*}(G,\F_{p})$ of compact $p$-adic Lie groups $G$ for specific cases of $G$. Since $p$-adic Lie groups are totally disconnected, working with them requires very different methods than what Chevalley and Eilenberg or Kac used for real Lie groups, and we have to follow the ideas of Lazard (see \cite{Laz}) and Serre. In particular we need a $p$-valuation on $G$ (and on the completed group algebras associated with $G$), and we work with the graded \enquote{Lazard} Lie algebra $\lie{g} = \F_{p} \otimes_{\F_{p}[\pi]} \gr G$ attached to $G$. We will repeatedly use that Sørensen (in \cite{Sor}) showed that $H^{*}(\lie{g},\F_{p})$ determines $H^{*}(G,\F_{p})$ via a multiplicative spectral sequence \[ E_{1}^{s,t} = H^{s,t}(\lie{g},\F_{p}) \Longrightarrow H^{s+t}(G,\F_{p}). \] When $G$ is equi-$p$-valuable, we get that $\lie{g}$ is concentrated in a single degree, and Lazard showed that $H^{*}(G,\F_{p}) \iso \bigwedge H^{1}(\lie{g},\F_{p})$, while Sørensen showed that this also follows from the above spectral sequence. We are interested in cases where $G$ is \emph{not} equi-$p$-valuable, and we note that the spectral sequence of Sørensen allows us to work purely with $G$ and $\lie{g}$ without having to worry about the completed group algebras $\Lambda(G) = \Z_{p}\llbracket G \rrbracket$ and $\Omega(G) = \F_{p}\llbracket G \rrbracket$.

Before describing our particular results in the following paragraph, we emphasize the following remark of Sørensen from \cite{Sor}: It is known (due to Lazard) that any compact $p$-adic Lie group contains an open equi-$p$-valuable subgroup (see \cite[Chap.~V~2.2.7.1]{Laz}), which gives the impression that the distinction between $p$-valued and equi-$p$-valued groups is somewhat nuanced, which is true for some questions. But there are many examples of naturally occurring $p$-valuable groups $G$ which are not equi-$p$-valuable, where detailed information about $H^{*}(G,\F_{p})$ is important. For example unipotent groups (i.e., the $\Z_{p}$-points of the unipotent radical of a Borel in a split reductive group), Serre's standard groups with $e>1$ as in \cite[Lem.~2.2.2]{Laz-complements}, pro-$p$ Iwahori subgroups for large enough $p$, and $1 + \idm_{D}$ where $D$ is the quaternion division algebra over $\Q_{p}$ for $p > 3$ (or more generally a central division algebra over $\Q_{p}$). Sørensen explicitly calculates $H^{*}\bigl( (1+\idm_{D})^{\Nrd = 1} , \F_{p} \bigr)$ for $p>3$ and uses it to describe $H^{*}(1+\idm_{D},\F_{p})$, and he notes that $1+\idm_{D}$ plays an important role both in number theory (in the Jacquet-Langlands correspondence for instance, see \cite{JL}) and algebraic topology, where $1+\idm_{D}$ is known as the (strict) Morava stabilizer in stable homotopy theory, and $H^{*}(1+\idm_{D},\F_{p})$ somehow controls certain localization functors with respect to Morava $K$-theory (see e.g.\ \cite{Henn}).

Our work in \Cref{cha:cohunigps} will build on ideas of Lazard and Serre from their more general (but not yet finished) description of the case when $G$ is not equi-$p$-valued, and especially the refinement of these ideas as described by Sørensen and Schneider in \cite{Sor} and \cite{Sch-notes}. We will focus on unipotent groups $N$ originating from split and connected reductive $\Z_p$-groups, which is similar to recent work in the case of $\Z_p$ coefficients by Ronchetti (cf.\ \cite{Ron}), and this work can be considered a slight refinement of \cite{GK} since we retain information about the cup product on $H^{*}(N,\F_{p})$.

In \Cref{cha:cohiwagps} we focus on the case of pro-$p$ Iwahori subgroups of $\SL_{n}$ and $\GL_{n}$ over $\Q_{p}$ for $n=2,3,4$ or over quadratic extensions $F/\Q_{p}$ for $n=2$. We explicitly calculate the algebra structure of $H^{*}(I,\F_{p})$ for the pro-$p$ Iwahori subgroups $I_{\SL_{2}(\Q_{p})} \subseteq \SL_{2}(\Z_{p})$ and $I_{\GL_{2}(\Q_{p})} \subseteq \GL_{2}(\Z_{p})$, and we note that these are isomorphic as algebras to $H^{*}\bigl( (1+\idm_{D})^{\Nrd = 1},\F_{p} \bigr)$ and $H^{*}(1+\idm_{D},\F_{p})$ respectively. We finish the chapter by mentioning some future research directions and a conjecture on the connection between the mod $p$ cohomology of $(1+\idm_{D})^{\Nrd = 1}$ (resp.\ $1+\idm_{D}$) for central division algebras and $I_{\SL_{n}(\Q_{p})}$ (resp.\ $I_{\GL_{n}(\Q_{p})}$).

Finally, the appendix will end with a very brief description of other research (all joint) that I have participated in.


%%% Local Variables:
%%% mode: latex
%%% TeX-master: "../main"
%%% End:
