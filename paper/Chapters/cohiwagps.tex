\section{Intoduction}%
\label{sec:cohiwagps-intro}

In this chapter we will calculate the cohomology over perfect fields $k$ (or just $k = \F_{p}$) of a collection of pro-$p$ Iwahori subgroups of $\SL_{n}$ and $\GL_{n}$ over $\Z_{p}$ or $\sO_{F}$ for quadratic extensions $F/\Q_{p}$.

\subsection{Background and motivation}%
\label{subsec:background-iwa}


In this chapter we will focus on describing the continuous mod $p$ cohomology of pro-$p$ Iwahori subgroups of $\SL_{n}$ and $\GL_{n}$ over $\Q_{p}$ for $n=2,3,4$ or over quadratic extensions $F/\Q_{p}$ for $n=2$.

We start by introducing the techniques we use throughout this chapter, and then we explicitly calculate the algebra structure of $H^{*}(I,\F_{p})$ for the pro-$p$ Iwahori subgroups $I_{\SL_{2}(\Q_{p})} \subseteq \SL_{2}(\Z_{p})$ and $I_{\GL_{2}(\Q_{p})} \subseteq \GL_{2}(\Z_{p})$, and we note that these are isomorphic as algebras to $H^{*}\bigl( (1+\idm_{D})^{\Nrd = 1},\F_{p} \bigr)$ and $H^{*}(1+\idm_{D},\F_{p})$ respectively. This part heavily relies on results of Sørensen and Fuks (see \cite{Sor} and \cite{Fuks}). Afterwards we fully describe the cohomological dimensions (but not the cup products, which are quite complicated) of the pro-$p$ Iwahori subgroups of $\SL_{3}(\Q_{p})$, $\GL_{3}(\Q_{p})$, $\SL_{2}(F)$ and $\GL_{2}(F)$, where $F$ is a quadratic extension of $\Q_{p}$. We also roughly describe the cohomological dimensions of the pro-$p$ Iwahori subgroups of $\SL_{4}(\Q_{p})$ and $\GL_{4}(\Z_{p})$, but we note that we run into some problems here. In particular the multiplicative spectral sequence \[ E_{1}^{s,t} = H^{s,t}(\lie{g},\F_{p}) \Longrightarrow H^{s+t}(I,\F_{p}) \] of Sørensen (see \cite{Sor}) collapses on the first page in all the previous examples, but for pro-$p$ Iwahori subgroups of $\SL_{4}(\Q_{p})$ or $\GL_{4}(\Q_{p})$ we no longer (trivially) get this. Here $\lie{g} = \F_{p} \otimes_{\F_{p}[\pi]} \gr I$ is the graded Lazard Lie algebra associated with $I$.

This work can be seen as a continuation of the recent work on the mod $p$ cohomology of pro-$p$ Iwahori subgroups. E.g.\ the work by Schneider and Olivier (see \cite{SchOll-modular,SchOll-torsion,Sch-smooth}) working with pro-$p$ Iwahori-Hecke modules and the work by Koziol (see \cite{Koziol}) computing $H^{1}(I,\pi)$ as a $\mathcal{H}$-algebra (where $\mathcal{H}$ is the pro-$p$ Iwahori-Hecke algebra and $\pi$ is a mod $p$ principal series representation of $\GL_{n}(F)$ for some $p$-adic field $F$). Work by Cornut and Ray (cf.\ \cite{Generators}) finding a minimal set of topological generators of the pro-$p$ Iwahori subgroup of a split reductive group over $\Z_{p}$ is also relevant, since the number of generators can be used to find the cohomological dimension of $H^{1}(I,\F_{p})$. Overall all of this work can be seen as part of the search for a mod $p$ and $p$-adic local Langlands program.

We finish the chapter by mentioning some observations on the nilpotency index of our calculated cohomology rings and discussing future research directions and a conjecture on the connection between the mod $p$ cohomology of $(1+\idm_{D})^{\Nrd = 1}$ (resp.\ $1+\idm_{D}$) for central division algebras and $I_{\SL_{n}(\Q_{p})}$ (resp.\ $I_{\GL_{n}(\Q_{p})}$). Finally we note that working with the Serre spectral sequence might allow us to generalize to all $I_{\SL_{n}(\Q_{p})}$ for all $n$.


\subsection{Setup and notation}%
\label{subsec:setup-iwa}

Let $p$ be an odd prime (further restricted later) and let $k$ be a perfect field of characteristic $p$.\nomiwa[k]{$k$}{a perfect field of characteristic $p$}

\paragraph{Field extension of $\Q_{p}$.} We fix a finite extension of $F/\Q_{p}$\nomiwa[F]{$F$}{a finite extension of $\Q_{p}$} of degree $D$\nomiwa[D]{$D$}{${} = [F : \Q_{p}]$, the degree of the extension $F/\Q_{p}$} with valuation ring $\sO_{F}$\nomiwa[OF]{$\sO_{F}$}{the valuation ring of $F$} and maximal ideal $\idm_{F} = (\varpi_{F}) \subseteq \sO_{F}$.\nomiwa[mF]{$\idm_{F}$}{maximal ideal of the valuation ring $\sO_{F}$}\nomiwa[piF]{$\varpi_{F}$}{a uniformizer of $F$} Let $e = e(F/\Q_{p})$ be the \emph{ramification index}\index{ramification index} and $f = f(F/\Q_{p})$ the \emph{inertia degree}\index{inertia degree} of the extension $F/\Q_{p}$. Let furthermore $v$ be the valuation on $F$ for which $v(p) = 1$, and thus $v(\varpi_{F}) = \frac{1}{e}$.

\paragraph{$\exp$ and $\log$.} Given finite field extension $F/\Q_{p}$ with valuation ring $\sO_{F}$ and maximal ideal $\idm_{F}$ with $p\sO_{F} = \idm_{F}^{e}$, we get by \cite[Prop.~(5.5)]{Neukirch} (noting that we will ensure that $1 > \frac{e}{p-1}$ later) that the power series
\begin{equation*}
  \exp(x) = 1 + x + \frac{x^{2}}{2!} + \frac{x^{3}}{3!} + \dotsb \quad \text{ and } \quad \log(1+z) = z - \frac{z^{2}}{2} + \frac{z^{3}}{3} - \dotsb,
\end{equation*}
are two mutually inverse isomorphisms (and homeomorphisms)
\[
  \begin{tikzcd}
    \idm_{F} \ar[r, yshift=0.7ex, "\exp"] & U_{F}^{(1)}. \ar[l, yshift=-0.7ex, "\log"]
  \end{tikzcd}
\]
Note that this implies that a $\Z_{p}$-basis of $\idm$ translates to a $\Z_{p}$-basis of $U_{F}^{(1)} = 1+\idm_{F}$ via $\exp$.

\paragraph{Big-$O$ notation.} For elements of $\sO_{F}$ we write $x = y + O(p^{r})$ if and only if $x-y \in p^{r}\sO_{F}$.\index{big-O notation@big-$O$ notation}\index{Opr@$O(p^{r})$}\nomiwa[Opr]{$O(p^{r})$}{for elements of $\sO_{F}$ we write $x = y + O(p^{r})$ if and only if $x-y \in p^{r}\sO_{F}$}

\paragraph{Matrices.} Let $E_{ij}$\nomiwa[Eij]{$E_{ij}$}{the matrix with $1$ in the $(i,j)$ entry and zeroes in all other entries} denote the matrix with $1$ in the $(i,j)$ entry, and zeroes in all other entries, and write $1_{n}$ for the identity matrix in $M_{n}(F)$. Let $A = (a_{ij})$. We write $A = \diag(a_{1},\dotsc,a_{n})$\nomiwa[diag]{$\diag(a_{1},\dotsc,a_{n})$}{(${} = (a_{ij})$) the diagonal matrix with entries $a_{ii} = a_{i}$} for the diagonal matrix in $M_{n}(F)$ with entries $a_{ii}=a_{i}$ in the diagonal, and $A = \diag_{i_{1},\dotsc,i_{k}}(a_{1},\dotsc,a_{k})$\nomiwa[diagi]{$\diag_{i_{1},\dotsc,i_{k}}(a_{1},\dotsc,)$}{(${} = (a_{ij})$) the matrix with entries $a_{i_{\ell}i_{\ell}} = a_{\ell}$ for $\ell=1,\dotsc,k$ and zeroes in all other entries} for the diagonal matrix in $M_{n}(F)$ with entries $a_{i_{\ell}i_{\ell}} = a_{\ell}$ for $\ell = 1,\dotsc,k$ and zeroes in all other entries. Finally, we write $A^{\top}$\nomiwa[T]{$(\edot)^{\top}$}{the transpose matrix} for the transpose matrix of $A$.

\paragraph{Dual basis.} Let $V$ be a $k$-vector space with basis $\basis = (e_{1},\dotsc,e_{d})$. Then we let $\basis^{*} = (e_{1}^{*},\dotsc,e_{d}^{*})$ be the dual basis of $\Hom_{k}(V,k)$ defined by $e_{i}^{*}(e_{i}) = \delta_{ij}$, where $\delta_{ij}$ is the Kronecker delta function. Now consider two vector spaces $V$ and $W$ with bases $\basis_{V}$ and $\basis_{W}$. Given a linear map $d \colon V \to W$ with matrix $A$ when described in these bases, it is a well known fact from linear algebra that the dual map $d^{*} \colon \Hom_{k}(W,k) \to \Hom_{k}(V,k)$ has matrix $A^{\top}$ when described in the dual bases $\basis_{V}^{*}$ and $\basis_{W}^{*}$. We will often use this without mention and abuse notation writing $d$ and $d^{\top}$ for these matrices.

\paragraph{Smith normal form.} Let $R$ be an integral domain and consider only non-zero matrices over $R$ in this paragraph. Given an $n \times m$ matrix $A$, there exist invertible $m \times m$ and $n \times n$ matrices $S$ and $T$ such that
\begin{equation*}
  SAT =
  \begingroup
  \linespread{1}\selectfont
  \begin{pNiceMatrix}[nullify-dots]
    a_{1} & 0 & \Cdots \hspace*{1mm} & & & & & 0 \\
    0 & a_{2} & \Ddots & & & & & \Vdots \\
    \Vdots & \Ddots & \Ddots \\
    & & & a_{r} \\
    & & & & 0 & \Ddots \hspace*{1mm} \\
    & & & & & \Ddots \\
    & & & & & & & 0 \\
    0 & \Cdots & & & & & 0 & 0
  \end{pNiceMatrix}
  \endgroup
\end{equation*}
and the diagonal entries $a_{i}$ satisfy $a_{i} \mid a_{i+1}$ for $i=1,\dotsc,r-1$. This matrix is called the Smith normal form\index{Smith normal form} of the matrix $A$. Given $n \times m$ matrices $A,B$, we write $A \snfsim B$ if $A$ and $B$ have the same Smith normal form. This notation will mainly be used when $B$ is already a matrix in Smith normal form. Finally we introduce the notation $A = \SNF_{n\times m}(a_{1},\dotsc,a_{r},0,\dotsc,0)$ for the $n \times m$ matrix with $a_{ii} = a_{i}$ for $i=1,\dotsc,r$ and zeroes in all other entries as above. In next subsection, we will note that the Smith normal form will be useful for our cohomology calculations.

\begin{remark}
  In the case $R = \Z$, the Smith normal form of a matrix can be found using the following row and column operations, which are invertible over $\Z$.
  \begin{align*}
    \text{(R1): } & \text{swap rows $R_{i}$ and $R_{j}$} & \text{(C1): } & \text{swap columns $C_{i}$ and $C_{j}$} \\
    \text{(R2): } & \text{multiply row $R_{i}$ by $-1$} & \text{(C2): } & \text{multiply column $C_{i}$ by $-1$} \\
    \text{(R3): } & \text{replace row $R_{i}$ by $R_{i} + kR_{j}$} & \text{(C3): } & \text{replace column $C_{i}$ by $C_{i} + kC_{j}$} \\
                  & \text{for some row $R_{j} \neq R_{i}$ and} && \text{for some column $C_{j} \neq C_{i}$ and} \\
    & k\in\Z && k\in\Z.
  \end{align*}

  \noindent We will not do these calculations by hand in this chapter, and will instead utilize implementations in Sage and SymPy that can find the Smith normal form of a matrix over $\Z$. Here it is important to note that the SymPy implementation does not allow the use of the rules (R2) and (C2), so we get a small difference between the results of the calculations in SymPy and Sage, but it will only be a difference of sign on some entries in the diagonal.
\end{remark}

\paragraph{Lazard theory.} For an introduction to Lazard theory see \Cref{sec:cohunigps-intro}, or \cite{Sch} for more details. In particular, note that the Lazard Lie algebra generalizes from $\F_{p}$ to general $k$ of characteristic $p$. We will let $\lie{g} = k \otimes_{\F_p[\pi]} \gr I$\nomiwa[g]{$\lie{g}$}{${} = k \otimes_{\F_{p}[\pi]} \gr I$, the Lazard Lie algebra corresponding to the pro-$p$ Iwahori subgroup $I$} be the Lazard Lie algebra corresponding to the pro-$p$ Iwahori subgroup $I$. Furthermore, recall that a sequence of elements $(g_1,\dotsc,g_r)$ in $G$ is called an \emph{ordered basis} of $(G,\omega)$\index{p-valued group!ordered basis} if the map $\Z_{p}^{r} \to G$ given by $(x_{1},\dotsc,x_{r}) \mapsto g_{1}^{x_{1}} \dotsb g_{r}^{x_{r}}$ is a bijection (and hence, by compactness, a homeomorphism) and
\begin{equation*}
  \omega(g_1^{x_1}\dotsb g_r^{x_r}) = \min_{1 \leq i \leq r}(\omega(g_i)+v(x_i)) \qquad \text{for any } x_1,\dotsc,x_r\in\Z_p.
\end{equation*}

\paragraph{Algebraic groups.} We will work with schemes using the functorial approach and notation described in \cite{Jan}. In particular, given an integral domain $R$, we note that a \emph{$R$-group functor}\index{R-group@$R$-group!functor} is a functor from the category of all $R$-algebras to the category of groups, a \emph{$R$-group scheme}\index{R-group@$R$-group!scheme} is a $R$-group functor that is an affine scheme over $R$ when considered as a $R$-functor, and an \emph{algebraic $R$-group}\index{R-group@$R$-group!algebraic}\index{algebraic R-group@algebraic $R$-group} is a $R$-group scheme that is algebraic as an affine scheme. For more in depth introduction to these concepts, we refer to \cite{Con-book} and \cite{Jan}.

\paragraph{Fixed groups and roots.} We fix a split and connected reductive algebraic $F$-group $\gs{G}$\nomiwa[G]{$\gs{G}$}{a (fixed) split and connected reductive algebraic $F$-group}\index{G}, and consider the locally profinite group $G = \gs{G}(F)$.\nomiwa[GF]{$G$}{${} = \gs{G}(F)$, a locally profinite group} We then fix split maximal torus $\gs{T} \subseteq \gs{G}$\nomiwa[T]{$\gs{T}$}{a (fixed) split maximal torus of $\gs{G}$} and let $T = \gs{T}(F)$.\nomiwa[TF]{$T$}{${} = \gs{T}(F)$}\index{maximal torus} In $T$ we have a maximal compact subgroup $T^{0}$ and its Sylow pro-$p$ subgroup $T^{1}$.

Let $\Phi = \Phi(\gs{G},\gs{T})$\nomiwa[Phi]{$\Phi$}{${} = \Phi(\gs{G},\gs{T})$, the root system of $\gs{G}$ with respect to $\gs{T}$} be the \emph{root system}\index{root!system} of $\gs{G}$ with respect to $\gs{T}$, and let $(X^{*}(T),\Phi,X_{*}(T),\Phi^{\vee})$\nomiwa[XTPhi]{$(X^{*}(T),\Phi,X_{*}(T),\Phi^{\vee})$}{the root datum associated with $\Phi = \Phi(\gs(G),\gs{T})$} be the associated root datum.\index{root!datum} Fix a system of positive roots $\Phi^{+}$ and let $\Delta \subseteq \Phi^{+}$ be the simple roots.\index{root!simple} For any $\alpha \in \Phi$ we have the root subgroup\index{root!subgroup} $\gs{U}_{\alpha} \subseteq \gs{G}$ with Lie algebra $\Lie \gs{U}_{\alpha} =  (\Lie \gs{G})_{\alpha}$. We let $U_{\alpha} = \gs{U}_{\alpha}(F)$ and choose an isomorphism $x_{\alpha} \colon F \xrightarrow{\iso} U_{\alpha}$\nomiwa[xa]{$x_{\alpha} \colon F \xrightarrow{\iso} U_{\alpha}$}{an isomorphism such that $tx_{\alpha}(x)t^{-1} = x_{\alpha}(\alpha(t)x)$ for $t \in T$ and $x \in F$} such that $tx_{\alpha}(x)t^{-1} = x_{\alpha}(\alpha(t)x)$ for $t \in T$ and $x \in F$. For $r \in \Z_{\geq 0}$ we let $U_{\alpha,r} = x_{\alpha}(\idm_{F}^{r})$.\nomiwa[Uar]{$U_{\alpha,r}$}{${} = x_{\alpha}(\idm_{F}^{r})$}

\begin{remark}
  In this chapter we write $\gs{U}$ instead of $\gs{N}$ since we try to stick to the notation of surrounding literature.
\end{remark}

\paragraph{Coxeter number and $p$.} Let $h$\nomiwa[h]{$h$}{the Coxeter number of $\gs{G}$} be the Coxeter number of $\gs{G}$ and assume from now on that $p-1 > eh$.\nomiwa[p]{$p$}{a prime, $p-1 \geq eh$, where $h$ is the Coxeter number of $\gs{G}$}

% We furthermore fix a basis $\Delta \subseteq \Phi$\nomiwa[Delta]{$\Delta$}{a (fixed) basis of the root system $\Phi$} of the root system,\index{root system} so we get a decomposition $\Phi = \Phi^+ \cup \Phi^-$\nomiwa[Phi+]{$\Phi^{+}$ / $\Phi^{-}$}{the positive/negative roots in $\Phi$ with respect to $\Delta$} into positive and negative roots. Let $\gs{B} = \gs{T}\gs{U}$\nomiwa[B]{$\gs{B}$ / $\gs{B}^{+}$}{(${} = \gs{T}\gs{U}$ / ${} = \gs{T}\gs{U}^{+}$) the Borel subgroups of $\gs{G}$ corresponding to $\Phi^{-}$ / $\Phi^{+}$} and $\gs{B}^+ = \gs{T}\gs{U}^+$ denote the Borel subgroups of $\gs{G}$ corresponding to $\Phi^-$ and $\Phi^+$, respectively, with unipotent radicals $\gs{U}$ and $\gs{U}^+$.\nomuni[U]{$\gs{U}$ / $\gs{U}^{+}$}{the unipotent radical of $\gs{B}$ / $\gs{B}^{+}$}\index{U@$\gs{U}$}

\paragraph{Pro-$p$ Iwahori subgroups.} We follow the definitions of \cite{SchOll-modular} with $\gs{G}, \gs{T}$ and $\gs{U}_{\alpha}$ as above. Let $I$ be the pro-$p$ Iwahori subgroup of $G$ (associated with a positive chamber as in \cite{SchOll-modular}, but we do not need the exact definition). We note by \cite[Lem.~2.1(i)]{SchOll-modular} and the proof of \cite[Lem.~2.3]{SchOll-modular} that $I$ has the following factorization: Multiplication defines a homeomorphism
\begin{equation}\label{eq:Iwahori-factor}
  \prod_{\alpha \in \Phi^{-}} U_{\alpha,1} \times T^{1} \times \prod_{\alpha \in \Phi^{+}} U_{\alpha,0} \xrightarrow{\iso} I,
\end{equation}
where the products are ordered in an arbitrarily chosen way. For a more detailed introduction to these pro-$p$ groups we refer to \cite{SchOll-modular}.

\paragraph{Pro-$p$ Iwahori subgroups of $\GL_{n}(F)$ and $\SL_{n}(F)$.} In this chapter, we will only work with pro-$p$ Iwahori subgroups of $\GL_{n}(F)$ or $\SL_{n}(F)$, which simplifies the definitions. When $\gs{G} = \GL_{n}$ or $\gs{G} = \SL_{n}$, we can always take $\gs{T}$ the diagonal maximal torus, and we can take $I$ to be the subgroup of $\gs{G}(\sO_{F})$ which is upper triangular and unipotent modulo $\varpi$. In this case we have that $U_{\alpha,1}$ for $\alpha \in \Phi^{-}$ correspond to entries below the diagonal and $U_{\alpha,0}$ for $\alpha \in \Phi^{+}$ corresponds to the entries above the diagonal.

\paragraph{$p$-valuation on $I$.} By a recent preprint by Lahiri and Sørensen (cf.\ \cite[Prop.~3.4]{IwaBasis}), we know (since $p-1 > eh$) that $I$ admits a $p$-valuation $\omega$ satisfying the properties:
\begin{enumerate}[(a)]
  \item $\omega$ is compatible with Iwahori factorization \eqref{eq:Iwahori-factor} of $I$ (cf.\ \cite[Def.~3.3]{IwaBasis}).
  \item $\omega(x_{\alpha}(x)) = v(x) + \frac{\Ht(\alpha)}{eh}$ where $\begin{dcases}
    x \in \idm_{F} & \text{if } \alpha \in \Phi^{-}, \\
    x \in \sO_{F} & \text{if } \alpha \in \Phi^{+}.
  \end{dcases}$
  \item $\omega(t) = \frac{1}{e} \cdot \sup\set{n \in \N : t \in T^{n}}$ for $t \in T^{1}$.
\end{enumerate}

\paragraph{Ordered basis of $I$.} Let $\set{b_{1},\dotsc,b_{\ell}}$ be a $\Z_{p}$-basis of $\sO_{F}$, %and let $\set{u_{1},\dotsc,u_{D}}$ be a $\Z_{p}$-basis of $U_{F}^{(1)} = 1+\idm_{F}$,
where $\ell = [F:\Q_{p}]$. Then $\bigl( x_{\alpha}(b_{1}), \dotsc, x_{\alpha}(b_{\ell}) \bigr)$ is an ordered basis for $U_{\alpha,0}$ when $\alpha \in \Phi^{+}$, and $\bigl( x_{\alpha}(\varpi_{F}b_{1}), \dotsc, x_{\alpha}(\varpi_{F}b_{\ell}) \bigr)$ is an ordered basis for $U_{\alpha,1}$ when $\alpha \in \Phi^{-}$. Furthermore, when $G$ is semisimple and simply connected, we have that the simple coroots\index{coroots} $\set{ \alpha^{\vee} : \alpha \in \Delta }$ form a $\Z$-basis of $X_{*}(T)$, and thus $\bigl( \alpha^{\vee}(\exp(\varpi_{F}b_{1})), \dotsc, \alpha^{\vee}(\exp(\varpi_{F}b_{\ell})) \bigr)_{\alpha \in \Delta}$ form an ordered basis of $T^{1}$. By \cite[Cor.~3.6]{IwaBasis}, given orderings of $\Phi^{+}$ and $\Phi^{-}$, and assuming that $G$ is semisimple and simply connected, we now get: the sequence of elements
\begin{enumerate}[$\bullet$]
  \item $\bigl( x_{\alpha}(\varpi_{F}b_{1}), \dotsc, x_{\alpha}(\varpi_{F}b_{\ell}) \bigr)_{\alpha \in \Phi^{-}}$,
  \item $\bigl( \alpha^{\vee}(\exp(\varpi_{F}b_{1})), \dotsc, \alpha^{\vee}(\exp(\varpi_{F}b_{\ell})) \bigr)_{\alpha \in \Delta}$,
  \item $\bigl( x_{\alpha}(b_{1}), \dotsc, x_{\alpha}(b_{\ell}) \bigr)_{\alpha \in \Phi^{+}}$
\end{enumerate}
forms an ordered basis of $(I,\omega)$ (with $\omega$ from the previous paragraph) which is a saturated $p$-valued group. Here, \cite{IwaBasis} notes that the $p$-valuation from the previous paragraph on this basis is given by (cf.\ \cite[Prop.~3.4]{IwaBasis})
\begin{equation}
  \label{eq:Iwa-p-val-basis}
  \begin{dcases}
    \omega\bigl( x_{\alpha}(\varpi_{F}b_{\ell}) \bigr) = \frac{1}{e} + \frac{\Ht(\alpha)}{eh}  & \alpha \in \Phi^{-} \\
    \omega\bigl( \alpha^{\vee}(u_{i}) \bigr) = \frac{1}{e} & \alpha \in \Delta \\
    \omega\bigl( x_{\alpha}(b_{\ell}) \bigr) = \frac{\Ht(\alpha)}{eh} & \alpha \in \Phi^{+}.
  \end{dcases}
\end{equation}

We note that the above argument uses that $\exp \colon \idm_{F} = (\varpi_{F}) \to U_{F}^{(1)} = 1 + \idm_{F}$ takes a basis to a basis, and noting that $\set{ \varpi_{F}b_{1}, \dotsc, \varpi_{F}b_{\ell} }$ is a $\Z_{p}$-basis of $\idm_{F} = \varpi_{F}\sO_{F}$. %, we see that we can take $u_{i} = \exp(\varpi_{F}b_{i})$ for $i=1,\dotsc,D$.

When $\gs{G} = \SL_{n}$, we have that $\Phi = \set{ \varepsilon_{i}-\varepsilon_{j} \given 1 \leq i,j \leq n, i\neq j }$ and can take
\begin{equation*}
  \Delta = \set{\alpha_{1}=\varepsilon_{1}-\varepsilon_{2}, \alpha_{2}=\varepsilon_{2}-\varepsilon_{3}, \dotsc, \alpha_{n-1}=\varepsilon_{n-1}-\varepsilon_{n}},
\end{equation*}
where $\varepsilon_{i}$ is the map that takes a diagonal matrix to its $i$-th diagonal entry. In this case $\alpha_{i}^{\vee}(u) = \diag(0,\dotsc,0,u,-u,0,\dotsc,0) = \diag_{i,i+1}(u,-u)$, where the non-zero entries are the $i$-th and $(i+1)$-th entries. This together with the above gives us the following ordered basis (in the listed order and with a chosen ordering of $\set{ (i,j) : 1 \leq i,j \leq n }$) in the case $\gs{G} = \SL_{n}$:
\begin{enumerate}[$\bullet$]
  \item $\bigl( 1_{n}+\varpi_{F}b_{1}E_{ij}, \dotsc, 1_{n}+\varpi_{F}b_{\ell}E_{ij} \bigr)_{1 \leq j < i \leq n}$,
  \item $\bigl( \diag_{i,i+1}(\exp(\varpi_{F}b_{1})), \dotsc, \diag_{i,i+1}(\exp(\varpi_{F}b_{\ell})) \bigr)_{i=1,\dotsc,n-1}$,
  \item $\bigl( 1_{n}+b_{1}E_{ij}, \dotsc, 1_{n}+b_{\ell}E_{ij} \bigr)_{1 \leq i < j \leq n}$.
\end{enumerate}
Here the $p$-valuation described in \eqref{eq:Iwa-p-val-basis} is given by
\begin{equation}\label{eq:Iwa-p-val-basis-SLn}
  \begin{dcases}
    \omega\bigl( 1_{n} + \varpi_{F}b_{m}E_{ij} \bigr) = \frac{1}{e} + \frac{j-i}{eh} & j < i, \\
    \omega\bigl( \diag_{i,i+1}(\exp(\varpi_{F}b_{m})) \bigr) = \frac{1}{e} & i = 1,\dotsc,n-1, \\
    \omega\bigl( 1_{n}+b_{m}E_{ij} \bigr) = \frac{j-i}{eh} & i < j
  \end{dcases}
\end{equation}
on the above ordered basis.

Finally note that an ordered basis of $\GL_{n}$ can be obtained from an ordered basis of $\SL_{n}$ by adding non-trivial elements of the center, which in the above corresponds to adding $\bigl(  \exp(\varpi_{F}b_{1})1_{n}, \dotsc, \exp(\varpi_{F}b_{\ell})1_{n} \bigr)$ to the middle item above (adding the root $\varepsilon_{1} + \dotsb + \varepsilon_{n}$), and the $p$-valuation on these is still $\frac{1}{e}$.

\paragraph{Cohomology.} We denote (using the Chevalley-Eilenberg complex) the Lie algebra cohomology\index{cohomology!Lie algebra} of any $k$-Lie algebra $\lie{g}$ by $H^{\bullet}(\lie{g}, \edot)$,\nomiwa[Hg]{$H^{\bullet}(\lie{g},\edot)$}{the cohomology of the Lie algebra $\lie{g}$} while we write $H^{\bullet}(G,\edot)$\nomiwa[HG]{$H^{\bullet}(G,\edot)$}{the continuous group cohomology of the topological group $G$} for the continuous group cohomology of a topological group $G$. Here we let the entries distinguish between different types of cohomology without any ambiguity. As in \Cref{sec:cohunigps-intro}, we introduce filtrations and then gradings on the cohomology and use the notation $H^{s,t} = \gr^{s}H^{s+t}$\nomiwa[Hst]{$H^{s,t}$}{${} = \gr^{s}H^{s+t}$ for some cohomology $H$} for any type of cohomology $H$.

\paragraph{Spectral sequences.} A cohomological spectral sequence\index{spectral sequence} is a choice of $r_0 \in \N$ and a collection of
\begin{enumerate}[$\bullet$]
  \item $k$-modules $E_r^{s,t}$ for each $s,t \in \Z$ and all integers $r \geq r_0$
  \item differentials $d_r^{s,t} \colon E_r^{s,t} \to E_r^{s+r,t+1-r}$ such that $d_r^2 = 0$ and $E_{r+1}$ is isomorphic to the homology of $(E_r,d_r)$, i.e.,
  \[
    E_{r+1}^{s,t} = \frac{\kernel(d_r^{s,t} \colon E_r^{s,t} \to E_r^{s+r,t+1-r})}{\image(d_r^{s-r,t+r-1} \colon E_r^{s-r,t+r-1} \to E_r^{s,t})}.
  \]
\end{enumerate}
For a given $r$, the collection $(E_r^{s,t},d_r^{s,t})_{s,t\in\Z}$ is called the $r$-th page. A spectral sequence \emph{converges}\index{spectral sequence!convergent} if $d_r$ vanishes on $E_r^{s,t}$ for any $s,t$ when $r\gg0$. In this case $E_r^{s,t}$ is independent of $r$ for sufficiently large $r$, we denote it by $E_{\infty}^{s,t}$ and write
  \[
    E_{r}^{s,t} \Longrightarrow E_\infty^{s+t}.
  \]
Also, we say that the spectral sequence collapses at the $r'$-th page if $E_{r} = E_{\infty}$ for all $r \geq r'$, but not for $r < r'$. Finally, when we have terms $E_\infty^{n}$  with a natural filtration $F^\bullet E_\infty^n$ (but no natural double grading), we set $E_\infty^{s,t} = \gr^{s} E_\infty^{s+t}= F^{s}E_\infty^{s+t}/F^{s+1}E_\infty^{s+t}$.

\subsection{Smith normal form and cohomology}%
\label{subsec:SNF-coh}

It is well known that the Smith normal form of matrices are useful when calculating (co)homology over $\Z$ as follows.

\begin{fact}\label{fact:SNF-Z-coh}
  Given a complex
  \[
    \begin{tikzcd}
      \Z^{n} \ar[r,"d_{1}"] & \Z^{m} \ar[r,"d_{2}"] & \Z^{\ell},
    \end{tikzcd}
  \]
  where $d_{1}$ and $d_{2}$ are $\Z$-linear maps with $d_{2} \circ d_{1} = 0$, the homology at the middle term is given by
  \begin{equation*}
    \kernel(d_{2})/\image(d_{1}) \iso \bigoplus_{i=1}^{r} \Z/a_{i}\Z \oplus \Z^{m-r-s}.
  \end{equation*}
  Here $r = \rank(d_{1})$, $s = \rank(d_{2})$ and $a_{1},\dotsc,a_{r}$ are the non-zero diagonal elements of the Smith normal form of $d_{1}$.
\end{fact}

We will not directly use this result, but instead we will follow the same ideas but reduce modulo $p$ to get matrices over $k$ (using the natural embedding $\F_{p} \hookrightarrow k$). Assuming that the non-zero diagonal entries $a_{i}$ of the Smith normal form of a matrix $d$ are in $\set{1,2,\dotsc,p-1}$ (or more generally $\gcd(a_{i},p) = 1$), we note that $a_{i} \pmod{p} \in k^{\times}$. So, given an $n \times m$ matrix $d$ with integer entries such that
\begin{equation*}
  d \snfsim \SNF_{n \times m}(a_{1},\dotsc,a_{r},0,\dotsc,0),
\end{equation*}
where $a_{1},\dotsc,a_{r}$ are non-zero and $\gcd(a_{i},p) = 1$, we get by considering $d$ as a matrix over $k$ that
\begin{equation}
  \label{eq:snf-dims}
  \begin{aligned}
    \dim_{k} \kernel(d) &= m-r, \\
    \dim_{k} \image(d) &= r, \\
    \dim_{k} \coker(d) &= n-r.
  \end{aligned}
\end{equation}

\begin{remark}
  Note that finding the Smith normal form of all matrices used in our (co)homology calculations, will thus allow us to calculate (co)homological dimensions for $p$ relatively prime to all non-zero diagonal entries of the Smith normal form matrices. This is what makes this method preferable to just calculating the rank of the matrices directly, since that would just allow us to find (co)homological dimensions for $p \gg 0$, but not give us the precise $p$ it will work for.
\end{remark}

\begin{remark}
  We assume here that $\lie{g}$ can be lifted to a Lie algebra $\lie{g}_{\Z}$ with the same Chevalley constants such that $\lie{g} = \lie{g}_{\Z} \otimes k$. In particular, we assume that these Chevalley constants are such that $\lie{g}_{\Z}$ satisfy Jacobi's identity. This will not be a problem in the following sections, since we are working with Lie algebras that are well defined mod $p$ for any large enough prime $p$ with coefficients independent of $p$. In \Cref{subsec:central-div-algs} we will see examples of Lie algebras where we need to work modulo a specific prime (we will do $p = 5$) and cannot lift easily to $\lie{g}_{\Z}$.
\end{remark}

So, when calculating dimensions of homology over $k$ of the middle term in a given complex
\[
  \begin{tikzcd}
    k^{n} \ar[r,"d_{1}"] & k^{m} \ar[r,"d_{2}"] & k^{\ell},
  \end{tikzcd}
\]
where $d_{1}$ and $d_{2}$ can be described by matrices with integer entries, and $d_{1} \snfsim \SNF_{m \times n}(a_{1},\dotsc,a_{r},0,\dotsc,0)$ and $d_{2} \snfsim \SNF_{\ell \times m}(b_{1},\dotsc,b_{r},0,\dotsc,0)$, then we can do it as follows:
\begin{enumerate}[align=left]
  \item[$n=0$:] The dimension of the homology of the middle term is $\dim_{k} \kernel(d_{2}) = m-s$.
  \item[$\ell=0$:] The dimension of the homology of the middle term is $\dim_{k} \coker(d_{1}) = m-r$.
  \item[$n,\ell \neq 0$:] The dimension of the homology of the middle term is $\dim_{k} \frac{\kernel(d_{2})}{\image(d_{1})} = m-s-r$, since $d_{2} \circ d_{1} = 0$.
\end{enumerate}

\begin{remark}
  Here the general formula is obviously just that the dimension of the homology of the middle term is $m-s-r$. Also, note that this is what we directly get from \Cref{fact:SNF-Z-coh} in the case $k = \F_{p}$, recalling that $\Z/m\Z \otimes \Z/n\Z \iso \Z/\gcd(n,m)\Z$.
\end{remark}

\begin{remark}
  When the dimensions of the vector spaces we work with get sufficiently large, the runtime of calculating the full Smith normal form of integer matrices becomes prohibitively high, so we can use an alternative solution. In this case, we can utilize that we know such a form exists, and that $\rank_{\Z}(A) = \rank_{\Z}(B)$ when $A \snfsim B$. Considering the $n \times m$ matrix $A$ with integer entries as a matrix over $\R$, we can then find the Singular value decomposition\index{Singular value decomposition} (SVD)\index{SVD} of $A$, i.e., complex matrices $U,\Sigma,V$ such that $A = U\Sigma V^{*}$. Here $U$ is an $n \times n$ unitary matrix, $\Sigma$ is a rectangular diagonal $n \times m$ matrix (a matrix like in the Smith normal form) with non-negative real numbers on the diagonal, and $V$ is an $m \times m$ unitary matrix. Now $\rank_{\R} \Sigma = \rank_{\Z} A$ allows us to find dimensions of (co)homology as in the case where we know the Smith normal form, but we use information about which $p$ exactly the calculations work for. Thus we will only be able to find the (co)homological dimensions for $p \gg 0$ in this case.
\end{remark}

\section{Techniques}%
\label{sec:tech-iwa}

In this section we will describe how to calculate information about the cohomology of a $p$-valuable group by using its Lazard Lie algebra. Note that this section uses a lot of concepts and notation from \Cref{subsec:Laz-theory}.

Let $(G, \omega)$ be a $p$-valuable group and let $k$ be a perfect field of characteristic $p$. In this section we will describe how the spectral sequence
\begin{equation}\label{eq:spec-sec-tech}
  E_{1}^{s,t} = H^{s,t}(\lie{g}, k) \Longrightarrow H^{s+t}(G, k)
\end{equation}
from \cite[§6.1]{Sor} can be used to calculate information about the dimensions of $H^{n}(G,k)$ for varying $n$ and information about the cup product on $H^{*}(G,k)$. After this, we will then briefly discuss how this applies to pro-$p$ Iwahori subgroups $I$ of $\GL_{n}$ or $\SL_{n}$.

Recall that $\lie{g}$ in the above spectral sequence is given by $\lie{g} = k \otimes_{\F_{p}[\pi]} \gr G$, so to describe $\lie{g}$, we first need a good description of the $\F_{p}[\pi]$-Lie algebra $\gr G$. To get this description, suppose that we have an ordered basis $(g_{1},\dotsc,g_{d})$ of $G$, so that $\omega(g) = \min_{i = 1,\dotsc,d} \bigl( \omega(g_{i}) + v_{p}(x_{i}) \bigr)$ for $g = g_{1}^{x_{1}} \dotsb g_{d}^{x_{d}}$, and recall that $\bigl( \sigma(g_{1}),\dotsc,\sigma(g_{d}) \bigr)$ is a basis of $\gr G$, where $\sigma(g) = gG_{\omega(g)+} \in \gr G$ for $g \neq 1$.

To understand the $\F_{p}[\pi]$-Lie algebra, we need to find $[\sigma(g_{i}),\sigma(g_{j})] = \sigma\bigl( [g_{i},g_{j}] \bigr)$ for all $i,j = 1,\dotsc,d$. We recall from \eqref{eq:sigma-gx} that $\sigma(g^{x}) = \overline{x}\pi^{v_{p}(x)} \act \sigma(g)$ for $g \in G \setminus\set{1}$ and $x \in \Z_{p}\setminus\set{0}$. Now, calculating $[g_{i},g_{j}]$ for all $i,j = 1,\dotsc,d$, we can find $x_{1},\dotsc,x_{d} \in \Z_{p}$ such that
\begin{equation*}
  [g_{i},g_{j}] = g_{1}^{x_{1}} \dotsb g_{d}^{x_{d}},
\end{equation*}
and thus
\begin{equation*}
  \bigl[ \sigma(g_{i}),\sigma(g_{j}) \bigr] = \sigma\bigl( [g_{i},g_{j}] \bigr) = \sum_{\ell=1}^{d} \overline{x}_{\ell}\pi^{v_{p}(x_{\ell})} \act \sigma(g_{\ell}).
\end{equation*}
See the proofs of \cite[Lem.~26.4 and Prop.~26.5]{Sch} for more details.

Let $\set{\ell_{1},\dotsc,\ell_{r}}$ be the subset of $\set{1,\dotsc,d}$ such that $v_{p}(x_{\ell_{s}}) = 0$ and $v_{p}(x_{\ell}) > 0$ for $\ell \notin \set{\ell_{1},\dotsc,\ell_{r}}$, and recall that $\lie{g} = k \otimes_{\F_{p}[\pi]} \gr G$ has basis $1 \otimes \sigma(g_{i})$. Since $\pi$ acts trivially on $k$ here, we see that
\begin{equation*}
  [\xi_{i},\xi_{j}] = \bigl[ 1\otimes\sigma(g_{i}),1\otimes\sigma(g_{j}) \bigr] = \sum_{s=1}^{r} \overline{x}_{\ell_{s}} \sigma(g_{\ell_{s}}).
\end{equation*}
Now we have a basis $(\xi_{1},\dotsc,\xi_{d})$ of $\lie{g} = k \otimes_{\F_{p}[\pi]} \gr G$, and we know all the structure constants.

\begin{remark}\label{rem:struc-consts-lift}
  Note that the structure constants are in $\F_{p} \subseteq k$ by the above, so we can lift them to structure constants in $\set{0,1,\dotsc,p-1} \subseteq \Z$, which will be useful later. Also note that we will often (but not always) be able to lift $\lie{g}$ to a $\Z$-Lie algebra $\lie{g}_{\Z}$ with $\lie{g} = \lie{g}_{\Z} \otimes k$.
\end{remark}

Assume from now on that the Lie algebra $\lie{g}$ is unitary,\index{unitary} i.e., that $[\xi_{i},\xi_{j}] = \sum_{\ell=1}^{d} c_{ij\ell} \xi_{\ell}$ has $\sum_{j=1}^{d} c_{ijj} = 0$. This will be the case for all Lie algebras, we will work with in this chapter. Suppose furthermore that $\lie{g}$ is a graded Lie algebra, graded by finitely many positive integers, $\lie{g} = \lie{g}^{1} \oplus \lie{g}^{2} \oplus \dotsb \oplus \lie{g}^{m}$, which will also be the case for all Lie algebras we work with in this chapter.

\begin{remark}\label{rem:g-Z-grading}
  Note that any $p$-valuable group $G$ admits a $p$-valuation $\omega$ with values in $\frac{1}{m}\Z$ for some $m \in \N$, cf.\ \cite[Cor.~33.3]{Sch}. Thus we can reindex the filtration of $G$ by letting $G^{i} = G_{\frac{i}{m}}$ for $i=0,1,\dotsc$, and this translates to $\gr^{i} G = \gr_{\frac{i}{m}} G$ and $\lie{g}^{i} = \lie{g}_{\frac{i}{m}}$ in general. In the cases we care about there will be no zero graded part, which allows us to make the above assumption.
\end{remark}

Then $\bigwedge^{n} \lie{g}$ is graded as well by letting
\begin{equation*}
  \gr^{j}\Bigl( \bigwedge^{n}\lie{g} \Bigr) = \bigoplus_{j_{1} + \dotsb + j_{n} = j} \lie{g}^{j_{1}} \wedge \dotsb \wedge \lie{g}^{j_{n}}.
\end{equation*}
We note that, since $\lie{g}$ is finite dimensional, there are only finitely many non-zero $\gr^{j}\bigl( \bigwedge^{n} \lie{g} \bigr)$ we are interested in, and we can find a basis of each of these using our basis $(\xi_{1},\dotsc,\xi_{d})$ of $\lie{g}$.

\begin{remark}
  When ordering the basis of $\gr^{j} \bigl( \bigwedge^{n}\lie{g} \bigr) = \bigoplus_{j_{1} + \dotsb + j_{n} = j} \lie{g}^{j_{1}} \wedge \dotsb \wedge \lie{g}^{j_{n}}$, we will do it as follows. First we order the $\lie{g}^{j_{1}} \wedge \dotsb \wedge \lie{g}^{j_{n}}$ by the lexicographical order on $(j_{1},\dotsc,j_{n})$. Then we order the basis of each $\lie{g}^{j_{1}} \wedge \dotsb \wedge \lie{g}^{j_{n}}$ by the lexicographical order on equal $j_{\ell}$'s, i.e., if $\lie{g}^{1} = \Span_{k}(\xi_{1},\xi_{3})$ and $\lie{g}^{2} = \Span_{k}(\xi_{2},\xi_{4})$, then $\lie{g}^{1} \wedge \lie{g}^{1} \wedge \lie{g}^{2}$ has basis $\xi_{1} \wedge \xi_{3} \wedge \xi_{2}, \xi_{1} \wedge \xi_{3} \wedge \xi_{4}$.
\end{remark}

Assuming furthermore that $k$ is $\Z$-graded (concentrated in degree $0$), the space $\Hom_{k}\bigl( \bigwedge^{n}\lie{g}, k \bigr)$ inherits the $\Z$-grading
\[
  \Hom_{k}\Bigl( \bigwedge^n \lie{g}, k \Bigr) = \bigoplus_{s \in \Z} \Hom_{k}^s\Bigl( \bigwedge^n\lie{g}, k \Bigr),
\]
where $\Hom_{k}^s$ denotes the homogeneous $k$-linear maps of degree $s$, cf.\ \cite[Lem.~4.2]{Fossum}. We note, by \cite[Chap.~1~§3.7]{Fuks}, that these gradings on the chain and cochain complexes transfer to gradings on the homology and cohomology. We write
\begin{equation*}
  H^{s,t} = H^{s,t}(\lie{g}, k) = H^{s+t}\Bigl( \gr^s \Hom_{k}\bigl(\bigwedge^{\bullet} \lie{g}, k\bigr) \Bigr).
\end{equation*}%
\nomiwa[Hst]{$H^{s,t}$}{${}= H^{s,t}(\lie{g},k) = H^{s+t}\bigl( \gr^{s} \Hom_{k}(\bigwedge^{\bullet}\lie{g}, k) \bigr)$}%

\begin{remark}
  We do not spend effort to describe the homology for a few reasons. First, we need the cohomology, not the homology, in our spectral sequence. Second, by \cite[Chap.~1~§3.6]{Fuks}, we have a version of Poincaré duality for Lie algebra cohomology, i.e., $H^{n}(\lie{g},k) \iso H_{n-d}(\lie{g},k)$, so we can easily describe the homology using the cohomology. Third, we will care about the cup product later, and we do not get a nice product in homology, cf.\ \cite[Chap.~1~§3.2]{Fuks}.
\end{remark}

Now we have bases of all $\gr^{j}\bigl( \bigwedge^{n}\lie{g} \bigr)$, and by \cite[Chap.~1~§3.7]{Fuks} we get graded chain complexes that we can use to find the homology of $\lie{g}$. Here the boundary maps of
\[
  \begin{tikzcd}
    \cdots \ar[r,"d_{4}"] & \bigwedge^{3}\lie{g} \ar[r,"d_{3}"] & \bigwedge^{2}\lie{g} \ar[r,"d_{2}"] & \lie{g} \ar[r,"d_{1}"] & k \ar[r] &  0,
  \end{tikzcd}
\]
are given by
\begin{equation*}
  d_{n}(x_{1} \wedge \dotsb \wedge x_{n}) = \sum_{i<j} (-1)^{i+j}[x_{i},x_{j}]\wedge x_{1} \wedge \dotsb \wedge \widehat{x}_{i} \wedge \dotsb \wedge \widehat{x}_{j} \wedge \dotsb \wedge x_{n},
\end{equation*}
and the coboundary maps
\[
  \begin{tikzcd}
    \cdots & \ar[l,"\partial_{3}"'] \Hom_{k}\Bigl( \bigwedge^{2}\lie{g}, k \Bigr) & \ar[l,"\partial_{2}"'] \Hom_{k}(\lie{g},k) & \ar[l,"\partial_{1}"'] \Hom_{k}(k, k) = k &  \ar[l] 0,
  \end{tikzcd}
\]
are the dual maps of the boundary maps (see \cite[Chap.~1~§3.1]{Fuks} for more details). Thus, if we use the dual basis of $\bigwedge^{n}\lie{g}$ in $\Hom_{k}\bigl( \bigwedge^{n} \lie{g},k \bigr)$, we get that $\partial_{n} = d_{n}^{\top}$ as matrices, where $(\edot)^{\top}$ is the transpose (cf.\ \Cref{sec:cohiwagps-intro}). Since we know bases and linear maps explicitly, and we know that the linear maps restrict to graded linear maps, we can now find matrices describing all graded linear maps
\begin{equation*}
  \gr^{j} \Bigl( \bigwedge^{n}\lie{g} \Bigr) \to \gr^{j} \Bigl( \bigwedge^{n-1}\lie{g} \Bigr),
\end{equation*}
and thus we can find matrices describing all graded linear maps
\begin{equation*}
  \Hom_{k}^{s}\Bigl( \bigwedge^{n-1}\lie{g},k \Bigr) \to \Hom_{k}^{s}\Bigl( \bigwedge^{n}\lie{g}, k \Bigr).
\end{equation*}

Noting that all the structure constants can be lifted to $\Z$ (in the examples we work with) and looking at the formula for the boundary maps, it is clear that the above matrices describing the (co)boundary maps can be lifted to $\Z$. Finding the Smith normal form of these lifts, we can calculate the cohomology over $k$ by \Cref{subsec:SNF-coh} for $p$ large enough. (In most examples, $p\geq5$ will be enough.)

Suppose now that we have found the dimensions of $H^{s,t} = H^{s,t}(\lie{g},k)$ for all $s,t$. To get information about the cohomology $H^{n}(G,k)$, we need to use the multiplicative spectral sequence \eqref{eq:spec-sec-tech}, i.e.,
\begin{equation*}
  E_{1}^{s,t} = H^{s,t}(\lie{g}, k) \Longrightarrow H^{s+t}(G,k)
\end{equation*}
and information about spectral sequences in general. We already know that this spectral sequence collapses at a finite page, and one can hope is that it will actually collapse at the first page. One way we can verify that the spectral sequence (in certain cases) collapses at the first page, is by considering the exact bidegrees of the differentials. We know that the differentials $d_{r}^{s,t} \colon E_{r}^{s,t} \to E_{r}^{s+r,t+1-r}$ have bidegree $(r,1-r)$, so if the non-zero modules on the first pages is distributed in such a way that all differentials $d_{r}^{s,t}$ are trivial for all $s,t$ and $r\geq1$, then we can be sure that the spectral sequence collapses on the first page. This will become clearer when we look at examples in the next few sections.

Now note, by \cite[Chap.~1~§3.7]{Fuks}, that the cup product is compatible with the gradings on the Lie algebra cohomology, in particular
\begin{equation}
H^{s,t} \cup H^{s',t'} \subseteq H^{s+s',t+t'},\label{eq:Hst-Hst-subset}
\end{equation}
where $H^{s,t} = H^{s,t}(\lie{g},k)$. Thus, since the spectral sequence is multiplicative, we can describe the cup product on $H^{*}(G,k)$ when the spectral sequence collapses on the first page. Some cup products will be trivially zero by \eqref{eq:Hst-Hst-subset}, and for the rest of the cup product, we can calculate them with an explicit basis using the following.

For $f \in \Hom_{k}\bigl( \bigwedge^{p}\lie{g}, k \bigr)$ and $g \in \Hom_{k}\bigl( \bigwedge^{q}\lie{g}, k \bigr)$, we know from \cite[Chap.~XIII, Sect.~8]{CartanHomAlg}, that the cup product in cohomology is induced by: $f \cup g \in \Hom_{k}\bigl( \bigwedge^{p+q}\lie{g}, k \bigr)$ defined by
\begin{equation}
  \label{eq:cup-prod-def}
  (f \cup g)(x_{1} \wedge \dotsb \wedge x_{p+q})  = \sum_{\mathclap{\substack{ \sigma \in S_{p+q} \\ \sigma(1) < \dotsb < \sigma(p) \\ \sigma(p+1) < \dotsb < \sigma(p+q) }}} \sign(\sigma) f(x_{\sigma(1)} \wedge \dotsb \wedge x_{\sigma(p)}) g(x_{\sigma(p+1)} \wedge \dotsb \wedge x_{\sigma(p+q)}).
\end{equation}

Even when the spectral sequence does not (necessarily) collapse on the first page, we can still get some bounds on the dimensions of $H^{n}(G,k)$ that will allow us to draw some conclusions about the structure of $H^{*}(G,k)$.

In the rest of this chapter we will focus on using the techniques described in this section to get as much as possible information about the cohomology of $H^{*}(I,k)$, where $I$ is the pro-$p$ Iwahori subgroup of $\SL_{n}(\Q_{p})$ or $\GL_{n}(\Q_{p})$ $n=2,3,4$ or the pro-$p$ Iwahori subgroup of $\SL_{2}(F)$ or $\GL_{2}(F)$ for $F/\Q_{p}$ a quadratic extension.



\section{\texorpdfstring{$I \subseteq \SL_{2}(\Z_{p})$}{I in SL2(Zp)}}%
\label{sec:Iwa-SL2}

In this section we will describe the continuous group cohomology of the pro-$p$ Iwahori subgroup $I$ of $\SL_{2}(\Q_{p})$.

When $I$ is the pro-$p$ Iwahori subgroup in $\SL_{2}(\Q_{p})$, we know by \Cref{sec:cohiwagps-intro} that we can take it to be of the form
\begin{equation*}
  I = \pmat{1+p\Z_{p} & \Z_{p} \\ p\Z_{p} & 1+p\Z_{p}}^{\!\!\det = 1} \subseteq \SL_{2}(\Z_{p}).
\end{equation*}
In this case, an obvious guess for an ordered basis (using that $(1+p)^{\Z_{p}} = 1+p\Z_{p}$) is
\begin{align*}
  g_{1}' &= \pmat{1 & 0 \\ p & 1}, & g_{2}' &= \pmat{1+p & 0 \\ 0 & (1+p)^{-1}}, & g_{3}' &= \pmat{1 & 1 \\ 0 & 1}.
\end{align*}
Because we want to be able to describe the commutators using this ordered basis, we will at one point need to solve for $x$ in equation of the form $(1+p)^{x} = y$. For this reason a better choice of ordered basis is (as described in \Cref{sec:cohiwagps-intro})
\begin{align}
  \label{eq:gis-SL2}
  g_{1} &= \pmat{1 & 0 \\ p & 1}, & g_{2} &= \pmat{\exp(p) & 0 \\ 0 & \exp(-p)}, & g_{3} &= \pmat{1 & 1 \\ 0 & 1}.
\end{align}
In this case the above equations to solve translate to solving for $x$ in $\exp(x) = y$, which we can easily do, as $x = \log(y)$, cf.\ \Cref{sec:cohiwagps-intro}.

\subsection{Finding the commutators \texorpdfstring{$[\xi_{i},\xi_{j}]$}{[xi-i,xi-j]}}%
\label{subsec:non-id-xi_ij-SL2}

Now write
\begin{equation}
  \label{eq:gixi-SL2}
  g_{1}^{x_{1}}g_{2}^{x_{2}}g_{3}^{x_{3}} = \pmat{\exp(px_{2}) & x_{3}\exp(px_{2}) \\ px_{1}\exp(px_{2}) & px_{1}x_{3}\exp(px_{2}) + \exp(-px_{2})} = \pmat{a_{11} & a_{12} \\ a_{21} & a_{22}}.
\end{equation}
Furthermore, write $g_{ij} = [g_{i},g_{j}]$\nomiwa[gij]{$g_{ij}$}{${} = [g_{i},g_{j}]$} and $\xi_{ij} = [\xi_{i},\xi_{j}]$.\nomiwa[xiij]{$\xi_{ij}$}{${} = [\xi_{i},\xi_{j}]$} Then we are ready to find $x_{1},x_{2},x_{3}$ such that $g_{ij} = g_{1}^{x_{1}}g_{2}^{x_{2}}g_{3}^{x_{3}}$ for different $i<j$. (In the following we use that $\frac{1}{p-1} = 1 + p + p^{2} + \dotsb$ and $\log(1-p) = -p - \frac{p^{2}}{2} - \frac{p^{3}}{3} - \dotsb$.)

We now list all non-identitiy commutators $g_{ij} = [g_{i},g_{j}]$ and find $\xi_{ij} = [\xi_{i},\xi_{j}]$ based on these. (For $g_{ij} = 1_{2}$ it is clear that $x_{1} = x_{2} = x_{3} = 0$, and thus $\xi_{ij} = 0$.)

\begin{description}
  \item[$g_{12} = \pmat{ 1 & 0 \\ p\bigl( 1 - \exp(-2p) \bigr) & 1 }$:] Comparing $g_{12}$ with \eqref{eq:gixi-SL2}, we see that $x_{2} = x_{3} = 0$. This leaves $a_{21} = px_{1} = p\bigl( 1 - \exp(-2p) \bigr) = 2p^{2} + O(p^{3})$, which implies that $x_{1} = 2p + O(p^{2})$. Hence $\sigma(g_{12}) = 2\pi \act \sigma(g_{1})$, which implies that $\xi_{12} = 0$.

  \item[$g_{13} = \pmat{ 1-p & p \\ -p^{2} & 1+p+p^{2} }$:] Comparing $g_{13}$ with \eqref{eq:gixi-SL2}, we see that
        \begin{align*}
          a_{11} &= \exp(px_{2}) = 1-p, \\
          a_{12} &= x_{3}\exp(px_{2}) = x_{3}(1-p) = p, \\
          a_{21} &= px_{1}\exp(px_{2}) = px_{1}(1-p) = -p^{2},
        \end{align*}
        and thus
        \begin{align*}
          x_{2} &= \dfrac{1}{p}\log(1-p) = \dfrac{1}{p}\bigl( (-p) + O(p^{2}) \bigr) = -1 + O(p), \\
          x_{3} &= \dfrac{p}{1-p} = p + O(p^{2}), \\
          x_{1} &= \dfrac{-p^{2}}{p(1-p)} = -p + O(p^{2}).
        \end{align*}
        Hence $\sigma(g_{13}) = -\pi \act \sigma(g_{1}) - \sigma(g_{2}) - \pi \act \sigma(g_{3})$, which implies that $\xi_{13} = -\xi_{2}$.

  \item[$g_{23} = \pmat{ 1 & \exp(2p)-1 \\ 0 & 1 }$:] Comparing $g_{23}$ with \eqref{eq:gixi-SL2}, we see that $x_{1} = x_{2} = 0$. This leaves $a_{12} = x_{3} = \exp(2p)-1 = 2p + O(p^{2})$. Hence $\sigma(g_{23}) = 2\pi \act \sigma(g_{3})$, which implies that $\xi_{23} = 0$.
\end{description}

To clarify, we found that
\begin{align*}
  \sigma(g_{12}) &= 2\pi \act \sigma(g_{1}), \\
  \sigma(g_{13}) &= -\pi \act \sigma(g_{1}) - \sigma(g_{2}) - \pi \act \sigma(g_{3}), \\
  \sigma(g_{23}) &= 2\pi \act \sigma(g_{3}),
\end{align*}
and recalling that $\xi_{i} = 1 \otimes \sigma(g_{i})$ in $k \otimes_{\F_{p}[\pi]} \gr G$, where $\pi$ acts trivially on $k$, we get that
\begin{align}\label{eq:xi_ij-SL2}
  \xi_{12} &= 0, & \xi_{13} &= -\xi_{2}, & \xi_{23} &= 0,
\end{align}
where $\xi_{ij} = [\xi_{i},\xi_{j}]$.

\subsection{Describing the graded chain complex, \texorpdfstring{$\gr^{j}\bigl(\bigwedge^{n}\lie{g}\bigr)$}{grj(wedge-n g)}}%
\label{subsec:graded-complex-SL2}

Looking at \eqref{eq:Iwa-p-val-basis-SLn} (with $e=1$ and $h=2$), we see that
\begin{align*}
  \omega(g_{1}) &= 1-\frac{1}{2} = \frac{1}{2}, & \omega(g_{2}) &= 1, & \omega(g_{3}) &= \frac{1}{2}.
\end{align*}
Hence $\lie{g}^{1} = \lie{g}_{\frac{1}{2}} = \Span_{k}(\xi_{1},\xi_{3})$ and $\lie{g}^{2} = \lie{g}_{1} = \Span_{k}(\xi_{2})$, cf.\ \Cref{rem:g-Z-grading}.

% So with $\xi_{i} = 1 \otimes \sigma(g_{i})$:
% \begin{align*}
%   [\xi_{1},\xi_{2}] &= 0, & [\xi_{1},\xi_{3}] &= -\xi_{2}, & [\xi_{2},\xi_{3}] &= 0.
% \end{align*}

Now we are ready to describe the graded chain complex
\begin{equation*}
  \gr^{j}\Bigl( \bigwedge^{n}\lie{g} \Bigr) = \bigoplus_{j_{1} + \dotsb + j_{n} = j} \lie{g}^{j_{1}} \wedge \dotsb \wedge \lie{g}^{j_{n}}
\end{equation*}
and its bases. We list the grading of $\bigwedge^{n}\lie{g}$ for all $n$.

\begin{description}
  \item[$n=0:$]
        \begin{equation*}
          \gr^{j}(k) =
          \begin{dcases}
            k & j=0, \\
            0 & \text{otherwise.}
          \end{dcases}
        \end{equation*}
        Bases:
        \begin{align*}
          k: \quad & 1.
        \end{align*}

   \item[$n=1:$]
        \begin{equation*}
          \gr^{j}(\lie{g}) =
          \begin{dcases}
            \lie{g}^{2} & j=2, \\
            \lie{g}^{1} & j=1, \\
            0          & \text{otherwise.}
          \end{dcases}
        \end{equation*}
        Bases:
        \begin{align*}
          \lie{g}^{1}: \quad & \xi_{1}, \xi_{3}, \\
          \lie{g}^{2}: \quad & \xi_{2}.
        \end{align*}

   \item[$n=2:$]
        \begin{equation*}
          \gr^{j}\Bigl( \bigwedge^{2}\lie{g} \Bigr) =
          \begin{dcases}
            \lie{g}^{1} \wedge \lie{g}^{2} & j=3, \\
            \lie{g}^{1} \wedge \lie{g}^{1} & j=2, \\
            0                             & \text{otherwise.}
          \end{dcases}
        \end{equation*}
        Bases:
        \begin{align*}
          \lie{g}^{1} \wedge \lie{g}^{2}: \quad & \xi_{1} \wedge \xi_{2}, \xi_{3} \wedge \xi_{2}, \\
          \lie{g}^{1} \wedge \lie{g}^{1}: \quad & \xi_{1} \wedge \xi_{3}.
        \end{align*}

  \item[$n=3:$]
        \begin{equation*}
          \gr^{j}\Bigl( \bigwedge^{3}\lie{g} \Bigr) =
          \begin{dcases}
            \lie{g}^{1} \wedge \lie{g}^{1} \wedge \lie{g}^{2} & j=4, \\ 0                                                & \text{otherwise.}
          \end{dcases}
        \end{equation*}
        Bases:
        \begin{align*}
          \lie{g}^{1} \wedge \lie{g}^{1} \wedge \lie{g}^{2}: \quad & \xi_{1} \wedge \xi_{3} \wedge \xi_{2}.
        \end{align*}

   \item[$n\geq4:$]
        \begin{equation*}
          \gr^{j}\Bigl( \bigwedge^{n}\lie{g} \Bigr) = 0 \text{ for all } j.
        \end{equation*}
\end{description}

\begin{table}[ht]
  \centering
  \caption[Graded complex dimensions for the $I \subseteq \SL_{2}(\Z_{p})$ case]{Dimensions of $\gr^{j}\bigl( \bigwedge^{n} \lie{g} \bigr)$ for the $I \subseteq \SL_{2}(\Z_{p})$ case.}
  \label{tab:graded-dims-SL2}
  $\begin{NiceArray}{*{6}{c}}[hvlines]
    \diagbox{n}{j} & 0 & 1 & 2 & 3 & 4 \\
    0 & 1 \\
    1 & & 2 & 1 \\
    2 & & & 1 & 2 \\
    3 & & & & & 1
  \end{NiceArray}$
\end{table}

We collect the above information about the dimensions of the chain complex of $\lie{g}$ in \Cref{tab:graded-dims-SL2}, and note that we only need to consider non-zero (non-empty) entries of the table, when we calculate  $H^{s,t} = H^{s,n-s}$ (where $H^{s,t} = H^{s,t}(\lie{g},k)$). Also, recalling that
\begin{equation*}
  \Hom_{k}\Bigl( \bigwedge^{n}\lie{g}, k \Bigr) = \bigoplus_{s \in \Z} \Hom_{k}^{s}\Bigl( \bigwedge^{n}\lie{g}, k \Bigr),
\end{equation*}
we see that, with $j=-s$, we get the same table for dimensions of the graded hom-spaces in the cochain complex.

\subsection{Finding the graded Lie algebra cohomology, \texorpdfstring{$H^{s,t}(\lie{g},k)$}{H(s,t)(g,k)}}%
\label{subsec:graded-coh-SL2}

\begin{remark}
  In this section we will calculate the cohomology directly instead of using the method described in \Cref{subsec:SNF-coh}, since the calculations are only with small matrices. To see how \Cref{subsec:SNF-coh} is used, we refer to \Cref{sec:Iwa-SL3}.
\end{remark}

We will now go through all different graded chain complexes one by one, using that $\gr^{j}$ in the chain complex corresponds to $\gr^{s}$ with $s = -j$ in the cochain complex. We note that the graded chain complex corresponds to vertical downwards arrows in \Cref{tab:graded-dims-SL2}, while the cochain complex corresponds to vertical upwards arrows. And finally, we reiterate that $H^{n} = H^{n}(\lie{g},k)$ and $H^{s,t} = H^{s,t}(\lie{g},k)$ in the following.

In grade $0$ we have the chain complex
\[
  \begin{tikzcd}
    0 \ar[r] & k \ar[r] & 0,
  \end{tikzcd}
\]
which gives us the grade $0$ cochain complex
\[
  \begin{tikzcd}
    0 & \ar[l] \Hom_{k}^{0}(k,k) & \ar[l] 0.
  \end{tikzcd}
\]
So $H^{0} = H^{0,0}$ with $\dim H^{0,0} = 1$.

In grade $1$ we have the chain complex
\[
  \begin{tikzcd}
    0 \ar[r] & \lie{g}^{1} \ar[r] & 0,
  \end{tikzcd}
\]
which gives us the grade $-1$ cochain complex
\[
  \begin{tikzcd}
    0 & \ar[l] \Hom_{k}^{-1}(\lie{g},k) & \ar[l] 0.
  \end{tikzcd}
\]
So $\dim H^{-1,2} = 2$ by \Cref{tab:graded-dims-SL2}.

In grade $2$ we have the chain complex
\[
  \begin{tikzcd}[ampersand replacement=\&]
    0 \ar[r] \& \lie{g}^{1} \wedge \lie{g}^{1} \ar[r, "{(1)}" {yshift=2pt}] \& \lie{g}^{2} \ar[r] \& 0,
  \end{tikzcd}
\]
since
\begin{align*}
  \lie{g}^{1} \wedge \lie{g}^{1} &\to \lie{g}^{2} \\
  \xi_{1} \wedge \xi_{3} &\mapsto -[\xi_{1},\xi_{3}] = \xi_{2}.
\end{align*}
This gives us the grade $-2$ cochain complex
\[
  \begin{tikzcd}[ampersand replacement=\&]
    0 \& \ar[l] \Hom_{k}^{-2}\bigl( \bigwedge^{2} \lie{g}, k \bigr) \& \ar[l, "{(1)}"' {yshift=2pt}] \Hom_{k}^{-2}(\lie{g},k) \& \ar[l] 0.
  \end{tikzcd}
\]
So with $d = (1)$, and comparing with \Cref{tab:graded-dims-SL2},
\begin{align*}
  \dim H^{-2,3} &= \dim \kernel(d) = 0, \\
  \dim H^{-2,4} &= \dim \coker(d) = 0.
\end{align*}

In grade $3$ we have the chain complex
\[
  \begin{tikzcd}
    0 \ar[r] & \lie{g}^{1} \wedge \lie{g}^{2} \ar[r] & 0,
  \end{tikzcd}
\]
which gives us the grade $-3$ cochain complex
\[
  \begin{tikzcd}
    0 & \ar[l] \Hom_{k}^{-3}\bigl( \bigwedge^{2}\lie{g}, k \bigr) & \ar[l] 0.
  \end{tikzcd}
\]
So $\dim H^{-3,5} = 2$ by \Cref{tab:graded-dims-SL2}.

In grade $4$ we have the chain complex
\[
  \begin{tikzcd}
    0 \ar[r] & \lie{g}^{1} \wedge \lie{g}^{1} \wedge \lie{g}^{2}  \ar[r] & 0,
  \end{tikzcd}
\]
which gives us the grade $-4$ cochain complex
\[
  \begin{tikzcd}
    0 & \ar[l] \Hom_{k}^{-4}\bigl( \bigwedge^{3}\lie{g}, k \bigr) & \ar[l] 0.
  \end{tikzcd}
\]
So $\dim H^{-4,7} = 1$ by \Cref{tab:graded-dims-SL2}.

\begin{table}[ht]
  \centering
  \caption[Graded cohomology dimensions for the $I \subseteq \SL_{2}(\Z_{p})$ case]{Dimensions of $E_{1}^{s,t} = H^{s,t}(\lie{g},k)$ for the $I \subseteq \SL_{2}(\Z_{p})$ case.}
  \label{tab:graded-coh-dims-SL2}
  \renewcommand{\arraystretch}{1.5}
  $\begin{NiceArray}{*{6}{c}}[hvlines, columns-width=auto]
    \diagbox{t}{s} & 0 & -1 & -2 & -3 & -4 \\
    0 & 1 \\
    1 & \\
    2 & & 2 \\
    3 & \\
    4 & \\
    5 & & & & 2 \\
    6 & \\
    7 & & & & & 1
  \end{NiceArray}$
  \renewcommand{\arraystretch}{1}
\end{table}

Altogether, we see that
\begin{equation}
  \label{eq:Hn-to-Hst-SL2}
  \begin{aligned}
    H^{0} &= H^{0,0}, \\
    H^{1} &= H^{-1,2}, \\
    H^{2} &= H^{-3,5}, \\
    H^{3} &= H^{-4,7},
  \end{aligned}
\end{equation}
with dimension as described in \Cref{tab:graded-coh-dims-SL2}.

\subsection{Describing the group cohomology, \texorpdfstring{$H^{n}(I,k)$}{Hn(I,k)}}%
\label{subsec:group-coh-SL2}

We note that all differentials $d_{r}^{s,t} \colon E_{r}^{s,t} \to E_{r}^{s+r,t+1-r}$ in \Cref{tab:graded-coh-dims-SL2} has bidegree $(r,1-r)$, i.e., they are all below the $(r,-r)$ arrow going $r$ to the left and $r$ up in the table, where $r \geq 1$. Looking at \Cref{tab:graded-coh-dims-SL2}, this clearly means that all differentials for $r \geq 1$ are trivial, and thus the spectral sequence collapses on the first page. Hence $H^{s,t}(\lie{g},k) = E_{1}^{s,t} \iso E_{\infty}^{s,t} = \gr^{s} H^{s+t}(I,k)$, and by \eqref{eq:Hn-to-Hst-SL2} and \Cref{tab:graded-coh-dims-SL2} we get that
\begin{equation}
  \label{eq:dim-HnI-SL2}
  \dim H^{n}(I,k) =
  \begin{dcases}
    1 & n=0, \\
    2 & n=1, \\
    2 & n=2, \\
    1 & n=3.
  \end{dcases}
\end{equation}

Recalling that the spectral sequence is multiplicative, we also note, by \Cref{tab:graded-coh-dims-SL2}, that $H^{s,t} \cup H^{s',t'} \subseteq H^{s+s',t+t'}$ implies that the cup products
\begin{equation*}
  \gr^{s} H^{n}(I,k) \otimes \gr^{s'} H^{n'}(I,k) \to \gr^{s+s'} H^{n+n'}(I,k)
\end{equation*}
are trivial, except for the obvious ones with $H^{0}(I,k)$ and $H^{1} \otimes H^{2} \to H^{3}$. We now want to describe the cup product $H^{1} \otimes H^{2} \to H^{3}$.

Let $e_{i_{1},\dotsc,i_{m}} = (\xi_{i_{1}} \wedge \dotsb \wedge \xi_{i_{m}})^{*}$ be the element of the dual basis of $\Hom_{k}\bigl( \bigwedge^{m}\lie{g},k \bigr)$ corresponding to $\xi_{i_{1}} \wedge \dotsb \wedge \xi_{i_{m}}$ in the basis of $\bigwedge^{m}\lie{g}$. Looking at the cochain complexes and descriptions of the maps above together with the known bases of the graded chain complexes, we get the following precise descriptions of the of the graded cohomology spaces $H^{s,t} = H^{s,t}(\lie{g},k)$:
\begin{equation}\label{eq:Hst-spaces-SL2}
  \begin{aligned}
    H^{-1,2} &= k[e_{1},e_{3}], \\
    H^{-3,5} &= k[e_{1,2},e_{3,2}], \\
    H^{-4,7} &= k[e_{1,3,2}].
  \end{aligned}
\end{equation}

For $f \in \Hom_{k}\bigl( \bigwedge^{p}\lie{g}, k \bigr)$ and $g \in \Hom_{k}\bigl( \bigwedge^{q}\lie{g}, k \bigr)$, we recall from \eqref{eq:cup-prod-def} that the cup product in cohomology is induced by: $f \cup g \in \Hom_{k}\bigl( \bigwedge^{p+q}\lie{g}, k \bigr)$ defined by
\begin{equation*}
  (f \cup g)(x_{1} \wedge \dotsb \wedge x_{p+q})  = \sum_{\mathclap{\substack{ \sigma \in S_{p+q} \\ \sigma(1) < \dotsb < \sigma(p) \\ \sigma(p+1) < \dotsb < \sigma(p+q) }}} \sign(\sigma) f(x_{\sigma(1)} \wedge \dotsb \wedge x_{\sigma(p)}) g(x_{\sigma(p+1)} \wedge \dotsb \wedge x_{\sigma(p+q)}).
\end{equation*}
So, when finding
\[
  \begin{tikzcd}
    H^{-1,2} \otimes H^{-3,5} \ar[r,"\cup"] & H^{-4,7},
  \end{tikzcd}
\]
we need to calculate $e_{1} \cup e_{1,2}$, $e_{1} \cup e_{3,2}$, $e_{3} \cup e_{1,2}$ and $e_{3} \cup e_{3,2}$ on the basis $\basis = (\xi_{1} \wedge \xi_{3} \wedge \xi_{2})$ of $\gr^{4} \bigwedge^{3} \lie{g}$.

We first note that \eqref{eq:cup-prod-def} simplifies to
\begin{equation*}
  (e_{i} \cup e_{j,k})(x_{1} \wedge x_{2} \wedge x_{3}) = \sum_{\substack{\sigma \in S_{3} \\ \sigma(2) < \sigma(3)}} \sign(\sigma) e_{i}(x_{\sigma(1)}) e_{j,k}(x_{\sigma(2)} \wedge x_{\sigma(3)})
\end{equation*}
in these cases. Here the terms of the sum on the right is only non-zero if $x_{\sigma(1)} = \xi_{i}$ and $x_{\sigma(2)} \wedge x_{\sigma(3)} = \xi_{j} \wedge \xi_{3}$ (up to constants). In the case $e_{1} \cup e_{1,2}$ (resp.\ $e_{3} \cup e_{3,2}$), we can only have this if $x_{1} \wedge x_{2} \wedge x_{3}$ contains two copies of $\xi_{1}$ (resp.\ $\xi_{3}$), which implies that $x_{1} \wedge x_{2} \wedge x_{3} = 0$. So $e_{1} \cup e_{1,2} = 0$ and $e_{3} \cup e_{3,2} = 0$. Alternatively, one can see this by plugging in $x_{1} \wedge x_{2} \wedge x_{3} = \xi_{1} \wedge \xi_{3} \wedge \xi_{2}$ and simply calculating the right side.

In the case $e_{1} \cup e_{3,2}$, \eqref{eq:cup-prod-def} simplifies to
\begin{equation*}
  (e_{1} \cup e_{3,2})(x_{1} \wedge x_{2} \wedge x_{3}) = \sum_{\substack{\sigma \in S_{3} \\ \sigma(2) < \sigma(3)}} \sign(\sigma) e_{1}(x_{\sigma(1)}) e_{3,2}(x_{\sigma(2)} \wedge x_{\sigma(3)}).
\end{equation*}
When $x_{1} \wedge x_{2} \wedge x_{3} = \xi_{1} \wedge \xi_{3} \wedge \xi_{2}$, we see that the terms on the right side are only non-zero if $x_{\sigma(1)} = \xi_{1}$, i.e., $\sigma(1)=1$, and thus $\sigma = (1)$ since $\sigma(2) < \sigma(3)$. So $x_{\sigma(1)} = \xi_{1}$, $x_{\sigma(2)} \wedge x_{\sigma(3)} = \xi_{3} \wedge \xi_{2}$ and $\sign(\sigma) = 1$, which gives us $(e_{1} \cup e_{3,2})(\xi_{1} \wedge \xi_{3} \wedge \xi_{2}) = 1$. Hence $e_{1} \cup e_{3,2} = e_{1,3,2}$.

In the case $e_{3} \cup e_{1,2}$, \eqref{eq:cup-prod-def} simplifies to
\begin{equation*}
  (e_{3} \cup e_{1,2})(x_{1} \wedge x_{2} \wedge x_{3}) = \sum_{\substack{\sigma \in S_{3} \\ \sigma(2) < \sigma(3)}} \sign(\sigma) e_{3}(x_{\sigma(1)}) e_{1,2}(x_{\sigma(2)} \wedge x_{\sigma(3)}).
\end{equation*}
When $x_{1} \wedge x_{2} \wedge x_{3} = \xi_{1} \wedge \xi_{3} \wedge \xi_{2}$, we see that the terms on the right side are only non-zero if $x_{\sigma(1)} = \xi_{3}$, i.e., $\sigma(1)=2$, and thus $\sigma = (1,2)$ since $\sigma(2) < \sigma(3)$. So $x_{\sigma(1)} = \xi_{3}$, $x_{\sigma(2)} \wedge x_{\sigma(3)} = \xi_{1} \wedge \xi_{2}$ and $\sign(\sigma) = -1$, which gives us $(e_{3} \cup e_{1,2})(\xi_{1} \wedge \xi_{3} \wedge \xi_{2}) = -1$. Hence $e_{3} \cup e_{1,2} = -e_{1,3,2}$.

In conclusion, all the non-trivial and non-zero cup products (up to graded commutativity) are:
\begin{equation}
  \label{eq:cup-products-SL2}
  \begin{aligned}
    e_{1} \cup e_{3,2} &= e_{1,3,2}, \\
    e_{3} \cup e_{1,2} &= -e_{1,3,2}.
  \end{aligned}
\end{equation}

Now, since the spectral sequence collapses on the first page, all of the above work on the cup product of the Lie algebra cohomology transfers to the cup product on $H^{*}(I,k)$ as described above. In particular, since all $H^{n}(I,k)$ only have one graded component, this is a clear description of the cup product on $H^{*}(I,k)$, and not just a graded cup product.

\begin{remark}\label{rem:quaternion}
  Let $D$ be the division quaternion algebra over $\Q_{p}$ for a prime $p>3$ and let $G = (1+\idm_{D})^{\Nrd = 1}$, where $\Nrd = \Nrd_{D/\Q_{p}}$ is the norm form. From \cite[Sect.~6.3]{Sor} (or from \cite[Prop.~7]{Henn}) we know that there is an isomorphism
  \begin{equation*}
    H^{*}(G,\F_{p}) \iso \F_{p} \oplus \F_{D} \oplus \F_{D} \oplus \F_{p}
  \end{equation*}
  of graded $\F_{p}$-algebras (where $\F_{D} \iso \F_{p^{2}}$ is viewed simply as a $\F_{p}$-vector space). I.e., $H^{n}(G,\F_{p})$ has the same dimensions as described in \eqref{eq:dim-HnI-SL2} (with $k=\F_{p}$). \cite{Sor} also shows that the only non-trivial and non-zero cup product is $H^{1}(G,\F_{p}) \times H^{2}(G,\F_{p}) \to H^{3}(G,\F_{p})$, which corresponds to the trace pairing $\F_{D} \times \F_{D} \to \F_{p}, (x,y) \mapsto \Tr(xy)$ (where $\Tr = \Tr_{\F_{D}/\F_{p}}$ from \cite[Def.~2.5]{Neukirch}). To be more explicit, let's assume that $p \equiv 3 \pmod{4}$. Then $x^{2}+1$ is irreducible over $\F_{p}$, so we can write $\F_{D} = \F_{p}[\alpha]$ with $\alpha^{2} = -1$, where $\F_{p}[\alpha]$ has $\F_{p}$-basis $1,\alpha$. Now, considering the maps $1\colon a+b\alpha \mapsto a+b\alpha$, $\alpha\colon a+b\alpha \mapsto -b + a\alpha$ and $\alpha^{2}=-1 \colon a+b\alpha \mapsto -a-b\alpha$, we see that the trace pairing is given by
  \begin{align*}
    \F_{D} \times \F_{D} &\to \F_{p} \\
    (1,1) &\mapsto \Tr(1) = 2, \\
    (1,\alpha) &\mapsto \Tr(\alpha) = 0, \\
    (\alpha,1) &\mapsto \Tr(\alpha) = 0, \\
    (\alpha,\alpha) &\mapsto \Tr(\alpha^{2}) = -2.
  \end{align*}
  This is (up to a multiple of $2$) the same as the description of the cup product on $H^{*}(I,\F_{p})$ above for $I \subseteq \SL_{2}(\Z_{p})$, so an interesting question is: Is there a nice relation between mod $p$ representations of $G = (1+\idm_{D})^{\Nrd = 1}$ and $I$? We already have bijections between \emph{certain} mod $p$ representations of $D^{\times}$ and $\GL_{2}(\Q_{p})$ from the Jacquet-Langlands correspondence for $\GL_{2}$ (cf.\ \cite{JL}), but by \cite[Rem.~4.5]{JL-remark} irreducible representations of $D^{\times}$ are trivial on $1+\idm_{D}$, so we need something new if we want a correspondence between mod $p$ representations of $G = (1+\idm_{D})^{\Nrd = 1}$ and $I \subseteq \SL_{2}(\Q_{p})$.

  As a continuation to the question in the $p \equiv 3 \pmod{4}$ case, we note that $\lie{g}_{D} = \F_{D} \oplus \F_{D}^{\Tr = 0}$ sitting in degree $1$ and $2$ has Lie bracket given by $[\overline{x},\overline{y}] = \overline{x}\overline{y}^{p} - \overline{y}\overline{x}^{p}$ for any $\overline{x},\overline{y} \in \F_{D}$ in degree $1$ (and $0$ otherwise) by \cite[(6.6)]{Sor}. So, with $\F_{D} = \F_{p}[\alpha]$, we note that
  \[
    [1,\alpha] = 1\cdot\alpha^{p} - \alpha\cdot1^{p} = -2\alpha,
  \]
  since $p \equiv 3 \pmod{4}$ and $\alpha^{2} = -1$. Thus we have an obvious relation between $\lie{g}$ and $\lie{g}_{D}$ given by $\xi_{1},\xi_{3} \leftrightarrow 1,\alpha$ in degree $1$ and $\xi_{2} \leftrightarrow 2\alpha$ in degree $2$. The question is whether we can lift this to a relation between (the representations of) $I$ and $G$. Which we will explore in more detail in \Cref{subsec:quat-algs}.
\end{remark}
  % Here it might be useful to note that we can assume that $i^{2} = -1$ and $j^{2} = p$ by \cite[Thm.~12.1.5]{Voight}, and furthermore that $\sO_{D}=\Z_{p}[i,j,k]$ (where $k=ij$) and $\idm_{D} = j\sO_{D}$, which has $\Z_{p}$-basis $p,pi,j,k$, by \cite[Thm.13.1.6]{Voight}. With this setup, we also have that
  % \begin{align*}
  %   \F_{D} = \sO_{D}/\idm_{D} \iso \idm_{D}^{n}/\idm_{D}^{n+1} &\xrightarrow{\iso} (1+\idm_{D}^{n})/(1+\idm_{D}^{n+1}) \\
  %   x &\mapsto 1+x
  % \end{align*}
  % is an isomorphism for all $n\geq1$, cf.\ \cite[13.5.8]{Voight}, which should help with translating from $G$ to $\lie{g}_{D}$. It only remains to find a (nice) ordered basis of $(1+\idm_{D})^{\Nrd = 1}$, and try to use all of the above to compare it with the basis $g_{1},g_{2},g_{3}$ of $I$ (or possibly with an alternative basis). One place to start might be, to try to use that $D^{\Nrd = 1} = [D^{\times},D^{\times}]$ by \cite[Exc.~30 of Chap.~7]{Voight}. Alternatively, it might be helpful to rewrite
  % \begin{equation*}
  %   D \iso \set[\bigg]{\pmat{ a+bi & c+di \\ p(c-di) & a-bi } \given a,b,c,d\in\Q_{p} } \subseteq M_{2}\bigl( \Q_{p}(i) \bigr),
  % \end{equation*}
  % using \cite[Cor.~13.3.14]{Voight}, since this presentation seems more directly related to the pro-$p$ Iwahori.


% \begin{conjecture}
%   There is some correspondence between mod $p$ representations of $(1+\idm_{D})^{\Nrd = 1}$ and $I \subseteq \SL_{2}(\Z_{p})$.
% \end{conjecture}

It might even be the case, that we have a group isomorphism between $(1+\idm_{D})^{\Nrd = 1}$ and $I$, which lifts the $\F_{p}$-Lie algebra isomorphism $\lie{g} \iso \lie{g}_{D}$ of \Cref{rem:quaternion}, but this seems less likely.

\subsection{Lower \texorpdfstring{$p$}{p}-series of \texorpdfstring{$I$}{I}}%
\label{subsec:lower-p-series-SL2}

One of the consequences of the cohomology calculations above is that $I$ is not a uniformly powerful group. To see this, note by the proof of \cite[Thm.~3.3.3]{Laz-complements} that a uniformly powerful pro-$p$ group is equi-$p$-valuable, and thus Lazard's famous isomorphism $H^{*}(G,k) \iso \bigwedge \Hom_{k}(\lie{g},k)$ for equi-$p$-valued groups $G$ can be applied to uniformly powerful groups. We refer to \cite[Cor.~6.3]{Sor} for a proof of the isomorphism with methods similar to what we use in this chapter. Now, since the dimensions from \eqref{eq:dim-HnI-SL2} do not match the dimensions of $\bigwedge \Hom_{k}(\lie{g},k)$ ($1,3,3,1$ since $\dim_{k} \lie{g} = 3$), we see that $I$ cannot be uniformly powerful (or equi-$p$-valuable). This leads to the interesting question of, whether we can describe the lower $p$-series of $I$, and see that $I$ is not uniformly powerful directly?

Before answering this question, we recall the following definitions, cf.\ \cite[Def.~1.15, Cor.~1.20, Def.~3.1 and Def.~4.1]{analytic_pro-p_groups}.
\begin{definition}
  A $p$-valued group $G$ is \emph{equi-$p$-valuable} if it admits a $p$-valuation $\omega$ and an ordered basis $(g_{1},\dotsc,g_{d})$ such that $\omega(g_{i}) = \omega(g_{j})$ for all $i,j$.\index{group!equi-p-valued@equi-$p$-valued}
\end{definition}

\begin{definition}
  Let $G$ be a finitely generated pro-$p$ group. The \emph{lower $p$-series}\index{lower p-series@lower $p$-series} $\dotsb \geq P_{3}(G) \geq P_{2}(G) \geq P_{1}(G)$ of $G$ is given by $P_{i}(G)$, where $P_{1}(G) = G$ and
  \begin{equation*}
    P_{i+1}(G) = P_{i}(G)^{p}\bigl[ P_{i}(G),G \bigr]
  \end{equation*}
  for $i \geq 1$.
\end{definition}

\begin{definition}
  Let $p$ be an odd prime. A pro-$p$ group $G$ is \emph{uniformly powerful}\index{group!uniformly powerful} (often written as \emph{uniform}) if
  \begin{enumerate}[(i)]
    \item $G$ is finitely generated,
    \item $G$ is \emph{powerful},\index{group!powerful} i.e., $G/\overline{G^{p}}$ is abelian, and
    \item for all $i$, $[P_{i}(G) : P_{i+1}(G)] = [G : P_{2}(G)]$.
  \end{enumerate}
\end{definition}

To show directly that $I$ is not uniformly powerful, we will calculate its lower $p$-series, for which we will introduce the notation $I_{i} = P_{i}(I)$ and work with generators of $I$. We already know that $I$ is generated by $g_{1},g_{2},g_{3}$ (note that this is similar to \cite[Thm.~2.4.1]{Generators} but with $\exp(p)$ instead of $1+p$ in the torus), and we now want to describe the generators of each $I_{i} = P_{i}(I)$. For this description we will use the following lemma.

\begin{lemma}\label{lem:J-I-com}
  Let $J$ be the subgroup of $I$ generated by $g_{1}^{p^{v_{1}}}, g_{2}^{p^{v_{2}}}, g_{3}^{p^{v_{3}}}$, and set $m_{1} = \min(v_{1},v_{2})$, $m_{2} = \min(v_{1},v_{3})$ and $m_{3} = \min(v_{2},v_{3})$. Then
  \begin{enumerate}[(i)]
    \item $J^{p}$ is the subgroup generated by $g_{1}^{p^{v_{1}+1}}, g_{2}^{p^{v_{2}+1}}, g_{3}^{p^{v_{3}+1}}$, and
    \item $[J,I]$ is the subgroup generated by $g_{1}^{p^{m_{1}+1}}, g_{2}^{p^{m_{2}}}, g_{3}^{p^{m_{3}+1}}$.
  \end{enumerate}
\end{lemma}
\begin{proof}
  By the definition of $J$, it is clear that $J^{p}$ is the subgroup generated by $g_{1}^{p^{v_{1}+1}}, g_{2}^{p^{v_{2}+1}}, g_{3}^{p^{v_{3}+1}}$, so we only need to show (ii).

  To find generators of $[J,I]$, it is enough to calculate the commutators of the generators of $J$ and $I$, and we note that $g$ and $g^{x}$ generate the same subgroup for $g \in I$ if $x \in \Z_{p}^{\times}$. Now
  \begin{align*}
    \bigl[g_{1}^{p^{v_{1}}},g_{2}\bigr] &= \pmat{ 1 & 0 \\ p^{v_{1}+1}\bigl( 1 - \exp(-2p) \bigr) & 0 } = g_{1}^{p^{v_{1}}(1-\exp(-2p))}, \\
    \bigl[g_{1},g_{2}^{p^{v_{2}}}\bigr] &= \pmat{ 1 & 0 \\ p\bigl( 1 - \exp(-2p^{v_{2}+1}) \bigr) } = g_{1}^{p^{v_{2}}(1-\exp(-2p^{v_{2}+1}))},
  \end{align*}
  where $\frac{1}{p}\bigl(1-\exp(-2p)\bigr) \in \Z_{p}^{\times}$ and $\frac{1}{p}\bigl( 1-\exp(-2p^{v_{2}+1}) \bigr) \in \Z_{p}^{\times}$ (since they both have $v_{p} = 0$), so these commutators are correspond to generators $g_{1}^{p^{v_{1}+1}}$ and $g_{1}^{p^{v_{2}+1}}$. Since $g_{1}^{p^{v_{1}+1}}$ and $g_{1}^{p^{v_{2}+1}}$ clearly generate the same group as $g_{1}^{p^{m_{1}+1}}$, we have the subgroup generated by $g_{1}^{p^{m_{1}+1}}$ in $[J,I]$.

  Similarly
  \begin{align*}
    \bigl[ g_{2}^{p^{v_{2}}},g_{3} \bigr] &= \pmat{ 1 & \exp(2p^{v_{2}+1}) - 1 \\ 0 & 1 } = g_{3}^{\exp(2p^{v_{2}+1})-1}, \\
    \bigl[ g_{2},g_{3}^{p^{v_{3}}} \bigr] &= \pmat{ 1 & p^{v_{3}}\bigl( \exp(2p)-1 \bigr) \\ 0 & 1 } = g_{3}^{p^{v_{3}}(\exp(2p)-1)},
  \end{align*}
  where $\frac{1}{p^{v_{2}+1}}\bigl( \exp(2p^{v_{2}+1})-1 \bigr) \in \Z_{p}^{\times}$ and $\frac{1}{p}\bigl( \exp(2p)-1 \bigr)$, so these commutators correspond to generators $g_{3}^{p^{v_{2}+1}}$ and $g_{3}^{p^{v_{3}+1}}$, which generate the same subgroup as $g_{3}^{p^{m_{3}+1}}$. Hence the subgroup generated by $g_{1}^{p^{m_{1}+1}}$ and $g_{3}^{p^{m_{3}+1}}$ is a subgroup of $J$.

  Finally, comparing
  \begin{equation*}
    \bigl[ g_{1}^{p^{v_{1}}},g_{3} \bigr] = \pmat{ 1 - p^{v_{1}+1} & p^{v_{1}+1} \\ -p^{2v_{1}+2} & p^{2v_{1}+2} + p^{v_{1}+1} + 1 } = g_{1}^{x_{1}}g_{2}^{x_{2}}g_{3}^{x_{3}}
  \end{equation*}
  with \eqref{eq:gixi-SL2}, we see that
  \begin{align*}
    a_{11} &= \exp(px_{2}) = 1-p^{v_{1}+1}, \\
    a_{12} &= x_{3}\exp(px_{2}) = x_{3}\bigl( 1-p^{v_{1}+1} \bigr) = p^{v_{1}+1}, \\
    a_{21} &= px_{1}\exp(px_{2}) = px_{1}\bigl( 1-p^{v_{1}+1} \bigr) = -p^{2v_{1}+2},
  \end{align*}
  and thus
  \begin{align*}
    x_{2} &= \frac{1}{p}\log(1-p^{v_{1}+1}) = -p^{v_{1}} + O(p^{v_{1}+1}), \\
    x_{3} &= \frac{p^{v_{1}+1}}{1-p^{v_{1}+1}} = p^{v_{1}+1} + O(p^{v_{1}+2}), \\
    x_{1} &= \frac{-p^{2v_{1}+2}}{p(1-p^{v_{1}+1})} = -p^{2v_{1}+1} + O(p^{2v_{1}+2}).
  \end{align*}
  Using that $g_{1}^{m_{1}+1}$ and $g_{3}^{m_{3}+1}$ generate a subgroup of $J$ and that $m_{1} \leq v_{1}$ and $m_{3} \leq v_{3}$, we can see that the above adds a generator $g_{2}^{\frac{1}{p}(\log(1-p^{v_{1}+1}))}$, and since
  \begin{equation*}
    \frac{\frac{1}{p}\log(1-p^{v_{1}+1})}{p^{v_{1}}} \in \Z_{p}^{\times},
  \end{equation*}
  we see that this is equivalent to adding a generator $g_{2}^{p^{v_{2}}}$. Similarly, comparing
  \begin{equation*}
    \bigl[ g_{1},g_{3}^{p^{v_{3}}} \bigr] = \pmat{ 1 - p^{v_{3}+1} & p^{2v_{3}+1} \\ -p^{v_{3}+2} & p^{2v_{3}+2} + p^{v_{3}+1} + 1 } = g_{1}^{x_{1}}g_{2}^{x_{2}}g_{3}^{x_{3}}
  \end{equation*}
  with \eqref{eq:gixi-SL2}, we see that
  \begin{align*}
    a_{11} &= \exp(px_{2}) = 1-p^{v_{3}+1}, \\
    a_{12} &= x_{3}\exp(px_{2}) = x_{3}\bigl( 1-p^{v_{3}+1} \bigr) = p^{2v_{3}+1}, \\
    a_{21} &= px_{1}\exp(px_{2}) = px_{1}\bigl( 1-p^{v_{3}+1} \bigr) = -p^{v_{3}+2},
  \end{align*}
  and thus
  \begin{align*}
    x_{2} &= \frac{1}{p}\log(1-p^{v_{3}+1}) = -p^{v_{3}} + O(p^{v_{3}+1}), \\
    x_{3} &= \frac{p^{2v_{3}+1}}{1-p^{v_{3}+1}} = p^{2v_{1}+1} + O(p^{2v_{1}+2}), \\
    x_{1} &= \frac{-p^{v_{3}+2}}{p(1-p^{v_{3}+1})} = -p^{v_{3}+1} + O(p^{v_{3}+2}).
  \end{align*}
  Using that $g_{1}^{m_{1}+1}$ and $g_{3}^{m_{3}+1}$ generate a subgroup of $J$ and that $m_{1}\leq v_{1}$ and $m_{3}\leq v_{3}$, we can see that the above adds a generator $g_{2}^{\frac{1}{p}(\log(1-p^{v_{3}+1}))}$, and since
  \begin{equation*}
    \frac{\frac{1}{p}\log(1-p^{v_{3}+1})}{p^{v_{3}}} \in \Z_{p}^{\times},
  \end{equation*}
  we see that this is equivalent to adding a generator $g_{2}^{p^{v_{3}}}$. So we have added the generators $g_{2}^{p^{v_{1}}}$ and $g_{2}^{p^{v_{3}}}$, which is equivalent to adding the generator $g_{2}^{p^{m_{2}}}$.

  Altogether, we see that $[J,I]$ is generated by $g_{1}^{p^{m_{1}+1}}$, $g_{2}^{p^{m_{2}}}$ and $g_{3}^{p^{m_{3}+1}}$.
\end{proof}

Now $I = I_{1}$ is generated by $g_{1},g_{2},g_{3}$, so by \Cref{lem:J-I-com} $I^{p}$ is generated by $g_{1}^{p},g_{2}^{p},g_{3}^{p}$ and $[I,I]$ is generated by $g_{1}^{p},g_{2},g_{3}^{p}$. Thus $I_{2} = I^{p}[I,I]$ is generated by $g_{1}^{p},g_{2},g_{3}^{p}$, and we see that $I_{2} = [I,I]$. Using \Cref{lem:J-I-com} again, we see that $[I,I]^{p}$ is generated by $g_{1}^{p^{2}},g_{2}^{p},g_{3}^{p^{2}}$ and $\bigl[ [I,I],I \bigr]$ is generated by $g_{1}^{p},g_{2}^{p},g_{3}^{p}$. So $I_{3}$ is generated by $g_{1}^{p},g_{2}^{p},g_{3}^{p}$, and we see that $I_{3} = I^{p}$. We now claim that
\begin{equation*}
  I_{i} =
  \begin{dcases*}
    I^{p^{n}} & if $i = 2n+1$, \\
    [I,I]^{p^{n-1}} & if $i = 2n$,
  \end{dcases*}
\end{equation*}
where $I_{2n+1}$ is generated by $g_{1}^{p^{n}},g_{2}^{p^{n}},g_{3}^{p^{n}}$ and $I_{2n}$ is generated by $g_{1}^{p^{n}},g_{2}^{p^{n-1}},g_{3}^{p^{n}}$. We will prove this by induction on $i$, where we already covered the base cases above. Assume first that $I_{2n} = [I,I]^{p^{n-1}}$ is generated by $g_{1}^{p^{n}},g_{2}^{p^{n-1}},g_{3}^{p^{n}}$. Then, by \Cref{lem:J-I-com}, $I_{2n}^{p}$ is generated by $g_{1}^{p^{n+1}},g_{2}^{p^{n}},g_{3}^{p^{n+1}}$ and $[I_{2n},I]$ is generated by $g_{1}^{p^{n}},g_{2}^{p^{n}},g_{3}^{p^{n}}$, so $I_{2n+1} = I_{2n}^{p}[I_{2n},I]$ is generated by $g_{1}^{p^{n}}, g_{2}^{p^{n}}, g_{3}^{p^{n}}$ and thus $I_{2n+1} = I^{p^{n}}$. Assume now, on the other hand, that $I_{2n+1} = I^{p^{n}}$ is generated by $g_{1}^{p^{n}}, g_{2}^{p^{n}}, g_{3}^{p^{n}}$. Then, by \Cref{lem:J-I-com}, $I_{2n+1}^{p}$ is generated by $g_{1}^{p^{n+1}}, g_{2}^{p^{n+1}}, g_{3}^{p^{n+1}}$ and $[I_{2n+1},I]$ is generated by $g_{1}^{p^{n+1}}, g_{2}^{p^{n}}, g_{3}^{p^{n+1}}$, so $I_{2n+2} = I_{2n+1}^{p}[I_{2n+1},I]$ is generated by $g_{1}^{p^{n+1}}, g_{2}^{p^{n}}, g_{3}^{p^{n+1}}$ and thus $I_{2n+2} = [I,I]^{p^{n}}$. Hence, by induction, we have proved:

\begin{theorem}
  Let $I$ be the pro-$p$ Iwahori subgroup of $\SL_{2}(\Q_{p})$ and let $g_{1},g_{2},g_{3}$ be the ordered basis of $I$ from \eqref{eq:gis-SL2}. Then the lower $p$-series is given by
  \begin{equation*}
    P_{i}(I) =
    \begin{dcases*}
      I^{p^{n}} & if $i = 2n+1$, \\
      [I,I]^{p^{n-1}} & if $i = 2n$,
    \end{dcases*}
  \end{equation*}
  where $P_{2n+1}(I) = I^{p^{n}}$ is the subgroup generated by $g_{1}^{p^{n}},g_{2}^{p^{n}},g_{3}^{p^{n}}$ and $P_{2n}(I) = [I,I]^{p^{n-1}}$ is the subgroup generated by $g_{1}^{p^{n}},g_{2}^{p^{n-1}},g_{3}^{p^{n}}$.

  Thus
  \begin{equation*}
    [P_{i}(G) : P_{i+1}(G)] =
    \begin{dcases*}
      1 & if $i = 2n$, \\
      2 & if $i = 2n+1$.
    \end{dcases*}
  \end{equation*}
  In particular, $I$ is not uniformly powerful.
\end{theorem}

\section{\texorpdfstring{$I \subseteq \GL_{2}(\Z_{p})$}{I in GL2(Zp)}}%
\label{sec:Iwa-GL2}

In this section we will describe the continuous group cohomology of the pro-$p$ Iwahori subgroup $I$ of $\GL_{2}(\Q_{p})$.

When $I$ is the pro-$p$ Iwahori subgroup in $\GL_{2}(\Q_{p})$, we know by \Cref{sec:cohiwagps-intro} that we can take it to be of the form
\begin{equation*}
  I = \pmat{1+p\Z_{p} & \Z_{p} \\ p\Z_{p} & 1+p\Z_{p}} \subseteq \GL_{2}(\Z_{p}),
\end{equation*}
and, by \Cref{sec:cohiwagps-intro}, we have an ordered basis
\begin{equation}
  \label{eq:gis-GL2}
  \begin{gathered}
    g_{1} = \pmat{1 & 0 \\ p & 1}, \qquad g_{2} = \pmat{\exp(p) & 0 \\ 0 & \exp(-p)}, \\
    g_{3} = \pmat{\exp(p) & 0 \\ 0 & \exp(p)}, \qquad g_{4} = \pmat{1 & 1 \\ 0 & 1}.
  \end{gathered}
\end{equation}
Since we just renamed some elements and added an element of the center of $\GL_{2}(\Z_{p})$ when comparing to the ordered basis of $I \subseteq \SL_{2}(\Z_{p})$ from \Cref{sec:Iwa-SL2}, it is clear from \Cref{eq:xi_ij-SL2} that the only non-zero commutator in $\lie{g} = k \otimes \gr G$ is
\begin{equation*}
  [\xi_{1},\xi_{4}] = -\xi_{2},
\end{equation*}
where $\xi_{i} = 1 \otimes \sigma(g_{i})$ as usual.

\subsection{Describing the graded chain complex, \texorpdfstring{$\gr^{j}\bigl(\bigwedge^{n}\lie{g}\bigr)$}{grj(wedge-n g)}}%
\label{subsec:graded-complex-GL2}

Looking at \eqref{eq:Iwa-p-val-basis-SLn} (with $e=1$ and $h=2$) and the note about the $\GL_{n}$ case after \eqref{eq:Iwa-p-val-basis-SLn}, we see that
\begin{align*}
  \omega(g_{1}) &= \frac{1}{2}, & \omega(g_{2}) &= 1, \\
  \omega(g_{3}) &= 1 & \omega(g_{4}) &= \frac{1}{2}.
\end{align*}
Hence $\lie{g}^{1} = \lie{g}_{\frac{1}{2}} = \Span_{k}(\xi_{1},\xi_{4})$ and $\lie{g}^{2} = \lie{g}_{1} = \Span_{k}(\xi_{2},\xi_{3})$, cf.\ \Cref{rem:g-Z-grading}.

Now we are ready to describe the graded chain complex
\begin{equation*}
  \gr^{j}\Bigl( \bigwedge^{n}\lie{g} \Bigr) = \bigoplus_{j_{1} + \dotsb + j_{n} = j} \lie{g}^{j_{1}} \wedge \dotsb \wedge \lie{g}^{j_{n}}
\end{equation*}
and its bases.

\begin{description}
  \item[$n=0:$]
        \begin{equation*}
          \gr^{j}(k) =
          \begin{dcases}
            k & j=0, \\
            0 & \text{otherwise.}
          \end{dcases}
        \end{equation*}
        Bases:
        \begin{align*}
          k: \quad & 1.
        \end{align*}

   \item[$n=1:$]
        \begin{equation*}
          \gr^{j}(\lie{g}) =
          \begin{dcases}
            \lie{g}^{2} & j=2, \\
            \lie{g}^{1} & j=1, \\
            0          & \text{otherwise.}
          \end{dcases}
        \end{equation*}
        Bases:
        \begin{align*}
          \lie{g}^{1}: \quad & \xi_{1}, \xi_{4}, \\
          \lie{g}^{2}: \quad & \xi_{2}, \xi_{3}.
        \end{align*}

   \item[$n=2:$]
        \begin{equation*}
          \gr^{j}\Bigl( \bigwedge^{2}\lie{g} \Bigr) =
          \begin{dcases}
            \lie{g}^{2} \wedge \lie{g}^{2} & j=4, \\
            \lie{g}^{1} \wedge \lie{g}^{2} & j=3, \\
            \lie{g}^{1} \wedge \lie{g}^{1} & j=2, \\
            0                             & \text{otherwise.}
          \end{dcases}
        \end{equation*}
        Bases:
        \begin{align*}
          \lie{g}^{2} \wedge \lie{g}^{2}: \quad & \xi_{2} \wedge \xi_{3}, \\
          \lie{g}^{1} \wedge \lie{g}^{2}: \quad & \xi_{1} \wedge \xi_{2}, \xi_{1} \wedge \xi_{3}, \xi_{4} \wedge \xi_{2}, \xi_{4} \wedge \xi_{3}, \\
          \lie{g}^{1} \wedge \lie{g}^{1}: \quad & \xi_{1} \wedge \xi_{4}.
        \end{align*}

  \item[$n=3:$]
        \begin{equation*}
          \gr^{j}\Bigl( \bigwedge^{3}\lie{g} \Bigr) =
          \begin{dcases}
            \lie{g}^{1} \wedge \lie{g}^{2} \wedge \lie{g}^{2} & j=5, \\
            \lie{g}^{1} \wedge \lie{g}^{1} \wedge \lie{g}^{2} & j=4, \\ 0                                                & \text{otherwise.}
          \end{dcases}
        \end{equation*}
        Bases:
        \begin{align*}
          \lie{g}^{1} \wedge \lie{g}^{2} \wedge \lie{g}^{2}: \quad & \xi_{1} \wedge \xi_{2} \wedge \xi_{3}, \xi_{4} \wedge \xi_{2} \wedge \xi_{3}, \\
          \lie{g}^{1} \wedge \lie{g}^{1} \wedge \lie{g}^{2}: \quad & \xi_{1} \wedge \xi_{4} \wedge \xi_{2}, \xi_{1} \wedge \xi_{4} \wedge \xi_{3}.
        \end{align*}

  \item[$n=4:$]
        \begin{equation*}
          \gr^{j}\Bigl( \bigwedge^{4}\lie{g} \Bigr) =
          \begin{dcases}
            \lie{g}^{1} \wedge \lie{g}^{1} \wedge \lie{g}^{2} \wedge \lie{g}^{2} & j=6, \\ 0                                                                  & \text{otherwise.}
          \end{dcases}
        \end{equation*}
        Bases:
        \begin{align*}
          \lie{g}^{1} \wedge \lie{g}^{1} \wedge \lie{g}^{2} \wedge \lie{g}^{4}: \quad & \xi_{1} \wedge \xi_{4} \wedge \xi_{2} \wedge \xi_{3}.
        \end{align*}

   \item[$n\geq5:$]
        \begin{equation*}
          \gr^{j}\Bigl( \bigwedge^{n}\lie{g} \Bigr) = 0 \text{ for all } j.
        \end{equation*}
\end{description}

\begin{table}[ht]
  \centering
  \caption[Graded complex dimensions for the $I \subseteq \GL_{2}(\Z_{p})$ case]{Dimensions of $\gr^{j}\bigl( \bigwedge^{n} \lie{g} \bigr)$ for the $I \subseteq \GL_{2}(\Z_{p})$ case.}
  \label{tab:graded-dims-GL2}
  $\begin{NiceArray}{*{8}{c}}[hvlines]
    \diagbox{n}{j} & 0 & 1 & 2 & 3 & 4 & 5 & 6 \\
    0 & 1 \\
    1 & & 2 & 2 \\
    2 & & & 1 & 4 & 1 \\
    3 & & & & & 2 & 2 \\
    4 & & & & & & & 1
  \end{NiceArray}$
\end{table}

We collect the above information about the dimensions of the chain complex of $\lie{g}$ in \Cref{tab:graded-dims-GL2}, and note that we only need to consider non-zero (non-empty) entries of the table, when we calculate  $H^{s,t} = H^{s,n-s}$ (where $H^{s,t} = H^{s,t}(\lie{g},k)$). Also, recalling that
\begin{equation*}
  \Hom_{k}\Bigl( \bigwedge^{n}\lie{g}, k \Bigr) = \bigoplus_{s \in \Z} \Hom_{k}^{s}\Bigl( \bigwedge^{n}\lie{g}, k \Bigr),
\end{equation*}
we see that, with $j=-s$, we get the same table for dimensions of the graded hom-spaces in the cochain complex.

\subsection{Finding the graded Lie algebra cohomology, \texorpdfstring{$H^{s,t}(\lie{g},k)$}{H(s,t)(g,k)}}%
\label{subsec:graded-coh-GL2}

We will now go through all different graded chain complexes one by one, using that $\gr^{j}$ in the chain complex corresponds to $\gr^{s}$ with $s = -j$ in the cochain complex. We note that the graded chain complex corresponds to vertical downwards arrows in \Cref{tab:graded-dims-GL2}, while the cochain complex corresponds to vertical upwards arrows. And finally, we reiterate that $H^{n} = H^{n}(\lie{g},k)$ and $H^{s,t} = H^{s,t}(\lie{g},k)$ in the following.

In grade $0$ we have the chain complex
\[
  \begin{tikzcd}
    0 \ar[r] & k \ar[r] & 0,
  \end{tikzcd}
\]
which gives us the grade $0$ cochain complex
\[
  \begin{tikzcd}
    0 & \ar[l] \Hom_{k}^{0}(k,k) & \ar[l] 0.
  \end{tikzcd}
\]
So $H^{0} = H^{0,0}$ with $\dim H^{0,0} = 1$.

In grade $1$ we have the chain complex
\[
  \begin{tikzcd}
    0 \ar[r] & \lie{g}^{1} \ar[r] & 0,
  \end{tikzcd}
\]
which gives us the grade $-1$ cochain complex
\[
  \begin{tikzcd}
    0 & \ar[l] \Hom_{k}^{-1}(\lie{g},k) & \ar[l] 0.
  \end{tikzcd}
\]
So $\dim H^{-1,2} = 2$ by \Cref{tab:graded-dims-SL2}.

In grade $2$ we have the chain complex
\[
  \begin{tikzcd}[ampersand replacement=\&]
    0 \ar[r] \& \lie{g}^{1} \wedge \lie{g}^{1} \ar[r, "{\begin{pmatrix} 1 \\ 0 \end{pmatrix}}" {yshift=7pt}] \& \lie{g}^{2} \ar[r] \& 0,
  \end{tikzcd}
\]
since
\begin{align*}
  \lie{g}^{1} \wedge \lie{g}^{1} &\to \lie{g}^{2} \\
  \xi_{1} \wedge \xi_{4} &\mapsto -[\xi_{1},\xi_{4}] = \xi_{2}.
\end{align*}
This gives us the grade $-2$ cochain complex
\[
  \begin{tikzcd}[ampersand replacement=\&]
    0 \& \ar[l] \Hom_{k}^{-2}\bigl( \bigwedge^{2} \lie{g}, k \bigr) \& \ar[l, "{\begin{pmatrix} 1 & 0 \end{pmatrix}}"' {yshift=7pt}] \Hom_{k}^{-2}(\lie{g},k) \& \ar[l] 0.
  \end{tikzcd}
\]
So with \[ d = \pmat{1 & 0},\] and comparing with \Cref{tab:graded-dims-GL2},
\begin{align*}
  \dim H^{-2,3} &= \dim \kernel(d) = 1, \\
  \dim H^{-2,4} &= \dim \coker(d) = 0.
\end{align*}

In grade $3$ we have the chain complex
\[
  \begin{tikzcd}
    0 \ar[r] & \lie{g}^{1} \wedge \lie{g}^{2} \ar[r] & 0,
  \end{tikzcd}
\]
which gives us the grade $-3$ cochain complex
\[
  \begin{tikzcd}
    0 & \ar[l] \Hom_{k}^{-3}\bigl( \bigwedge^{2}\lie{g}, k \bigr) & \ar[l] 0.
  \end{tikzcd}
\]
So $\dim H^{-3,5} = 2$ by \Cref{tab:graded-dims-SL2}.

In grade $4$ we have the chain complex
\[
  \begin{tikzcd}[ampersand replacement=\&]
    0 \ar[r] \& \lie{g}^{1} \wedge \lie{g}^{1} \wedge \lie{g}^{2}  \ar[r,"{\begin{pmatrix} 0 & 1 \end{pmatrix}}" {yshift=7pt}] \& \lie{g}^{2} \wedge \lie{g}^{2} \ar[r] \& 0,
  \end{tikzcd}
\]
since
\begin{align*}
  \lie{g}^{1} \wedge \lie{g}^{1} \wedge \lie{g}^{2} &\to \lie{g}^{2} \wedge \lie{g}^{2} \\
  \xi_{1} \wedge \xi_{4} \wedge \xi_{2} &\mapsto -[\xi_{1},\xi_{4}] \wedge \xi_{2} + [\xi_{1},\xi_{2}] \wedge \xi_{4} - [\xi_{4},\xi_{2}] \wedge \xi_{1} = \xi_{2} \wedge \xi_{2} = 0, \\
  \xi_{1} \wedge \xi_{4} \wedge \xi_{3} &\mapsto -[\xi_{1},\xi_{4}] \wedge \xi_{3} + [\xi_{1},\xi_{3}] \wedge \xi_{4} - [\xi_{4},\xi_{3}] \wedge \xi_{1} = \xi_{2} \wedge \xi_{3}.
\end{align*}
This gives us the grade $-4$ cochain complex
\[
  \begin{tikzcd}[ampersand replacement=\&]
    0 \& \ar[l] \Hom_{k}^{-4}\bigl( \bigwedge^{3}\lie{g}, k \bigr) \& \ar[l,"{\begin{pmatrix} 0 \\ 1 \end{pmatrix}}"' {yshift=7pt}] \Hom_{k}^{-4}\bigl( \bigwedge^{2}\lie{g}, k \bigr) \& \ar[l] 0.
  \end{tikzcd}
\]
So with \[ d = \pmat{0 \\ 1},\] and comparing with \Cref{tab:graded-dims-GL2},
\begin{align*}
  \dim H^{-4,6} &= \dim \kernel(d) = 0, \\
  \dim H^{-4,7} &= \dim \coker(d) = 1.
\end{align*}

In grade $5$ we have the chain complex
\[
  \begin{tikzcd}
    0 \ar[r] & \lie{g}^{1} \wedge \lie{g}^{2} \wedge \lie{g}^{2} \ar[r] & 0,
  \end{tikzcd}
\]
which gives us the grade $-5$ cochain complex
\[
  \begin{tikzcd}
    0 & \ar[l] \Hom_{k}^{-5}\bigl( \bigwedge^{3}\lie{g}, k \bigr) & \ar[l] 0.
  \end{tikzcd}
\]
So $\dim H^{-5,8} = 2$ by \Cref{tab:graded-dims-SL2}.

In grade $6$ we have the chain complex
\[
  \begin{tikzcd}
    0 \ar[r] & \lie{g}^{1} \wedge \lie{g}^{1} \wedge \lie{g}^{2} \wedge \lie{g}^{2} \ar[r] & 0,
  \end{tikzcd}
\]
which gives us the grade $-6$ cochain complex
\[
  \begin{tikzcd}
    0 & \ar[l] \Hom_{k}^{-6}\bigl( \bigwedge^{4}\lie{g}, k \bigr) & \ar[l] 0.
  \end{tikzcd}
\]
So $\dim H^{-6,10} = 1$ by \Cref{tab:graded-dims-SL2}.

\begin{table}[ht]
  \centering
  \caption[Graded cohomology dimensions for the $I \subseteq \GL_{2}(\Z_{p})$ case]{Dimensions of $E_{1}^{s,t} = H^{s,t}(\lie{g},k)$ for the $I \subseteq \GL_{2}(\Z_{p})$ case.}
  \label{tab:graded-coh-dims-GL2}
  \renewcommand{\arraystretch}{1.5}
  $\begin{NiceArray}{*{8}{c}}[hvlines, columns-width=auto]
    \diagbox{t}{s} & 0 & -1 & -2 & -3 & -4 & -5 & -6 \\
    0 & 1 \\
    1 & \\
    2 & & 2 \\
    3 & & & 1 \\
    4 & \\
    5 & & & & 4 \\
    6 & \\
    7 & & & & & 1 \\
    8 & & & & & & 2 \\
    9 & \\
    10 & & & & & & & 1
  \end{NiceArray}$
  \renewcommand{\arraystretch}{1}
\end{table}

Altogether, we see that
\begin{equation}
  \label{eq:Hn-to-Hst-GL2}
  \begin{aligned}
    H^{0} &= H^{0,0}, \\
    H^{1} &= H^{-1,2} \oplus H^{-2,3}, \\
    H^{2} &= H^{-3,5}, \\
    H^{3} &= H^{-4,7} \oplus H^{-5,8}, \\
    H^{4} &= H^{-6,10},
  \end{aligned}
\end{equation}
with dimension as described in \Cref{tab:graded-coh-dims-GL2}.

\subsection{Describing the group cohomology, \texorpdfstring{$H^{n}(I,k)$}{Hn(I,k)}}%
\label{subsec:group-coh-GL2}

We note that all differentials $d_{r}^{s,t} \colon E_{r}^{s,t} \to E_{r}^{s+r,t+1-r}$ in \Cref{tab:graded-coh-dims-GL2} have bidegree $(r,1-r)$, i.e., they are all below the $(r,-r)$ arrow going $r$ to the left and $r$ up in the table, where $r \geq 1$. Looking at \Cref{tab:graded-coh-dims-GL2}, this clearly means that all differentials for $r \geq 1$ are trivial, and thus the spectral sequence collapses on the first page. Hence $H^{s,t}(\lie{g},k) = E_{1}^{s,t} \iso E_{\infty}^{s,t} = \gr^{s} H^{s+t}(I,k)$, and by \eqref{eq:Hn-to-Hst-GL2} and \Cref{tab:graded-coh-dims-GL2} we get that
\begin{equation}
  \label{eq:dim-HnI-GL2}
  \dim H^{n}(I,k) =
  \begin{dcases}
    1 & n=0, \\
    3 & n=1, \\
    4 & n=2, \\
    3 & n=3, \\
    1 & n=4.
  \end{dcases}
\end{equation}

Recalling that the spectral sequence is multiplicative, we also note, by \Cref{tab:graded-coh-dims-GL2}, that $H^{s,t} \cup H^{s',t'} \subseteq H^{s+s',t+t'}$ implies that the cup products
\begin{equation*}
  \gr^{s} H^{n}(I,k) \otimes \gr^{s'} H^{n'}(I,k) \to \gr^{s+s'} H^{n+n'}(I,k)
\end{equation*}
are trivial except for the cases with $H^{0}$ and
\begin{equation}\label{eq:non-triv-cups-GL2}
  \begin{aligned}
    H^{-1,2} \cup H^{-2,3} &\subseteq H^{-3,5}, \\
    H^{-1,2} \cup H^{-3,5} &\subseteq H^{-4,7}, \\
    H^{-1,2} \cup H^{-5,8} &\subseteq H^{-6,10}, \\
    H^{-2,3} \cup H^{-3,5} &\subseteq H^{-5,8}, \\
    H^{-2,3} \cup H^{-4,7} &\subseteq H^{-6,10}, \\
    H^{-3,5} \cup H^{-3,5} &\subseteq H^{-6,10},
  \end{aligned}
\end{equation}
and the reverse of the above (which we can find using graded commutativity).

Next we want to describe these cup products.

Let $e_{i_{1},\dotsc,i_{m}} = (\xi_{i_{1}} \wedge \dotsb \wedge \xi_{i_{m}})^{*}$\nomiwa[ei1i2]{$e_{i_{1},\dotsc,i_{m}}$}{${} = (\xi_{i_{1}} \wedge \dotsb \wedge \xi_{i_{m}})^{*}$, the element of the dual basis of $\Hom_{k}\bigl( \bigwedge^{m}\lie{g},k \bigr)$ corresponding to $\xi_{i_{1}} \wedge \dotsb \wedge \xi_{i_{m}}$ in the basis of $\bigwedge^{m}\lie{g}$} be the element of the dual basis of $\Hom_{k}\bigl( \bigwedge^{m}\lie{g},k \bigr)$ corresponding to $\xi_{i_{1}} \wedge \dotsb \wedge \xi_{i_{m}}$ in the basis of $\bigwedge^{m}\lie{g}$. Looking at the cochain complexes and descriptions of the maps above together with the known bases of the graded chain complexes, we get the following precise descriptions of the of the graded cohomology spaces $H^{s,t} = H^{s,t}(\lie{g},k)$:
\begin{equation}\label{eq:Hst-spaces-GL2}
  \begin{aligned}
    H^{-1,2} &= k[e_{1},e_{4}], \\
    H^{-2,3} &= \kernel \pmat{1 & 0} = k[e_{3}], \\
    H^{-3,5} &= k[e_{1,2},e_{1,3},e_{4,2},e_{4,3}], \\
    H^{-4,7} &= \frac{k[e_{1,4,2},e_{1,4,3}]}{\image \pmat{0 \\ 1}} = k[e_{1,4,2}], \\
    H^{-5,8} &= k[e_{1,2,3},e_{4,2,3}], \\
    H^{-6,10} &= k[e_{1,4,2,3}].
  \end{aligned}
\end{equation}

For $f \in \Hom_{k}\bigl( \bigwedge^{p}\lie{g}, k \bigr)$ and $g \in \Hom_{k}\bigl( \bigwedge^{q}\lie{g}, k \bigr)$, we recall from \eqref{eq:cup-prod-def} that the cup product in cohomology is induced by: $f \cup g \in \Hom_{k}\bigl( \bigwedge^{p+q}\lie{g}, k \bigr)$ defined by
\begin{equation*}
  (f \cup g)(x_{1} \wedge \dotsb \wedge x_{p+q})  = \sum_{\mathclap{\substack{ \sigma \in S_{p+q} \\ \sigma(1) < \dotsb < \sigma(p) \\ \sigma(p+1) < \dotsb < \sigma(p+q) }}} \sign(\sigma) f(x_{\sigma(1)} \wedge \dotsb \wedge x_{\sigma(p)}) g(x_{\sigma(p+1)} \wedge \dotsb \wedge x_{\sigma(p+q)}).
\end{equation*}

We will now find all the cup products in \eqref{eq:non-triv-cups-GL2} by working with our given bases and the \eqref{eq:cup-prod-def}.
% In the following it will be important to be aware that we use $\sigma \in S_{n}$ to describe a \enquote{positional} permutation when writing it as an action, and not a permutation of the numbers $\set{1,\dotsc,n}$. We will write e.g. $\sigma \act (1,4,2) = (4,1,2)$ for $\sigma=(1,2)$, when using $\sigma$ in this way, while $\sigma(1)=2$, $\sigma(2) = 1$, $\sigma(3) = 1$ will be written as usual. \dknote{Be more clear about the notation.}

We will start by finding
\[
  \begin{tikzcd}
    H^{-1,2} \otimes H^{-2,3} \ar[r,"\cup"] & H^{-3,5}.
  \end{tikzcd}
\]
Looking at \eqref{eq:Hst-spaces-GL2}, we need to describe the maps $e_{1} \cup e_{3}$ and $e_{4} \cup e_{3}$ on the basis of $\gr^{3} \bigwedge^{2}\lie{g}$, i.e., on $\basis = (\xi_{1}\wedge\xi_{2}, \xi_{1} \wedge \xi_{3},\xi_{4} \wedge \xi_{2},\xi_{4} \wedge \xi_{3})$. In the case of $e_{1} \cup e_{3}$, \eqref{eq:cup-prod-def} simplifies to
\begin{equation*}
  (e_{1} \cup e_{3})(x_{1} \wedge x_{2}) = \sum_{\sigma \in S_{2}} \sign(\sigma) e_{1}(x_{\sigma(1)}) e_{3}(x_{\sigma(2)}),
\end{equation*}
which is zero on all of $\basis$ except $\xi_{1} \wedge \xi_{3}$ with $\sigma=(1)$, where we get (using that $\sign\bigl( (1) \bigr) = 1$)
\begin{equation*}
  (e_{1} \cup e_{3})(\xi_{1} \wedge \xi_{3}) = 1.
\end{equation*}
Hence $e_{1} \cup e_{3} = e_{1,3}$. In the case of $e_{4} \cup e_{3}$, \eqref{eq:cup-prod-def} simplifies to
\begin{equation*}
  (e_{4} \cup e_{3})(x_{1} \wedge x_{2}) = \sum_{\sigma \in S_{2}} \sign(\sigma) e_{4}(x_{\sigma(1)}) e_{3}(x_{\sigma(2)}),
\end{equation*}
which is zero on all of $\basis$ except $\xi_{4} \wedge \xi_{3}$ with $\sigma=(1)$, where we get (using that $\sign\bigl( (1) \bigr) = 1$)
\begin{equation*}
  (e_{4} \cup e_{3})(\xi_{4} \wedge \xi_{3}) = 1.
\end{equation*}
Hence $e_{4} \cup e_{3} = e_{4,3}$. Looking at \eqref{eq:Hst-spaces-GL2}, we see that $e_{1,3}$ and $e_{4,3}$ are the second and forth basis elements of $H^{-3,5}$, so the above calculation caries over to the above cup product in cohomology.

We will now describe
\[
  \begin{tikzcd}
    H^{-1,2} \otimes H^{-3,5} \ar[r,"\cup"] & H^{-4,7}.
  \end{tikzcd}
\]
Looking at \eqref{eq:Hst-spaces-GL2}, we need to describe the maps
\begin{align*}
  &e_{1} \cup e_{1,2}, e_{1} \cup e_{1,3}, e_{1} \cup e_{4,2}, e_{1} \cup e_{4,3}, \\
  &e_{4} \cup e_{1,2}, e_{4} \cup e_{1,3}, e_{4} \cup e_{4,2}, e_{4} \cup e_{4,3}
\end{align*}
on the basis of $\gr^{4} \bigwedge^{3}\lie{g}$, i.e., on $\basis = (\xi_{1} \wedge \xi_{4} \wedge \xi_{2}, \xi_{1} \wedge \xi_{4} \wedge \xi_{3})$. In any case with repeat numbers, it is clear from \eqref{eq:cup-prod-def} (and the fact that there are no repeats in $\basis$) that the cup product will be zero, so we only need to consider $e_{1} \cup e_{4,2}$, $e_{1} \cup e_{4,3}$, $e_{4} \cup e_{1,2}$ and $e_{4} \cup e_{1,3}$. In all cases of $e_{i} \cup e_{j,k}$, \eqref{eq:cup-prod-def} simplifies to
\begin{equation*}
  (e_{i} \cup e_{j,k})(x_{1} \wedge x_{2} \wedge x_{3}) = \sum_{\substack{\sigma \in S_{3} \\ \sigma(2) < \sigma(3)}} \sign(\sigma) e_{i}(x_{\sigma(1)}) e_{j,k}(x_{\sigma(2)} \wedge x_{\sigma(3)}),
\end{equation*}
i.e., the sum is over $\sigma \in \set{(1), (1,2), (1,3,2)}$. When $(i,j,k) = (1,4,2)$ or $(i,j,k) = (1,4,3)$, this sum is zero on all of $\basis$ except $\xi_{i} \wedge \xi_{j} \wedge \xi_{k}$ with $\sigma=(1)$, since $\sigma$ needs to fix $1$ and $\basis$ uses the same ordering. Here we get (using that $\sign\bigl((1)\bigr) = 1$)
\begin{equation*}
  (e_{i} \cup e_{j,k})(\xi_{i} \wedge \xi_{j} \wedge \xi_{k}) = 1.
\end{equation*}
Hence $e_{1} \cup e_{4,2} = e_{1,4,2}$ and $e_{1} \cup e_{4,3} = e_{1,4,3}$. When $(i,j,k) = (4,1,2)$ or $(i,j,k) = (4,1,3)$, the sum is zero on all of $\basis$ except $\xi_{j} \wedge \xi_{i} \wedge \xi_{k}$ with $\sigma=(1,2)$, since the order of the first and second elements of $(i,j,k)$ are swapped compared to in $\basis$. Here we get (using that $\sign\bigl((1,2)\bigr) = -1$)
\begin{equation*}
  (e_{i} \cup e_{j,k})(\xi_{i} \wedge \xi_{j} \wedge \xi_{k}) = -1.
\end{equation*}
Hence $e_{4} \cup e_{1,2} = -e_{1,4,2}$ and $e_{4} \cup e_{1,3} = -e_{1,4,3}$. Looking at \eqref{eq:Hst-spaces-GL2}, we see that $e_{1,4,3}$ reduces to zero in $H^{-4,7}$, while $e_{1,4,2}$ is part of the basis. So in the cup product on the cohomology, the only nontrivial products are $e_{1} \cup e_{4,2} = e_{1,4,2}$ and $e_{4} \cup e_{1,2} = -e_{1,4,2}$.

At this point, it should be clear how to skip some of the details, so we will proceed with less justification than above.

Now consider
\[
  \begin{tikzcd}
    H^{-1,2} \otimes H^{-5,8} \ar[r,"\cup"] & H^{-6,10}.
  \end{tikzcd}
\]
Looking at \eqref{eq:Hst-spaces-GL2}, the only nontrivial maps we need to describe are $e_{1} \cup e_{4,2,3}$ and $e_{4} \cup e_{1,2,3}$ on the basis of $\gr^{6} \bigwedge^{4}\lie{g}$, i.e., on $\basis = (\xi_{1} \wedge \xi_{4} \wedge \xi_{2} \wedge \xi_{3})$. For $e_{1} \cup e_{4,2,3}$ to be non-zero, we need $\sigma \in S_{4}$ that fixes $1$ and satisfies $\sigma(2) < \sigma(3) < \sigma(4)$, which is only true for $\sigma = (1)$. For $e_{4} \cup e_{1,2,3}$ to be non-zero, we need $\sigma \in S_{4}$ that swaps $1$ and $2$ and satisfies $\sigma(2) < \sigma(3) < \sigma(4)$, which is only true for $\sigma = (1,2)$. Since $\sign\bigl((1)\bigr) = 1$ and $\sign\bigl((1,2)\bigr) = -1$, we get that $e_{1} \cup e_{4,2,3} = e_{1,4,2,3}$ and $e_{4} \cup e_{1,2,3} = -e_{1,4,2,3}$. Looking at \eqref{eq:Hst-spaces-GL2}, we see that $e_{1,4,2,3}$ it the basis elements of $H^{-6,10}$, so the above calculation caries over to the above cup products in cohomology.

Continue with
\[
  \begin{tikzcd}
    H^{-2,3} \otimes H^{-3,5} \ar[r,"\cup"] & H^{-5,8}.
  \end{tikzcd}
\]
Looking at \eqref{eq:Hst-spaces-GL2}, the only nontrivial maps we need to describe are $e_{3} \cup e_{1,2}$ and $e_{3} \cup e_{4,2}$ on the basis of $\gr^{5} \bigwedge^{3}\lie{g}$, i.e., on $\basis = (\xi_{1} \wedge \xi_{2} \wedge \xi_{3}, \xi_{4} \wedge \xi_{2} \wedge \xi_{3})$.  For $e_{3} \cup e_{1,2}$ or $e_{3} \cup e_{4,2}$ to be non-zero, we need $\sigma \in S_{3}$ that satisfies $\sigma(1) = 3$ (putting $\xi_{3}$ first) and $\sigma(2) < \sigma(3)$, which is only true for $\sigma = (1,3,2)$. Since $\sign\bigl(((1,3,2)\bigr) = 1$, we get that $e_{3} \cup e_{1,2} = e_{1,2,3}$ and $e_{3} \cup e_{4,2} = e_{4,2,3}$. Looking at \eqref{eq:Hst-spaces-GL2}, we see that $e_{1,2,3}$ and $e_{4,2,3}$ are the basis elements of $H^{-5,8}$, so the above calculation caries over to the above cup products in cohomology.

Continue with
\[
  \begin{tikzcd}
    H^{-2,3} \otimes H^{-4,7} \ar[r,"\cup"] & H^{-6,10}.
  \end{tikzcd}
\]
Looking at \eqref{eq:Hst-spaces-GL2}, the only map we need to describe is $e_{3} \cup e_{1,4,2}$ on the basis of $\gr^{6} \bigwedge^{4}\lie{g}$, i.e., on $\basis = (\xi_{1} \wedge \xi_{4} \wedge \xi_{2} \wedge \xi_{3})$.  For $e_{3} \cup e_{1,4,2}$ to be non-zero, we need $\sigma \in S_{4}$ that satisfies $\sigma(1) = 4$ (putting $\xi_{3}$ first) and $\sigma(2) < \sigma(3) < \sigma(4)$, which is only true for $\sigma = (1,4,3,2)$. Since $\sign\bigl(((1,4,3,2)\bigr) = -1$, we get that $e_{3} \cup e_{1,4,2} = -e_{1,4,2,3}$. Looking at \eqref{eq:Hst-spaces-GL2}, we see that $e_{1,4,2,3}$ it the basis elements of $H^{-6,10}$, so the above calculation caries over to the above cup product in cohomology.

Finally, consider
\[
  \begin{tikzcd}
    H^{-3,5} \otimes H^{-3,5} \ar[r,"\cup"] & H^{-6,10}.
  \end{tikzcd}
\]
Looking at \eqref{eq:Hst-spaces-GL2}, the only nontrivial maps we need to describe are $e_{1,2} \cup e_{4,3}$ and $e_{1,3} \cup e_{4,2}$ (getting the rest by graded-commutativity) on the basis of $\gr^{6} \bigwedge^{4}\lie{g}$, i.e., on $\basis = (\xi_{1} \wedge \xi_{4} \wedge \xi_{2} \wedge \xi_{3})$.  For $e_{1,2} \cup e_{4,3}$ to be non-zero, we need $\sigma \in S_{4}$ that satisfies
\begin{enumerate}[$\bullet$]
  \item $\set{\sigma(1),\sigma(2)} = \set{1,3}$ (putting $\xi_{1}$ and $\xi_{2}$ in $e_{1,2}$),
  \item $\set{\sigma(3),\sigma(4)} = \set{2,4}$ (putting $\xi_{4}$ and $\xi_{3}$ in $e_{4,3}$),
  \item $\sigma(1) < \sigma(2)$ and $\sigma(3) < \sigma(4)$,
\end{enumerate}
which is only true for $\sigma = (2,3)$. Since $\sign\bigl((2,3)\bigr) = -1$, we get that $e_{1,2} \cup e_{4,3} = -e_{1,4,2,3}$. For $e_{1,3} \cup e_{4,2}$ to be non-zero, we need $\sigma \in S_{4}$ that satisfies
\begin{enumerate}[$\bullet$]
  \item $\set{\sigma(1),\sigma(2)} = \set{1,4}$ (putting $\xi_{1}$ and $\xi_{3}$ in $e_{1,3}$),
  \item $\set{\sigma(3),\sigma(4)} = \set{2,3}$ (putting $\xi_{4}$ and $\xi_{2}$ in $e_{4,2}$),
  \item $\sigma(1) < \sigma(2)$ and $\sigma(3) < \sigma(4)$,
\end{enumerate}
which is only true for $\sigma = (2,4,3)$. Since $\sign\bigl( (2,4,3) \bigr) = 1$, we get that $e_{1,3} \cup e_{4,2} = e_{1,4,2,3}$. Looking at \eqref{eq:Hst-spaces-GL2}, we see that $e_{1,4,2,3}$ it the basis elements of $H^{-6,10}$, so the above calculation caries over to the above cup products in cohomology. Also, since $H^{-3,5} = H^{2}$, we get, by graded commutativity of the cup product, that $e_{4,2} \cup e_{1,3} = (-1)^{2\times2} e_{1,3} \cup e_{4,2} = e_{1,4,2,3}$ and $e_{4,3} \cup e_{1,2} = (-1)^{2\times2} e_{1,2} \cup e_{4,3} = -e_{1,4,2,3}$.

In conclusion, all the non-trivial and non-zero cup products (up to graded commutativity) are:
\begin{equation}
  \label{eq:cup-products-GL2}
  \begin{aligned}
    e_{1} \cup e_{3} &= e_{1,3}, \\
    e_{4} \cup e_{3} &= e_{4,3}, \\
    e_{1} \cup e_{4,2} &= e_{1,4,2}, \\
    e_{4} \cup e_{1,2} &= -e_{1,4,2}, \\
    e_{1} \cup e_{4,2,3} &= e_{1,4,2,3}, \\
    e_{4} \cup e_{1,2,3} &= -e_{1,4,2,3}, \\
    e_{3} \cup e_{1,2} &= e_{1,2,3}, \\
    e_{3} \cup e_{4,2} &= e_{4,2,3}, \\
    e_{3} \cup e_{1,4,2} &= -e_{1,4,2,3}, \\
    e_{1,3} \cup e_{4,2} &= e_{1,4,2,3}, \\
    e_{1,2} \cup e_{4,3} &= -e_{1,4,2,3}.
  \end{aligned}
\end{equation}

Now, since the spectral sequence collapses on the first page, all of the above work on the cup product of the Lie algebra cohomology transfers to the cup product on $H^{*}(I,k)$ as described above.

\begin{remark}\label{rem:GL2-SL2-coh-iso}
  Let $\F_{p}[\varepsilon]$ denote the dual numbers ($\varepsilon^{2} = 0$), where $\varepsilon$ sits in grade $-2$. The above cup product calculations show that $H^{*}(I_{\SL_{2}(\Q_{p})}, \F_{p}) \otimes_{\F_{p}} \F_{p}[\varepsilon] \iso H^{*}(I_{\GL_{2}(\Q_{p})},\F_{p})$ as algebras, where $I_{G}$ is the pro-$p$ Iwahori subgroup of $G$. To see this, note that the both algebras are $12$ dimensional, and on $H^{*}(I_{\SL_{2}(\Q_{p})}, \F_{p})$, and we know from \eqref{eq:Hst-spaces-SL2} that
  \begin{align*}
    H^{0} &= \F_{p}, \\
    H^{1} &= H^{-1,2} = \F_{p}[e_{1},e_{3}] = \F_{p}[x_{1},x_{2}], \\
    H^{2} &= H^{-3,5} = \F_{p}[e_{1,2},e_{3,2}] = \F_{p}[y_{1},y_{2}], \\
    H^{3} &= H^{-4,7} = \F_{p}[e_{1,3,2}] = \F_{p}[z],
  \end{align*}
  where we write $x_{1}=e_{1}, x_{2}=e_{3}, y_{1} = e_{1,2}, y_{2} = e_{3,2}, z = e_{1,3,2}$, and from \eqref{eq:cup-products-SL2} that
  \begin{align*}
    x_{1} \cup y_{2} &= z, & x_{2} \cup y_{1} &= -z, \\
    y_{2} \cup x_{1} &= (-1)^{1\times2}x_{1} \cup y_{2} = z, & y_{1} \cup x_{2} = (-1)^{1\times2}x_{2} \cup y_{1} &= -z.
  \end{align*}
  are the only non-trivial and non-zero cup products. Now $\F_{p}[\varepsilon] \iso \F_{p} \oplus \F_{p}\varepsilon$ and
  \begin{align*}
    &H^{*}(I_{\SL_{2}(\Q_{p})},\F_{p}) \otimes_{\F_{p}} \F_{p}[\varepsilon] \\
    &= \bigl( \F_{p} \oplus \F_{p}[x_{1},x_{2}] \oplus \F_{p}[y_{1},y_{2}] \oplus \F_{p}[z] \bigr) \otimes_{\F_{p}} \F_{p}[\varepsilon] \\
    &\iso \F_{p}[\varepsilon] \oplus \F_{p}[\varepsilon][x_{1},x_{2}] \oplus \F_{p}[\varepsilon][y_{1},y_{2}] \oplus \F_{p}[\varepsilon][z] \\
    &\iso \underbrace{\F_{p}}_{H^{0,0}} \oplus \underbrace{\F_{p}\varepsilon}_{H^{-1,2}} \oplus \underbrace{\F_{p}x_{1} \oplus \F_{p}x_{2}}_{H^{-1,2}} \oplus \underbrace{\F_{p}x_{1}\varepsilon \oplus \F_{p}x_{2}\varepsilon \oplus \F_{p}y_{1} \oplus \F_{p}y_{2}}_{H^{-3,5}} \\*
    &\phantom{{}={}} \oplus \underbrace{\F_{p}y_{1} \oplus \F_{p}y_{2}}_{H^{-5,8}} \oplus \underbrace{\F_{p}y_{1}\varepsilon \oplus \F_{p}y_{2}\varepsilon}_{H^{-5,8}} \oplus \underbrace{\F_{p}z}_{H^{-4,7}} \oplus \underbrace{\F_{p}z\varepsilon}_{H^{-6,10}},
  \end{align*}
  and the map
  \begin{align*}
    H^{*}(I_{\SL_{2}(\Q_{p})},\F_{p}) \otimes_{\F_{p}} \F_{p}[\varepsilon] &\to H^{*}(I_{\GL_{2}(\Q_{p})},\F_{p}) \\
    1 &\mapsto 1, \\
    \varepsilon &\mapsto e_{3}, \\
    x_{1} &\mapsto e_{1}, \\
    x_{2} &\mapsto e_{4}, \\
    x_{1}\varepsilon &\mapsto e_{1,3}, \\
    x_{1}\varepsilon &\mapsto e_{4,3}, \\
    y_{1} &\mapsto e_{1,2}, \\
    y_{2} &\mapsto e_{4,2}, \\
    y_{1}\varepsilon &\mapsto e_{1,2,3}, \\
    y_{2}\varepsilon &\mapsto e_{4,2,3}, \\
    z &\mapsto e_{1,4,2}, \\
    z\varepsilon &\mapsto e_{1,4,2,3},
  \end{align*}
  is an isomorphism of algebras (cf.\ \eqref{eq:Hst-spaces-GL2}) since the above and \eqref{eq:cup-products-GL2} gives us (writing $\times$ for the product in the algebra on the left)
  \begin{align*}
    x_{1} \times \varepsilon &= x_{1}\varepsilon & e_{1} \cup e_{3} &= e_{1,3}, \\
    x_{2} \times \varepsilon &= x_{2}\varepsilon & e_{4} \cup e_{3} &= e_{4,3}, \\
    x_{1} \times y_{1} &= z & e_{1} \cup e_{4,2} &= e_{1,4,2}, \\
    x_{2} \times y_{1} &= -z & e_{4} \cup e_{1,2} &= -e_{1,4,2}, \\
    x_{1} \times y_{1}\varepsilon &= z\varepsilon & e_{1} \cup e_{4,2,3} &= e_{1,4,2,3}, \\
    x_{2} \times y_{1}\varepsilon &= -z\varepsilon & e_{4} \cup e_{1,2,3} &= -e_{1,4,2,3}, \\
    \varepsilon \times y_{1} = (-1)^{1\times2} y_{1} \times \varepsilon &= y_{1}\varepsilon & e_{3} \cup e_{1,2} &= e_{1,2,3}, \\
    \varepsilon \times y_{2} = (-1)^{1\times2} y_{2} \times \varepsilon &= y_{2}\varepsilon & e_{3} \cup e_{4,2} &= e_{4,2,3}, \\
    \varepsilon \times z = (-1)^{1\times3} z \times \varepsilon &= -z\varepsilon & e_{3} \cup e_{1,4,2} &= -e_{1,4,2,3}, \\
    y_{1} \times x_{2}\varepsilon &= -z\varepsilon & e_{1,2} \cup e_{4,3} &= -e_{1,4,2,3}, \\
    y_{2} \times x_{1}\varepsilon &= z\varepsilon & e_{4,2} \cup e_{1,3} &= e_{1,4,2}.
  \end{align*}
\end{remark}

\section{\texorpdfstring{$I \subseteq \SL_{3}(\Z_{p})$}{I in SL3(Zp)}}%
\label{sec:Iwa-SL3}

In this section we will describe the continuous group cohomology of the pro-$p$ Iwahori subgroup $I$ of $\SL_{3}(\Q_{p})$.

When $I$ is the pro-$p$ Iwahori subgroup in $\SL_{3}(\Q_{p})$, we know by \Cref{sec:cohiwagps-intro} that we can take it to be of the form
\begin{equation*}
  I = \pmat{1+p\Z_{p} & \Z_{p} & \Z_{p} \\ p\Z_{p} & 1+p\Z_{p} & \Z_{p} \\ p\Z_{p} & p\Z_{p} & 1+p\Z_{p}}^{\!\!\det = 1} \subseteq \SL_{3}(\Z_{p}),
\end{equation*}
and, by \Cref{sec:cohiwagps-intro}, we have an ordered basis
\begin{equation}
  \label{eq:gis-SL3}
  \begin{gathered}
    g_{1} = \pmat{ 1 \\ & 1 \\ p && 1 }, \quad g_{2} = \pmat{ 1 \\ p & 1 \\ && 1 }, \quad g_{3} = \pmat{ 1 \\ & 1 \\ & p & 1 }, \\
    g_{4} = \pmat{ \exp(p) \\ & \exp(-p) \\ && 1 }, \quad g_{5} = \pmat{ 1 \\ & \exp(p) \\ && \exp(-p) }, \\
    g_{6} = \pmat{ 1 \\ & 1 & 1 \\ && 1 }, \quad g_{7} = \pmat{ 1 & 1 \\ & 1 \\ && 1 }, \quad g_{8} = \pmat{ 1 && 1 \\ & 1 \\ && 1 }.
  \end{gathered}
\end{equation}
Here we write any zeros as blank space in matrices, to make the notation easier to read for the bigger matrices.

\subsection{Finding the commutators \texorpdfstring{$[\xi_{i},\xi_{j}]$}{[xi-i,xi-j]}}%
\label{subsec:non-id-xi_ij-SL3}

Now
\begin{equation*}
    g_{1}^{x_{1}}g_{2}^{x_{2}}g_{3}^{x_{3}}g_{4}^{x_{4}}g_{5}^{x_{5}}g_{6}^{x_{6}}g_{7}^{x_{7}}g_{8}^{x_{8}} = \pmat{ a_{11} & a_{12} & a_{13} \\ a_{21} & a_{22} & a_{23} \\ a_{31} & a_{32} & a_{33}},
\end{equation*}
where
\begin{equation}
  \label{eq:gixi-SL3}
  \begin{aligned}
    a_{11} &= \exp(px_{4}), \\
    a_{12} &= x_{7}\exp(px_{4}), \\
    a_{13} &= x_{8}\exp(px_{4}), \\
    a_{21} &= px_{2}\exp(px_{4}), \\
    a_{22} &= px_{2}x_{7}\exp(px_{4}) + \exp\bigl( p(x_{5}-x_{4}) \bigr), \\
    a_{23} &= px_{2}x_{8}\exp(px_{4}) + x_{6}\exp\bigl( p(x_{5}-x_{4}) \bigr), \\
    a_{31} &= px_{1}\exp(px_{4}), \\
    a_{32} &= px_{1}x_{7}\exp(px_{4}) + px_{3}\exp\bigl( p(x_{5}-x_{4}) \bigr), \\
    a_{33} &= px_{1}x_{8}\exp(px_{4}) + px_{3}x_{6}\exp\bigl( p(x_{5}-x_{4}) \bigr) + \exp(-px_{5}).
  \end{aligned}
\end{equation}

Writing $g_{ij} = [g_{i},g_{j}]$ and $\xi_{ij} = [\xi_{i},\xi_{j}]$, we are now ready to find $x_{1},\dotsc,x_{8}$ such that $g_{ij} = g_{1}^{x_{1}} \dotsb g_{8}^{x_{8}}$ for different $i<j$. (In the following we use that $\frac{1}{p-1} = 1 + p + p^{2} + \dotsb$ and $\log(1-p) = -p - \frac{p^{2}}{2} - \frac{p^{3}}{3} - \dotsb$.) Also, except in the first case, we will note that $x_{k} \in p\Z_{p}$ implies that the coefficient on $\xi_{k}$ in $\xi_{ij}$ is zero.

We now list all non-identity commutators $g_{ij} = [g_{i},g_{j}]$ and find $\xi_{ij} = [\xi_{i},\xi_{j}]$ based on these. (For $g_{ij} = 1_{3}$ it is clear that $x_{1} = \cdots = x_{8} = 0$, and thus $\xi_{ij} = 0$.)

\begin{description}
  \item[$g_{14} = \pmat{1 \\ & 1 \\ p\bigl( 1-\exp(-p) \bigr) && 1}$:] Comparing $g_{14}$ with \eqref{eq:gixi-SL3}, we see that $x_{2} = x_{4} = x_{7} = x_{8} = 0$, and thus also $x_{3} = x_{5} = x_{6} = 0$. This leaves $a_{31} = px_{1} = p\bigl( 1-\exp(-p) \bigr) = p^{2} + O(p^{3})$, which implies that $x_{1} = p + O(p^{2})$. Hence $\sigma(g_{14}) = \pi \act \sigma(g_{1})$, which implies that $\xi_{14} = 0$.

  \item[$g_{15} = \pmat{1 \\ & 1 \\ p\bigl( 1-\exp(-p) \bigr) && 1}$:] Since $g_{15} = g_{14}$, the above calculation shows that $\xi_{15} = 0$.

  \item[$g_{16} = \pmat{1 \\ -p & 1 \\ && 1}$:] Comparing $g_{16}$ with \eqref{eq:gixi-SL3}, we see that $x_{1} = x_{4} = x_{7} = x_{8} = 0$, and thus also $x_{3} = x_{5} = x_{6} = 0$. This leaves $a_{21} = px_{2} = -p$, which implies that $x_{2} = -1$. Hence $\sigma(g_{16}) = -\sigma(g_{2})$, which implies that $\xi_{16} = -\xi_{2}$.

  \item[$g_{17} = \pmat{1 \\ & 1 \\ & p & 1}$:] Comparing $g_{17}$ with \eqref{eq:gixi-SL3}, we see that $x_{1} = x_{2} = x_{4} = x_{7} = x_{8} = 0$, and thus also $x_{5} = x_{6} = 0$. This leaves $a_{32} = px_{3} = p$, which implies that $x_{3} = 1$. Hence $\sigma(g_{17}) = \sigma(g_{3})$, which implies that $\xi_{17} = \xi_{3}$.

  \item[$g_{18} = \pmat{1-p && p \\ & 1 \\ -p^{2} && 1+p+p^{2}}$:] Comparing $g_{18}$ with \eqref{eq:gixi-SL3}, we see that $x_{2} = x_{7} = 0$, and thus also $x_{3} = x_{6} = 0$ and $x_{4} = x_{5}$. Using
        \begin{align*}
          a_{11} &= \exp(px_{4}) = 1-p, \\
          a_{13} &= x_{8}\exp(px_{4}) = x_{8}(1-p) = p, \\
          a_{31} &= px_{1}\exp(px_{4}) = px_{1}(1-p) = -p^{2},
        \end{align*}
        we get that
        \begin{align*}
          x_{4} &= \dfrac{1}{p}\log(1-p) = \dfrac{1}{p}\bigl( (-p) + O(p^{2}) \bigr) = -1 + O(p), \\
          x_{8} &= \dfrac{p}{1-p} = p + O(p^{2}), \\
          x_{1} &= \dfrac{-p^{2}}{p(1-p)} = -p + O(p^{2}).
        \end{align*}
        Hence $\sigma(g_{18}) = -\pi \act \sigma(g_{1}) - \sigma(g_{4}) - \sigma(g_{5}) + \pi \act \sigma(g_{8})$, which implies that $\xi_{18} = -(\xi_{4}+\xi_{5})$.

  \item[$g_{23} = \pmat{1 \\ & 1 \\ -p^{2} && 1}$:] Comparing $g_{23}$ with \eqref{eq:gixi-SL3}, we see that $x_{2} = x_{4} = x_{7} = x_{8} = 0$, and thus also $x_{3} = x_{5} = x_{6} = 0$. This leaves $a_{31} = px_{1} = -p^{2}$, which implies that $x_{1} = -p$. Hence $\sigma(g_{23}) = -\pi \act \sigma(g_{1})$, which implies that $\xi_{23} = 0$.

  \item[$g_{24} = \pmat{1 \\ p\bigl( 1-\exp(-2p) \bigr) & 1 \\ && 1}$:] Comparing $g_{24}$ with \eqref{eq:gixi-SL3}, we see that $x_{1} = x_{4} = x_{7} = x_{8} = 0$, and thus also $x_{3} = x_{5} = x_{6} = 0$. This leaves $a_{21} = px_{2} = p\bigl( 1-\exp(-2p) \bigr) = p\bigl( 1-\bigl( 1+(-2p)+O(p^{2}) \bigr) \bigr) = 2p^{2} + O(p^{3})$, which implies that $x_{2} = 2p + O(p^{2})$. Hence $\sigma(g_{24}) = 2\pi \act \sigma(g_{1})$, which implies that $\xi_{24} = 0$.

  \item[$g_{25} = \pmat{1 \\ p\bigl( 1-\exp(p) \bigr) & 1 \\ && 1}$:] Except a factor $-2$ in the exponential, which clearly does not change the final result, we have the same calculation as for $g_{24}$. Thus $\xi_{25} = 0$.

  \item[$g_{27} = \pmat{ 1-p & p \\ -p^{2} & 1+p+p^{2} \\ && 1}$:] Comparing $g_{27}$ with \eqref{eq:gixi-SL3}, we see that $x_{1} = x_{8} = 0$, and thus also $x_{3} = x_{6} = 0$, so $x_{5} = 0$. Using
        \begin{align*}
          a_{11} &= \exp(px_{4}) = 1-p, \\
          a_{12} &= x_{7}\exp(px_{4}) = x_{8}(1-p) = p, \\
          a_{21} &= px_{2}\exp(px_{4}) = px_{2}(1-p) = -p^{2},
        \end{align*}
        we get that
        \begin{align*}
          x_{4} &= \dfrac{1}{p}\log(1-p) = \dfrac{1}{p}\bigl( (-p) + O(p^{2}) \bigr) = -1 + O(p), \\
          x_{7} &= \dfrac{p}{1-p} = p + O(p^{2}), \\
          x_{2} &= \dfrac{-p^{2}}{p(1-p)} = -p + O(p^{2}).
        \end{align*}
        Hence $\sigma(g_{27}) = -\pi \act \sigma(g_{2}) - \sigma(g_{4}) + \pi \act \sigma(g_{7})$, which implies that $\xi_{27} = -\xi_{4}$.

  \item[$g_{28} = \pmat{ 1 \\ & 1 & p \\ && 1}$:] Comparing $g_{28}$ with \eqref{eq:gixi-SL3}, we see that $x_{1} = x_{2} = x_{4} = x_{7} = x_{8} = 0$, and thus also $x_{3} = x_{5} = 0$. This leaves $a_{23} = x_{6} = p$. Hence $\sigma(g_{28}) = \pi \act \sigma(g_{6})$, which implies that $\xi_{28} = 0$.

  \item[$g_{34} = \pmat{ 1 \\ & 1 \\ & p\bigl( 1-\exp(p) \bigr) & 1}$:] Comparing $g_{34}$ with \eqref{eq:gixi-SL3}, we see that $x_{1} = x_{2} = x_{4} = x_{7} = x_{8} = 0$, and thus also $x_{5} = x_{6} = 0$. This leaves $a_{32} = px_{3} = p\bigl( 1-\exp(p) \bigr) = p\bigl( 1-\bigl( 1+p+O(p^{2}) \bigr) \bigr) = -p^{2} + O(p^{3})$, which implies that $x_{3} = -p + O(p^{2})$. Hence $\sigma(g_{34}) = -\pi \act \sigma(g_{3})$, which implies that $\xi_{34} = 0$.

  \item[$g_{35} = \pmat{ 1 \\ & 1 \\ & p\bigl( 1-\exp(-2p) \bigr) & 1}$:] Except a factor $-2$ in the exponential, which clearly does not change the final result, we have the same calculation as for $g_{34}$. Thus $\xi_{35} = 0$.

  \item[$g_{36} = \pmat{ 1 \\ & 1-p & p \\ & -p^{2} & 1+p+p^{2}}$:] Comparing $g_{36}$ with \eqref{eq:gixi-SL3}, we see that $x_{1} = x_{2} = x_{4} = x_{7} = x_{8} = 0$. Using
        \begin{align*}
          a_{22} &= \exp(px_{5}) = 1-p, \\
          a_{23} &= x_{6}\exp(px_{5}) = x_{6}(1-p) = p, \\
          a_{32} &= px_{3}\exp(px_{5}) = px_{3}(1-p) = -p^{2},
        \end{align*}
        we get that
        \begin{align*}
          x_{5} &= \dfrac{1}{p}\log(1-p) = \dfrac{1}{p}\bigl( (-p) + O(p^{2}) \bigr) = -1 + O(p), \\
          x_{6} &= \dfrac{p}{1-p} = p + O(p^{2}), \\
          x_{3} &= \dfrac{-p^{2}}{p(1-p)} = -p + O(p^{2}).
        \end{align*}
        Hence $\sigma(g_{36}) = -\pi \act \sigma(g_{3}) - \sigma(g_{5}) + \pi \act \sigma(g_{6})$, which implies that $\xi_{36} = -\xi_{5}$.

  \item[$g_{38} = \pmat{ 1 & -p \\ & 1 \\ && 1}$:] Comparing $g_{38}$ with \eqref{eq:gixi-SL3}, we see that $x_{1} = x_{2} = x_{4} = x_{8} = 0$, and thus also $x_{3} = x_{5} = x_{6} = 0$. This leaves $a_{12} = x_{7} = -p$. Hence $\sigma(g_{38}) = -\pi \act \sigma(g_{3})$, which implies that $\xi_{38} = 0$.

  \item[$g_{46} = \pmat{ 1 \\ & 1 & \exp(-p)-1 \\ && 1}$:] Comparing $g_{46}$ with \eqref{eq:gixi-SL3}, we see that $x_{1} = x_{2} = x_{4} = x_{7} = x_{8} = 0$, and thus also $x_{3} = x_{5} = 0$. This leaves $a_{23} = x_{6} = \exp(-p) - 1 = -p + O(p^{2})$. Hence $\sigma(g_{46}) = -\pi \act \sigma(g_{6})$, which implies that $\xi_{46} = 0$.

  \item[$g_{47} = \pmat{ 1 & \exp(2p)-1 \\ & 1 \\ && 1}$:] Comparing $g_{47}$ with \eqref{eq:gixi-SL3}, we see that $x_{1} = x_{2} = x_{4} = x_{8} = 0$, and thus also $x_{3} = x_{5} = x_{6} = 0$. This leaves $a_{12} = x_{7} = \exp(2p) - 1 = 2p + O(p^{2})$. Hence $\sigma(g_{47}) = 2\pi \act \sigma(g_{7})$, which implies that $\xi_{47} = 0$.

  \item[$g_{48} = \pmat{ 1 && \exp(p)-1 \\ & 1 \\ && 1}$:] Comparing $g_{48}$ with \eqref{eq:gixi-SL3}, we see that $x_{1} = x_{2} = x_{4} = x_{7} = 0$, and thus also $x_{3} = x_{5} = x_{6} = 0$. This leaves $a_{13} = x_{8} = \exp(p) - 1 = p + O(p^{2})$. Hence $\sigma(g_{48}) = \pi \act \sigma(g_{8})$, which implies that $\xi_{48} = 0$.

  \item[$g_{56} = \pmat{ 1 \\ & 1 & \exp(2p)-1 \\ && 1}$:] Except a factor $-2$ in the exponential, which clearly does not change the final result, we have the same calculation as for $g_{46}$. Thus $\xi_{56} = 0$.

  \item[$g_{57} = \pmat{ 1 & \exp(-p)-1 \\ & 1 \\ && 1}$:] Except a factor $-2$ in the exponential, which clearly does not change the final result, we have the same calculation as for $g_{47}$. Thus $\xi_{57} = 0$.

  \item[$g_{58} = \pmat{ 1 && \exp(p)-1 \\ & 1 \\ && 1}$:] Since $g_{58} = g_{48}$, the above calculation shows that $\xi_{58} = 0$.

  \item[$g_{67} = \pmat{ 1 && -1 \\ & 1 \\ && 1}$:] Comparing $g_{67}$ with \eqref{eq:gixi-SL3}, we see that $x_{1} = x_{2} = x_{4} = x_{7} = 0$, and thus also $x_{3} = x_{5} = x_{6} = 0$. This leaves $a_{13} = x_{8} = -1$. Hence $\sigma(g_{67}) = -\sigma(g_{8})$, which implies that $\xi_{67} = -\xi_{8}$.
\end{description}

Thus the non-zero commutators $[\xi_{i},\xi_{j}]$ with $i<j$ are:
\begin{equation}
  \label{eq:xi_ij-SL3}
  \begin{aligned}
    [\xi_{1},\xi_{6}] &= -\xi_{2}, & [\xi_{1},\xi_{7}] &= \xi_{3}, & [\xi_{1},\xi_{8}] &= -(\xi_{4}+\xi_{5}), \\
    [\xi_{2},\xi_{7}] &= -\xi_{4}, & [\xi_{3},\xi_{6}] &= -\xi_{5}, & [\xi_{6},\xi_{7}] &= -\xi_{8}.
  \end{aligned}
\end{equation}

\subsection{Describing the graded chain complex, \texorpdfstring{$\gr^{j}\bigl(\bigwedge^{n}\lie{g}\bigr)$}{grj(wedge-n g)}}%
\label{subsec:graded-complex-SL3}

Looking at \eqref{eq:Iwa-p-val-basis-SLn} (with $e=1$ and $h=3$), we see that
\begin{align*}
  \omega(g_{1}) &= 1-\frac{2}{3} = \frac{1}{3}, & \omega(g_{2}) &= 1-\frac{1}{3} = \frac{2}{3}, & \omega(g_{3}) &= 1-\frac{1}{3} = \frac{2}{3}, \\
  \omega(g_{4}) &= 1, & \omega(g_{5}) &= 1, & \omega(g_{6}) &= \frac{1}{3}, \\
  \omega(g_{7}) &= \frac{1}{3}, & \omega(g_{8}) &= \frac{2}{3}.
\end{align*}
Hence
\begin{equation*}
  \lie{g} = k \otimes_{\F_{p}[\pi]} \gr I = \Span_{k}(\xi_{1},\dotsc,\xi_{8}) = \lie{g}^{1} \oplus \lie{g}^{2} \oplus \lie{g}^{3},
\end{equation*}
where
\begin{align*}
  \lie{g}^{1} &= \lie{g}_{\frac{1}{3}} = \Span_{k}(\xi_{1},\xi_{6},\xi_{7}), \\
  \lie{g}^{2} &= \lie{g}_{\frac{2}{3}} = \Span_{k}(\xi_{2},\xi_{3},\xi_{8}), \\
  \lie{g}^{3} &= \lie{g}_{1} = \Span_{k}(\xi_{4},\xi_{5}).
\end{align*}
See \Cref{rem:g-Z-grading} for more details.

% \begin{equation}
%   \label{eq:5}
%   [\lie{g}^{i},\lie{g}^{j}] =
%   \begin{dcases*}
%     \lie{g}^{2} & if $i=j=1$, \\
%     \lie{g}^{3} & if $(i,j)\in \set{(1,2),(2,1)}$, \\
%     0 & otherwise.
%   \end{dcases*}
% \end{equation}

Now we are ready to describe the graded chain complex
\begin{equation*}
  \gr^{j}\Bigl( \bigwedge^{n}\lie{g} \Bigr) = \bigoplus_{j_{1} + \dotsb + j_{n} = j} \lie{g}^{j_{1}} \wedge \dotsb \wedge \lie{g}^{j_{n}},
\end{equation*}
but we will skip the description of the bases this time. For a description of the basis, we refer to the supplemental files of \cite{code}. We list the grading of $\bigwedge^{n}\lie{g}$ for all $n$.

\begin{description}
  \item[$n=0:$]
        \begin{equation*}
          \gr^{j}(k) =
          \begin{dcases}
            k & j=0, \\
            0 & \text{otherwise.}
          \end{dcases}
        \end{equation*}

   \item[$n=1:$]
        \begin{equation*}
          \gr^{j}(\lie{g}) =
          \begin{dcases}
            \lie{g}^{3} & j=3, \\
            \lie{g}^{2} & j=2, \\
            \lie{g}^{1} & j=1, \\
            0          & \text{otherwise.}
          \end{dcases}
        \end{equation*}

  \item[$n=2:$]
        \begin{equation*}
          \gr^{j}\Bigl( \bigwedge^{2}\lie{g} \Bigr) =
          \begin{dcases}
            \lie{g}^{3} \wedge \lie{g}^{3} & j=6,                                                                             \\
            \lie{g}^{2} \wedge \lie{g}^{3} & j=5,                                                                             \\
            \!\begin{aligned} & \lie{g}^{1} \wedge \lie{g}^{3} \\ & \oplus \lie{g}^{2} \wedge \lie{g}^{2} \end{aligned} & j=4, \\
            \lie{g}^{1} \wedge \lie{g}^{2}                                                                             & j=3, \\
            \lie{g}^{1} \wedge \lie{g}^{1}                                                                             & j=2, \\
            0                                                                                                         & \text{otherwise.}
          \end{dcases}
        \end{equation*}

  \item[$n=3:$]
        \begin{equation*}
          \gr^{j}\Bigl( \bigwedge^{3}\lie{g} \Bigr) =
          \begin{dcases}
            \lie{g}^{2} \wedge \lie{g}^{3} \wedge \lie{g}^{3} & j=8,                                                                                                                   \\
            \!\begin{aligned} & \lie{g}^{1} \wedge \lie{g}^{3} \wedge \lie{g}^{3} \\ & \oplus \lie{g}^{2} \wedge \lie{g}^{2} \wedge \lie{g}^{3} \end{aligned} & j=7, \\
            \!\begin{aligned} & \lie{g}^{1} \wedge \lie{g}^{2} \wedge \lie{g}^{3} \\ & \oplus \lie{g}^{2} \wedge \lie{g}^{2} \wedge \lie{g}^{2} \end{aligned} & j=6, \\
            \!\begin{aligned} & \lie{g}^{1} \wedge \lie{g}^{1} \wedge \lie{g}^{3} \\ & \oplus \lie{g}^{1} \wedge \lie{g}^{2} \wedge \lie{g}^{2} \end{aligned} & j=5, \\
            \lie{g}^{1} \wedge \lie{g}^{1} \wedge \lie{g}^{2}                                                                                                & j=4, \\
            \lie{g}^{1} \wedge \lie{g}^{1} \wedge \lie{g}^{1}                                                                                                & j=3, \\
            0                                                                                                                                              & \text{otherwise.}
          \end{dcases}
        \end{equation*}

  \item[$n=4:$]
        \begin{equation*}
          \gr^{j}\Bigl( \bigwedge^{4}\lie{g} \Bigr) =
          \begin{dcases}
            \lie{g}^{2} \wedge \lie{g}^{2} \wedge \lie{g}^{3} \wedge \lie{g}^{3} & j=10,                                                                                                                   \\
            \!\begin{aligned} & \lie{g}^{1} \wedge \lie{g}^{2} \wedge \lie{g}^{3} \wedge \lie{g}^{3} \\ & \oplus \lie{g}^{2} \wedge \lie{g}^{2} \wedge \lie{g}^{2} \wedge \lie{g}^{3} \end{aligned} & j=9, \\
            \!\begin{aligned} & \lie{g}^{1} \wedge \lie{g}^{1} \wedge \lie{g}^{3} \wedge \lie{g}^{3} \\ & \oplus \lie{g}^{1} \wedge \lie{g}^{2} \wedge \lie{g}^{2} \wedge \lie{g}^{3} \end{aligned} & j=8, \\
            \!\begin{aligned} & \lie{g}^{1} \wedge \lie{g}^{1} \wedge \lie{g}^{2} \wedge \lie{g}^{3} \\ & \oplus \lie{g}^{1} \wedge \lie{g}^{2} \wedge \lie{g}^{2} \wedge \lie{g}^{2} \end{aligned} & j=7, \\
            \!\begin{aligned} & \lie{g}^{1} \wedge \lie{g}^{1} \wedge \lie{g}^{1} \wedge \lie{g}^{3} \\ & \oplus \lie{g}^{1} \wedge \lie{g}^{1} \wedge \lie{g}^{2} \wedge \lie{g}^{2} \end{aligned} & j=6, \\
            \lie{g}^{1} \wedge \lie{g}^{1} \wedge \lie{g}^{1} \wedge \lie{g}^{2}                                                                                                                  & j=5, \\
            0                                                                                                                                                                                   & \text{otherwise.}
          \end{dcases}
        \end{equation*}

  \item[$n=5:$]
        \begin{equation*}
          \gr^{j}\Bigl( \bigwedge^{5}\lie{g} \Bigr) =
          \begin{dcases}
            \lie{g}^{2} \wedge \lie{g}^{2} \wedge \lie{g}^{2} \wedge \lie{g}^{3} \wedge \lie{g}^{3} & j=12,                                                                                                                   \\
            \lie{g}^{1} \wedge \lie{g}^{2} \wedge \lie{g}^{2} \wedge \lie{g}^{3} \wedge \lie{g}^{3} & j=11, \\
            \!\begin{aligned} & \lie{g}^{1} \wedge \lie{g}^{1} \wedge \lie{g}^{2} \wedge \lie{g}^{3} \wedge \lie{g}^{3} \\ & \oplus \lie{g}^{1} \wedge \lie{g}^{2} \wedge \lie{g}^{2} \wedge \lie{g}^{2} \wedge \lie{g}^{3} \end{aligned} & j=10, \\
            \!\begin{aligned} & \lie{g}^{1} \wedge \lie{g}^{1} \wedge \lie{g}^{1} \wedge \lie{g}^{3} \wedge \lie{g}^{3} \\ & \oplus \lie{g}^{1} \wedge \lie{g}^{1} \wedge \lie{g}^{2} \wedge \lie{g}^{2} \wedge \lie{g}^{3} \end{aligned} & j=9, \\
            \!\begin{aligned} & \lie{g}^{1} \wedge \lie{g}^{1} \wedge \lie{g}^{1} \wedge \lie{g}^{2} \wedge \lie{g}^{3} \\ & \oplus \lie{g}^{1} \wedge \lie{g}^{1} \wedge \lie{g}^{2} \wedge \lie{g}^{2} \wedge \lie{g}^{2} \end{aligned} & j=8, \\
            \lie{g}^{1} \wedge \lie{g}^{1} \wedge \lie{g}^{1} \wedge \lie{g}^{2} \wedge \lie{g}^{2}                                                                                                                  & j=7, \\
            0                                                                                                                                                                                                      & \text{otherwise.}
          \end{dcases}
        \end{equation*}


  \item[$n=6:$]
        \begin{equation*}
          \gr^{j}\Bigl( \bigwedge^{6}\lie{g} \Bigr) =
          \begin{dcases}
            \lie{g}^{1} \wedge \lie{g}^{2} \wedge \lie{g}^{2} \wedge \lie{g}^{2} \wedge \lie{g}^{3} \wedge \lie{g}^{3} & j=13,                                                                                                                   \\
            \lie{g}^{1} \wedge \lie{g}^{1} \wedge \lie{g}^{2} \wedge \lie{g}^{2} \wedge \lie{g}^{3} \wedge \lie{g}^{3}                                                                                                                   & j=12, \\
            \!\begin{aligned} & \lie{g}^{1} \wedge \lie{g}^{1} \wedge \lie{g}^{1} \wedge \lie{g}^{2} \wedge \lie{g}^{3} \wedge \lie{g}^{3} \\ & \oplus \lie{g}^{1} \wedge \lie{g}^{1} \wedge \lie{g}^{2} \wedge \lie{g}^{2} \wedge \lie{g}^{2} \wedge \lie{g}^{3} \end{aligned} & j=11, \\
            \lie{g}^{1} \wedge \lie{g}^{1} \wedge \lie{g}^{1} \wedge \lie{g}^{2} \wedge \lie{g}^{2} \wedge \lie{g}^{3}                                                                                                                   & j=10, \\
            \lie{g}^{1} \wedge \lie{g}^{1} \wedge \lie{g}^{1} \wedge \lie{g}^{2} \wedge \lie{g}^{2} \wedge \lie{g}^{2}                                                                                                                   & j=9,  \\
            0                                                                                                                                                                                                & \text{otherwise.}
          \end{dcases}
        \end{equation*}


  \item[$n=7:$]
        \begin{equation*}
          \gr^{j}\Bigl( \bigwedge^{7}\lie{g} \Bigr) =
          \begin{dcases}
            \lie{g}^{1} \wedge \lie{g}^{1} \wedge \lie{g}^{2} \wedge \lie{g}^{2} \wedge \lie{g}^{2} \wedge \lie{g}^{3} \wedge \lie{g}^{3}                    & j=14, \\
            \lie{g}^{1} \wedge \lie{g}^{1} \wedge \lie{g}^{1} \wedge \lie{g}^{2} \wedge \lie{g}^{2} \wedge \lie{g}^{3} \wedge \lie{g}^{3}                    & j=13, \\
            \lie{g}^{1} \wedge \lie{g}^{1} \wedge \lie{g}^{1} \wedge \lie{g}^{2} \wedge \lie{g}^{2} \wedge \lie{g}^{2} \wedge \lie{g}^{3}                    & j=12, \\
            0                                                                                                               & \text{otherwise.}
          \end{dcases}
        \end{equation*}

  \item[$n=8:$]
        \begin{equation*}
          \gr^{j}\Bigl( \bigwedge^{8}\lie{g} \Bigr) =
          \begin{dcases}
            \lie{g}^{1} \wedge \lie{g}^{1} \wedge \lie{g}^{1} \wedge \lie{g}^{2} \wedge \lie{g}^{2} \wedge \lie{g}^{2} \wedge \lie{g}^{3} \wedge \lie{g}^{3} & j=15, \\
            0                                                                                                         & \text{otherwise.}
          \end{dcases}
        \end{equation*}

   \item[$n\geq9:$]
        \begin{equation*}
          \gr^{j}\Bigl( \bigwedge^{n}\lie{g} \Bigr) = 0 \text{ for all } j.
        \end{equation*}
\end{description}

\begin{table}[ht]
  \centering
  \caption[Graded complex dimensions for the $I \subseteq \SL_{3}(\Z_{p})$ case]{Dimensions of $\gr^{j}\bigl( \bigwedge^{n} \lie{g} \bigr)$.}
  \label{tab:graded-dims-SL3}
  $\begin{NiceArray}{*{17}{c}}[hvlines]
    \diagbox{n}{j} & 0 & 1 & 2 & 3 & 4 & 5 & 6 & 7 & 8 & 9 & 10 & 11 & 12 & 13 & 14 & 15 \\
    0 & 1 \\
    1 & & 3 & 3 & 2 \\
    2 & & & 3 & 9 & 9 & 6 & 1 \\
    3 & & & & 1 & 9 & 15 & 19 & 9 & 3 \\
    4 & & & & & & 3 & 11 & 21 & 21 & 11 & 3 \\
    5 & & & & & & & & 3 & 9 & 19 & 15 & 9 & 1 \\
    6 & & & & & & & & & & 1 & 6 & 9 & 9 & 3 \\
    7 & & & & & & & & & & & & & 2 & 3 & 3 \\
    8 & & & & & & & & & & & & & & & & 1
  \end{NiceArray}$
\end{table}

We collect the above information about the dimensions of the chain complex of $\lie{g}$ in \Cref{tab:graded-dims-SL3}, and note that we only need to consider non-zero (non-empty) entries of the table, when we calculate  $H^{s,t} = H^{s,n-s}$ (where $H^{s,t} = H^{s,t}(\lie{g},k)$). Also, recalling that
\begin{equation*}
  \Hom_{k}\Bigl( \bigwedge^{n}\lie{g}, k \Bigr) = \bigoplus_{s \in \Z} \Hom_{k}^{s}\Bigl( \bigwedge^{n}\lie{g}, k \Bigr),
\end{equation*}
we see that, with $j=-s$, we get the same table for dimensions of the graded hom-spaces in the cochain complex.

\subsection{Finding the graded Lie algebra cohomology, \texorpdfstring{$H^{s,t}(\lie{g},k)$}{H(s,t)(g,k)}}%
\label{subsec:graded-coh-SL3}

We will now go through all different graded chain complexes one by one, using that $\gr^{j}$ in the chain complex corresponds to $\gr^{s}$ with $s = -j$ in the cochain complex. We note that the graded chain complex corresponds to vertical downwards arrows in \Cref{tab:graded-dims-SL3}, while the cochain complex corresponds to vertical upwards arrows. And finally, we reiterate that $H^{n} = H^{n}(\lie{g},k)$ and $H^{s,t} = H^{s,t}(\lie{g},k)$ in the following.

\begin{remark}
  We will repeatedly use that, if
  \begin{equation*}
    d \snfsim \SNF_{n \times m}(a_{1},\dotsc,a_{r},0,\dotsc,0)
  \end{equation*}
  with $a_{1},\dotsc,a_{r}$ non-zero (in $\F_{p}$), then
  \begin{align*}
    \dim \kernel(d) &= m-r, \\
    \dim \image(d) &= r, \\
    \dim \coker(d) &= n-r,
  \end{align*}
  as described in \Cref{subsec:SNF-coh}.
\end{remark}

In grade $0$ we have the chain complex
\[
  \begin{tikzcd}
    0 \ar[r] & k \ar[r] & 0
  \end{tikzcd}
\]
which gives us the grade $0$ cochain complex
\[
  \begin{tikzcd}
    0 & \ar[l] \Hom_{k}^{0}(k,k) & \ar[l] 0
  \end{tikzcd}
\]
So $H^{0} = H^{0,0}$ with $\dim H^{0,0} = 1$.

In grade $1$ we have the chain complex
\[
  \begin{tikzcd}
    0 \ar[r] & \lie{g}^{1} \ar[r] & 0
  \end{tikzcd}
\]
which gives us the grade $-1$ cochain complex
\[
  \begin{tikzcd}
    0 & \ar[l] \Hom_{k}^{-1}(\lie{g},k) & \ar[l] 0
  \end{tikzcd}
\]
So $\dim H^{-1,2} = 3$ by \Cref{tab:graded-dims-SL3}.

In grade $2$ we have the chain complex
\[
  \begin{tikzcd}[ampersand replacement=\&]
    0 \ar[r] \& \lie{g}^{1} \wedge \lie{g}^{1} \ar[r, "{\begin{pmatrix} 1 & 0 & 0 \\ 0 & -1 & 0 \\ 0 & 0 & 1 \end{pmatrix}}" {yshift=7pt}] \& \lie{g}^{2} \ar[r] \& 0
  \end{tikzcd}
\]
since
\begin{align*}
  \lie{g}^{1} \wedge \lie{g}^{1} &\to \lie{g}^{2} \\
  \xi_{1} \wedge \xi_{6} &\mapsto -[\xi_{1},\xi_{6}] = \xi_{2} \\
  \xi_{1} \wedge \xi_{7} &\mapsto -[\xi_{1},\xi_{7}] = -\xi_{3} \\
  \xi_{6} \wedge \xi_{7} &\mapsto -[\xi_{6},\xi_{7}] = \xi_{8}.
\end{align*}
This gives us the grade $-2$ cochain complex
\[
  \begin{tikzcd}[ampersand replacement=\&]
    0 \& \ar[l] \Hom_{k}^{-2}\bigl( \bigwedge^{2} \lie{g}, k \bigr) \& \ar[l, "{\begin{pmatrix} 1 & 0 & 0 \\ 0 & -1 & 0 \\ 0 & 0 & 1 \end{pmatrix}}"' {yshift=7pt}] \Hom_{k}^{-2}(\lie{g},k) \& \ar[l] 0,
  \end{tikzcd}
\]
where
\begin{equation*}
  d = \pmat{1&0&0 \\ 0&-1&0 \\ 0&0&1} \snfsim  \SNF_{3\times3}(1,1,1).
\end{equation*}
So
\begin{align*}
  \dim H^{-2,3} &= \dim \kernel(d) = 0, \\
  \dim H^{-2,4} &= \dim \coker(d) = 0.
\end{align*}

In grade $3$ we have the chain complex
\[
  \begin{tikzcd}[ampersand replacement=\&, column sep=4em]
    0 \ar[r] \& \lie{g}^{1} \wedge \lie{g}^{1} \wedge \lie{g}^{1} \ar[r, "{\begin{pmatrix} 0 & 0 & -1 & 0 & -1 & 0 & -1 & 0 & 0 \end{pmatrix}^{\top}}" {yshift=7pt}] \& \lie{g}^{1} \wedge \lie{g}^{2} \ar[r, "{\begin{pmatrix} 0 & 0 & 1 & 0 & 0 & 0 & -1 & 0 & 0 \\ 0 & 0 & 1 & 0 & -1 & 0 & 0 & 0 & 0 \end{pmatrix}}"' {yshift=-7pt}] \& \lie{g}^{3} \ar[r] \& 0
  \end{tikzcd}
\]
which gives us the grade $-3$ cochain complex
\[
  \begin{tikzcd}[ampersand replacement=\&, column sep=1em]
    0 \& \ar[l] \Hom_{k}^{-3}\bigl( \bigwedge^{3}\lie{g}, k \bigr) \& \ar[l, "{\begin{pmatrix} 0 & 0 & -1 & 0 & -1 & 0 & -1 & 0 & 0 \end{pmatrix}}"' {yshift=7pt}] \Hom_{k}^{-3}\bigl( \bigwedge^{2}\lie{g}, k \bigr) \& \ar[l, "{\begin{pmatrix} 0 & 0 & 1 & 0 & 0 & 0 & -1 & 0 & 0 \\ 0 & 0 & 1 & 0 & -1 & 0 & 0 & 0 & 0 \end{pmatrix}^{\top}}" {yshift=-7pt}] \Hom_{k}^{-3}(\lie{g}, k) \& \ar[l] 0,
  \end{tikzcd}
\]
where
\begin{align*}
  d_{1} = \pmat{0&0 \\ 0&0 \\ 1&1 \\ 0&0 \\ 0&-1 \\ 0&0 \\ 0&-1 \\ 0&0 \\ 0&0} &\snfsim \SNF_{9\times2}(1,1), \\
  d_{2} = \pmat{0&0&-1&0&-1&0&-1&0&0} &\snfsim \SNF_{1\times9}(1).
\end{align*}
So
\begin{align*}
  \dim H^{-3,4} &= \dim \kernel(d_{1}) = 2-2 = 0, \\
  \dim H^{-3,5} &= \dim \dfrac{\kernel(d_{2})}{\image(d_{1})} = (9-1) - 2 = 6, \\
  \dim H^{-3,6} &= \dim \coker(d_{2}) = 1 - 1 = 0.
\end{align*}

In grade $4$ we have the chain complex
\[
  \begin{tikzcd}[ampersand replacement=\&]
    0 \ar[r] \& \lie{g}^{1} \wedge \wedge \lie{g}^{1} \wedge \lie{g}^{2} \ar[r,"d^{\top}"] \& \begin{aligned} &\lie{g}^{1} \wedge \lie{g}^{3} \\ &\oplus \lie{g}^{2} \wedge \lie{g}^{2} \end{aligned} \ar[r] \& 0
  \end{tikzcd}
\]
which gives us the grade $-4$ cochain complex
\[
  \begin{tikzcd}
    0 & \ar[l] \Hom_{k}^{-4}\bigl( \bigwedge^{3}\lie{g}, k \bigr) & \ar[l,"d"'] \Hom_{k}^{-4}\bigl( \bigwedge^{2}\lie{g}, k \bigr) &  \ar[l] 0
  \end{tikzcd}
\]
where
\begin{equation*}
  d \snfsim \SNF_{9\times9}(1,1,1,1,1,1,0,0,0).
\end{equation*}
So
\begin{align*}
  \dim H^{-4,6} &= \dim \kernel(d) = 9-6 = 3, \\
  \dim H^{-4,7} &= \dim \coker(d) = 9-6 = 3.
\end{align*}

In grade $5$ we have the chain complex
\[
  \begin{tikzcd}[ampersand replacement=\&]
    0 \ar[r] \& \lie{g}^{1} \wedge \lie{g}^{1} \wedge \lie{g}^{1} \wedge \lie{g}^{2} \ar[r,"d_{2}^{\top}"] \& \begin{aligned} &\lie{g}^{1} \wedge \lie{g}^{1} \wedge \lie{g}^{3} \\ &\oplus \lie{g}^{1} \wedge \lie{g}^{2} \wedge \lie{g}^{2} \end{aligned} \ar[r,"d_{1}^{\top}"] \& \lie{g}^{2} \wedge \lie{g}^{3} \ar[r] \& 0
  \end{tikzcd}
\]
which gives us the grade $-5$ cochain complex
\[
  \begin{tikzcd}[column sep=1em]
    0 & \ar[l] \Hom_{k}^{-5}\bigl( \bigwedge^{4}\lie{g}, k \bigr) & \ar[l,"d_{2}"' {yshift=2pt}] \Hom_{k}^{-5}\bigl( \bigwedge^{3}\lie{g}, k \bigr) & \ar[l,"d_{1}"' {yshift=2pt}] \Hom_{k}^{-5}\bigl( \bigwedge^{2}\lie{g}, k \bigr) & \ar[l] 0,
  \end{tikzcd}
\]
where
\begin{align*}
  d_{1} &\snfsim \SNF_{15\times6}(1,1,1,1,1,1), \\
  d_{2} &\snfsim \SNF_{3\times15}(1,1,1).
\end{align*}
So
\begin{align*}
  \dim H^{-5,7} &= \dim \kernel(d_{1}) = 6-6 = 0, \\
  \dim H^{-5,8} &= \dim \dfrac{\kernel(d_{2})}{\image(d_{1})} = (15-3) - 6 = 6, \\
  \dim H^{-5,9} &= \dim \coker(d_{2}) = 3 - 3 = 0.
\end{align*}

In grade $6$ we have the chain complex
\[
  \begin{tikzcd}[ampersand replacement=\&]
    0 \ar[r] \& \begin{aligned} &\lie{g}^{1} \wedge \lie{g}^{1} \wedge \lie{g}^{1} \wedge \lie{g}^{3} \\ &\oplus \lie{g}^{1} \wedge \lie{g}^{1} \wedge \lie{g}^{2} \wedge \lie{g}^{2} \end{aligned} \ar[r,"d_{2}^{\top}"] \& \begin{aligned} &\lie{g}^{1} \wedge \lie{g}^{2} \wedge \lie{g}^{3} \\ &\oplus \lie{g}^{2} \wedge \lie{g}^{2} \wedge \lie{g}^{2} \end{aligned} \ar[r,"d_{1}^{\top}"] \& \lie{g}^{3} \wedge \lie{g}^{3} \ar[r] \& 0
  \end{tikzcd}
\]
which gives us the grade $-6$ cochain complex
\[
  \begin{tikzcd}[column sep=1em]
    0 & \ar[l] \Hom_{k}^{-6}\bigl( \bigwedge^{4}\lie{g}, k \bigr) & \ar[l,"d_{2}"' {yshift=2pt}] \Hom_{k}^{-6}\bigl( \bigwedge^{3}\lie{g}, k \bigr) & \ar[l,"d_{1}"' {yshift=2pt}] \Hom_{k}^{-6}\bigl( \bigwedge^{2}\lie{g}, k \bigr) & \ar[l] 0,
  \end{tikzcd}
\]
where
\begin{align*}
  d_{1} &\snfsim \SNF_{19\times1}(1), \\
  d_{2} &\snfsim \SNF_{11\times19}(1,1,1,1,1,1,1,1,1,1,2).
\end{align*}
So
\begin{align*}
  \dim H^{-6,8} &= \dim \kernel(d_{1}) = 1-1 = 0, \\
  \dim H^{-6,9} &= \dim \dfrac{\kernel(d_{2})}{\image(d_{1})} = (19-11) - 1 = 7, \\
  \dim H^{-6,10} &= \dim \coker(d_{2}) = 11 - 11 = 0.
\end{align*}

In grade $7$ we have the chain complex
\[
  \begin{tikzcd}[ampersand replacement=\&, column sep=1em]
    0 \ar[r] \& \lie{g}^{1} \wedge \lie{g}^{1} \wedge \lie{g}^{1} \wedge \lie{g}^{2} \wedge \lie{g}^{2} \ar[r,"d_{2}^{\top}" {yshift=2pt}] \& \begin{aligned} &\lie{g}^{1} \wedge \lie{g}^{1} \wedge \lie{g}^{2} \wedge \lie{g}^{3} \\ &\oplus \lie{g}^{1} \wedge \lie{g}^{2} \wedge \lie{g}^{2} \wedge \lie{g}^{2} \end{aligned} \ar[r,"d_{1}^{\top}" {yshift=2pt}] \& \begin{aligned} &\lie{g}^{1} \wedge \lie{g}^{3} \wedge \lie{g}^{3} \\ &\oplus \lie{g}^{2} \wedge \lie{g}^{2} \wedge \lie{g}^{3} \end{aligned}\ar[r] \& 0
  \end{tikzcd}
\]
which gives us the grade $-7$ cochain complex
\[
  \begin{tikzcd}[column sep=1em]
    0 & \ar[l] \Hom_{k}^{-7}\bigl( \bigwedge^{5}\lie{g}, k \bigr) & \ar[l,"d_{2}"' {yshift=2pt}] \Hom_{k}^{-7}\bigl( \bigwedge^{4}\lie{g}, k \bigr) & \ar[l,"d_{1}"' {yshift=2pt}] \Hom_{k}^{-7}\bigl( \bigwedge^{3}\lie{g}, k \bigr) & \ar[l] 0,
  \end{tikzcd}
\]
where
\begin{align*}
  d_{1} &\snfsim \SNF_{21\times9}(1,1,1,1,1,1,1,1,1), \\
  d_{2} &\snfsim \SNF_{3\times21}(1,1,1).
\end{align*}
So
\begin{align*}
  \dim H^{-7,10} &= \dim \kernel(d_{1}) = 9-9 = 0, \\
  \dim H^{-7,11} &= \dim \dfrac{\kernel(d_{2})}{\image(d_{1})} = (21-3) - 9 = 9, \\
  \dim H^{-7,12} &= \dim \coker(d_{2}) = 3 - 3 = 0.
\end{align*}

By \cite[Chap~1 §3.6 and §3.7]{Fuks}, we can now find the rest of the cohomology using a version of Poincaré duality for Lie algebra cohomology. But we keep the sketch work to make it clear that this works. We refer to \cite{code} for the calculations.

In grade $-8$ we get coboundary maps
\begin{align*}
  d_{1} &\snfsim \SNF_{21\times3}(1,1,1), \\
  d_{2} &\snfsim \SNF_{9\times21}(1,1,1,1,1,1,1,1,1).
\end{align*}
So
\begin{align*}
  \dim H^{-8,11} &= \dim \kernel(d_{1}) = 3-3 = 0, \\
  \dim H^{-8,12} &= \dim \dfrac{\kernel(d_{2})}{\image(d_{1})} = (21-9) - 3 = 9, \\
  \dim H^{-8,13} &= \dim \coker(d_{2}) = 9 - 9 = 0.
\end{align*}


In grade $-9$ we get coboundary maps
\begin{align*}
  d_{1} &\snfsim \SNF_{19\times11}(1,1,1,1,1,1,1,1,1,1,1), \\
  d_{2} &\snfsim \SNF_{1\times19}(1).
\end{align*}
So
\begin{align*}
  \dim H^{-9,13} &= \dim \kernel(d_{1}) = 11-11 = 0, \\
  \dim H^{-9,14} &= \dim \dfrac{\kernel(d_{2})}{\image(d_{1})} = (19-1) - 11 = 7, \\
  \dim H^{-9,15} &= \dim \coker(d_{2}) = 1 - 1 = 0.
\end{align*}

In grade $-10$ we get coboundary maps
\begin{align*}
  d_{1} &\snfsim \SNF_{15\times3}(1,1,1), \\
  d_{2} &\snfsim \SNF_{6\times15}(1,1,1,1,1,1).
\end{align*}
So
\begin{align*}
  \dim H^{-10,14} &= \dim \kernel(d_{1}) = 3-3 = 0, \\
  \dim H^{-10,15} &= \dim \dfrac{\kernel(d_{2})}{\image(d_{1})} = (15-6) - 3 = 6, \\
  \dim H^{-10,16} &= \dim \coker(d_{2}) = 6-6 = 0.
\end{align*}

In grade $-11$ we get coboundary maps
\begin{equation*}
  d \snfsim \SNF_{9\times9}(1,1,1,1,1,1,0,0,0).
\end{equation*}
So
\begin{align*}
  \dim H^{-11,16} &= \dim \kernel(d) = 9-6 = 3, \\
  \dim H^{-11,17} &= \dim \coker(d) = 9-6 = 3.
\end{align*}

In grade $-12$ we get coboundary maps
\begin{align*}
  d_{1} &\snfsim \SNF_{9\times1}(1), \\
  d_{2} &\snfsim \SNF_{2\times9}(1,1).
\end{align*}
So
\begin{align*}
  \dim H^{-12,17} &= \dim \kernel(d_{1}) = 1-1 = 0, \\
  \dim H^{-12,18} &= \dim \dfrac{\kernel(d_{2})}{\image(d_{1})} = (9-2) - 1 = 6, \\
  \dim H^{-12,19} &= \dim \coker(d_{2}) = 2-2 = 0.
\end{align*}

In grade $-13$ we get coboundary maps
\begin{equation*}
  d \snfsim \SNF_{3\times3}(1,1,1).
\end{equation*}
So
\begin{align*}
  \dim H^{-13,19} &= \dim \kernel(d) = 3-3 = 0, \\
  \dim H^{-13,20} &= \dim \coker(d) = 3-3 = 0.
\end{align*}

In grade $14$ we have the chain complex
\[
  \begin{tikzcd}
    0 \ar[r] & \lie{g}^{1} \wedge \lie{g}^{1} \wedge \lie{g}^{2} \wedge \lie{g}^{2} \wedge \lie{g}^{2} \wedge \lie{g}^{3} \wedge \lie{g}^{3}  \ar[r] & 0
  \end{tikzcd}
\]
which gives us the grade $-14$ cochain complex
\[
  \begin{tikzcd}
    0 & \ar[l] \Hom_{k}^{-14}\bigl( \bigwedge^{7}\lie{g},k \bigr) & \ar[l] 0
  \end{tikzcd}
\]
So $\dim H^{-14,21} = 3$ by \Cref{tab:graded-dims-SL3}.

In grade $15$ we have the chain complex
\[
  \begin{tikzcd}
    0 \ar[r] & \lie{g}^{1} \wedge \lie{g}^{1} \wedge \lie{g}^{1} \wedge \lie{g}^{2} \wedge \lie{g}^{2} \wedge \lie{g}^{2} \wedge \lie{g}^{3} \wedge \lie{g}^{3}  \ar[r] & 0
  \end{tikzcd}
\]
which gives us the grade $-15$ cochain complex
\[
  \begin{tikzcd}
    0 & \ar[l] \Hom_{k}^{-15}\bigl( \bigwedge^{8}\lie{g},k \bigr) & \ar[l] 0
  \end{tikzcd}
\]
So $H^{8} = H^{-15,23}$ with $\dim H^{-15,23} = 1$ by \Cref{tab:graded-dims-SL3}.

\begin{table}[ht]
  \centering
  \caption[Graded cohomology dimensions for the $I \subseteq \SL_{3}(\Z_{p})$ case]{Dimensions of $E_{1}^{s,t} = H^{s,t} = \gr^{s} H^{s+t}(\lie{g},k)$ for the $I \subseteq \SL_{3}(\Z_{p})$ case.}
  \label{tab:graded-coh-dims-SL3}
  \renewcommand{\arraystretch}{1.7}
  \scalebox{0.7}{%
    $\begin{NiceArray}{*{17}{c}}[hvlines, columns-width=auto]
      \diagbox{t}{s} & 0 & -1 & -2 & -3 & -4 & -5 & -6 & -7 & -8 & -9 & -10 & -11 & -12 & -13 & -14 & -15 \\
      0 & 1\\
      1 \\
      2 && 3 \\
      3 \\
      4 \\
      5 &&&& 6\\
      6 &&&&& 3 \\
      7 &&&&& 3\\
      8 &&&&&& 6\\
      9 &&&&&&& 7\\
      10 \\
      11 &&&&&&&& 9 \\
      12 &&&&&&&&& 9 \\
      13 \\
      14 &&&&&&&&&& 7 \\
      15 &&&&&&&&&&& 6 \\
      16 &&&&&&&&&&&& 3 \\
      17 &&&&&&&&&&&& 3 \\
      18 &&&&&&&&&&&&& 6 \\
      19 \\
      20 \\
      21 &&&&&&&&&&&&&&& 3 \\
      22 \\
      23 &&&&&&&&&&&&&&&& 1
    \end{NiceArray}$%
  }
  \renewcommand{\arraystretch}{1}
\end{table}

\clearpage

Altogether, we see that
\begin{equation}
  \label{eq:Hn-to-Hst-SL3}
  \begin{aligned}
    H^{0} &= H^{0,0}, \\
    H^{1} &= H^{-1,2}, \\
    H^{2} &= H^{-3,5} \oplus H^{-4,6}, \\
    H^{3} &= H^{-4,7} \oplus H^{-5,8} \oplus H^{-6,9}, \\
    H^{4} &= H^{-7,11} \oplus H^{-8,12}, \\
    H^{5} &= H^{-9,14} \oplus H^{-10,15} \oplus H^{-11,16}, \\
    H^{6} &= H^{-11,17} \oplus H^{-12,18}, \\
    H^{7} &= H^{-14,21}, \\
    H^{8} &= H^{-15,23}
  \end{aligned}
\end{equation}
with dimension as described in \Cref{tab:graded-coh-dims-SL3}.

\subsection{Describing the group cohomology, \texorpdfstring{$H^{n}(I,k)$}{Hn(I,k)}}%
\label{subsec:group-coh-SL3}

We note that all differentials $d_{r}^{s,t} \colon E_{r}^{s,t} \to E_{r}^{s+r,t+1-r}$ in \Cref{tab:graded-coh-dims-SL3} has bidegree $(r,1-r)$, i.e., they are all below the $(r,-r)$ arrow going $r$ to the left and $r$ up in the table, where $r \geq 1$. Looking at \Cref{tab:graded-coh-dims-SL3}, this clearly means that all differentials for $r \geq 1$ are trivial, and thus the spectral sequence collapses on the first page. Hence $H^{s,t}(\lie{g},k) = E_{1}^{s,t} \iso E_{\infty}^{s,t} = \gr^{s} H^{s+t}(I,k)$, and by \eqref{eq:Hn-to-Hst-SL3} and \Cref{tab:graded-coh-dims-SL3} we get that
\begin{equation}
  \label{eq:dim-HnI-SL3}
  \dim H^{n}(I,k) =
  \begin{dcases}
    1 & n=0, \\
    3 & n=1, \\
    9 & n=2, \\
    16 & n=3, \\
    18 & n=4, \\
    16 & n=5, \\
    9 & n=6, \\
    3 & n=7, \\
    1 & n=8.
  \end{dcases}
\end{equation}

Recalling that the spectral sequence is multiplicative, we also note, by \Cref{tab:graded-coh-dims-SL3}, that $H^{s,t} \cup H^{s',t'} \subseteq H^{s+s',t+t'}$ implies that the cup products
\begin{equation*}
  \gr^{s} H^{n}(I,k) \otimes \gr^{s'} H^{n'}(I,k) \to \gr^{s+s'} H^{n+n'}(I,k)
\end{equation*}
are trivial. But, since the spectral sequence collapses on the first page, we also have \eqref{eq:Hn-to-Hst-SL3} for $H^{n}(I,k)$, and thus the cup product is trivial.

\section{\texorpdfstring{$I \subseteq \GL_{3}(\Z_{p})$}{I in GL3(Zp)}}%
\label{sec:Iwa-GL3}

In this section we will describe the continuous group cohomology of the pro-$p$ Iwahori subgroup $I$ of $\GL_{3}(\Q_{p})$.

When $I$ is the pro-$p$ Iwahori subgroup in $\GL_{3}(\Q_{p})$, we know by \Cref{sec:cohiwagps-intro} that we can take it to be of the form
\begin{equation*}
  I = \pmat{1+p\Z_{p} & \Z_{p} & \Z_{p} \\ p\Z_{p} & 1+p\Z_{p} & \Z_{p} \\ p\Z_{p} & p\Z_{p} & 1+p\Z_{p}} \subseteq \GL_{3}(\Z_{p}),
\end{equation*}
and, by \Cref{sec:cohiwagps-intro}, we have an ordered basis
\begin{equation}
  \label{eq:gis-GL3}
  \begin{gathered}
    g_{1} = \pmat{ 1 \\ & 1 \\ p && 1 }, \quad g_{2} = \pmat{ 1 \\ p & 1 \\ && 1 }, \quad g_{3} = \pmat{ 1 \\ & 1 \\ & p & 1 }, \\
    g_{4} = \pmat{ \exp(p) \\ & \exp(-p) \\ && 1 }, \quad g_{5} = \pmat{ 1 \\ & \exp(p) \\ && \exp(-p) }, \\
    g_{6} = \pmat{ \exp(p) \\ & \exp(p) \\ && \exp(p) },  \\
    g_{7} = \pmat{ 1 \\ & 1 & 1 \\ && 1 }, \quad g_{8} = \pmat{ 1 & 1 \\ & 1 \\ && 1 }, \quad g_{9} = \pmat{ 1 && 1 \\ & 1 \\ && 1 }.
  \end{gathered}
\end{equation}
Since we just renamed some elements and added an element of the center of $\GL_{3}(\Z_{p})$ when comparing to the ordered basis of $I \subseteq \SL_{3}(\Z_{p})$ from \Cref{sec:Iwa-SL3}, it is clear from \Cref{eq:xi_ij-SL3} that the only non-zero commutators $[\xi_{i},\xi_{j}]$ with $i<j$ are:
\begin{equation}
  \label{eq:xi_ij-GL3}
  \begin{aligned}
    [\xi_{1},\xi_{7}] &= -\xi_{2}, & [\xi_{1},\xi_{8}] &= \xi_{3}, & [\xi_{1},\xi_{9}] &= -(\xi_{4}+\xi_{5}), \\
    [\xi_{2},\xi_{8}] &= -\xi_{4}, & [\xi_{3},\xi_{7}] &= -\xi_{5}, & [\xi_{7},\xi_{8}] &= -\xi_{9}.
  \end{aligned}
\end{equation}

Looking at \Cref{subsec:graded-coh-SL3}, we easily see that
\begin{align*}
  \lie{g}^{1} &= \lie{g}_{\frac{1}{3}} = \Span_{k}(\xi_{1},\xi_{7},\xi_{8}), \\
  \lie{g}^{2} &= \lie{g}_{\frac{2}{3}} = \Span_{k}(\xi_{2},\xi_{3},\xi_{9}), \\
  \lie{g}^{3} &= \lie{g}_{1} = \Span_{k}(\xi_{4},\xi_{5},\xi_{6}).
\end{align*}

This is enough to calculate the graded mod $p$ cohomology of $\lie{g}$, see \cite{code} for the details. We write the result in \Cref{tab:graded-coh-dims-GL3}.

\begin{table}[ht]
  \centering
  \caption[Graded cohomology dimensions for the $I \subseteq \GL_{3}(\Z_{p})$ case]{Dimensions of $E_{1}^{s,t} = H^{s,t} = \gr^{s} H^{s+t}(\lie{g},k)$ for the $I \subseteq \GL_{3}(\Z_{p})$ case.}
  \label{tab:graded-coh-dims-GL3}
  \renewcommand{\arraystretch}{1.7}
  \scalebox{0.6}{%
    $\begin{NiceArray}{*{20}{c}}[hvlines, columns-width=auto]
      \diagbox{t}{s} & 0 & -1 & -2 & -3 & -4 & -5 & -6 & -7 & -8 & -9 & -10 & -11 & -12 & -13 & -14 & -15 & -16 & -17 & -18 \\
      0 & 1\\
      1 \\
      2 && 3 \\
      3 \\
      4 &&&& 1 \\
      5 &&&& 6 \\
      6 &&&&& 6 \\
      7 &&&&& 3\\
      8 &&&&&& 6 \\
      9 &&&&&&& 13 \\
      10 &&&&&&&& 3 \\
      11 &&&&&&&& 12 \\
      12 &&&&&&&&& 15 \\
      13 &&&&&&&&&& 7 \\
      14 &&&&&&&&&& 7 \\
      15 &&&&&&&&&&& 15 \\
      16 &&&&&&&&&&&& 12 \\
      17 &&&&&&&&&&&& 3 \\
      18 &&&&&&&&&&&&& 13 \\
      19 &&&&&&&&&&&&&& 6 \\
      20 &&&&&&&&&&&&&&& 3 \\
      21 &&&&&&&&&&&&&&& 6 \\
      22 &&&&&&&&&&&&&&&& 6 \\
      23 &&&&&&&&&&&&&&&& 1 \\
      24 \\
      25 &&&&&&&&&&&&&&&&&& 3 \\
      26 \\
      27 &&&&&&&&&&&&&&&&&&& 1
    \end{NiceArray}$%
  }
  \renewcommand{\arraystretch}{1}
\end{table}

\section[\texorpdfstring{$I \subseteq \SL_{4}(\Z_{p}),\GL_{4}(\Z_{p})$}{I in SL4(Zp) and GL4()}]{\texorpdfstring{$I \subseteq \SL_{4}(\Z_{p})$ and $I \subseteq \GL_{4}(\Z_{p})$}{I in SL4(Zp) and GL4(Zp)}}%
\label{sec:Iwa-SL4-GL4}

In this section we will briefly describe the problem with finding continuous group cohomology of the pro-$p$ Iwahori subgroup $I$ of $\SL_{4}(\Q_{p})$ and $\GL_{4}(\Q_{p})$.

We leave all the calculations of commutators and $p$-valuations to the appendix, cf.\ \Cref{sec:SL4-calc} and \Cref{sec:GL4-calc}, and note that the dimensions of the graded cohomology $H^{s,t}(\lie{g},k)$ for $\lie{g} = k \otimes_{\F_{p}[\pi]} \gr I$ are shown in \Cref{tab:graded-coh-dims-SL4} and \Cref{tab:graded-coh-dims-GL4}.

Looking at \Cref{tab:graded-coh-dims-SL4}, we see that it is no longer clear that the spectral sequence collapses on the first page. To see this, recall that all differentials on the first page are of the form $d_{1}^{s,t} \colon E_{1}^{s,t} \to E_{1}^{s+1,t}$, so we have maps like $d_{1}^{-5,7} \colon H^{-5,7}(\lie{g},k) \to H^{-4,7}(\lie{g},k)$ that are not obviously trivial, since $\dim_{k} H^{-5,7} = 4$ and $\dim_{k} H^{-4,7} = 4$. To figure out at what page the spectral sequence collapses in this case, one needs to look more carefully at how exactly the spectral sequence is obtained in \cite{Sor}, which is much more complicated than what we have done so far.

\begin{table}[ht]
  \centering
  \caption[Graded cohomology dimensions for the $I \subseteq \SL_{4}(\Z_{p})$ case]{Dimensions of $E_{1}^{s,t} = H^{s,t} = \gr^{s} H^{s+t}(\lie{g},k)$ for the $I \subseteq \SL_{4}(\Z_{p})$ case. This table only shows the graded dimensions of $H^{0},\dotsc,H^{7}$. We note that the graded dimensions of $H^{8},\dotsc,H^{15}$ can be found using Poincaré duality, which gives us $\dim_{k} H^{s,t} = \dim_{k} H^{-36-s,51-t}$ for $(s,t)$ from the table.}
  \label{tab:graded-coh-dims-SL4}
  \renewcommand{\arraystretch}{1.7}
  \scalebox{0.6}{%
    $\begin{NiceArray}{*{21}{c}}[hvlines, columns-width=auto]
      \diagbox{t}{s} & 0 & -1 & -2 & -3 & -4 & -5 & -6 & -7 & -8 & -9 & -10 & -11 & -12 & -13 & -14 & -15 & -16 & -17 & -18 & -19 \\
      0 & 1\\
      1 \\
      2 && 4 \\
      3 \\
      4 &&& 2 \\
      5 &&&& 8 \\
      6 &&&&& \\
      7 &&&&& 4 & 4 \\
      8 &&&&&& 8 \\
      9 &&&&&&& 20 \\
      10 &&&&&&&& 8 \\
      11 &&&&&&&& 20 & 1 \\
      12 &&&&&&&&& 34 \\
      13 &&&&&&&&&& 16 \\
      14 &&&&&&&&&& 12 & 18 \\
      15 &&&&&&&&&&& 26 \\
      16 &&&&&&&&&&& 4 & 76 \\
      17 &&&&&&&&&&&&& 39 \\
      18 &&&&&&&&&&&&& 28 & 8 \\
      19 &&&&&&&&&&&&&& 68 \\
      20 &&&&&&&&&&&&&&& 72 \\
      21 &&&&&&&&&&&&&&& 12 & 68 \\
      22 &&&&&&&&&&&&&&&& 24 & 8 \\
      23 &&&&&&&&&&&&&&&&& 121 \\
      24 &&&&&&&&&&&&&&&&&& 80 \\
      25 &&&&&&&&&&&&&&&&&&& 54 \\
      26 &&&&&&&&&&&&&&&&&&&& 12
    \end{NiceArray}$%
  }
  \renewcommand{\arraystretch}{1}
\end{table}

\begin{table}[ht]
  \centering
  \caption[Graded cohomology dimensions for the $I \subseteq \GL_{4}(\Z_{p})$ case]{Dimensions of $E_{1}^{s,t} = H^{s,t} = \gr^{s} H^{s+t}(\lie{g},k)$ for the $I \subseteq \GL_{4}(\Z_{p})$ case. This table only shows the graded dimensions of $H^{0},\dotsc,H^{7}$. We note that the graded dimensions of $H^{8},\dotsc,H^{15}$ can be found using Poincaré duality, which gives us $\dim_{k} H^{s,t} = \dim_{k} H^{-40-s,56-t}$ for $(s,t)$ from the table.}
  \label{tab:graded-coh-dims-GL4}
  \renewcommand{\arraystretch}{1.7}
  \scalebox{0.5}{%
    $\begin{NiceArray}{*{25}{c}}[hvlines, columns-width=auto]
      \diagbox{t}{s} & 0 & -1 & -2 & -3 & -4 & -5 & -6 & -7 & -8 & -9 & -10 & -11 & -12 & -13 & -14 & -15 & -16 & -17 & -18 & -19 & -20 & -21 & -22 & -23 \\
      0 & 1\\
      1 \\
      2 && 4 \\
      3 \\
      4 &&& 2 \\
      5 &&&& 8 & 1 \\
      6 &&&&& \\
      7 &&&&& 4 & 8 \\
      8 &&&&&& 8 \\
      9 &&&&&&& 22 \\
      10 &&&&&&&& 16 \\
      11 &&&&&&&& 20 & 1 \\
      12 &&&&&&&&& 38 & 4 \\
      13 &&&&&&&&&& 24 \\
      14 &&&&&&&&&& 12 & 38 \\
      15 &&&&&&&&&&& 26 & 8 \\
      16 &&&&&&&&&&& 4 & 96 & 1 \\
      17 &&&&&&&&&&&&& 73 \\
      18 &&&&&&&&&&&&& 28 & 24 \\
      19 &&&&&&&&&&&&&& 80 & 18 \\
      20 &&&&&&&&&&&&&&& 98 \\
      21 &&&&&&&&&&&&&&& 16 & 144 \\
      22 &&&&&&&&&&&&&&&& 24 & 47 \\
      23 &&&&&&&&&&&&&&&&& 149 & 8 \\
      24 &&&&&&&&&&&&&&&&&& 148 \\
      25 &&&&&&&&&&&&&&&&&& 12 & 126 \\
      26 &&&&&&&&&&&&&&&&&&& 66 & 80 \\
      27 &&&&&&&&&&&&&&&&&&&& 104 & 8 \\
      28 &&&&&&&&&&&&&&&&&&&&& 242 \\
      29 &&&&&&&&&&&&&&&&&&&&&& 104 \\
      30 &&&&&&&&&&&&&&&&&&&&&&& 66 \\
      31 &&&&&&&&&&&&&&&&&&&&&&&& 12
    \end{NiceArray}$%
  }
  \renewcommand{\arraystretch}{1}
\end{table}



\section[\texorpdfstring{$I \subseteq \SL_{2}(\sO_{F})$}{I in SL2(OF)}, quadratic]{\texorpdfstring{$I \subseteq \SL_{2}(\sO_{F})$}{I in SL3(OF)} for quadratic extensions \texorpdfstring{$F/\Q_{p}$}{F/Qp}}%
\label{sec:Iwa-SL2-F}

In this section we will describe the continuous group cohomology of the pro-$p$ Iwahori subgroup $I$ of $\SL_{2}(F)$ for quadratic extensions $F/\Q_{p}$.

We write $F = \Q_{p}(\alpha)$, and we will focus on the cases $\alpha = i$ (when $p \equiv 3 \pmod{4}$) and $\alpha = \sqrt{p}$.

When $I$ is the pro-$p$ Iwahori subgroup in $\SL_{2}(F)$, we know by \Cref{sec:cohiwagps-intro} that we can take it to be of the form
\begin{equation*}
  I = \pmat{1+\varpi_{F}\sO_{F} & \sO_{F} \\ \varpi_{F}\sO_{F} & 1+\varpi_{F}\sO_{F}}^{\!\!\det = 1} \subseteq \SL_{2}(\sO_{F}),
\end{equation*}
where $\varpi_{F} = p$ when $F = \Q_{p}(i)$ and $\varpi_{F} = \sqrt{p}$ when $F = \Q_{p}(\sqrt{p})$. By \Cref{sec:cohiwagps-intro}, we have an ordered basis
\begin{equation}
  \label{eq:gis-SL2-F}
  \begin{gathered}
    g_{1} = \pmat{ 1 & 0 \\ \varpi_{F} & 1 }, \quad g_{2} = \pmat{ 1 & 0 \\ \varpi_{F}\alpha & 1 }, \\
    g_{3} = \pmat{ \exp(\varpi_{F}) & 0 \\ 0 & \exp(-\varpi_{F}) }, \quad g_{4} = \pmat{ \exp(\varpi_{F}\alpha) & 0 \\ 0 & \exp(-\varpi_{F}\alpha) }, \\
    g_{5} = \pmat{ 1 & 1 \\ 0 & 1 }, \quad g_{6} = \pmat{ 1 & \alpha \\ 1 & 1 },
  \end{gathered}
\end{equation}
since $1,\alpha$ is a $\Z_{p}$-basis of $\sO_{F}$.


\subsection{Finding the commutators \texorpdfstring{$[\xi_{i},\xi_{j}]$}{[xi-i,xi-j]}}%
\label{subsec:non-id-xi_ij-SL2-F}

Now
\begin{equation*}
    g_{1}^{x_{1}}g_{2}^{x_{2}}g_{3}^{x_{3}}g_{4}^{x_{4}}g_{5}^{x_{5}}g_{6}^{x_{6}} = \pmat{ a_{11} & a_{12}  \\ a_{21} & a_{22} },
\end{equation*}
where
\begin{equation}
  \label{eq:gixi-SL2-F}
  \begin{aligned}
    a_{11} &= \exp\bigl( \varpi_{F}(x_{3} + \alpha x_{4}) \bigr), \\
    a_{12} &= (x_{5} + \alpha x_{6}) \exp\bigl( \varpi_{F}(x_{3} + \alpha x_{4}) \bigr), \\
    a_{21} &= \varpi_{F}(x_{1} + \alpha x_{2}) \exp\bigl( \varpi_{F}(x_{3} + \alpha x_{4}) \bigr), \\
    a_{22} &= \varpi_{F}(x_{1}+\alpha x_{2})(x_{5} + \alpha x_{6})\exp\bigl( \varpi_{F}(x_{3} + \alpha x_{4}) \bigr) + \exp\bigl( -\varpi_{F}(x_{3} + \alpha x_{4}) \bigr).
  \end{aligned}
\end{equation}

Writing $g_{ij} = [g_{i},g_{j}]$ and $\xi_{ij} = [\xi_{i},\xi_{j}]$, we are now ready to find $x_{1},\dotsc,x_{8}$ such that $g_{ij} = g_{1}^{x_{1}} \dotsb g_{6}^{x_{6}}$ for different $i<j$. (In the following we use that $\frac{1}{1-x} = 1 + x + x^{2} + \dotsb$ and $\log(1-x) = -x - \frac{x^{2}}{2} - \frac{x^{3}}{3} - \dotsb$ for $x \in (\varpi_{F})$.)

We now list all non-identity commutators $g_{ij} = [g_{i},g_{j}]$ and find $\xi_{ij} = [\xi_{i},\xi_{j}]$ based on these. (For $g_{ij} = 1_{2}$ it is clear that $x_{1} = \cdots = x_{6} = 0$, and thus $\xi_{ij} = 0$.) To avoid confusion, we will do the $F = \Q_{p}(i)$ case (with $p \equiv 3 \pmod{4}$) first, and then the $F = \Q_{p}(\sqrt{p})$ case afterwards. So for now $\varpi_{F} = p$ and $\alpha = i$. Also, note that we adapt the $O(p)$ notation to $\sO_{F}$ in the obvious way.

\begin{description}
  \item[$g_{13} = \pmat{ 1 & 0 \\ p\bigl( 1 - \exp(-2p) \bigr) & 1 }$:] Comparing $g_{13}$ with \eqref{eq:gixi-SL2-F}, we see that $x_{3} = x_{4} = x_{5} = x_{6} = 0$. This leaves $a_{21} = p(x_{1} + i x_{2}) = p\bigl( 1 - \exp(-2p) \bigr) = 2p^{2} + O(p^{3})$, which implies that $x_{1}, x_{2} \in p\Z_{p}$. Hence $\xi_{13} = 0$.

  \item[$g_{14} = \pmat{ 1 & 0 \\ p\bigl( 1 - \exp(-2pi) \bigr) & 1 }$:] Comparing $g_{14}$ with \eqref{eq:gixi-SL2-F}, we see that $x_{3} = x_{4} = x_{5} = x_{6} = 0$. This leaves $a_{21} = p(x_{1} + i x_{2}) = p\bigl( 1 - \exp(-2pi) \bigr) = 2p^{2}i + O(p^{3})$, which implies that $x_{1}, x_{2} \in p\Z_{p}$. Hence $\xi_{14} = 0$.

  \item[$g_{15} = \pmat{ 1-p & p \\ -p^{2} & 1+p+p^{2} }$:] Comparing $g_{15}$ with \eqref{eq:gixi-SL2-F}, we see that
        \begin{align*}
          a_{11} &= \exp\bigl( p(x_{3} + i x_{4}) \bigr) = 1-p, \\
          a_{12} &= (x_{5} + i x_{6}) \exp\bigl( p(x_{3} + i x_{4}) \bigr) = (x_{5} + i x_{6})(1-p) = p, \\
          a_{21} &= p(x_{1} + ix_{2}) \exp\bigl( p(x_{3} + i x_{4}) \bigr) = p(x_{1} + i x_{2})(1-p) = -p^{2},
        \end{align*}
        and thus
        \begin{align*}
          x_{3} + ix_{4} &= \dfrac{1}{p}\log(1-p) = \dfrac{1}{p}\bigl( (-p) + O(p^{2}) \bigr) = -1 + O(p), \\
          x_{5} + ix_{6} &= \dfrac{p}{1-p} = p + O(p^{2}), \\
          x_{1} + ix_{2} &= \dfrac{-p^{2}}{p(1-p)} = -p + O(p^{2}).
        \end{align*}
        Hence $x_{1},x_{2},x_{4}, x_{5},x_{6} \in p\Z_{p}$ and $x_{3} \in -1 + p\Z_{p}$, which implies that $\xi_{15} = -\xi_{3}$.

  \item[$g_{16} = \pmat{ 1-pi & pi^{2} \\ -p^{2}i & 1+pi+p^{2}i^{2} }$:] Comparing $g_{16}$ with \eqref{eq:gixi-SL2-F}, we see that
        \begin{align*}
          a_{11} &= \exp\bigl( p(x_{3} + i x_{4}) \bigr) = 1-pi, \\
          a_{12} &= (x_{5} + i x_{6}) \exp\bigl( p(x_{3} + i x_{4}) \bigr) = (x_{5} + i x_{6})(1-pi) = -p, \\
          a_{21} &= p(x_{1} + ix_{2}) \exp\bigl( p(x_{3} + i x_{4}) \bigr) = p(x_{1} + i x_{2})(1-pi) = -p^{2}i,
        \end{align*}
        and thus
        \begin{align*}
          x_{3} + ix_{4} &= \dfrac{1}{p}\log(1-pi) = \dfrac{1}{p}\bigl( (-pi) + O(p^{2}) \bigr) = -i + O(p), \\
          x_{5} + ix_{6} &= \dfrac{-p}{1-p} = -p + O(p^{2}), \\
          x_{1} + ix_{2} &= \dfrac{-p^{2}i}{p(1-pi)} = -pi + O(p^{2}).
        \end{align*}
        Hence $x_{1},x_{2},x_{3}, x_{5},x_{6} \in p\Z_{p}$ and $x_{4} \in -1 + p\Z_{p}$, which implies that $\xi_{16} = -\xi_{4}$.

  \item[$g_{23} = \pmat{ 1 & 0 \\ pi\bigl( 1 - \exp(-2p) \bigr) & 1 }$:] Comparing $g_{23}$ with $g_{13}$, it is not hard to see that $\xi_{23} = 0$.

  \item[$g_{24} = \pmat{ 1 & 0 \\ pi\bigl( 1 - \exp(-2pi) \bigr) & 1 }$:] Comparing $g_{24}$ with $g_{14}$, it is not hard to see that $\xi_{24} = 0$.

  \item[$g_{25} = \pmat{ 1-pi & pi \\ -p^{2}i^{2} & 1+pi+p^{2}i^{2} }$:] Comparing $g_{25}$ with $g_{16}$, it is not hard to see that $\xi_{25} = -\xi_{4}$.

  \item[$g_{26} = \pmat{ 1-pi^{2} & pi^{3} \\ -p^{2}i^{3} & 1+pi^{2}+p^{2}i^{4} }$:] Comparing $g_{26}$ with $g_{15}$ (noting that $i^{2} = -1$, so $1-pi^{2} = 1+p$), it is not hard to see that $\xi_{25} = \xi_{3}$.

  \item[$g_{35} = \pmat{ 1 & \exp(2p)-1 \\ 0 & 1 }$:] Comparing $g_{35}$ with \eqref{eq:gixi-SL2-F}, we see that $x_{1} = x_{2} = x_{3} = x_{4} = 0$. This leaves $a_{12} = x_{5} + i x_{6} = \exp(2p)-1 = 2p + O(p^{2})$, which implies that $x_{5}, x_{6} \in p\Z_{p}$. Hence $\xi_{35} = 0$.

  \item[$g_{36} = \pmat{ 1 & i\bigl(\exp(2p)-1\bigr) \\ 0 & 1 }$:] Comparing $g_{36}$ with $g_{35}$, it is not hard to see that $\xi_{36} = 0$.

  \item[$g_{45} = \pmat{ 1 & \exp(2pi)-1 \\ 0 & 1 }$:] Comparing $g_{45}$ with \eqref{eq:gixi-SL2-F}, we see that $x_{1} = x_{2} x_{3} = x_{4} = 0$. This leaves $a_{12} = x_{5} + i x_{6} = \exp(2pi)-1 = 2pi + O(p^{2})$, which implies that $x_{5}, x_{6} \in p\Z_{p}$. Hence $\xi_{45} = 0$.

  \item[$g_{46} = \pmat{ 1 & i\bigl( \exp(2pi)-1 \bigr) \\ 0 & 1 }$:] Comparing $g_{46}$ with $g_{45}$, it is not hard to see that $\xi_{45} = 0$.
\end{description}

Let's now do the calculations for $F = \Q_{p}(\sqrt{p})$, and note that we again adapt the $O(p)$ notation in the obvious way. Now $\varpi_{F} = \sqrt{p}$ and $\alpha = \sqrt{p}$.

\begin{description}
  \item[$g_{13} = \pmat{ 1 & 0 \\ \sqrt{p}\bigl( 1 - \exp(-2\sqrt{p}) \bigr) & 1 }$:] Comparing $g_{13}$ with \eqref{eq:gixi-SL2-F}, we see that $x_{3} = x_{4} = x_{5} = x_{6} = 0$. This leaves $a_{21} = \sqrt{p}(x_{1} + \sqrt{p} x_{2}) = \sqrt{p}\bigl( 1 - \exp(-2\sqrt{p}) \bigr) = 2p + O(p^{3/2})$, which implies that $x_{1} \in p\Z_{p}$ and $x_{2} \in 2 + p\Z_{p}$. Hence $\xi_{13} = 2\xi_{2}$.

  \item[$g_{14} = \pmat{ 1 & 0 \\ \sqrt{p}\bigl( 1 - \exp(-2p) \bigr) & 1 }$:] Comparing $g_{14}$ with \eqref{eq:gixi-SL2-F}, we see that $x_{3} = x_{4} = x_{5} = x_{6} = 0$. This leaves $a_{21} = \sqrt{p}(x_{1} + \sqrt{p} x_{2}) = \sqrt{p}\bigl( 1 - \exp(-2p) \bigr) = 2p\sqrt{p} + O(p^{2})$, which implies that $x_{1}, x_{2} \in p\Z_{p}$. Hence $\xi_{14} = 0$.

  \item[$g_{15} = \pmat{ 1-\sqrt{p} & \sqrt{p} \\ -p & 1+\sqrt{p}+p }$:] Comparing $g_{15}$ with \eqref{eq:gixi-SL2-F}, we see that
        \begin{align*}
          a_{11} &= \exp\bigl( \sqrt{p}(x_{3} + \sqrt{p} x_{4}) \bigr) = 1-\sqrt{p}, \\
          a_{12} &= (x_{5} + \sqrt{p} x_{6}) \exp\bigl( \sqrt{p}(x_{3} + \sqrt{p} x_{4}) \bigr) = (x_{5} + \sqrt{p} x_{6})(1-\sqrt{p}) = \sqrt{p}, \\
          a_{21} &= \sqrt{p}(x_{1} + \sqrt{p}x_{2}) \exp\bigl( \sqrt{p}(x_{3} + \sqrt{p} x_{4}) \bigr) = \sqrt{p}(x_{1} + \sqrt{p} x_{2})(1-\sqrt{p}) = -p,
        \end{align*}
        and thus
        \begin{align*}
          x_{3} + \sqrt{p} x_{4} &= \dfrac{1}{\sqrt{p}}\log(1-\sqrt{p}) = \dfrac{1}{\sqrt{p}}\bigl( (-\sqrt{p}) - \frac{p}{2} + O(p^{3/2}) \bigr) = -1 - \frac{\sqrt{p}}{2} + O(p), \\
          x_{5} + \sqrt{p} x_{6} &= \dfrac{\sqrt{p}}{1-\sqrt{p}} = \sqrt{p} + O(p), \\
          x_{1} + \sqrt{p} x_{2} &= \dfrac{-p}{\sqrt{p}(1-\sqrt{p})} = -\sqrt{p} + O(p).
        \end{align*}
        Hence $x_{1}, x_{5} \in p\Z_{p}$, $x_{2}, x_{3}, x_{4} \in -1 + p\Z_{p}$ and $x_{6} \in 1 + p\Z_{p}$, which implies that $\xi_{15} = -\xi_{2} - \xi_{3} - \frac{1}{2}\xi_{4} + \xi_{6}$. (Here $-\frac{1}{2} = \frac{p-1}{2}$ since the characteristic of $k$ is $p$.)

  \item[$g_{16} = \pmat{ 1-p & p\sqrt{p} \\ -p\sqrt{p} & 1+p+p^{2} }$:] Comparing $g_{16}$ with \eqref{eq:gixi-SL2-F}, we see that
        \begin{align*}
          a_{11} &= \exp\bigl( \sqrt{p}(x_{3} + \sqrt{p} x_{4}) \bigr) = 1-p, \\
          a_{12} &= (x_{5} + \sqrt{p} x_{6}) \exp\bigl( \sqrt{p}(x_{3} + \sqrt{p} x_{4}) \bigr) = (x_{5} + \sqrt{p} x_{6})(1-p) = p\sqrt{p}, \\
          a_{21} &= \sqrt{p}(x_{1} + \sqrt{p} x_{2}) \exp\bigl( \sqrt{p}(x_{3} + \sqrt{p} x_{4}) \bigr) = \sqrt{p}(x_{1} + \sqrt{p} x_{2})(1-p) = -p\sqrt{p},
        \end{align*}
        and thus
        \begin{align*}
          x_{3} + \sqrt{p} x_{4} &= \dfrac{1}{\sqrt{p}}\log(1-p) = \dfrac{1}{\sqrt{p}}\bigl( (-p) + O(p^{3/2}) \bigr) = -\sqrt{p} + O(p), \\
          x_{5} + \sqrt{p} x_{6} &= \dfrac{p\sqrt{p}}{1-p} = p\sqrt{p} + O(p^{2}), \\
          x_{1} + \sqrt{p} x_{2} &= \dfrac{-p\sqrt{p}}{\sqrt{p}(1-p)} = -p + O(p^{3/2}).
        \end{align*}
        Hence $x_{1},x_{2},x_{3}. x_{5},x_{6} \in p\Z_{p}$ and $x_{4} \in -1 + p\Z_{p}$, which implies that $\xi_{16} = -\xi_{4}$.

  \item[$g_{23} = \pmat{ 1 & 0 \\ p\bigl( 1 - \exp(-2\sqrt{p}) \bigr) & 1 }$:] Comparing $g_{23}$ with $g_{13}$, it is not hard to see that $\xi_{23} = 0$.

  \item[$g_{24} = \pmat{ 1 & 0 \\ p\bigl( 1 - \exp(-2p) \bigr) & 1 }$:] Comparing $g_{24}$ with $g_{14}$, it is not hard to see that $\xi_{24} = 0$.

  \item[$g_{25} = \pmat{ 1-p & p \\ -p^{2} & 1+p+p^{2} }$:] Comparing $g_{25}$ with \eqref{eq:gixi-SL2-F}, we see that
        \begin{align*}
          a_{11} &= \exp\bigl( \sqrt{p}(x_{3} + \sqrt{p} x_{4}) \bigr) = 1-p, \\
          a_{12} &= (x_{5} + \sqrt{p} x_{6}) \exp\bigl( \sqrt{p}(x_{3} + \sqrt{p} x_{4}) \bigr) = (x_{5} + \sqrt{p} x_{6})(1-p) = p, \\
          a_{21} &= \sqrt{p}(x_{1} + \sqrt{p} x_{2}) \exp\bigl( \sqrt{p}(x_{3} + \sqrt{p} x_{4}) \bigr) = \sqrt{p}(x_{1} + \sqrt{p} x_{2})(1-p) = -p^{2},
        \end{align*}
        and thus
        \begin{align*}
          x_{3} + \sqrt{p} x_{4} &= \dfrac{1}{\sqrt{p}}\log(1-p) = \dfrac{1}{\sqrt{p}}\bigl( (-p) + O(p^{3/2}) \bigr) = -\sqrt{p} + O(p), \\
          x_{5} + \sqrt{p} x_{6} &= \dfrac{p}{1-p} = p + O(p^{2}), \\
          x_{1} + \sqrt{p} x_{2} &= \dfrac{-p^{2}}{\sqrt{p}(1-p)} = -p^{3/2} + O(p^{2}).
        \end{align*}
        Hence $x_{1},x_{2},x_{3}. x_{5},x_{6} \in p\Z_{p}$ and $x_{4} \in -1 + p\Z_{p}$, which implies that $\xi_{25} = -\xi_{4}$.

  \item[$g_{26} = \pmat{ 1-p\sqrt{p} & p^{2} \\ -p^{2}\sqrt{p} & 1+p\sqrt{p}+p^{3} }$:] Comparing $g_{26}$ with \eqref{eq:gixi-SL2-F}, we see that
        \begin{align*}
          a_{11} &= \exp\bigl( \sqrt{p}(x_{3} + \sqrt{p} x_{4}) \bigr) = 1-p\sqrt{p}, \\
          a_{12} &= (x_{5} + \sqrt{p} x_{6}) \exp\bigl( \sqrt{p}(x_{3} + \sqrt{p} x_{4}) \bigr) = (x_{5} + \sqrt{p} x_{6})(1-p\sqrt{p}) = p^{2}, \\
          a_{21} &= \sqrt{p}(x_{1} + \sqrt{p}x_{2}) \exp\bigl( \sqrt{p}(x_{3} + \sqrt{p} x_{4}) \bigr) = \sqrt{p}(x_{1} + \sqrt{p} x_{2})(1-p\sqrt{p}) = -p^{2}\sqrt{p},
        \end{align*}
        and thus
        \begin{align*}
          x_{3} + \sqrt{p} x_{4} &= \dfrac{1}{\sqrt{p}}\log(1-p\sqrt{p}) = \dfrac{1}{\sqrt{p}}\bigl( (-p\sqrt{p}) + O(p^{2}) \bigr) = -p + O(p^{3/2}), \\
          x_{5} + \sqrt{p} x_{6} &= \dfrac{p^{2}}{1-p\sqrt{p}} = p^{2} + O(p^{5/2}), \\
          x_{1} + \sqrt{p} x_{2} &= \dfrac{-p^{2}\sqrt{p}}{\sqrt{p}(1-p\sqrt{p})} = -p^{2} + O(p^{5/2}).
        \end{align*}
        Hence $x_{1},x_{2},x_{3},x_{4}, x_{5},x_{6} \in p\Z_{p}$, which implies that $\xi_{26} = 0$.

  \item[$g_{35} = \pmat{ 1 & \exp(2\sqrt{p})-1 \\ 0 & 1 }$:] Comparing $g_{35}$ with \eqref{eq:gixi-SL2-F}, we see that $x_{1} = x_{2} = x_{3} = x_{4} = 0$. This leaves $a_{12} = x_{5} + i x_{6} = \exp(2\sqrt{p})-1 = 2\sqrt{p} + O(p)$, which implies that $x_{5} \in p\Z_{p}$ and $x_{6} \in 2 + p\Z_{p}$. Hence $\xi_{35} = 2\xi_{6}$.

  \item[$g_{36} = \pmat{ 1 & \sqrt{p}\bigl(\exp(2\sqrt{p})-1\bigr) \\ 0 & 1 }$:] Comparing $g_{36}$ with $g_{35}$, it is not hard to see that $\xi_{36} = 0$.

  \item[$g_{45} = \pmat{ 1 & \exp(2p)-1 \\ 0 & 1 }$:] Comparing $g_{45}$ with \eqref{eq:gixi-SL2-F}, we see that $x_{1} = x_{2} = x_{3} = x_{4} = 0$. This leaves $a_{12} = x_{5} + i x_{6} = \exp(2p)-1 = 2p + O(p^{2})$, which implies that $x_{5}, x_{6} \in p\Z_{p}$. Hence $\xi_{45} = 0$.

  \item[$g_{46} = \pmat{ 1 & \sqrt{p}\bigl( \exp(2p)-1 \bigr) \\ 0 & 1 }$:] Comparing $g_{46}$ with $g_{45}$, it is not hard to see that $\xi_{45} = 0$.
\end{description}

In summary, the only non-zero commutators $[\xi_{i},\xi_{j}]$ with $i<j$ are
\begin{equation}
  \label{eq:xi_ij-SL2-F-unram}
  \begin{aligned}
    [\xi_{1},\xi_{5}] &= -\xi_{3}, & [\xi_{1},\xi_{6}] &= -\xi_{4}, \\
    [\xi_{2},\xi_{5}] &= -\xi_{4}, & [\xi_{2},\xi_{6}] &= \xi_{3},
  \end{aligned}
\end{equation}
when $p\equiv3 \pmod{4}$ and $F = \Q_{p}(i)$, and
\begin{align*}
  [\xi_{1},\xi_{3}] &= 2\xi_{2} & [\xi_{1},\xi_{5}] &= -\xi_{2} - \xi_{3} - \frac{1}{2}\xi_{4} + \xi_{6}, \\
  [\xi_{1},\xi_{6}] &= -\xi_{4}, & [\xi_{2},\xi_{5}] &= -\xi_{4}, \\
  [\xi_{3},\xi_{5}] &= 2\xi_{6},
\end{align*}
when $F = \Q_{p}(\sqrt{p})$. To make the $F = \Q_{p}(\sqrt{p})$ case easier to work with, we make a base change $\xi_{5}' = \xi_{5} - \frac{1}{2}\xi_{6}$ and $\xi_{i}' = \xi_{i}$ for $i \neq 5$, which gives us commutators
\begin{equation}
  \label{eq:xi_ij-SL2-F-ram}
  \begin{aligned}
    [\xi_{1}',\xi_{3}'] &= 2\xi_{2}' & [\xi_{1}',\xi_{5}'] &= -\xi_{2}' - \xi_{3}' + \xi_{6}', \\
    [\xi_{1}',\xi_{6}'] &= -\xi_{4}', & [\xi_{2}',\xi_{5}'] &= -\xi_{4}', \\
    [\xi_{3}',\xi_{5}'] &= 2\xi_{6}'.
  \end{aligned}
\end{equation}


\subsection{Finding the cohomology}%
\label{subsec:graded-coh-SL2-F}

Looking at \eqref{eq:Iwa-p-val-basis-SLn} in the $p\equiv 3\pmod{4}$ and $F = \Q_{p}(i)$ case (with $e=1$ and $h=2$), we see that
\begin{align*}
  \omega(g_{1}) &= 1-\frac{1}{2} = \frac{1}{2}, & \omega(g_{2}) &= 1-\frac{1}{2} = \frac{1}{2}, \\
  \omega(g_{3}) &= 1, & \omega(g_{4}) &= 1, \\
  \omega(g_{5}) &= \frac{1}{2}, & \omega(g_{6}) &= \frac{1}{2}.
\end{align*}
Hence
\begin{equation*}
  \lie{g} = k \otimes_{\F_{p}[\pi]} \gr I = \Span_{k}(\xi_{1},\dotsc,\xi_{6}) = \lie{g}^{1} \oplus \lie{g}^{2},
\end{equation*}
where
\begin{align*}
  \lie{g}^{1} &= \lie{g}_{\frac{1}{2}} = \Span_{k}(\xi_{1},\xi_{2},\xi_{5},\xi_{6}), \\
  \lie{g}^{2} &= \lie{g}_{1} = \Span_{k}(\xi_{3},\xi_{4}).
\end{align*}

In the $F = \Q_{p}(\sqrt{p})$ case (with $e=2$ and $h=2$) \eqref{eq:Iwa-p-val-basis-SLn} gives us
\begin{align*}
  \omega(g_{1}) &= \frac{1}{2}\Bigl( 1-\frac{1}{2} \Bigr) = \frac{1}{4}, & \omega(g_{2}) &= \frac{1}{2}\Bigl( 1-\frac{1}{2} \Bigr) = \frac{1}{4}, \\
  \omega(g_{3}) &= \frac{1}{2}, & \omega(g_{4}) &= \frac{1}{2}, \\
  \omega(g_{5}) &= \frac{1}{4}, & \omega(g_{6}) &= \frac{1}{4}.
\end{align*}
Hence
\begin{equation*}
  \lie{g} = k \otimes_{\F_{p}[\pi]} \gr I = \Span_{k}(\xi_{1}',\dotsc,\xi_{6}') = \lie{g}^{1} \oplus \lie{g}^{2},
\end{equation*}
where
\begin{align*}
  \lie{g}^{1} &= \lie{g}_{\frac{1}{4}} = \Span_{k}(\xi_{1}',\xi_{2}',\xi_{5}',\xi_{6}'), \\
  \lie{g}^{2} &= \lie{g}_{\frac{1}{2}} = \Span_{k}(\xi_{3}',\xi_{4}').
\end{align*}
See \Cref{rem:g-Z-grading} for more details.

This is enough to calculate the graded mod $p$ cohomology of $\lie{g}$, see \cite{code} for the details. We write the result in \Cref{tab:graded-coh-dims-GL2-F}.

\begin{table}[ht]
  \centering
  \caption[Graded cohomology dimensions for the $I \subseteq \SL_{2}(\sO_{F})$, quadratic case]{Dimensions of $E_{1}^{s,t} = H^{s,t}(\lie{g},\F_{p})$ for the $I \subseteq \SL_{2}(\sO_{F})$ case, where $F/\Q_{p}$ is a quadratic extension (either $F = \Q_{p}(i)$ or $F = \Q_{p}(\sqrt{p})$).}
  \label{tab:graded-coh-dims-SL2-F}
  \renewcommand{\arraystretch}{1.5}
  $\begin{NiceArray}{*{10}{c}}[hvlines, columns-width=auto]
    \diagbox{t}{s} & 0 & -1 & -2 & -3 & -4 & -5 & -6 & -7 & -8 \\
    0 & 1 \\
    1 \\
    2 & & 4 \\
    3 \\
    4 & & & 4 \\
    5 & & & & 4 \\
    6 & \\
    7 & & & & & 10 \\
    8 \\
    9 & & & & & & 4 \\
    10 & & & & & & & 4 \\
    11 \\
    12 & & & & & & & & 4 \\
    13 \\
    14 & & & & & & & & & 1
  \end{NiceArray}$
  \renewcommand{\arraystretch}{1}
\end{table}

Altogether, we see that
\begin{equation}
  \label{eq:Hn-to-Hst-SL2-F}
  \begin{aligned}
    H^{0} &= H^{0,0}, \\
    H^{1} &= H^{-1,2}, \\
    H^{2} &= H^{-2,4} \oplus H^{-3,5}, \\
    H^{3} &= H^{-4,7}, \\
    H^{4} &= H^{-5,9} \oplus H^{-6,10}, \\
    H^{5} &= H^{-7,12}, \\
    H^{6} &= H^{-8,14},
  \end{aligned}
\end{equation}
with dimension as described in \Cref{tab:graded-coh-dims-SL2-F} in both the $F = \Q_{p}(i)$ and the $F = \Q_{p}(\sqrt{p})$ case. I.e., the mod $p$ cohomology does not depend on the field extension (among the above ones) in this case.

We note that all differentials $d_{r}^{s,t} \colon E_{r}^{s,t} \to E_{r}^{s+r,t+1-r}$ in \Cref{tab:graded-coh-dims-SL2-F} have bidegree $(r,1-r)$, i.e., they are all below the $(r,-r)$ arrow going $r$ to the left and $r$ up in the table, where $r \geq 1$. Looking at \Cref{tab:graded-coh-dims-SL2-F}, this clearly means that all differentials for $r \geq 1$ are trivial, and thus the spectral sequence collapses on the first page. Hence $H^{s,t}(\lie{g},k) = E_{1}^{s,t} \iso E_{\infty}^{s,t} = \gr^{s} H^{s+t}(I,k)$, and by \eqref{eq:Hn-to-Hst-SL2-F} and \Cref{tab:graded-coh-dims-SL2-F} we get that
\begin{equation}
  \label{eq:dim-HnI-SL2-F}
  \dim H^{n}(I,k) =
  \begin{dcases}
    1 & n=0, \\
    4 & n=1, \\
    8 & n=2, \\
    10 & n=3, \\
    8 & n=4, \\
    4 & n=5, \\
    1 & n=6.
  \end{dcases}
\end{equation}


\section[\texorpdfstring{$I \subseteq \GL_{2}(\sO_{F})$}{I in GL2(OF)}, quadratic]{\texorpdfstring{$I \subseteq \GL_{2}(\sO_{F})$}{I in GL2(OF)}}%
\label{sec:Iwa-GL2-F}

In this section we will describe the continuous group cohomology of the pro-$p$ Iwahori subgroup $I$ of $\GL_{2}(F)$ for quadratic extensions $F/\Q_{p}$.

We again write $F = \Q_{p}(\alpha)$ and focus on the cases $\alpha = i$ (when $p \equiv 3 \pmod{4}$) and $\alpha = \sqrt{p}$.

When $I$ is the pro-$p$ Iwahori subgroup in $\GL_{2}(F)$, we know by \Cref{sec:cohiwagps-intro} that we can take it to be of the form
\begin{equation*}
  I = \pmat{1+\varpi_{F}\sO_{F} & \sO_{F} \\ \varpi_{F}\sO_{F} & 1+\varpi_{F}\sO_{F}} \subseteq \GL_{2}(\sO_{F}),
\end{equation*}
where $\varpi_{F} = p$ when $F = \Q_{p}(i)$ and $\varpi_{F} = \sqrt{p}$ when $F = \Q_{p}(\sqrt{p})$. By \Cref{sec:cohiwagps-intro}, we have an ordered basis
\begin{equation}
  \label{eq:gis-GL2-F}
  \begin{gathered}
    g_{1} = \pmat{ 1 & 0 \\ \varpi_{F} & 1 }, \quad g_{2} = \pmat{ 1 & 0 \\ \varpi_{F}\alpha & 1 }, \\
    g_{3} = \pmat{ \exp(\varpi_{F}) & 0 \\ 0 & \exp(-\varpi_{F}) }, \quad g_{4} = \pmat{ \exp(\varpi_{F}\alpha) & 0 \\ 0 & \exp(-\varpi_{F}\alpha) }, \\
    g_{5} = \pmat{ \exp(\varpi_{F}) & 0 \\ 0 & \exp(\varpi_{F}) }, \quad g_{6} = \pmat{ \exp(\varpi_{F}\alpha) & 0 \\ 0 & \exp(\varpi_{F}\alpha) }, \\
    g_{7} = \pmat{ 1 & 1 \\ 0 & 1 }, \quad g_{8} = \pmat{ 1 & \alpha \\ 1 & 1 },
  \end{gathered}
\end{equation}
since $1,\alpha$ is a $\Z_{p}$-basis of $\sO_{F}$.

Since we just renamed some elements and added an element of the center of $\GL(\sO_{F})$ when comparing to the ordered basis of $I \subseteq \SL_{2}(\sO_{F})$ from \Cref{sec:Iwa-SL2-F}, it is clear from \Cref{eq:xi_ij-SL2-F-unram} and \Cref{eq:xi_ij-SL2-F-ram} that the only non-zero commutators $[\xi_{i},\xi_{j}]$  (with $i<j$) in  $\lie{g} = k \otimes \gr G$ are
\begin{equation}
  \label{eq:xi_ij-GL2-F-unram}
  \begin{aligned}
    [\xi_{1},\xi_{7}] &= -\xi_{3}, & [\xi_{1},\xi_{8}] &= -\xi_{4}, \\
    [\xi_{2},\xi_{7}] &= -\xi_{4}, & [\xi_{2},\xi_{8}] &= \xi_{3},
  \end{aligned}
\end{equation}
when $p\equiv3 \pmod{4}$ and $F = \Q_{p}(i)$, and
\begin{align*}
  [\xi_{1},\xi_{3}] &= 2\xi_{2} & [\xi_{1},\xi_{7}] &= -\xi_{2} - \xi_{3} - \frac{1}{2}\xi_{4} + \xi_{8}, \\
  [\xi_{1},\xi_{8}] &= -\xi_{4}, & [\xi_{2},\xi_{7}] &= -\xi_{4}, \\
  [\xi_{3},\xi_{7}] &= 2\xi_{8},
\end{align*}
when $F = \Q_{p}(\sqrt{p})$. To make the $F = \Q_{p}(\sqrt{p})$ case easier to work with, we again make a base change $\xi_{7}' = \xi_{7} - \frac{1}{2}\xi_{8}$ and $\xi_{i}' = \xi_{i}$ for $i \neq 6$, which gives us commutators
\begin{equation}
  \label{eq:xi_ij-GL2-F-ram}
  \begin{aligned}
    [\xi_{1}',\xi_{3}'] &= 2\xi_{2}' & [\xi_{1}',\xi_{7}'] &= -\xi_{2}' - \xi_{3}' + \xi_{8}', \\
    [\xi_{1}',\xi_{8}'] &= -\xi_{4}', & [\xi_{2}',\xi_{7}'] &= -\xi_{4}', \\
    [\xi_{3}',\xi_{7}'] &= 2\xi_{8}'.
  \end{aligned}
\end{equation}

Looking at \Cref{subsec:graded-coh-SL2-F}, we easily see that
\begin{align*}
  \lie{g}^{1} &= \Span_{k}(\xi_{1},\xi_{2},\xi_{7},\xi_{8}), \\
  \lie{g}^{2} &= \Span_{k}(\xi_{3},\xi_{4},\xi_{5},\xi_{6}),
\end{align*}
in both cases.

This is enough to calculate the graded mod $p$ cohomology of $\lie{g}$, see \cite{code} for the details. We write the result in \Cref{tab:graded-coh-dims-GL2-F}.

\begin{table}[ht]
  \centering
  \caption[Graded cohomology dimensions for the $I \subseteq \GL_{2}(\sO_{F})$, quadratic case]{Dimensions of $E_{1}^{s,t} = H^{s,t}(\lie{g},\F_{p})$ for the $I \subseteq \GL_{2}(\sO_{F})$ case, where $F/\Q_{p}$ is a quadratic extension (either $F = \Q_{p}(i)$ or $F = \Q_{p}(\sqrt{p})$).}
  \label{tab:graded-coh-dims-GL2-F}
  \renewcommand{\arraystretch}{1.7}
  \scalebox{0.8}{%
  $\begin{NiceArray}{*{14}{c}}[hvlines, columns-width=auto]
    \diagbox{t}{s} & 0 & -1 & -2 & -3 & -4 & -5 & -6 & -7 & -8 & -9 & -10 & -11 & -12 \\
    0 & 1 \\
    1 \\
    2 & & 4 \\
    3 & & & 2 \\
    4 & & & 4 \\
    5 & & & & 12 \\
    6 & & & & & 1 \\
    7 & & & & & 18 \\
    8 & & & & & & 12 \\
    9 & & & & & & 4 \\
    10 & & & & & & & 28 \\
    11 & & & & & & & & 4 \\
    12 & & & & & & & & 12 \\
    13 & & & & & & & & & 18 \\
    14 & & & & & & & & & 1 \\
    15 & & & & & & & & & & 12 \\
    16 & & & & & & & & & & & 4 \\
    17 & & & & & & & & & & & 2 \\
    18 & & & & & & & & & & & & 4 \\
    19 \\
    20 & & & & & & & & & & & & & 1
  \end{NiceArray}$%
  }
  \renewcommand{\arraystretch}{1}
\end{table}

\clearpage
Altogether, we see that
\begin{equation}
  \label{eq:Hn-to-Hst-GL2-F}
  \begin{aligned}
    H^{0} &= H^{0,0}, \\
    H^{1} &= H^{-1,2} \oplus H^{-2,3}, \\
    H^{2} &= H^{-2,4} \oplus H^{-3,5} \oplus H^{-4,6}, \\
    H^{3} &= H^{-4,7} \oplus H^{-5,8}, \\
    H^{4} &= H^{-5,9} \oplus H^{-6,10} \oplus H^{-7,11}, \\
    H^{5} &= H^{-7,12} \oplus H^{-8,13}, \\
    H^{6} &= H^{-8,14} \oplus H^{-9,15} \oplus H^{-10,16}, \\
    H^{7} &= H^{-10,17} \oplus H^{-11,18}, \\
    H^{8} &= H^{-12,20}
  \end{aligned}
\end{equation}
with dimension as described in \Cref{tab:graded-coh-dims-GL2-F} in both the $F = \Q_{p}(i)$ and the $F = \Q_{p}(\sqrt{p})$ case. I.e., the mod $p$ cohomology does not depend on the field extension (among the above ones) in this case.

We note that all differentials $d_{r}^{s,t} \colon E_{r}^{s,t} \to E_{r}^{s+r,t+1-r}$ in \Cref{tab:graded-coh-dims-GL2-F} have bidegree $(r,1-r)$, i.e., they are all below the $(r,-r)$ arrow going $r$ to the left and $r$ up in the table, where $r \geq 1$. Looking at \Cref{tab:graded-coh-dims-GL2-F}, this clearly means that all differentials for $r \geq 1$ are trivial, and thus the spectral sequence collapses on the first page. Hence $H^{s,t}(\lie{g},k) = E_{1}^{s,t} \iso E_{\infty}^{s,t} = \gr^{s} H^{s+t}(I,k)$, and by \eqref{eq:Hn-to-Hst-GL2-F} and \Cref{tab:graded-coh-dims-GL2-F} we get that
\begin{equation}
  \label{eq:dim-HnI-GL2-F}
  \dim H^{n}(I,k) =
  \begin{dcases}
    1 & n=0, \\
    6 & n=1, \\
    17 & n=2, \\
    30 & n=3, \\
    36 & n=4, \\
    30 & n=5, \\
    17 & n=6, \\
    6 & n=7, \\
    1 & n=8.
  \end{dcases}
\end{equation}

\section{Nilpotency index}%
\label{sec:nilp-index}

Before ending this chapter with a brief discussion of future research directions, we will mention an interesting consequence of our above calculations.

Given any cohomology theory $H$ (say over $k$), one can always think of the ring $H^{*}$ with the cup product as $H^{*} = k \oplus H^{+}$, where $k = H^{0}$ and $H^{+} = \bigoplus_{n>0} H^{n}$. Assuming that only finitely many $H^{n}$ are non-zero and that each $H^{n}$ is finite dimensional, one can note that $H^{+}$ must be nilpotent. Thus an interesting question becomes: what is the nilpotency index of $H^{+}$? I.e., what is the smallest positive integer $m$ such that $(H^{+})^{m} = 0$? In continuation of this, another slightly easier question to answer is, what is the nilpotency index of $H^{1}$? I.e., what is the smallest positive integer $m$ such that $(H^{1})^{m} = 0$.

% Furthermore, one can consider the exterior algebra $\Lambda(H^{*})$, which by definition is $T(H^{*})/I$, where $T(H^{*}) = \bigoplus_{\ell=0}^{\infty} (H^{*})^{\otimes \ell}$ is the tensor algebra with multiplication given by the canonical isomorphisms $(H^{*})^{\otimes m} \otimes (H^{*})^{\otimes \ell} \to (H^{*})^{\otimes(m+\ell)}$, and $I$ is the two-sided ideal generated by all elements of the form $x \otimes x$. Now another interesting question is, what's the smallest number of wedges needed to ensure that $H^{*} \wedge \dotsb \wedge H^{*} = 0$? Or simpler, what's the smallest number of wedges needed to ensure that $H^{1} \wedge \dotsb \wedge H^{1} = 0$? \dknote{Is this the correct setup? Otherwise, how to change it? Answer this part later. Should be fine for $\SL_{2}(\Z_{p})$ and $\GL_{2}(\Z_{p})$ cases.}

We will now try to answer the above questions for the group cohomology $H^{*}(I,k)$ in each of the cases we have discussed in this chapter. Before beginning, recall that
\begin{equation}\label{eq:graded-coh-inc}
  H^{s,t} \cup H^{s',t'} \subseteq H^{s+s',t+t'}
\end{equation}
by \cite[Chap.~1~§3.7]{Fuks}. This will be useful for finding upper bounds for the nilpotency index. Also, note that we write $H^{+} \cup H^{+}$ for the image of $\cup \colon H^{+} \times H^{+} \to H^{+}$, and similarly for $H^{1}$.

In the $I \subseteq \SL_{2}(\Z_{p})$ case, we saw in \eqref{eq:dim-HnI-SL2} that the cup product is trivial except for $\cup \colon H^{1} \times H^{2} \to H^{3}$, so $H^{1} \cup H^{1} = 0$ and
\begin{align*}
  H^{+} \cup H^{+} &\neq 0, & H^{+} \cup H^{+} \cup H^{+} &= 0.
\end{align*}

In the $I \subseteq \GL_{2}(\Z_{p})$ case, we completely described the (graded) cup product in \eqref{eq:dim-HnI-GL2}, which should be enough to answer the questions. Looking at \Cref{tab:graded-coh-dims-GL2} and using \eqref{eq:graded-coh-inc}, we see that an upper bound for $H^{1}$ is that
\begin{align*}
  H^{1} \cup H^{1} \cup H^{1} &\neq 0, & H^{1} \cup H^{1} \cup H^{1} \cup H^{1} &= 0,
\end{align*}
by starting with $H^{-1,2} \cup H^{-2,3} \subseteq H^{-3,5} \neq 0$ and then using that $H^{-3,5} \cup H^{-1,2} \subseteq H^{-4,7} \neq 0$ or $H^{-3,5} \cup H^{-2,3} \subseteq H^{-5,8} \neq 0$, and finally  $H^{-4,7} \cup H^{-2,3} \subseteq H^{-6,10} \neq 0$ or $H^{-5,8} \cup H^{-1,2} \subseteq H^{-6,10}\neq0$. The question is whether we can follow those steps with non-zero cup products. We note by \eqref{eq:cup-products-GL2} that
\begin{align*}
  e_{1} \cup e_{3} &= e_{1,3}, & e_{4} \cup e_{3} &= e_{4,3},
\end{align*}
are the only non-zero cup product we can do in $H^{1} = H^{-1,2} \oplus H^{-2,3}$. But $H^{-1,2} = k[e_{1},e_{4}]$ and $H^{-2,3} = k[e_{3}]$ by \eqref{eq:Hst-spaces-GL2}, and we already noted in \Cref{subsec:group-coh-GL2} that $e_{i_{1},\dotsc,i_{m}} \cup e_{j_{1},\dotsc,j_{\ell}}$ if $\set{i_{1},\dotsc,i_{m}} \cap \set{j_{1},\dotsc,j_{\ell}} \neq \emptyset$, so we clearly cannot cup with anything from $H^{1}$ without getting zero. Thus
\begin{align*}
  H^{1} \cup H^{1} &\neq 0, & H^{1} \cup H^{1} \cup H^{1} &= 0.
\end{align*}
Now, having only four possible numbers in the subscript and using the above equation, we note that we can only ever hope to have two cup products before getting zero (cf.\ \eqref{eq:Hst-spaces-GL2}). By \eqref{eq:cup-products-GL2}
\begin{equation*}
  e_{3} \cup (e_{1} \cup e_{4,2}) = e_{3} \cup e_{1,4,2} = -e_{1,4,2,3} \neq 0,
\end{equation*}
so
\begin{align*}
  H^{+} \cup H^{+} \cup H^{+} &\neq 0, & H^{+} \cup H^{+} \cup H^{+} \cup H^{+} &= 0,
\end{align*}
for $I \subseteq \GL_{2}(\Z_{p})$.

In the $I \subseteq \SL_{3}(\Z_{p})$ case, we have not described the cup product in detail, but we can tell purely from \eqref{eq:graded-coh-inc} and \Cref{tab:graded-coh-dims-SL3}, that $H^{1} \cup H^{1} = 0$. Going through \Cref{tab:graded-coh-dims-SL3}, we also note that an upper bound for $H^{+}$ is
\begin{align*}
  H^{+} \cup H^{+} \cup H^{+} \cup H^{+} &\neq 0, & H^{+} \cup H^{+} \cup H^{+} \cup H^{+} \cup H^{+} &= 0,
\end{align*}
which possibly can be achieved through (cf.\ \cite{code})
\begin{align*}
  H^{-4,6} \cup H^{-6,9} &\subseteq H^{-10,15}, & H^{-10,15} \cup H^{-4,6} &\subseteq H^{-14,21}, \\
  H^{-14,21} \cup H^{-1,2} &\subseteq H^{-15,23}.
\end{align*}
It still remains to check whether such a series of non-zero cup products exist, which we will not do here. (This would require a lot of extra computations by hand, or hopefully better automation for computing cup products than what has been achieved so far.)

\begin{remark}
  To give estimates for the upper bounds of $m$ such that $(H^{+})^{m} \neq 0$, we can build a directed graph with nodes $(s,t)$ for $(s,t)$ such that $H^{s,t} \neq 0$ and arrows $(s,t) \to (s+s',t+t')$ (and $(s',t') \to (s+s',t+t')$) for $(s',t')$ such that $H^{s',t'} \neq 0$ and $H^{s+s',t+t'} \neq 0$ (so that $H^{s,t} \cup H^{s',t'} \subseteq H^{s+s',t+t'}$ has a chance of being non-zero). Then standard algorithms for finding the longest path in a directed graph can quickly give us a longest path as above.

  Note that we sometime can be more restrictive than just finding the longest path by also considering dimension arguments. This is the case when working with $I \subseteq \SL_{4}(\Z_{p})$ in the following, where $\dim_{k} H^{-5,7} = 4$, so if $H^{-5,7}$ is used in the sequence of cup products as part of the longest path more than $4$ times, then we know the cup product must be zero.
\end{remark}

In the $I \subseteq \GL_{3}(\Z_{p})$ case, we also have not described the cup product in detail, but we can tell purely from \eqref{eq:graded-coh-inc} and \Cref{tab:graded-coh-dims-GL3}, that
\begin{align*}
  H^{1} \cup H^{1} \cup H^{1} &\neq 0, & H^{1} \cup H^{1} \cup H^{1} \cup H^{1} &= 0,
\end{align*}
is an upper bound, since $\dim_{k} H^{-3,4} = 1$, so $H^{-3,4}$ can only be used once in a non-zero cup product.This upper bound might be achieved through
\begin{align*}
  H^{-3,4} \cup H^{-1,2} &\subseteq H^{-4,6}, & H^{-4,6} \cup H^{-1,2} \subset H^{-5,8},
\end{align*}
but it still remains to check whether such a non-zero cup product exists. Going through \Cref{tab:graded-coh-dims-GL3}, we also note that an upper bound for $H^{+}$ is
\begin{align*}
  (H^{+})^{5} &\neq 0, & (H^{+})^{6} &= 0,
\end{align*}
which possibly can be achieved through (cf.\ \cite{code})
\begin{align*}
  H^{-3,4} \cup H^{-1,2} &\subseteq H^{-4,6}, & H^{-4,6} \cup H^{-1,2} &\subseteq H^{-5,8}, \\
  H^{-5,8} \cup H^{-3,4} &\subseteq H^{-8,12}, & H^{-8,12} \cup H^{-1,2} &\subseteq HH^{-9,14}, \\
  H^{-9,14} \cup H^{-6,9} &\subseteq H^{-15,23}.
\end{align*}
Again it still remains to check whether such a series of non-zero cup products exist.

In the $I \subseteq \SL_{4}(\Z_{p})$ case, we also have not described the cup product in detail, and we do not even fully know the cohomology of $I$ in this case, but from \eqref{eq:graded-coh-inc} and \Cref{tab:graded-coh-dims-SL4}, we at least know enough to see that
\begin{align*}
  H^{1} \cup H^{1} &\neq 0, & H^{1} \cup H^{1} \cup H^{1} &= 0,
\end{align*}
is an upper bound, since the dimension of the entries $E_{r}^{s,t}$ are non-increasing when $r$ increases. This upper bound might be achieved through
\begin{equation*}
  H^{-1,2} \cup H^{-1,2} \subseteq H^{-2,4},
\end{equation*}
but it still remains to check whether such a non-zero cup product exists. Going through \Cref{tab:graded-coh-dims-SL4}, we also note that an upper bound for $H^{+}$ is
\begin{align*}
  (H^{+})^{7} &\neq 0, & (H^{+})^{8} &= 0,
\end{align*}
which possibly can be achieved through (cf.\ \cite{code})
\begin{align*}
  H^{-1,2} \cup H^{-5,7} &\subseteq H^{-6,9}, & H^{-6,9} \cup H^{-5,7} &\subseteq H^{-11,16}, \\
  H^{-11,16} \cup H^{-5,7} &\subseteq H^{-16,23}, & H^{-16,23} \cup H^{-5,7} &\subseteq HH^{-21,30}, \\
  H^{-21,30} \cup H^{-7,10} &\subseteq H^{-28,40}, & H^{-28,40} \cup H^{-8,11} &\subseteq H^{-36,51}.
\end{align*}

Similarly, in the $I \subseteq \GL_{4}(\Z_{p})$ case, we also have not described the cup product or fully know the cohomology of $I$ in this case, but from \eqref{eq:graded-coh-inc} and \Cref{tab:graded-coh-dims-GL4}, we at least know enough to see that
\begin{align*}
  (H^{1})^{4} &\neq 0, & (H^{1})^{5} &= 0,
\end{align*}
is an upper bound, since the dimension of the entries $E_{r}^{s,t}$ are non-increasing when $r$ increases. This upper bound might be achieved through
\begin{align*}
  H^{-4,5} \cup H^{-1,2} &\subseteq H^{-5,7}, & H^{-5,7} \cup H^{-1,2} &\subseteq H^{-6,9}, \\
  H^{-6,9} \cup H^{-1,2} &\subseteq H^{-7,11},
\end{align*}
but it still remains to check whether such a non-zero cup product exists. Going through \Cref{tab:graded-coh-dims-GL4}, we also note that an upper bound for $H^{+}$ is
\begin{align*}
  (H^{+})^{10} &\neq 0, & (H^{+})^{11} &= 0,
\end{align*}
which possibly can be achieved through (cf.\ \cite{code})
\begin{align*}
  H^{-4,5} \cup H^{-1,2} &\subseteq H^{-5,7}, & H^{-5,7} \cup H^{-1,2} &\subseteq H^{-6,9}, \\
  H^{-6,9} \cup H^{-1,2} &\subseteq H^{-7,11}, & H^{-7,11} \cup H^{-4,5} &\subseteq HH^{-11,16}, \\
  H^{-11,16} \cup H^{-1,2} &\subseteq H^{-12,18}, & H^{-12,18} \cup H^{-5,7} &\subseteq H^{-17,25}, \\
  H^{-17,25} \cup H^{-9,12} &\subseteq H^{-26,37}, & H^{-26,37} \cup H^{-5,7} &\subseteq H^{-31,44}, \\
  H^{-31,44} \cup H^{-5,7} &\subseteq H^{-36,51}.
\end{align*}

In the $I \subseteq \SL_{2}(\sO_{F})$ (with $F/\Q_{p}$ quadratic) case, we also have not described the cup product in detail, but from \eqref{eq:graded-coh-inc} and \Cref{tab:graded-coh-dims-SL2-F}, we at least know enough to see that
\begin{align*}
  H^{1} \cup H^{1} &\neq 0, & H^{1} \cup H^{1} \cup H^{1} &= 0,
\end{align*}
is an upper bound. This upper bound might be achieved through
\begin{equation*}
  H^{-1,2} \cup H^{-1,2} \subseteq H^{-2,4},
\end{equation*}
but it still remains to check whether such a non-zero cup product exists. Going through \Cref{tab:graded-coh-dims-SL2-F}, we also note that an upper bound for $H^{+}$ is
\begin{align*}
  (H^{+})^{4} &\neq 0, & (H^{+})^{5} &= 0,
\end{align*}
which possibly can be achieved through (cf.\ \cite{code})
\begin{align*}
  H^{-3,5} \cup H^{-3,5} &\subseteq H^{-6,10}, & H^{-6,10} \cup H^{-1,2} &\subseteq H^{-7,12}, \\
  H^{-7,12} \cup H^{-1,2} &\subseteq H^{-8,14}.
\end{align*}

In the $I \subseteq \GL_{2}(\sO_{F})$ (with $F/\Q_{p}$ quadratic) case, we also have not described the cup product in detail, but from \eqref{eq:graded-coh-inc} and \Cref{tab:graded-coh-dims-GL2-F}, we at least know enough to see that
\begin{align*}
  (H^{1})^{4} &\neq 0, & (H^{1})^{5} &= 0,
\end{align*}
is an upper bound. This upper bound might be achieved through
\begin{align*}
  H^{-2,3} \cup H^{-1,2} &\subseteq H^{-3,5}, & H^{-3,5} \cup H^{-1,2} &\subseteq H^{-4,7}, \\
  H^{-4,7} \cup H^{-1,2} &\subseteq H^{-5,9},
\end{align*}
but it still remains to check whether such a non-zero cup product exists. Going through \Cref{tab:graded-coh-dims-GL2-F}, we also note that an upper bound for $H^{+}$ is
\begin{align*}
  (H^{+})^{6} &\neq 0, & (H^{+})^{7} &= 0,
\end{align*}
which possibly can be achieved through (cf.\ \cite{code})
\begin{align*}
  H^{-2,3} \cup H^{-1,2} &\subseteq H^{-3,5}, & H^{-3,5} \cup H^{-1,2} &\subseteq H^{-4,7}, \\
  H^{-4,7} \cup H^{-1,2} &\subseteq H^{-5,9}, & H^{-5,9} \cup H^{-2,3} &\subseteq H^{-7,12}, \\
  H^{-7,12} \cup H^{-1,2} &\subseteq H^{-8,14}.
\end{align*}

We collect our nilpotency index upper bounds in \Cref{tab:nilp-ind}.

\begin{table}[ht]
  \centering
  \caption[Nilpotency index upper bounds for mod $p$ cohomology of pro-$p$ Iwahori subgroups]{The upper bound for the nilpotency index of mod $p$ cohomology for each pro-$p$ Iwahori subgroup of $\SL_{n}$ and $\GL_{n}$ that we have found. Confirmed nilpotency indices are bolded, and pure upper bounds are not bolded.}
  \label{tab:nilp-ind}
  \begin{NiceTabular}{rccc}[hvlines]
    $n$ & 2 & 3 & 4 \\
    $I \subseteq \SL_{n}(\Z_{p})$, $H^{1}$ & \textbf{2} & \textbf{2} & 3 \\
    $I \subseteq \SL_{n}(\Z_{p})$, $H^{+}$ & \textbf{3} & 5 & 8  \\
    $I \subseteq \GL_{n}(\Z_{p})$, $H^{1}$ & \textbf{3} & 4 & 5 \\
    $I \subseteq \GL_{n}(\Z_{p})$, $H^{+}$ & \textbf{4} & 7 & 11 \\
    $I \subseteq \SL_{n}(\sO_{F})$ (quadratic), $H^{1}$ & 3 \\
    $I \subseteq \SL_{n}(\sO_{F})$ (quadratic), $H^{+}$ & 4 \\
    $I \subseteq \GL_{n}(\sO_{F})$ (quadratic), $H^{1}$ & 5 \\
    $I \subseteq \GL_{n}(\sO_{F})$ (quadratic), $H^{+}$ & 7
  \end{NiceTabular}
\end{table}


\section{Future work}%
\label{sec:future}

In this section we will discuss some interesting future directions of research. We will assume for the whole section that $k = \F_{p}$.

\subsection{Quaternion algebras}%
\label{subsec:quat-algs}

In this subsection, we will further assume that $p>5$ is a prime of the form $p \equiv 3 \pmod{4}$, so that $\Q_{p^{2}} = \Q_{p}(i)$ is the unique unramified quadratic extension of $\Q_{p}$, and $\F_{p^{2}} = \F_{p}[i]$ is the unique quadratic extension of $\F_{p}$.

Let $D$ be the division quaternion algebra over $\Q_{p}$ and let $\tilde{G} = 1+\idm_{D}$ and $G = (1+\idm_{D})^{\Nrd = 1}$, where $\Nrd = \Nrd_{D/\Q_{p}}$ is the norm form. By \cite[Thm.~12.1.5]{Voight} we can assume that $i^{2} = -1$ and $j^{2} = p$ (i.e., we have a tower $D / \Q_{p}(i) / \Q_{p}$), and that $\sO_{D}=\Z_{p}[i,j,k]$ (where $k=ij$) and $\idm_{D} = j\sO_{D} = \sO_{D}j$ (i.e., $\varpi_{D} = j$), which has $\Z_{p}$-basis $p,pi,j,k$, by \cite[Thm.13.1.6]{Voight}.

Now let $\sigma \colon \Q_{p}(i) \to \Q_{p}(i)$ be the complex conjugate and note that $\gen{\sigma} = \Gal\bigl( \Q_{p}(i)/\Q_{p} \bigr)$, so
\begin{equation*}
  D \iso \set[\bigg]{\pmat{ a+bi & c+di \\ p(c-di) & a-bi } \given a,b,c,d\in\Q_{p} } \subseteq M_{2}\bigl( \Q_{p}(i) \bigr)
\end{equation*}
by \cite[Cor.~13.3.14]{Voight}. Hence we have an embedding
\begin{align*}
  D &\hookrightarrow M_{2}\bigl( \Q_{p}(i) \bigr) \\
  a + bi + cj + dk &\mapsto \pmat{ a+bi & c+di \\ p(c-di) & a-bi }
\end{align*}
with
\begin{equation*}
  \Nrd(a+bi+cj+dk) = a^{2} + b^{2} - pc^{2} - pd^{2} = \det\Bigl( \pmat{ a+bi & c+di \\ p(c-di) & a-bi } \Bigr).
\end{equation*}
We note furthermore that $\sO_{D} = \Z_{p}[i] \oplus \Z_{p}[i]j$ and $\idm_{D} = \sO_{D}j = p\Z_{p}[i] \oplus \Z_{p}[i]j$ gives us
\begin{equation*}
  \idm_{D} \iso \set[\bigg]{\pmat{ p(a+bi) & c+di \\ p(c-di) & p(a-bi) } \given a,b,c,d\in\Q_{p} },
\end{equation*}
so $1 + \idm_{D} \subseteq I_{\GL_{2}(\Q_{p}(i))}$, where we denote by $I_{G}$\nomiwa[IG]{$I_{G}$}{the pro-$p$ Iwahori subgroup of $G$} the (standard choice of) pro-$p$ Iwahori subgroup of $G$ (cf.\ \Cref{subsec:setup-iwa}). Altogether, we get a commutative diagram
\begin{equation}\label{eq:quat-diagram}
  \begin{tikzcd}
    (1+\idm_{D})^{\Nrd = 1} \ar[d, hook] \ar[r, hook] & I_{\SL_{2}(\Q_{p}(i))} \ar[d, hook] & \ar[l, hook'] I_{\SL_{2}(\Q_{p})} \ar[d, hook] \\
    (\sO_{D}^{\times})^{\Nrd = 1} \ar[d, hook] \ar[r, hook] & \SL_{2}\bigl( \Z_{p}[i] \bigr) \ar[d, hook] & \ar[l, hook'] \SL_{2}(\Z_{p}) \ar[d, hook] \\
    (D^{\times})^{\Nrd = 1} \ar[d, hook] \ar[r, hook] & \SL_{2}\bigl( \Q_{p}(i) \bigr) \ar[d, hook] & \ar[l, hook'] \SL_{2}(\Q_{p}) \ar[d, hook] \\
    D^{\times} \ar[r, hook] & \GL_{2}\bigl( \Q_{p}(i) \bigr) & \ar[l, hook'] \GL_{2}(\Q_{p}) \\
    \sO_{D}^{\times} \ar[u, hook] \ar[r, hook] & \GL_{2}\bigl( \Z_{p}[i] \bigr) \ar[u, hook] & \ar[l, hook'] \GL_{2}(\Z_{p}) \ar[u, hook] \\
    1+\idm_{D} \ar[u, hook] \ar[r, hook] & I_{\GL_{2}(\Q_{p}(i))} \ar[u, hook] & \ar[l, hook'] I_{\GL_{2}(\Q_{p})}. \ar[u, hook]
  \end{tikzcd}
\end{equation}

We saw in \Cref{rem:quaternion} that $H^{*}(G,\F_{p}) \iso H^{*}(I_{\SL_{2}(\Q_{p})},\F_{p})$, and in \Cref{rem:GL2-SL2-coh-iso} we noted that $H^{*}(I_{\SL_{2}(\Q_{p})},\F_{p}) \otimes_{F_{p}} \F_{p}[\varepsilon] \iso H^{*}(I_{\GL_{2}(\Q_{p})},\F_{p})$ (where $\varepsilon^{2} = 0$), while \cite[Sect.~6.3]{Sor} notes that $H^{*}(\tilde{G},\F_{p}) \iso H^{*}(G,\F_{p}) \otimes_{\F_{p}} \F_{p}[\varepsilon]$, so $H^{*}(\tilde{G},\F_{p}) \iso H^{*}(I_{\GL_{2}(\Q_{p})},\F_{p})$. (Recall that $G = (1+\idm_{D})^{\Nrd = 1}$ and $\tilde{G} = 1+\idm_{D}$.) Furthermore, \cite[Sect.~6.3]{Sor} argues that $H^{*}(\sO_{D}^{\times},\F_{p}) \iso H^{*}(\tilde{G},\F_{p})^{\F_{D}^{\times}}$, using that we can factor $\sO_{D}^{\times}$ as a semi-direct product $\tilde{G} \rtimes \F_{D}^{\times}$. Here the $\F_{D}^{\times}$-action on $H^{*}(\tilde{G},\F_{p})$ is understood and non-trivial, cf.\ \cite[Prop.~7~(b)]{Henn}. An interesting question is, if the comparison between cohomology of the left side and right side of \eqref{eq:quat-diagram} can be somehow continued?

\begin{remark}
  To see that $H^{*}(\tilde{G},\F_{p}) \iso H^{*}(I_{\GL_{2}(\Q_{p})},\F_{p})$ for $p\geq5$ in general, and not just for $p \equiv 3 \pmod{4}$, one can compare the basis and structure of $H^{*}(\tilde{G},\F_{p})$ described in \cite[Thm.~3.2]{Ravenel} with the basis and structure we describe in \eqref{eq:Hst-spaces-GL2} and \eqref{eq:cup-products-GL2}.
\end{remark}

Another interesting direction of research is to note that we already have bijections between \emph{certain} mod $p$ representations of $D^{\times}$ and $\GL_{2}(\Q_{p})$ from the Jacquet-Langlands correspondence for $\GL_{2}$ (cf.\ \cite{JL}), and we can ask whether there are similar relations in between the left and right side of the other rows of \eqref{eq:quat-diagram}. Here we note that by \cite[Rem.~4.5]{JL-remark} irreducible representations of $D^{\times}$ are trivial on $1+\idm_{D}$, so we need something new if we want a correspondence between certain mod $p$ representations of $G = (1+\idm_{D})^{\Nrd = 1}$ and $I_{\SL_{2}(\Q_{p})}$ or between certain mod $p$ representations of $\tilde{G} = 1+\idm_{D}$ and $I_{\GL_{2}(\Q_{p})}$.

Finally, although we already have isomorphisms $H^{*}(G,\F_{p}) \iso H^{*}(I_{\SL_{2}(\Q_{p})},\F_{p})$ and $H^{*}(\tilde{G},\F_{p}) \iso H^{*}(I_{\GL_{2}(\Q_{p})},\F_{p})$, we note that these were obtained by concrete calculations, and we would really prefer to have canonical isomorphisms (possibly obtained by working with the corresponding row of \eqref{eq:quat-diagram}).

In pursuit of the canonical isomorphisms mentioned above, we note that one can show by explicit calculations (with bases) that the inclusions of \eqref{eq:quat-diagram} give inclusions
\begin{equation*}
  \begin{tikzcd}
    \gr (1+\idm_{D})^{\Nrd = 1} \ar[r, hook] & \gr I_{\SL_{2}(\Q_{p}(i))} & \ar[l, hook'] \gr I_{\SL_{2}(\Q_{p})}, \\
    \gr (1+\idm_{D}) \ar[r, hook] & \gr I_{\GL_{2}(\Q_{p}(i))} & \ar[l, hook'] \gr I_{\GL_{2}(\Q_{p})},
  \end{tikzcd}
\end{equation*}
where the pro-$p$ Iwahori subgroups are graded as usual (start with $\gr I = \bigoplus_{\nu>0} \gr_{\nu} I$ where $\gr_{\nu} I = I_{\nu}/I_{\nu+}$ and translate to $\gr^{i} I$), $1+\idm_{D}$ is graded by $\gr^{i}(1+\idm_{D}) = (1+\idm_{D}^{i})/(1+\idm_{D}^{i+1})$, and $(1+\idm_{D})^{\Nrd = 1}$ is graded by $\gr^{i}(1+\idm_{D})^{\Nrd = 1} = (1+\idm_{D}^{i})^{\Nrd = 1}/(1+\idm_{D}^{i+1})^{\Nrd = 1}$. These inclusions further translate to inclusions
\begin{equation*}
  \begin{tikzcd}
    \lie{g}_{(1+\idm_{D})^{\Nrd = 1}} \ar[r, hook] & \lie{g}_{I_{\SL_{2}(\Q_{p}(i))}} & \ar[l, hook'] \lie{g}_{I_{\SL_{2}(\Q_{p})}}, \\
    \lie{g}_{(1+\idm_{D})} \ar[r, hook] & \lie{g}_{I_{\GL_{2}(\Q_{p}(i))}} & \ar[l, hook'] \lie{g}_{I_{\GL_{2}(\Q_{p})}},
  \end{tikzcd}
\end{equation*}
where $\lie{g}_{G} = \F_{p} \otimes_{\F_{p}[\pi]} \gr G$ is the Lazard Lie algebra of $G$. We note that these inclusions do not have the same images, but we noted in \Cref{rem:quaternion} that $\lie{g}_{(1+\idm_{D})^{\Nrd = 1}} \iso \lie{g}_{I_{\SL_{2}(\Q_{p}(i))}}$, so we might be able to come up with a canonical isomorphism through these somehow.

% With this setup, we also have that
% \begin{align*}
%   \F_{D} = \sO_{D}/\idm_{D} \iso \idm_{D}^{n}/\idm_{D}^{n+1} &\xrightarrow{\iso} (1+\idm_{D}^{n})/(1+\idm_{D}^{n+1}) \\
%   x &\mapsto 1+x
% \end{align*}
% is an isomorphism for all $n\geq1$, cf.\ \cite[13.5.8]{Voight}, which should help with translating from $G$ to $\lie{g}_{D}$.

\subsection{Central division algebras}%
\label{subsec:central-div-algs}

Let $D$ be the central division algebra over $\Q_{p}$ of dimension $n^{2}$ and invariant $\frac{1}{n}$. Recall the following setup from \cite[Sect.~6.3]{Sor}: The valuation $v_{p}$ on $\Q_{p}$ extends uniquely to a valuation $\tilde{v} \colon D^{\times} \to \frac{1}{n}\Z$ by the formula $\tilde{v}(x) = \frac{1}{n}v\bigl(\Nrd_{D/\Q_{p}}(x)\bigr)$, and the valuation ring $\sO_{D} = \set{ x : \tilde{v}(x) >0 }$ is the maximal compact subring of $D$. It is local with maximal ideal $\idm_{D} = \set{ x : \tilde{v}(x) > 0 }$ and residue field $\F_{D} \iso \F_{p^{n}}$. Furthermore, we can pick $\varpi_{D}$ such that $\tilde{v}(\varpi_{D}) = \frac{1}{n}$, $\idm_{D} = \varpi_{D}\sO_{D} = \sO_{D}\varpi_{D}$ and $p = \varpi_{D}^{n}$.

When $p > n+1$, we also saw in \cite[Sect.~6.3]{Sor} that $\tilde{G} = 1 + \idm_{D}$ has Lazard Lie algebra $\tilde{\lie{g}} = \F_{D} \oplus \dotsb \oplus \F_{D}$ concentrated in degrees $1,2,\dotsc,n$ with Lie bracket given by the formula
\begin{equation}\label{eq:central-div-alg-com}
  [x,y] = xy^{p^{i}} - yx^{p^{j}}
\end{equation}
for $x \in \tilde{\lie{g}}^{i} \iso \F_{D}$ and $y \in \tilde{\lie{g}}^{j}$. Furthermore $G = (1 + \idm_{D})^{\Nrd = 1}$ has Lazard Lie algebra $\lie{g} = \F_{D} \oplus \dotsb \oplus \F_{D} \oplus \F_{D}^{\Tr = 0}$ concentrated in degrees $1,2,\dotsc,n$ with Lie bracket given by \eqref{eq:central-div-alg-com}. (Note that one can easily check that $[x,y]$ has trace zero when $i+j = n$.) Here $\F_{D}^{\Tr = 0}$ is the kernel of the trace $\Tr_{\F_{D}/\F_{p}}$ and $\lie{g} \subseteq \tilde{\lie{g}}$ is a codimension one Lie subalgebra.

In the previous subsection we focused on the case $n = 2$ (and $p \equiv 3 \pmod{4}$), but one can ask if some of the ideas work in more general cases. For the remainder of this subsection we will focus on the case $n = 3$ and $p = 5$.

We note that $x^{3} + 3x + 3$ is an irreducible polynomial in $\F_{5}[x]$, so $\F_{D} \iso \F_{5^{3}} \iso \F_{p}[\alpha]$ where $\alpha^{3} = -3\alpha-3 = 2\alpha + 2$. Now let $\xi_{1} = 1, \xi_{2} = \alpha, \xi_{3} = \alpha^{2}$ be the basis of $\tilde{\lie{g}}^{1} \iso \F_{D}$, let $\xi_{4} = 1, \xi_{5} = \alpha, \xi_{6} = \alpha^{2}$ be the basis of $\tilde{\lie{g}}^{2} \iso \F_{D}$, and let $\xi_{7} = 1, \xi_{8} = \alpha, \xi_{9} = \alpha^{2}$ be the basis of $\tilde{\lie{g}}^{3} \iso \F_{D}$. Using \eqref{eq:central-div-alg-com}, we see that
\begin{equation}\label{eq:central-div-alg-first-com}
  \begin{aligned}
    [\xi_{1},\xi_{2}] &= 4\xi_{4} + 3\xi_{5} + 2\xi_{6}, & [\xi_{1},\xi_{3}] &= 3\xi_{4} + 2\xi_{5} + 4\xi_{6}, \\
    [\xi_{1},\xi_{5}] &= 4\xi_{7} + 3\xi_{8} + 2\xi_{9}, & [\xi_{1},\xi_{6}] &= 3\xi_{7} + 2\xi_{8} + 4\xi_{9}, \\
    [\xi_{2},\xi_{3}] &= 2\xi_{4} + \xi_{5} + 4\xi_{6}, & [\xi_{2},\xi_{4}] &= 4\xi_{7} + \xi_{8} + 2\xi_{9}, \\
    [\xi_{2},\xi_{5}] &= 3\xi_{7} + \xi_{8} + 4\xi_{9}, & [\xi_{2},\xi_{6}] &= 2\xi_{8}, \\
    [\xi_{3},\xi_{4}] &= 4\xi_{7} + 2\xi_{8} + 2\xi_{9}, & [\xi_{3},\xi_{5}] &= 3\xi_{8}, \\
    [\xi_{3},\xi_{6}] &= 3\xi_{7} + 4\xi_{9}.
  \end{aligned}
\end{equation}
Here
\begin{equation*}
  \tilde{\lie{g}}^{1} = \Span_{\F_{5}}(\xi_{1},\xi_{2},\xi_{3}), \quad \tilde{\lie{g}}^{2} = \Span_{\F_{5}}(\xi_{4},\xi_{5},\xi_{6}), \quad \tilde{\lie{g}}^{3} = \Span_{\F_{5}}(\xi_{7},\xi_{8},\xi_{9}),
\end{equation*}
so we order the basis by the index of the $\xi_{i}$'s. Calculating the cohomology as in the previous sections with this information, we get \Cref{tab:graded-coh-dims-central-div-alg-3}.

\begin{table}[ht]
  \centering
  \caption[Graded cohomology dimensions for the Lazard Lie algebra of $1+\idm_{D}$ in the $n=3$ and $p=5$ case]{Dimensions of $E_{1}^{s,t} = H^{s,t} = \gr^{s} H^{s+t}(\tilde{\lie{g}},\F_{5})$ for $\tilde{G} = 1+\idm_{D}$ in the $n=3$ and $p=5$ case.}
  \label{tab:graded-coh-dims-central-div-alg-3}
  \renewcommand{\arraystretch}{1.7}
  \scalebox{0.6}{%
    $\begin{NiceArray}{*{20}{c}}[hvlines, columns-width=auto]
      \diagbox{t}{s} & 0 & -1 & -2 & -3 & -4 & -5 & -6 & -7 & -8 & -9 & -10 & -11 & -12 & -13 & -14 & -15 & -16 & -17 & -18 \\
      0 & 1\\
      1 \\
      2 && 3 \\
      3 \\
      4 &&&& 1 \\
      5 &&&& 6 \\
      6 &&&&& 6 \\
      7 &&&&& 3\\
      8 &&&&&& 6 \\
      9 &&&&&&& 13 \\
      10 &&&&&&&& 3 \\
      11 &&&&&&&& 12 \\
      12 &&&&&&&&& 15 \\
      13 &&&&&&&&&& 7 \\
      14 &&&&&&&&&& 7 \\
      15 &&&&&&&&&&& 15 \\
      16 &&&&&&&&&&&& 12 \\
      17 &&&&&&&&&&&& 3 \\
      18 &&&&&&&&&&&&& 13 \\
      19 &&&&&&&&&&&&&& 6 \\
      20 &&&&&&&&&&&&&&& 3 \\
      21 &&&&&&&&&&&&&&& 6 \\
      22 &&&&&&&&&&&&&&&& 6 \\
      23 &&&&&&&&&&&&&&&& 1 \\
      24 \\
      25 &&&&&&&&&&&&&&&&&& 3 \\
      26 \\
      27 &&&&&&&&&&&&&&&&&&& 1
    \end{NiceArray}$%
  }
  \renewcommand{\arraystretch}{1}
\end{table}

\begin{remark}
  Note that when calculating the cohomology here, we need to do all calculations modulo $5$ since \eqref{eq:central-div-alg-first-com} do not lift to a Lie algebra over $\Z$ with these Chevalley constants. See \cite{code} for the details.
\end{remark}

Comparing \Cref{tab:graded-coh-dims-GL3} and \Cref{tab:graded-coh-dims-central-div-alg-3}, we see that $H^{*}(I,\F_{5})$ for $I \subseteq \GL_{3}(\Z_{p})$ and $H^{*}(1+\idm_{D},\F_{5})$ have the same graded cohomology dimensions, and it would be interesting to investigate whether $H^{*}(I,\F_{5}) \iso H^{*}(1+\idm_{D},\F_{5})$ as graded algebras. More generally, is $H^{*}(I,\F_{p}) \iso H^{*}(1+\idm_{D},\F_{p})$ as graded algebras for $p \geq 5$?

In a similar vein, we can recall that $\Tr_{\F_{D}\F_{5}}(x) = x + x^{5} + x^{5^{2}}$ for $x \in \F_{D} \iso \F_{5^{3}}$, so
\begin{align*}
  \Tr_{\F_{D}/\F_{5}}(1) &= 3, & \Tr_{\F_{D}/\F_{5}}(\alpha) &= 0, & \Tr_{\F_{D}/\F_{5}}(\alpha^{2}) &= 4,
\end{align*}
since $\alpha^{3} = 2\alpha + 2$. Thus $\F_{D}^{\Tr = 0}$ has basis $\alpha, 4+2\alpha^{2}$. Now let $\xi_{1}' = 1, \xi_{2}' = \alpha, \xi_{3}' = \alpha^{2}$ be the basis of $\lie{g}^{1} \iso \F_{D}$, let $\xi_{4}' = 1, \xi_{5}' = \alpha, \xi_{6}' = \alpha^{2}$ be the basis of $\lie{g}^{2} \iso \F_{D}$, and let $\xi_{7}' = \alpha, \xi_{8}' = 4+2\alpha$ be the basis of $\lie{g}^{3} \iso \F_{D}^{\Tr = 0}$. Using \eqref{eq:central-div-alg-com}, we see that
\clearpage
\begin{equation}\label{eq:central-div-alg-second-com}
  \begin{aligned}
    [\xi_{1}',\xi_{2}'] &= 4\xi_{4}' + 3\xi_{5}' + 2\xi_{6}', & [\xi_{1}',\xi_{3}'] &= 3\xi_{4}' + 2\xi_{5}' + 4\xi_{6}', \\
    [\xi_{1}',\xi_{5}'] &= 3\xi_{7}' + \xi_{8}', & [\xi_{1}',\xi_{6}'] &= 2\xi_{7}' + 2\xi_{8}', \\
    [\xi_{2}',\xi_{3}'] &= 2\xi_{4}' + \xi_{5}' + 4\xi_{6}', & [\xi_{2}',\xi_{4}'] &= \xi_{7}' + \xi_{8}', \\
    [\xi_{2}',\xi_{5}'] &= \xi_{7}' + 2\xi_{8}', & [\xi_{2}',\xi_{6}'] &= 2\xi_{7}', \\
    [\xi_{3}',\xi_{4}'] &= 2\xi_{7}' + \xi_{8}', & [\xi_{3}',\xi_{5}'] &= 3\xi_{7}', \\
    [\xi_{3}',\xi_{6}'] &= 2\xi_{8}'.
  \end{aligned}
\end{equation}
Here
\begin{equation*}
  \lie{g}^{1} = \Span_{\F_{5}}(\xi_{1}',\xi_{2}',\xi_{3}'), \quad \lie{g}^{2} = \Span_{\F_{5}}(\xi_{4}',\xi_{5}',\xi_{6}'), \quad \lie{g}^{3} = \Span_{\F_{5}}(\xi_{7}',\xi_{8}'),
\end{equation*}
so we order the basis by the index of the $\xi_{i}'$'s. Calculating the cohomology as in the previous sections with this information, we get \Cref{tab:graded-coh-dims-central-div-alg-3-prime}

\begin{table}[ht]
  \centering
  \caption[Graded cohomology dimensions for the Lazard Lie algebra of $(1+\idm_{D})^{\Nrd = 1}$ in the $n=3$ and $p=5$ case]{Dimensions of $E_{1}^{s,t} = H^{s,t} = \gr^{s} H^{s+t}(\lie{g},\F_{5})$ for $G = (1+\idm_{D})^{\Nrd = 1}$ in the $n=3$ and $p=5$ case.}
  \label{tab:graded-coh-dims-central-div-alg-3-prime}
  \renewcommand{\arraystretch}{1.7}
  \scalebox{0.7}{%
    $\begin{NiceArray}{*{17}{c}}[hvlines, columns-width=auto]
      \diagbox{t}{s} & 0 & -1 & -2 & -3 & -4 & -5 & -6 & -7 & -8 & -9 & -10 & -11 & -12 & -13 & -14 & -15 \\
      0 & 1\\
      1 \\
      2 && 3 \\
      3 \\
      4 \\
      5 &&&& 6\\
      6 &&&&& 3 \\
      7 &&&&& 3\\
      8 &&&&&& 6\\
      9 &&&&&&& 7\\
      10 \\
      11 &&&&&&&& 9 \\
      12 &&&&&&&&& 9 \\
      13 \\
      14 &&&&&&&&&& 7 \\
      15 &&&&&&&&&&& 6 \\
      16 &&&&&&&&&&&& 3 \\
      17 &&&&&&&&&&&& 3 \\
      18 &&&&&&&&&&&&& 6 \\
      19 \\
      20 \\
      21 &&&&&&&&&&&&&&& 3 \\
      22 \\
      23 &&&&&&&&&&&&&&&& 1
    \end{NiceArray}$%
  }
  \renewcommand{\arraystretch}{1}
\end{table}

Again, comparing \Cref{tab:graded-coh-dims-SL3} and \Cref{tab:graded-coh-dims-central-div-alg-3-prime}, we see that $H^{*}(I,\F_{5})$ for $I \subseteq \SL_{3}(\Z_{p})$ and $H^{*}\bigl((1+\idm_{D})^{\Nrd = 1},\F_{5}\bigr)$ have the same graded cohomology dimensions, and it would be interesting to investigate whether $H^{*}(I,\F_{5}) \iso H^{*}\bigl((1+\idm_{D})^{\Nrd = 1},\F_{5}\bigr)$ as graded algebras. More generally, is $H^{*}(I,\F_{p}) \iso H^{*}\bigl((1+\idm_{D})^{\Nrd = 1},\F_{p}\bigr)$ as graded algebras for $p \geq 5$?

Another interesting observation is that $H^{*}(\tilde{G},\F_{p}) \iso H^{*}(G,\F_{p}) \otimes_{\F_{p}} \F_{p}[\varepsilon]$ as graded algebras (with $\varepsilon^{2} = 0$) by \cite[Sect.~6.3]{Sor}, so an interesting question is whether $H^{*}(I_{\GL_{n}(\Q_{p})},\F_{p}) \iso H^{*}(I_{\SL_{n}(\Q_{p})},\F_{p}) \otimes_{\F_{p}} \F_{p}[\varepsilon]$ as graded algebras.

Altogether, the above seems to hint at the following conjecture:

\begin{conjecture}
  Let $D$ be the central division algebra over $\Q_{p}$ of dimension $n^{2}$ and invariant $\frac{1}{n}$. Let $\sO_{D}$ be the maximal compact (local) subring of $D$ with maximal ideal $\idm_{D}$ and residue field $\F_{D} \iso \F_{p^{n}}$. If $p > n+1$ then
  \begin{enumerate}[$\bullet$]
    \item $H^{*}\bigl( I_{\GL_{n}(\Q_{p})}, \F_{p} \bigr) \iso H^{*}\bigl( 1+\idm_{D}, \F_{p} \bigr)$ as graded algebras, and
    \item $H^{*}\bigl( I_{\SL_{n}(\Q_{p})}, \F_{p} \bigr) \iso H^{*}\bigl( (1+\idm_{D})^{\Nrd = 1}, \F_{p} \bigr)$ as graded algebras.
  \end{enumerate}
  In particular, this implies by \cite[Sect.~6.3]{Sor} that \[ H^{*}( I_{\GL_{n}(\Q_{p})}, \F_{p} ) \iso H^{*}( I_{\SL_{n}(\Q_{p})}, \F_{p} ) \otimes_{\F_{p}} \F_{p}[\varepsilon] \] as graded algebras, where $\F_{p}[\varepsilon]$ denote the dual numbers ($\varepsilon^{2} = 0$).
\end{conjecture}

\subsection{Serre spectral sequence}%
\label{subsec:Serre-spec-seq}

Another interesting research direction is to try to work with the Serre spectral sequence in the following way.

Assume we have the \enquote{standard} setup with $\gs{G} = \SL_{n}$, $\gs{U}$ unipotent upper triangular matrices and $\gs{T}$ diagonal matrices with determinant $1$. Let also $I \subseteq \SL_{n}(\Z_{p})$ be the pro-$p$ Iwahori subgroup of $\SL_{n}(\Q_{p})$ which is upper triangular and unipotent modulo $p$, and let
\begin{equation*}
  K \coloneqq \kernel\bigl( \red \colon \gs{G}(\Z_{p}) \to \gs{G}(\F_{p}) \bigr),
\end{equation*}
where $\red \colon \gs{G}(\Z_{p}) \to \gs{G}(\F_{p})$ is the reduction map. (Note that $I = \set{g \in \gs{G}(\Z_{p}) : \red(g) \in \gs{U}(\F_{p})}$ in this case, cf.\ \cite{Generators}.) Then
\begin{equation*}
  I/K \iso \gs{U}(\F_{p}),
\end{equation*}
and thus we get the Serre spectral sequence
\begin{equation*}
  E_{2}^{i,j} = H^{i}\bigl( \gs{U}(\F_{p}), H^{j}(K,\F_{p}) \bigr) \Longrightarrow H^{i+j}(I,\F_{p}),
\end{equation*}
which is also a multiplicative spectral sequence. Since $K$ is a uniformly powerful group (cf.\ \cite[Prop.~7.6]{SchOll-modular}), we know by \cite[p.~183]{Laz} that
\begin{equation*}
  H^{j}(K,\F_{p}) \iso \bigwedge^{j} \Hom_{\F_{p}}(K,\F_{p}).
\end{equation*}
Now we can let $\SL_{n}(\Z_{p})$ act by
\begin{equation*}
  (g \act f)(x) = f(g^{-1}xg)
\end{equation*}
for $g \in \SL_{n}(\Z_{p})$, $f \colon K \to \F_{p}$ and $x \in K$, and hope to split $\bigwedge^{j} \Hom_{\F_{p}}(K,\F_{p})$ into a direct sum of Verma modules $\bigoplus_{\lambda} V(\lambda)$ for $p$-restricted $\lambda$ ($\lambda$ with $0 \leq \inner{\lambda}{\alpha^{\vee}} \leq p-1$), similarly to what is done in \cite{PT} (as we used in \Cref{cha:cohunigps}). This description might be easier to generalize than what we have worked with in this chapter, but it is harder to get started with since the spectral sequence is more complicated. One can hope that the difference in the spectral sequence might make it so that it will always collapse on the second page (the starting page in this case).

%\dknote{Maybe add a note about dimension of Lie algebra center conjecture.}


%%% Local Variables:
%%% mode: latex
%%% TeX-master: "../main"
%%% End:
