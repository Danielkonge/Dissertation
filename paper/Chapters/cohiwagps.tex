\section{Intoduction}%
\label{sec:cohiwagps}

\section{\texorpdfstring{$I \subseteq \SL_{2}(\Z_{p})$}{I in SL2(Zp)}}%
\label{sec:Iwa-SL2}

\begin{equation*}
  I = \pmat{1+p\Z_{p} & \Z_{p} \\ p\Z_{p} & 1+p\Z_{p}} \subseteq \SL_{2}(\Z_{p}).
\end{equation*}

Obvious try (using that $(1+p)^{\Z_{p}} = 1+p\Z_{p}$):
\begin{align*}
  g_{1}' &= \pmat{1 & 0 \\ p & 1}, & g_{2}' &= \pmat{1+p & 0 \\ 0 & (1+p)^{-1}}, & g_{3}' &= \pmat{1 & 1 \\ 0 & 1}.
\end{align*}

Better:
\begin{align}
  \label{eq:gis-SL2}
  g_{1} &= \pmat{1 & 0 \\ p & 1}, & g_{2} &= \pmat{\exp(p) & 0 \\ 0 & \exp(-p)}, & g_{3} &= \pmat{1 & 1 \\ 0 & 1}.
\end{align}

For $g = (a_{ij})$
\begin{equation*}
  \omega(g) \defeq \min\bigl( v_{p}(a_{11}-1), \tfrac{1}{2} + v_{p}(a_{12}), -\tfrac{1}{2} + v_{p}(a_{21}), v_{p}(a_{22}-1) \bigr).
\end{equation*}

\begin{equation}
  \label{eq:gixi-SL2}
  g_{1}^{x_{1}}g_{2}^{x_{2}}g_{3}^{x_{3}} = \pmat{\exp(px_{2}) & x_{3}\exp(px_{2}) \\ px_{1}\exp(px_{2}) & px_{1}x_{3}\exp(px_{2}) + \exp(-px_{2})} = \pmat{a_{11} & a_{12} \\ a_{21} & a_{22}}.
\end{equation}

$g_{ij} = [g_{i},g_{j}]$

In the following we use that $\frac{1}{p-1} = 1 + p + p^{2} + \dotsb$ and $\log(1-p) = -p - \frac{p^{2}}{2} - \frac{p^{3}}{3} - \dotsb$.

\begin{description}
  \item[$g_{12} = \pmat{ 1 & 0 \\ p\bigl( 1 - \exp(-2p) \bigr) }$:] Comparing $g_{12}$ with \eqref{eq:gixi-SL2}, we see that $x_{2} = x_{3} = 0$. This leaves $a_{21} = px_{1} = p\bigl( 1 - \exp(-2p) \bigr) = 2p^{2} + O(p^{3})$, which implies that $x_{1} = 2p + O(p^{2})$. Hence $\sigma(g_{12}) = 2\pi \act \sigma(g_{1})$, which implies that $\xi_{12} = 0$.

  \item[$g_{13} = \pmat{ 1-p & p \\ -p^{2} & 1+p+p^{2} }$:] Comparing $g_{13}$ with \eqref{eq:gixi-SL2}, we see that
        \begin{align*}
          a_{11} &= \exp(px_{2}) = 1-p, \\
          a_{12} &= x_{3}\exp(px_{2}) = x_{3}(1-p) = p, \\
          a_{21} &= px_{1}\exp(px_{2}) = px_{1}(1-p) = -p^{2},
        \end{align*}
        and thus
        \begin{align*}
          x_{2} &= \dfrac{1}{p}\log(1-p) = \dfrac{1}{p}\bigl( (-p) + O(p^{2}) \bigr) = -1 + O(p), \\
          x_{3} &= \dfrac{p}{1-p} = p + O(p^{2}), \\
          x_{1} &= \dfrac{-p^{2}}{p(1-p)} = -p + O(p^{2}).
        \end{align*}
        Hence $\sigma(g_{13}) = -\pi \act \sigma(g_{1}) - \sigma(g_{2}) - \pi \act \sigma(g_{3})$, which implies that $\xi_{13} = -\xi_{2}$.

  \item[$g_{23} = \pmat{ 1 & \exp(2p)-1 \\ 0 & 1 }$:] Comparing $g_{23}$ with \eqref{eq:gixi-SL2}, we see that $x_{1} = x_{2} = 0$. This leaves $a_{12} = x_{3} = \exp(2p)-1 = 2p + O(p^{2})$. Hence $\sigma(g_{23}) = 2\pi \act \sigma(g_{3})$, which implies that $\xi_{23} = 0$.
\end{description}


\begin{align*}
  \sigma(g_{12}) &= 2\pi \act \sigma(g_{1}), \\
  \sigma(g_{13}) &= \pi \act \sigma(g_{1}) + (p-1)\sigma(g_{2}) + \pi \act \sigma(g_{3}), \\
  \sigma(g_{23}) &= \pi \act \sigma(g_{3}).
\end{align*}


So with $\xi_{i} = 1 \otimes \sigma(g_{i})$:
\begin{align*}
  [\xi_{1},\xi_{2}] &= 0, & [\xi_{1},\xi_{3}] &= -\xi_{2}, & [\xi_{2},\xi_{3}] &= 0.
\end{align*}

\section{\texorpdfstring{$I \subseteq \GL_{2}(\Z_{p})$}{I in GL2(Zp)}}%
\label{sec:Iwa-GL2}

\begin{equation}
  \label{eq:gis-GL2}
  \begin{gathered}
    g_{1} = \pmat{1 & 0 \\ p & 1}, \qquad g_{2} = \pmat{\exp(p) & 0 \\ 0 & \exp(-p)}, \\
    g_{3} = \pmat{\exp(p) & 0 \\ 0 & \exp(p)}, \qquad g_{4} = \pmat{1 & 1 \\ 0 & 1}.
  \end{gathered}
\end{equation}

\begin{align}
  & g_{1}^{x_{1}}g_{2}^{x_{2}}g_{3}^{x_{3}}g_{4}^{x_{4}}  \notag \\
                &= \pmat{ \exp\bigl( p(x_{2}+x_{3}) \bigr) & \exp\bigl( p(x_{2}+x_{3}) \bigr)x_{4} \\ px_{1}\exp\bigl( p(x_{2}+x_{3}) \bigr) & \exp\bigl( p(x_{2}+x_{3}) \bigr)px_{1}x_{4} + \exp\bigl( p(x_{3}-x_{2}) \bigr) } \\
                &= \pmat{ a_{11} & a_{12} \\ a_{21} & a_{22} }. \notag
\end{align}

$g_{ij} = [g_{i},g_{j}]$
\begin{align*}
  \sigma(g_{12}) &= (p-2)\pi \act \sigma(g_{1}), \\
  \sigma(g_{14}) &= (p-1)\pi \act \sigma(g_{1}) + (p-1)\sigma(g_{2}) + \pi \act \sigma(g_{3}), \\
  \sigma(g_{24}) &= (p-2)\pi \act \sigma(g_{3}), \\
  \sigma(g_{13}) &= \sigma(g_{23}) = \sigma(g_{24}) = 0.
\end{align*}

So with $\xi_{i} = 1 \otimes \sigma(g_{i})$:
\begin{equation*}
  [\xi_{1},\xi_{4}] = -\xi_{2}
\end{equation*}
is the only non-zero commutator.

\section{\texorpdfstring{$I \subseteq \SL_{3}(\Z_{p})$}{I in SL3(Zp)}}%
\label{sec:Iwa-SL3}

To make the notation easier to read for the bigger matrices, we will write any zeros as blank space in matrices in this section.
\begin{equation}
  \label{eq:gis-SL3}
  \begin{gathered}
    g_{1} = \pmat{ 1 \\ & 1 \\ p && 1 }, \quad g_{2} = \pmat{ 1 \\ p & 1 \\ && 1 }, \quad g_{3} = \pmat{ 1 \\ & 1 \\ & p & 1 }, \\
    g_{4} = \pmat{ \exp(p) \\ & \exp(-p) \\ && 1 }, \quad g_{5} = \pmat{ 1 \\ & \exp(p) \\ && \exp(-p) }, \\
    g_{6} = \pmat{ 1 \\ & 1 & 1 \\ && 1 }, \quad g_{7} = \pmat{ 1 & 1 \\ & 1 \\ && 1 }, \quad g_{8} = \pmat{ 1 && 1 \\ & 1 \\ && 1 }.
  \end{gathered}
\end{equation}

\begin{equation*}
    g_{1}^{x_{1}}g_{2}^{x_{2}}g_{3}^{x_{3}}g_{4}^{x_{4}}g_{5}^{x_{5}}g_{6}^{x_{6}}g_{7}^{x_{7}}g_{8}^{x_{8}} = \pmat{ a_{11} & a_{12} & a_{13} \\ a_{21} & a_{22} & a_{23} \\ a_{31} & a_{32} & a_{33}}
\end{equation*}
where
\begin{equation}
  \label{eq:gixi-SL3}
  \begin{aligned}
    a_{11} &= \exp(px_{4}), \\
    a_{12} &= x_{7}\exp(px_{4}), \\
    a_{13} &= x_{8}\exp(px_{4}), \\
    a_{21} &= px_{2}\exp(px_{4}), \\
    a_{22} &= px_{2}x_{7}\exp(px_{4}) + \exp\bigl( p(x_{5}-x_{4}) \bigr), \\
    a_{23} &= px_{2}x_{8}\exp(px_{4}) + x_{6}\exp\bigl( p(x_{5}-x_{4}) \bigr), \\
    a_{31} &= px_{1}\exp(px_{4}), \\
    a_{32} &= px_{1}x_{7}\exp(px_{4}) + px_{3}\exp\bigl( p(x_{5}-x_{4}) \bigr), \\
    a_{33} &= px_{1}x_{8}\exp(px_{4}) + px_{3}x_{6}\exp\bigl( p(x_{5}-x_{4}) \bigr) + \exp(-px_{5}).
  \end{aligned}
\end{equation}


\subsection{\texorpdfstring{Non-identiy $[g_{i},g_{j}]$}{Non-identity [gi,gj]}}%
\label{subsec:non-id-gij-SL3}

$g_{ij} = [g_{i},g_{j}]$

Except in the first case, we will note that $x_{i} \in p\Z_{p}$ implies that the coefficient on $\xi_{k}$ in $\xi_{ij}$ is zero. \dknote{Introduce $O(p^{k})$ notation.}

Note that we repeatedly use that $-1 = (p-1) + (p-1)p + (p-1)p^{2} + \dotsb$ in $\Z_{p}$ and $-1 = p-1$ in $\F_{p}$.

\begin{description}
  \item[$g_{14} = \pmat{1 \\ & 1 \\ p\bigl( 1-\exp(-p) \bigr) && 1}$:] Comparing $g_{14}$ with \eqref{eq:gixi-SL3}, we see that $x_{2} = x_{4} = x_{7} = x_{8} = 0$, and thus also $x_{3} = x_{5} = x_{6} = 0$. This leaves $a_{31} = px_{1} = p\bigl( 1-\exp(-p) \bigr) = p^{2} + O(p^{3})$, which implies that $x_{1} = p + O(p^{2})$. Hence $\sigma(g_{14}) = \pi \act \sigma(g_{1})$, which implies that $\xi_{14} = 0$.

  \item[$g_{15} = \pmat{1 \\ & 1 \\ p\bigl( 1-\exp(-p) \bigr) && 1}$:] Since $g_{15} = g_{14}$, the above calculation shows that $\xi_{15} = 0$.

  \item[$g_{16} = \pmat{1 \\ -p & 1 \\ && 1}$:] Comparing $g_{16}$ with \eqref{eq:gixi-SL3}, we see that $x_{1} = x_{4} = x_{7} = x_{8} = 0$, and thus also $x_{3} = x_{5} = x_{6} = 0$. This leaves $a_{21} = px_{2} = -p$, which implies that $x_{2} = -1$. Hence $\sigma(g_{16}) = -\sigma(g_{2})$, which implies that $\xi_{16} = -\xi_{2}$.

  \item[$g_{17} = \pmat{1 \\ & 1 \\ & p & 1}$:] Comparing $g_{17}$ with \eqref{eq:gixi-SL3}, we see that $x_{1} = x_{2} = x_{4} = x_{7} = x_{8} = 0$, and thus also $x_{5} = x_{6} = 0$. This leaves $a_{32} = px_{3} = p$, which implies that $x_{3} = 1$. Hence $\sigma(g_{17}) = \sigma(g_{3})$, which implies that $\xi_{17} = \xi_{3}$.

  \item[$g_{18} = \pmat{1-p && p \\ & 1 \\ -p^{2} && 1+p+p^{2}}$:] Comparing $g_{18}$ with \eqref{eq:gixi-SL3}, we see that $x_{2} = x_{7} = 0$, and thus also $x_{3} = x_{6} = 0$ and $x_{4} = x_{5}$. Using
        \begin{align*}
          a_{11} &= \exp(px_{4}) = 1-p, \\
          a_{13} &= x_{8}\exp(px_{4}) = x_{8}(1-p) = p, \\
          a_{31} &= px_{1}\exp(px_{4}) = px_{1}(1-p) = -p^{2},
        \end{align*}
        we get that
        \begin{align*}
          x_{4} &= \dfrac{1}{p}\log(1-p) = \dfrac{1}{p}\bigl( (-p) + O(p^{2}) \bigr) = -1 + O(p), \\
          x_{8} &= \dfrac{p}{1-p} = p + O(p^{2}), \\
          x_{1} &= \dfrac{-p^{2}}{p(1-p)} = -p + O(p^{2}).
        \end{align*}
        Hence $\sigma(g_{18}) = -\pi \act \sigma(g_{1}) - \sigma(g_{4}) - \sigma(g_{5}) + \pi \act \sigma(g_{8})$, which implies that $\xi_{18} = -(\xi_{4}+\xi_{5})$.

  \item[$g_{23} = \pmat{1 \\ & 1 \\ -p^{2} && 1}$:] Comparing $g_{23}$ with \eqref{eq:gixi-SL3}, we see that $x_{2} = x_{4} = x_{7} = x_{8} = 0$, and thus also $x_{3} = x_{5} = x_{6} = 0$. This leaves $a_{31} = px_{1} = -p^{2}$, which implies that $x_{1} = -p$. Hence $\sigma(g_{23}) = -\pi \act \sigma(g_{1})$, which implies that $\xi_{23} = 0$.

  \item[$g_{24} = \pmat{1 \\ p\bigl( 1-\exp(-2p) \bigr) & 1 \\ && 1}$:] Comparing $g_{24}$ with \eqref{eq:gixi-SL3}, we see that $x_{1} = x_{4} = x_{7} = x_{8} = 0$, and thus also $x_{3} = x_{5} = x_{6} = 0$. This leaves $a_{21} = px_{2} = p\bigl( 1-\exp(-2p) \bigr) = p\bigl( 1-\bigl( 1+(-2p)+O(p^{2}) \bigr) \bigr) = 2p^{2} + O(p^{3})$, which implies that $x_{2} = 2p + O(p^{2})$. Hence $\sigma(g_{24}) = 2\pi \act \sigma(g_{1})$, which implies that $\xi_{24} = 0$.

  \item[$g_{25} = \pmat{1 \\ p\bigl( 1-\exp(p) \bigr) & 1 \\ && 1}$:] Except a factor $-2$ in the exponential, which clearly doesn't change the final result, we have the same calculation as for $g_{24}$. Thus $\xi_{25} = 0$.

  \item[$g_{27} = \pmat{ 1-p & p \\ -p^{2} & 1+p+p^{2} \\ && 1}$:] Comparing $g_{27}$ with \eqref{eq:gixi-SL3}, we see that $x_{1} = x_{8} = 0$, and thus also $x_{3} = x_{6} = 0$, so $x_{5} = 0$. Using
        \begin{align*}
          a_{11} &= \exp(px_{4}) = 1-p, \\
          a_{12} &= x_{7}\exp(px_{4}) = x_{8}(1-p) = p, \\
          a_{21} &= px_{2}\exp(px_{4}) = px_{2}(1-p) = -p^{2},
        \end{align*}
        we get that
        \begin{align*}
          x_{4} &= \dfrac{1}{p}\log(1-p) = \dfrac{1}{p}\bigl( (-p) + O(p^{2}) \bigr) = -1 + O(p), \\
          x_{7} &= \dfrac{p}{1-p} = p + O(p^{2}), \\
          x_{2} &= \dfrac{-p^{2}}{p(1-p)} = -p + O(p^{2}).
        \end{align*}
        Hence $\sigma(g_{27}) = -\pi \act \sigma(g_{2}) - \sigma(g_{4}) + \pi \act \sigma(g_{7})$, which implies that $\xi_{27} = -\xi_{4}$.

  \item[$g_{28} = \pmat{ 1 \\ & 1 & p \\ && 1}$:] Comparing $g_{28}$ with \eqref{eq:gixi-SL3}, we see that $x_{1} = x_{2} = x_{4} = x_{7} = x_{8} = 0$, and thus also $x_{3} = x_{5} = 0$. This leaves $a_{23} = x_{6} = p$. Hence $\sigma(g_{28}) = \pi \act \sigma(g_{6})$, which implies that $\xi_{28} = 0$.

  \item[$g_{34} = \pmat{ 1 \\ & 1 \\ & p\bigl( 1-\exp(p) \bigr) & 1}$:] Comparing $g_{34}$ with \eqref{eq:gixi-SL3}, we see that $x_{1} = x_{2} = x_{4} = x_{7} = x_{8} = 0$, and thus also $x_{5} = x_{6} = 0$. This leaves $a_{32} = px_{3} = p\bigl( 1-\exp(p) \bigr) = p\bigl( 1-\bigl( 1+p+O(p^{2}) \bigr) \bigr) = -p^{2} + O(p^{3})$, which implies that $x_{3} = -p + O(p^{2})$. Hence $\sigma(g_{34}) = -\pi \act \sigma(g_{3})$, which implies that $\xi_{34} = 0$.

  \item[$g_{35} = \pmat{ 1 \\ & 1 \\ & p\bigl( 1-\exp(-2p) \bigr) & 1}$:] Except a factor $-2$ in the exponential, which clearly doesn't change the final result, we have the same calculation as for $g_{34}$. Thus $\xi_{35} = 0$.

  \item[$g_{36} = \pmat{ 1 \\ & 1-p & p \\ & -p^{2} & 1+p+p^{2}}$:] Comparing $g_{36}$ with \eqref{eq:gixi-SL3}, we see that $x_{1} = x_{2} = x_{4} = x_{7} = x_{8} = 0$. Using
        \begin{align*}
          a_{22} &= \exp(px_{5}) = 1-p, \\
          a_{23} &= x_{6}\exp(px_{5}) = x_{6}(1-p) = p, \\
          a_{32} &= px_{3}\exp(px_{5}) = px_{3}(1-p) = -p^{2},
        \end{align*}
        we get that
        \begin{align*}
          x_{5} &= \dfrac{1}{p}\log(1-p) = \dfrac{1}{p}\bigl( (-p) + O(p^{2}) \bigr) = -1 + O(p), \\
          x_{6} &= \dfrac{p}{1-p} = p + O(p^{2}), \\
          x_{3} &= \dfrac{-p^{2}}{p(1-p)} = -p + O(p^{2}).
        \end{align*}
        Hence $\sigma(g_{36}) = -\pi \act \sigma(g_{3}) - \sigma(g_{5}) + \pi \act \sigma(g_{6})$, which implies that $\xi_{36} = -\xi_{5}$.

  \item[$g_{38} = \pmat{ 1 & -p \\ & 1 \\ && 1}$:] Comparing $g_{38}$ with \eqref{eq:gixi-SL3}, we see that $x_{1} = x_{2} = x_{4} = x_{8} = 0$, and thus also $x_{3} = x_{5} = x_{6} = 0$. This leaves $a_{12} = x_{7} = -p$. Hence $\sigma(g_{38}) = -\pi \act \sigma(g_{3})$, which implies that $\xi_{38} = 0$.

  \item[$g_{46} = \pmat{ 1 \\ & 1 & \exp(-p)-1 \\ && 1}$:] Comparing $g_{46}$ with \eqref{eq:gixi-SL3}, we see that $x_{1} = x_{2} = x_{4} = x_{7} = x_{8} = 0$, and thus also $x_{3} = x_{5} = 0$. This leaves $a_{23} = x_{6} = \exp(-p) - 1 = -p + O(p^{2})$. Hence $\sigma(g_{46}) = -\pi \act \sigma(g_{6})$, which implies that $\xi_{46} = 0$.

  \item[$g_{47} = \pmat{ 1 & \exp(2p)-1 \\ & 1 \\ && 1}$:] Comparing $g_{47}$ with \eqref{eq:gixi-SL3}, we see that $x_{1} = x_{2} = x_{4} = x_{8} = 0$, and thus also $x_{3} = x_{5} = x_{6} = 0$. This leaves $a_{12} = x_{7} = \exp(2p) - 1 = 2p + O(p^{2})$. Hence $\sigma(g_{47}) = 2\pi \act \sigma(g_{7})$, which implies that $\xi_{47} = 0$.

  \item[$g_{48} = \pmat{ 1 && \exp(p)-1 \\ & 1 \\ && 1}$:] Comparing $g_{48}$ with \eqref{eq:gixi-SL3}, we see that $x_{1} = x_{2} = x_{4} = x_{7} = 0$, and thus also $x_{3} = x_{5} = x_{6} = 0$. This leaves $a_{13} = x_{8} = \exp(p) - 1 = p + O(p^{2})$. Hence $\sigma(g_{48}) = \pi \act \sigma(g_{8})$, which implies that $\xi_{48} = 0$.

  \item[$g_{56} = \pmat{ 1 \\ & 1 & \exp(2p)-1 \\ && 1}$:] Except a factor $-2$ in the exponential, which clearly doesn't change the final result, we have the same calculation as for $g_{46}$. Thus $\xi_{56} = 0$.

  \item[$g_{57} = \pmat{ 1 & \exp(-p)-1 \\ & 1 \\ && 1}$:] Except a factor $-2$ in the exponential, which clearly doesn't change the final result, we have the same calculation as for $g_{47}$. Thus $\xi_{57} = 0$.

  \item[$g_{58} = \pmat{ 1 && \exp(p)-1 \\ & 1 \\ && 1}$:] Since $g_{58} = g_{48}$, the above calculation shows that $\xi_{58} = 0$.

  \item[$g_{67} = \pmat{ 1 && -1 \\ & 1 \\ && 1}$:] Comparing $g_{67}$ with \eqref{eq:gixi-SL3}, we see that $x_{1} = x_{2} = x_{4} = x_{7} = 0$, and thus also $x_{3} = x_{5} = x_{6} = 0$. This leaves $a_{13} = x_{8} = -1$. Hence $\sigma(g_{67}) = -\sigma(g_{8})$, which implies that $\xi_{67} = -\xi_{8}$.
\end{description}


The non-zero commutators are:
\begin{equation}
  \label{eq:xi_ij-SL3}
  \begin{aligned}
    [\xi_{1},\xi_{6}] &= -\xi_{2}, & [\xi_{1},\xi_{7}] &= \xi_{3}, & [\xi_{1},\xi_{8}] &= -(\xi_{4}+\xi_{5}), \\
    [\xi_{2},\xi_{7}] &= -\xi_{4}, & [\xi_{3},\xi_{6}] &= -\xi_{5}, & [\xi_{6},\xi_{7}] &= -\xi_{8}.
  \end{aligned}
\end{equation}

\begin{equation*}
  \lie{g} = k \otimes_{\F_{p}[\pi]} \gr I = \Span\set{\xi_{1},\dotsc,\xi_{8}} = \lie{g}_{\frac{1}{3}} \oplus \lie{g}_{\frac{2}{3}} \oplus \lie{g}_{1} = \lie{g}^{1} \oplus \lie{g}^{2} \oplus \lie{g}^{3}.
\end{equation*}

\begin{equation}
  \label{eq:5}
  [\lie{g}^{i},\lie{g}^{j}] =
  \begin{dcases*}
    \lie{g}^{2} & if $i=j=1$, \\
    \lie{g}^{3} & if $(i,j)\in \set{(1,2),(2,1)}$, \\
    0 & otherwise.
  \end{dcases*}
\end{equation}

\begin{equation*}
  \gr^{j}\Bigl( \bigwedge^{n}\lie{g} \Bigr) = \bigoplus_{j_{1} + \dotsb + j_{n} = j} \lie{g}^{j_{1}} \wedge \dotsb \wedge \lie{g}^{j_{n}}.
\end{equation*}

\begin{description}
  \item[$n\geq9:$]
        \begin{equation*}
          \gr^{j}\Bigl( \bigwedge^{n}\lie{g} \Bigr) = 0 \text{ for all } j.
        \end{equation*}
  \item[$n=8:$]
        \begin{equation*}
          \gr^{j}\Bigl( \bigwedge^{8}\lie{g} \Bigr) =
          \begin{dcases}
            \lie{g}^{1} \wedge \lie{g}^{1} \wedge \lie{g}^{1} \wedge \lie{g}^{2} \wedge \lie{g}^{2} \wedge \lie{g}^{2} \wedge \lie{g}^{3} \wedge \lie{g}^{3} & j=15, \\
            0                                                                                                         & \text{otherwise.}
          \end{dcases}
        \end{equation*}
  \item[$n=7:$]
        \begin{equation*}
          \gr^{j}\Bigl( \bigwedge^{7}\lie{g} \Bigr) =
          \begin{dcases}
            \lie{g}^{1} \wedge \lie{g}^{1} \wedge \lie{g}^{2} \wedge \lie{g}^{2} \wedge \lie{g}^{2} \wedge \lie{g}^{3} \wedge \lie{g}^{3}                    & j=14, \\
            \lie{g}^{1} \wedge \lie{g}^{1} \wedge \lie{g}^{1} \wedge \lie{g}^{2} \wedge \lie{g}^{2} \wedge \lie{g}^{3} \wedge \lie{g}^{3}                    & j=13, \\
            \lie{g}^{1} \wedge \lie{g}^{1} \wedge \lie{g}^{1} \wedge \lie{g}^{2} \wedge \lie{g}^{2} \wedge \lie{g}^{2} \wedge \lie{g}^{3}                    & j=12, \\
            0                                                                                                               & \text{otherwise.}
          \end{dcases}
        \end{equation*}

  \item[$n=6:$]
        \begin{equation*}
          \gr^{j}\Bigl( \bigwedge^{6}\lie{g} \Bigr) =
          \begin{dcases}
            \lie{g}^{1} \wedge \lie{g}^{2} \wedge \lie{g}^{2} \wedge \lie{g}^{2} \wedge \lie{g}^{3} \wedge \lie{g}^{3} & j=13,                                                                                                                   \\
            \lie{g}^{1} \wedge \lie{g}^{1} \wedge \lie{g}^{2} \wedge \lie{g}^{2} \wedge \lie{g}^{3} \wedge \lie{g}^{3}                                                                                                                   & j=12, \\
            \!\begin{aligned} & \lie{g}^{1} \wedge \lie{g}^{1} \wedge \lie{g}^{1} \wedge \lie{g}^{2} \wedge \lie{g}^{3} \wedge \lie{g}^{3} \\ & \oplus \lie{g}^{1} \wedge \lie{g}^{1} \wedge \lie{g}^{2} \wedge \lie{g}^{2} \wedge \lie{g}^{2} \wedge \lie{g}^{3} \end{aligned} & j=11, \\
            \lie{g}^{1} \wedge \lie{g}^{1} \wedge \lie{g}^{1} \wedge \lie{g}^{2} \wedge \lie{g}^{2} \wedge \lie{g}^{3}                                                                                                                   & j=10, \\
            \lie{g}^{1} \wedge \lie{g}^{1} \wedge \lie{g}^{1} \wedge \lie{g}^{2} \wedge \lie{g}^{2} \wedge \lie{g}^{2}                                                                                                                   & j=9,  \\
            0                                                                                                                                                                                                & \text{otherwise.}
          \end{dcases}
        \end{equation*}

  \item[$n=5:$]
        \begin{equation*}
          \gr^{j}\Bigl( \bigwedge^{5}\lie{g} \Bigr) =
          \begin{dcases}
            \lie{g}^{2} \wedge \lie{g}^{2} \wedge \lie{g}^{2} \wedge \lie{g}^{3} \wedge \lie{g}^{3} & j=12,                                                                                                                   \\
            \lie{g}^{1} \wedge \lie{g}^{2} \wedge \lie{g}^{2} \wedge \lie{g}^{3} \wedge \lie{g}^{3} & j=11, \\
            \!\begin{aligned} & \lie{g}^{1} \wedge \lie{g}^{1} \wedge \lie{g}^{2} \wedge \lie{g}^{3} \wedge \lie{g}^{3} \\ & \oplus \lie{g}^{1} \wedge \lie{g}^{2} \wedge \lie{g}^{2} \wedge \lie{g}^{2} \wedge \lie{g}^{3} \end{aligned} & j=10, \\
            \!\begin{aligned} & \lie{g}^{1} \wedge \lie{g}^{1} \wedge \lie{g}^{1} \wedge \lie{g}^{3} \wedge \lie{g}^{3} \\ & \oplus \lie{g}^{1} \wedge \lie{g}^{1} \wedge \lie{g}^{2} \wedge \lie{g}^{2} \wedge \lie{g}^{3} \end{aligned} & j=9, \\
            \!\begin{aligned} & \lie{g}^{1} \wedge \lie{g}^{1} \wedge \lie{g}^{1} \wedge \lie{g}^{2} \wedge \lie{g}^{3} \\ & \oplus \lie{g}^{1} \wedge \lie{g}^{1} \wedge \lie{g}^{2} \wedge \lie{g}^{2} \wedge \lie{g}^{2} \end{aligned} & j=8, \\
            \lie{g}^{1} \wedge \lie{g}^{1} \wedge \lie{g}^{1} \wedge \lie{g}^{2} \wedge \lie{g}^{2}                                                                                                                  & j=7, \\
            0                                                                                                                                                                                                      & \text{otherwise.}
          \end{dcases}
        \end{equation*}

  \item[$n=4:$]
        \begin{equation*}
          \gr^{j}\Bigl( \bigwedge^{4}\lie{g} \Bigr) =
          \begin{dcases}
            \lie{g}^{2} \wedge \lie{g}^{2} \wedge \lie{g}^{3} \wedge \lie{g}^{3} & j=10,                                                                                                                   \\
            \!\begin{aligned} & \lie{g}^{1} \wedge \lie{g}^{2} \wedge \lie{g}^{3} \wedge \lie{g}^{3} \\ & \oplus \lie{g}^{2} \wedge \lie{g}^{2} \wedge \lie{g}^{2} \wedge \lie{g}^{3} \end{aligned} & j=9, \\
            \!\begin{aligned} & \lie{g}^{1} \wedge \lie{g}^{1} \wedge \lie{g}^{3} \wedge \lie{g}^{3} \\ & \oplus \lie{g}^{1} \wedge \lie{g}^{2} \wedge \lie{g}^{2} \wedge \lie{g}^{3} \end{aligned} & j=8, \\
            \!\begin{aligned} & \lie{g}^{1} \wedge \lie{g}^{1} \wedge \lie{g}^{2} \wedge \lie{g}^{3} \\ & \oplus \lie{g}^{1} \wedge \lie{g}^{2} \wedge \lie{g}^{2} \wedge \lie{g}^{2} \end{aligned} & j=7, \\
            \!\begin{aligned} & \lie{g}^{1} \wedge \lie{g}^{1} \wedge \lie{g}^{1} \wedge \lie{g}^{3} \\ & \oplus \lie{g}^{1} \wedge \lie{g}^{1} \wedge \lie{g}^{2} \wedge \lie{g}^{2} \end{aligned} & j=6, \\
            \lie{g}^{1} \wedge \lie{g}^{1} \wedge \lie{g}^{1} \wedge \lie{g}^{2}                                                                                                                  & j=5, \\
            0                                                                                                                                                                                   & \text{otherwise.}
          \end{dcases}
        \end{equation*}

  \item[$n=3:$]
        \begin{equation*}
          \gr^{j}\Bigl( \bigwedge^{3}\lie{g} \Bigr) =
          \begin{dcases}
            \lie{g}^{2} \wedge \lie{g}^{3} \wedge \lie{g}^{3} & j=8,                                                                                                                   \\
            \!\begin{aligned} & \lie{g}^{1} \wedge \lie{g}^{3} \wedge \lie{g}^{3} \\ & \oplus \lie{g}^{2} \wedge \lie{g}^{2} \wedge \lie{g}^{3} \end{aligned} & j=7, \\
            \!\begin{aligned} & \lie{g}^{1} \wedge \lie{g}^{2} \wedge \lie{g}^{3} \\ & \oplus \lie{g}^{2} \wedge \lie{g}^{2} \wedge \lie{g}^{2} \end{aligned} & j=6, \\
            \!\begin{aligned} & \lie{g}^{1} \wedge \lie{g}^{1} \wedge \lie{g}^{3} \\ & \oplus \lie{g}^{1} \wedge \lie{g}^{2} \wedge \lie{g}^{2} \end{aligned} & j=5, \\
            \lie{g}^{1} \wedge \lie{g}^{1} \wedge \lie{g}^{2}                                                                                                & j=4, \\
            \lie{g}^{1} \wedge \lie{g}^{1} \wedge \lie{g}^{1}                                                                                                & j=3, \\
            0                                                                                                                                              & \text{otherwise.}
          \end{dcases}
        \end{equation*}

  \item[$n=2:$]
        \begin{equation*}
          \gr^{j}\Bigl( \bigwedge^{2}\lie{g} \Bigr) =
          \begin{dcases}
            \lie{g}^{3} \wedge \lie{g}^{3} & j=6,                                                                             \\
            \lie{g}^{2} \wedge \lie{g}^{3} & j=5,                                                                             \\
            \!\begin{aligned} & \lie{g}^{1} \wedge \lie{g}^{3} \\ & \oplus \lie{g}^{2} \wedge \lie{g}^{2} \end{aligned} & j=4, \\
            \lie{g}^{1} \wedge \lie{g}^{2}                                                                             & j=3, \\
            \lie{g}^{1} \wedge \lie{g}^{1}                                                                             & j=2, \\
            0                                                                                                         & \text{otherwise.}
          \end{dcases}
        \end{equation*}

  \item[$n=1:$]
        \begin{equation*}
          \gr^{j}(\lie{g}) =
          \begin{dcases}
            \lie{g}^{3} & j=3, \\
            \lie{g}^{2} & j=2, \\
            \lie{g}^{1} & j=1, \\
            0          & \text{otherwise.}
          \end{dcases}
        \end{equation*}

  \item[$n=0:$]
        \begin{equation*}
          \gr^{j}(k) =
          \begin{dcases}
            k & j=0, \\
            0 & \text{otherwise.}
          \end{dcases}
        \end{equation*}
\end{description}

\begin{table}[h]
  \centering
  $\begin{NiceArray}{*{17}{c}}[hvlines]
    \diagbox{n}{j} & 0 & 1 & 2 & 3 & 4 & 5 & 6 & 7 & 8 & 9 & 10 & 11 & 12 & 13 & 14 & 15 \\
    0 & 1 \\
    1 & & 3 & 3 & 2 \\
    2 & & & 3 & 9 & 9 & 6 & 1 \\
    3 & & & & 1 & 9 & 15 & 19 & 9 & 3 \\
    4 & & & & & & 3 & 11 & 21 & 21 & 11 & 3 \\
    5 & & & & & & & & 3 & 9 & 19 & 15 & 9 & 1 \\
    6 & & & & & & & & & & 1 & 6 & 9 & 9 & 3 \\
    7 & & & & & & & & & & & & & 2 & 3 & 3 \\
    8 & & & & & & & & & & & & & & & & 1
  \end{NiceArray}$
  \caption[Graded complex dimensions for $I \subseteq \SL_{3}(\Z_{p})$]{Dimensions of $\gr^{j}\bigl( \bigwedge^{n} \lie{g} \bigr)$.}
  \label{tab:graded-dims-SL3}
\end{table}

\begin{equation*}
  \Hom_{k}\Bigl( \bigwedge^{n}\lie{g}, k \Bigr) = \bigoplus_{s \in \Z} \Hom_{k}^{s}\Bigl( \bigwedge^{n}\lie{g}, k \Bigr)
\end{equation*}

With $j=-s$, we get the same table for dimensions of the graded hom-spaces.

Note that when finding cohomology, we only need to consider $H^{s,t} = H^{s,n-s}$ for the non-zero entries of \Cref{tab:graded-dims-SL3}.

We repeatedly use that, if
\begin{equation*}
  d \snfsim \SNF^{n,m}(a_{1},\dotsc,a_{r},0,\dotsc,0)
\end{equation*}
with $a_{1},\dotsc,a_{r}$ non-zero (in $\F_{p}$), then
\begin{align*}
  \dim \kernel(d) &= m-r, \\
  \dim \image(d) &= r, \\
  \dim \coker(d) &= n-r.
\end{align*}

$\gr^{0}:$
\[
  \begin{tikzcd}
    0 \ar[r] & k \ar[r] & 0
  \end{tikzcd}
\]



\[
  \begin{tikzcd}
    0 & \ar[l] \Hom_{k}^{0}(k,k) & \ar[l] 0
  \end{tikzcd}
\]

So $H^{0} = H^{0,0}$ with $\dim H^{0,0} = 1$.

$\gr^{1}:$
\[
  \begin{tikzcd}
    0 \ar[r] & \lie{g}^{1} \ar[r] & 0
  \end{tikzcd}
\]

\[
  \begin{tikzcd}
    0 & \ar[l] \Hom_{k}^{-1}(\lie{g},k) & \ar[l] 0
  \end{tikzcd}
\]

So $\dim H^{-1,2} = 3$ by \Cref{tab:graded-dims-SL3}.

$\gr^{2}:$
\[
  \begin{tikzcd}[ampersand replacement=\&]
    0 \ar[r] \& \lie{g}^{1} \wedge \lie{g}^{1} \ar[r, "{\begin{pmatrix} 1 & 0 & 0 \\ 0 & -1 & 0 \\ 0 & 0 & 1 \end{pmatrix}}" {yshift=7pt}] \& \lie{g}^{2} \ar[r] \& 0
  \end{tikzcd}
\]

\begin{align*}
  \lie{g}^{1} \wedge \lie{g}^{1} &\to \lie{g}^{2} \\
  \xi_{1} \wedge \xi_{6} &\mapsto -[\xi_{1},\xi_{6}] = \xi_{2} \\
  \xi_{1} \wedge \xi_{7} &\mapsto -[\xi_{1},\xi_{7}] = -\xi_{3} \\
  \xi_{6} \wedge \xi_{7} &\mapsto -[\xi_{6},\xi_{7}] = \xi_{8}.
\end{align*}

\[
  \begin{tikzcd}[ampersand replacement=\&]
    0 \& \ar[l] \Hom_{k}^{-2}\bigl( \bigwedge^{2} \lie{g}, k \bigr) \& \ar[l, "{\begin{pmatrix} 1 & 0 & 0 \\ 0 & -1 & 0 \\ 0 & 0 & 1 \end{pmatrix}}"' {yshift=7pt}] \Hom_{k}^{-2}(\lie{g},k) \& \ar[l] 0
  \end{tikzcd}
\]

\begin{equation*}
  d = \pmat{1&0&0 \\ 0&-1&0 \\ 0&0&1} \snfsim  \SNF^{3\times3}(1,-1,1).
\end{equation*}

So
\begin{align*}
  \dim H^{-2,3} &= \dim \kernel(d) = 0, \\
  \dim H^{-2,4} &= \dim \coker(d) = 0.
\end{align*}

$\gr^{3}:$
\[
  \begin{tikzcd}[ampersand replacement=\&, column sep=4em]
    0 \ar[r] \& \lie{g}^{1} \wedge \lie{g}^{1} \wedge \lie{g}^{1} \ar[r, "{\begin{pmatrix} 0 & 0 & -1 & 0 & -1 & 0 & -1 & 0 & 0 \end{pmatrix}^{\top}}" {yshift=7pt}] \& \lie{g}^{1} \wedge \lie{g}^{2} \ar[r, "{\begin{pmatrix} 0 & 0 & 1 & 0 & 0 & 0 & -1 & 0 & 0 \\ 0 & 0 & 1 & 0 & -1 & 0 & 0 & 0 & 0 \end{pmatrix}}"' {yshift=-7pt}] \& \lie{g}^{3} \ar[r] \& 0
  \end{tikzcd}
\]

\[
  \begin{tikzcd}[ampersand replacement=\&, column sep=1em]
    0 \& \ar[l] \Hom_{k}^{-3}\bigl( \bigwedge^{3}\lie{g}, k \bigr) \& \ar[l, "{\begin{pmatrix} 0 & 0 & -1 & 0 & -1 & 0 & -1 & 0 & 0 \end{pmatrix}}"' {yshift=7pt}] \Hom_{k}^{-3}\bigl( \bigwedge^{2}\lie{g}, k \bigr) \& \ar[l, "{\begin{pmatrix} 0 & 0 & 1 & 0 & 0 & 0 & -1 & 0 & 0 \\ 0 & 0 & 1 & 0 & -1 & 0 & 0 & 0 & 0 \end{pmatrix}^{\top}}" {yshift=-7pt}] \Hom_{k}^{-3}(\lie{g}, k) \& \ar[l] 0
  \end{tikzcd}
\]

\begin{align*}
  d_{1} = \pmat{0&0 \\ 0&0 \\ 1&1 \\ 0&0 \\ 0&-1 \\ 0&0 \\ 0&-1 \\ 0&0 \\ 0&0} &\snfsim \SNF^{9\times2}(1,-1), \\
  d_{2} = \pmat{0&0&-1&0&-1&0&-1&0&0} &\snfsim \SNF^{1\times9}(-1).
\end{align*}

So
\begin{align*}
  \dim H^{-3,4} &= \dim \kernel(d_{1}) = 2-2 = 0, \\
  \dim H^{-3,5} &= \dim \dfrac{\kernel(d_{2})}{\image(d_{1})} = (9-1) - 2 = 6, \\
  \dim H^{-3,6} &= \dim \coker(d_{2}) = 1 - 1 = 0.
\end{align*}

$\gr^{4}:$
\[
  \begin{tikzcd}[ampersand replacement=\&]
    0 \ar[r] \& \lie{g}^{1} \wedge \wedge \lie{g}^{1} \wedge \lie{g}^{2} \ar[r,"d^{\top}"] \& \begin{aligned} &\lie{g}^{1} \wedge \lie{g}^{3} \\ &\oplus \lie{g}^{2} \wedge \lie{g}^{2} \end{aligned} \ar[r] \& 0
  \end{tikzcd}
\]

\[
  \begin{tikzcd}
    0 & \ar[l] \Hom_{k}^{-4}\bigl( \bigwedge^{3}\lie{g}, k \bigr) & \ar[l,"d"'] \Hom_{k}^{-4}\bigl( \bigwedge^{2}\lie{g}, k \bigr) &  \ar[l] 0
  \end{tikzcd}
\]

\begin{equation*}
  d \snfsim \SNF^{9\times9}(1,1,1,-1,1,-1,0,0,0)
\end{equation*}

So
\begin{align*}
  \dim H^{-4,6} &= \dim \kernel(d) = 9-6 = 3, \\
  \dim H^{-4,7} &= \dim \coker(d) = 9-6 = 3.
\end{align*}

$\gr^{5}:$
\[
  \begin{tikzcd}[ampersand replacement=\&]
    0 \ar[r] \& \lie{g}^{1} \wedge \lie{g}^{1} \wedge \lie{g}^{1} \wedge \lie{g}^{2} \ar[r,"d_{2}^{\top}"] \& \begin{aligned} &\lie{g}^{1} \wedge \lie{g}^{1} \wedge \lie{g}^{3} \\ &\oplus \lie{g}^{1} \wedge \lie{g}^{2} \wedge \lie{g}^{2} \end{aligned} \ar[r,"d_{1}^{\top}"] \& \lie{g}^{2} \wedge \lie{g}^{3} \ar[r] \& 0
  \end{tikzcd}
\]

\[
  \begin{tikzcd}[column sep=1em]
    0 & \ar[l] \Hom_{k}^{-5}\bigl( \bigwedge^{4}\lie{g}, k \bigr) & \ar[l,"d_{2}"' {yshift=2pt}] \Hom_{k}^{-5}\bigl( \bigwedge^{3}\lie{g}, k \bigr) & \ar[l,"d_{1}"' {yshift=2pt}] \Hom_{k}^{-5}\bigl( \bigwedge^{2}\lie{g}, k \bigr) & \ar[l] 0
  \end{tikzcd}
\]

\begin{align*}
  d_{1} &\snfsim \SNF^{15\times6}(1,1,-1,-1,1,1), \\
  d_{2} &\snfsim \SNF^{3\times15}(-1,1,1).
\end{align*}

So
\begin{align*}
  \dim H^{-5,7} &= \dim \kernel(d_{1}) = 6-6 = 0, \\
  \dim H^{-5,8} &= \dim \dfrac{\kernel(d_{2})}{\image(d_{1})} = (15-3) - 6 = 6, \\
  \dim H^{-5,9} &= \dim \coker(d_{2}) = 3 - 3 = 0.
\end{align*}

$\gr^{6}:$
\[
  \begin{tikzcd}[ampersand replacement=\&]
    0 \ar[r] \& \begin{aligned} &\lie{g}^{1} \wedge \lie{g}^{1} \wedge \lie{g}^{1} \wedge \lie{g}^{3} \\ &\oplus \lie{g}^{1} \wedge \lie{g}^{1} \wedge \lie{g}^{2} \wedge \lie{g}^{2} \end{aligned} \ar[r,"d_{2}^{\top}"] \& \begin{aligned} &\lie{g}^{1} \wedge \lie{g}^{2} \wedge \lie{g}^{3} \\ &\oplus \lie{g}^{2} \wedge \lie{g}^{2} \wedge \lie{g}^{2} \end{aligned} \ar[r,"d_{1}^{\top}"] \& \lie{g}^{3} \wedge \lie{g}^{3} \ar[r] \& 0
  \end{tikzcd}
\]

\[
  \begin{tikzcd}[column sep=1em]
    0 & \ar[l] \Hom_{k}^{-6}\bigl( \bigwedge^{4}\lie{g}, k \bigr) & \ar[l,"d_{2}"' {yshift=2pt}] \Hom_{k}^{-6}\bigl( \bigwedge^{3}\lie{g}, k \bigr) & \ar[l,"d_{1}"' {yshift=2pt}] \Hom_{k}^{-6}\bigl( \bigwedge^{2}\lie{g}, k \bigr) & \ar[l] 0
  \end{tikzcd}
\]

\begin{align*}
  d_{1} &\snfsim \SNF^{19\times1}(-1), \\
  d_{2} &\snfsim \SNF^{11\times19}(-1,1,-1,1,-1,-1,-1,1,1,1,-2).
\end{align*}

So
\begin{align*}
  \dim H^{-6,8} &= \dim \kernel(d_{1}) = 1-1 = 0, \\
  \dim H^{-6,9} &= \dim \dfrac{\kernel(d_{2})}{\image(d_{1})} = (19-11) - 1 = 7, \\
  \dim H^{-6,10} &= \dim \coker(d_{2}) = 11 - 11 = 0.
\end{align*}

$\gr^{7}:$
\[
  \begin{tikzcd}[ampersand replacement=\&, column sep=1em]
    0 \ar[r] \& \lie{g}^{1} \wedge \lie{g}^{1} \wedge \lie{g}^{1} \wedge \lie{g}^{2} \wedge \lie{g}^{2} \ar[r,"d_{2}^{\top}" {yshift=2pt}] \& \begin{aligned} &\lie{g}^{1} \wedge \lie{g}^{1} \wedge \lie{g}^{2} \wedge \lie{g}^{3} \\ &\oplus \lie{g}^{1} \wedge \lie{g}^{2} \wedge \lie{g}^{2} \wedge \lie{g}^{2} \end{aligned} \ar[r,"d_{1}^{\top}" {yshift=2pt}] \& \begin{aligned} &\lie{g}^{1} \wedge \lie{g}^{3} \wedge \lie{g}^{3} \\ &\oplus \lie{g}^{2} \wedge \lie{g}^{2} \wedge \lie{g}^{3} \end{aligned}\ar[r] \& 0
  \end{tikzcd}
\]

\[
  \begin{tikzcd}[column sep=1em]
    0 & \ar[l] \Hom_{k}^{-7}\bigl( \bigwedge^{5}\lie{g}, k \bigr) & \ar[l,"d_{2}"' {yshift=2pt}] \Hom_{k}^{-7}\bigl( \bigwedge^{4}\lie{g}, k \bigr) & \ar[l,"d_{1}"' {yshift=2pt}] \Hom_{k}^{-7}\bigl( \bigwedge^{3}\lie{g}, k \bigr) & \ar[l] 0
  \end{tikzcd}
\]

\begin{align*}
  d_{1} &\snfsim \SNF^{21\times9}(-1,-1,-1,1,1,1,1,-1,1), \\
  d_{2} &\snfsim \SNF^{3\times21}(1,1,-1).
\end{align*}

So
\begin{align*}
  \dim H^{-7,10} &= \dim \kernel(d_{1}) = 9-9 = 0, \\
  \dim H^{-7,11} &= \dim \dfrac{\kernel(d_{2})}{\image(d_{1})} = (21-3) - 9 = 9, \\
  \dim H^{-7,12} &= \dim \coker(d_{2}) = 3 - 3 = 0.
\end{align*}

The following calculations are not necessary, since we can get the results using a version of Poincaré duality for Lie algebra cohomology, but we keep the sketch work to make it clear that nothing goes wrong.

$\gr^{8}:$

\begin{align*}
  d_{1} &\snfsim \SNF^{21\times3}(1,-1,1), \\
  d_{2} &\snfsim \SNF^{9\times21}(-1,-1,-1,1,1,-1,-1,1,-1).
\end{align*}

So
\begin{align*}
  \dim H^{-8,11} &= \dim \kernel(d_{1}) = 3-3 = 0, \\
  \dim H^{-8,12} &= \dim \dfrac{\kernel(d_{2})}{\image(d_{1})} = (21-9) - 3 = 9, \\
  \dim H^{-8,13} &= \dim \coker(d_{2}) = 9 - 9 = 0.
\end{align*}


$\gr^{9}:$

\begin{align*}
  d_{1} &\snfsim \SNF^{19\times11}(-1,-1,1,-1,1,-1,-1,-1,-1,1,-1), \\
  d_{2} &\snfsim \SNF^{1\times19}(-1).
\end{align*}

So
\begin{align*}
  \dim H^{-9,13} &= \dim \kernel(d_{1}) = 11-11 = 0, \\
  \dim H^{-9,14} &= \dim \dfrac{\kernel(d_{2})}{\image(d_{1})} = (19-1) - 11 = 7, \\
  \dim H^{-9,15} &= \dim \coker(d_{2}) = 1 - 1 = 0.
\end{align*}

$\gr^{10}:$

\begin{align*}
  d_{1} &\snfsim \SNF^{15\times3}(1,1,-1), \\
  d_{2} &\snfsim \SNF^{6\times15}(-1,1,1,-1,1,1).
\end{align*}

So
\begin{align*}
  \dim H^{-10,14} &= \dim \kernel(d_{1}) = 3-3 = 0, \\
  \dim H^{-10,15} &= \dim \dfrac{\kernel(d_{2})}{\image(d_{1})} = (15-6) - 3 = 6, \\
  \dim H^{-10,16} &= \dim \coker(d_{2}) = 6-6 = 0.
\end{align*}

$\gr^{11}:$

\begin{equation*}
  d \snfsim \SNF^{9\times9}(1,1,-1,-1,-1,-1,0,0,0).
\end{equation*}

So
\begin{align*}
  \dim H^{-11,16} &= \dim \kernel(d) = 9-6 = 3, \\
  \dim H^{-11,17} &= \dim \coker(d) = 9-6 = 3.
\end{align*}

$\gr^{12}:$

\begin{align*}
  d_{1} &\snfsim \SNF^{9\times1}(1), \\
  d_{2} &\snfsim \SNF^{2\times9}(1,-1).
\end{align*}

So
\begin{align*}
  \dim H^{-12,17} &= \dim \kernel(d_{1}) = 1-1 = 0, \\
  \dim H^{-12,18} &= \dim \dfrac{\kernel(d_{2})}{\image(d_{1})} = (9-2) - 1 = 6, \\
  \dim H^{-12,19} &= \dim \coker(d_{2}) = 2-2 = 0.
\end{align*}

$\gr^{13}:$

\begin{equation*}
  d \snfsim \SNF^{3\times3}(-1,1,-1).
\end{equation*}

So
\begin{align*}
  \dim H^{-13,19} &= \dim \kernel(d) = 3-3 = 0, \\
  \dim H^{-13,20} &= \dim \coker(d) = 3-3 = 0.
\end{align*}

$\gr^{14}:$
\[
  \begin{tikzcd}
    0 \ar[r] & \lie{g}^{1} \wedge \lie{g}^{1} \wedge \lie{g}^{2} \wedge \lie{g}^{2} \wedge \lie{g}^{2} \wedge \lie{g}^{3} \wedge \lie{g}^{3}  \ar[r] & 0
  \end{tikzcd}
\]

\[
  \begin{tikzcd}
    0 & \ar[l] \Hom_{k}^{-14}\bigl( \bigwedge^{7}\lie{g},k \bigr) & \ar[l] 0
  \end{tikzcd}
\]

So $\dim H^{-14,21} = 3$ by \Cref{tab:graded-dims-SL3}.

$\gr^{15}:$
\[
  \begin{tikzcd}
    0 \ar[r] & \lie{g}^{1} \wedge \lie{g}^{1} \wedge \lie{g}^{1} \wedge \lie{g}^{2} \wedge \lie{g}^{2} \wedge \lie{g}^{2} \wedge \lie{g}^{3} \wedge \lie{g}^{3}  \ar[r] & 0
  \end{tikzcd}
\]

\[
  \begin{tikzcd}
    0 & \ar[l] \Hom_{k}^{-15}\bigl( \bigwedge^{8}\lie{g},k \bigr) & \ar[l] 0
  \end{tikzcd}
\]

So $H^{8} = H^{-15,23}$ with $\dim H^{-15,23} = 1$ by \Cref{tab:graded-dims-SL3}.

Altogether:
\begin{align*}
  H^{0} &= H^{0,0}, \\
  H^{1} &= H^{-1,2}, \\
  H^{2} &= H^{-3,5} \oplus H^{-4,6}, \\
  H^{3} &= H^{-4,7} \oplus H^{-5,8} \oplus H^{-6,9}, \\
  H^{4} &= H^{-7,11} \oplus H^{-8,12}, \\
  H^{5} &= H^{-9,14} \oplus H^{-10,15} \oplus H^{-11,16}, \\
  H^{6} &= H^{-11,17} \oplus H^{-12,18}, \\
  H^{7} &= H^{-14,21}, \\
  H^{8} &= H^{-15,23}
\end{align*}
and we have the following table:
\begin{table}[ht]
  \centering
  \scalebox{0.8}{%
  $\begin{NiceArray}{*{17}{c}}[hvlines]
    \diagbox{t}{s} & 0 & -1 & -2 & -3 & -4 & -5 & -6 & -7 & -8 & -9 & -10 & -11 & -12 & -13 & -14 & -15 \\
    0 & 1\\
    1 \\
    2 && 3 \\
    3 \\
    4 \\
    5 &&&& 6\\
    6 &&&&& 3 \\
    7 &&&&& 3\\
    8 &&&&&& 6\\
    9 &&&&&&& 7\\
    10 \\
    11 &&&&&&&& 9 \\
    12 &&&&&&&&& 9 \\
    13 \\
    14 &&&&&&&&&& 7 \\
    15 &&&&&&&&&&& 6 \\
    16 &&&&&&&&&&&& 3 \\
    17 &&&&&&&&&&&& 3 \\
    18 &&&&&&&&&&&&& 6 \\
    19 \\
    20 \\
    21 &&&&&&&&&&&&&&& 3 \\
    22 \\
    23 &&&&&&&&&&&&&&&& 1
  \end{NiceArray}$%
  }
  \caption[Graded cohomology dimensions for $I \subseteq \SL_{3}(\Z_{p})$]{Dimensions of $E_{1}^{s,t} = H^{s,t} = \gr^{s} H^{s+t}(\lie{g},k)$.}
  \label{tab:graded-coh-dims-SL3}
\end{table}
Thus
\begin{equation*}
  \dim H^{i} =
  \begin{dcases}
    1 & i=0, \\
    3 & i=1, \\
    9 & i=2, \\
    16 & i=3, \\
    18 & i=4, \\
    16 & i=5, \\
    9 & i=6, \\
    3 & i=7, \\
    1 & i=8.
  \end{dcases}
\end{equation*}

\section{\texorpdfstring{$I \subseteq \GL_{3}(\Z_{p})$}{I in GL3(Zp)}}%
\label{sec:Iwa-GL3}

\begin{equation}
  \label{eq:gis-GL3}
  \begin{gathered}
    g_{1} = \pmat{ 1 \\ & 1 \\ p && 1 }, \quad g_{2} = \pmat{ 1 \\ p & 1 \\ && 1 }, \quad g_{3} = \pmat{ 1 \\ & 1 \\ & p & 1 }, \\
    g_{4} = \pmat{ \exp(p) \\ & \exp(-p) \\ && 1 }, \quad g_{5} = \pmat{ 1 \\ & \exp(p) \\ && \exp(-p) }, \\
    g_{6} = \pmat{ \exp(p) \\ & \exp(p) \\ && \exp(p) },  \\
    g_{7} = \pmat{ 1 \\ & 1 & 1 \\ && 1 }, \quad g_{8} = \pmat{ 1 & 1 \\ & 1 \\ && 1 }, \quad g_{9} = \pmat{ 1 && 1 \\ & 1 \\ && 1 }.
  \end{gathered}
\end{equation}


\begin{table}[ht]
  \centering
  \scalebox{0.7}{%
  $\begin{NiceArray}{*{20}{c}}[hvlines]
    \diagbox{t}{s} & 0 & -1 & -2 & -3 & -4 & -5 & -6 & -7 & -8 & -9 & -10 & -11 & -12 & -13 & -14 & -15 & -16 & -17 & -18 \\
    0 & 1\\
    1 \\
    2 && 3 \\
    3 \\
    4 &&&& 1 \\
    5 &&&& 6 \\
    6 &&&&& 6 \\
    7 &&&&& 3\\
    8 &&&&&& 6 \\
    9 &&&&&&& 13 \\
    10 &&&&&&&& 3 \\
    11 &&&&&&&& 12 \\
    12 &&&&&&&&& 15 \\
    13 &&&&&&&&&& 7 \\
    14 &&&&&&&&&& 7 \\
    15 &&&&&&&&&&& 15 \\
    16 &&&&&&&&&&&& 12 \\
    17 &&&&&&&&&&&& 3 \\
    18 &&&&&&&&&&&&& 13 \\
    19 &&&&&&&&&&&&&& 6 \\
    20 &&&&&&&&&&&&&&& 3 \\
    21 &&&&&&&&&&&&&&& 6 \\
    22 &&&&&&&&&&&&&&&& 6 \\
    23 &&&&&&&&&&&&&&&& 1 \\
    24 \\
    25 &&&&&&&&&&&&&&&&&& 3 \\
    26 \\
    27 &&&&&&&&&&&&&&&&&&& 1
  \end{NiceArray}$%
  }
  \caption[Graded cohomology dimensions for $I \subseteq \GL_{3}(\Z_{p})$]{Dimensions of $E_{1}^{s,t} = H^{s,t} = \gr^{s} H^{s+t}(\lie{g},k)$.}
  \label{tab:graded-coh-dims-GL3}
\end{table}

%%% Local Variables:
%%% mode: latex
%%% TeX-master: "../main"
%%% End:
