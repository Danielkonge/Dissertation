% \usepackage{etoolbox} %gives a lot of useful tools, e.g. \newbool{...}, \setbool{...}{...}, ...

%% BEGIN : SETUP OF LANGUAGE AND FONTS %%

\usepackage[utf8]{inputenc} %use Unicode
\usepackage[T1]{fontenc} %European fonts

\usepackage[english=american]{csquotes} %set up \enquote to do American quoting style

\usepackage[english]{babel} %use English as the main language

%% END : SETUP OF LANGUAGE AND FONTS %%

%% BEGIN : STANDARD PACKAGES %%

\usepackage{%
  amsmath,       %some math tools
  amssymb,       %math symbols
  graphicx,      %enhanced graphics options
  mathtools,     %extension of amsmath
  microtype,     %small typographic effects
  bm,            %bold math symbols
  % todonotes,   %adds the option \todo{...} (use fixme instead)
  stmaryrd,      %some more math symbolswork
  % nicematrix,    %nicer matrix controls
  mathrsfs,      %more math fonts
}

\usepackage[footnote]{nicematrix}

\usepackage[shortlabels]{enumitem} %better lists, e.g. enumerate

\usepackage[citestyle=alphabetic,bibstyle=alphabetic,uniquename=init,autolang=hyphen,backend=biber,sortcites=true]{biblatex} %used to make bibliography
% \renewcommand*{\labelalphaothers}{}
\addbibresource{ref.bib}
% \DeclareLabelalphaTemplate{
%   \labelelement{
%     \field[final]{shorthand}
%     \field{label}
%     \field[strwidth=3,strside=left,ifnames=1]{labelname}
%     \field[strwidth=1,strside=left]{labelname}
%   }
% }
% \nobibintoc %removes the bibliography from the table of contents
% \nocite{*} %include all bibliography entries in the printed bibliography

\usepackage[thmmarks,amsmath]{ntheorem} %package for theorems, definitions, ...

\usepackage{varioref}

\usepackage{url}
\urlstyle{same}

\usepackage[colorlinks=false]{hyperref} %make links to references

\usepackage[nameinlink]{cleveref} %clever referencing (uses varioref and hyperref above)
% make some aliases for later
\crefalias{theorembreak}{theorem}
\crefalias{definitionbreak}{definition}
\crefalias{examplebreak}{example}
\crefalias{examplesbreak}{examples}
\crefalias{corollarybreak}{corollary}
\crefalias{propositionbreak}{proposition}
\crefalias{lemmabreak}{lemma}
\crefalias{remarkbreak}{remark}
\crefalias{proofbreak}{proof}
%

\usepackage[layout=margin,final]{fixme}
%\fxsetup{mode=multiuser}
\FXRegisterAuthor{dk}{andk}{DK} %register a user DK, and provide \dknote, \dkwarning, \dkerror and \dkfatal

\usepackage{calc} %used for calculating (useful when using tikz)

\usepackage{tikz} %used for drawing and making commutative diagrams
\usetikzlibrary{calc,shapes.geometric,decorations.markings,decorations.pathmorphing,arrows,cd,quotes,babel}
\usepackage{pgfplots} %for making graphs and plotting functions
\pgfplotsset{compat=1.18} %small fix to some compatibility issue

%\usepackage[left]{showlabels} %show labels in left margin while editing
\usepackage[notref,notcite,final]{showkeys}

%% END : STANDARD PACKAGES %%

%% BEGIN : SETUP INDEX AND NOMENCLATURE %%

%make index using memoir
\makeindex %create the default index

\usepackage[intoc]{nomencl} %used for nomenclature
\makenomenclature
\renewcommand*{\nomname}{List of Symbols}%name the nomenclature
\renewcommand*{\nomlabel}[1]{\hfil #1\hfil}%center symbol
%\renewcommand*{\pagedeclaration}[1]{\dotfill\hyperpage{#1}} %pagenumber in nomenclature + link
\renewcommand{\nomgroup}[1]{%divide nomenclature into groups (can only use capital letters)
\item[\bfseries%
  \ifstrequal{#1}{A}{Cohomology of Unipotent groups}{%
    \ifstrequal{#1}{B}{Cohomology of pro-$p$ Iwahori subgroups}{%
      \ifstrequal{#1}{C}{List-Decodable Mean Estimation and Clustering}{}}}%
  ]}
\newcommand*{\nomuni}[1][]{\nomenclature[A#1]} %nomenclature for Cohomology of Unipotent groups
\newcommand*{\nomiwa}[1][]{\nomenclature[B#1]} %nomenclature for Cohomology of pro-p Iwahori subgroups
\newcommand*{\nomlis}[1][]{\nomenclature[C#1]} %nomenclature for List-Decodable Mean Estimation and Clustering

%% END : SETUP INDEX AND NOMENCLATURE %%

%% BEGIN : THEOREMS %%

\theoremseparator{.}

\theoremstyle{plain}
\theoremsymbol{\ensuremath{\clubsuit}}
\newtheorem{theorem}{Theorem}[chapter]
\newtheorem{corollary}[theorem]{Corollary}
\newtheorem{proposition}[theorem]{Proposition}
\newtheorem{lemma}[theorem]{Lemma}
\newtheorem{fact}[theorem]{Fact}


\theorembodyfont{\normalfont}
\theoremsymbol{\ensuremath{\spadesuit}}
\newtheorem{definition}[theorem]{Definition}

\theoremsymbol{\ensuremath{\scriptstyle\bigcirc}}
\newtheorem{example}[theorem]{Example}
\newtheorem{examples}[theorem]{Examples}
\newtheorem{conjecture}[theorem]{Conjecture}

\theoremheaderfont{\itshape}
\theoremsymbol{\ensuremath{\triangle}}
\newtheorem{remark}[theorem]{Remark}

\theoremstyle{nonumberplain}
\theoremsymbol{\ensuremath{\square}}
\newtheorem{proof}{Proof}

%corresponding break versions:

% \theoremsymbol{}
\theoremheaderfont{\normalfont\bfseries}
\theorembodyfont{\itshape}
\theoremstyle{break}
\theoremsymbol{\ensuremath{\clubsuit}}
\newtheorem{theorembreak}[theorem]{Theorem}
\newtheorem{corollarybreak}[theorem]{Corollary}
\newtheorem{propositionbreak}[theorem]{Proposition}
\newtheorem{lemmabreak}[theorem]{Lemma}

\theorembodyfont{\normalfont}
\theoremsymbol{\ensuremath{\spadesuit}}
\newtheorem{definitionbreak}[theorem]{Definition}

\theoremsymbol{\ensuremath{\scriptstyle\bigcirc}}
\newtheorem{examplebreak}{Example}
\newtheorem{examplesbreak}{Examples}

\theoremheaderfont{\itshape}
\theoremsymbol{\ensuremath{\triangle}}
\newtheorem{remarkbreak}{Remark}

\theoremstyle{nonumberbreak}
\theoremsymbol{\ensuremath{\square}}
\newtheorem{proofbreak}{Proof}

% no number versions:

\theoremsymbol{}
\theoremstyle{nonumberplain}
\theoremheaderfont{\normalfont\bfseries}
\theorembodyfont{\normalfont}
\newtheorem{notheorem}{Theorem}
\newtheorem{nocorollary}{Corollary}
\newtheorem{noproposition}{Proposition}
\newtheorem{nolemma}{Lemma}

\newtheorem{nodefinition}{Definition}

\theoremsymbol{\ensuremath{\scriptstyle\bigcirc}}
\newtheorem{noexample}{Example}
\newtheorem{noexamples}{Examples}

\theoremheaderfont{\itshape}
\theoremsymbol{\ensuremath{\triangle}}
\newtheorem{noremark}{Remark}

%corresponding no number break versions:

\theoremsymbol{}
\theoremheaderfont{\normalfont\bfseries}
\theorembodyfont{\normalfont}
\theoremstyle{nonumberbreak}
\newtheorem{notheorembreak}{Theorem}
\newtheorem{nocorollarybreak}{Corollary}
\newtheorem{nopropositionbreak}{Proposition}
\newtheorem{nolemmabreak}{Lemma}

\newtheorem{nodefinitionbreak}{Definition}

\theoremsymbol{\ensuremath{\scriptstyle\bigcirc}}
\newtheorem{noexamplebreak}{Example}
\newtheorem{noexamplesbreak}{Examples}

\theoremheaderfont{\itshape}
\theoremsymbol{\ensuremath{\triangle}}
\newtheorem{noremarkbreak}{Remark}

% special theorems:

% \theoremseparator{.}
% \theoremstyle{plain}
% \theoremheaderfont{\normalfont\bfseries}
% \theorembodyfont{\itshape}
% \theoremsymbol{}
% \newtheorem{fact}[theorem]{Fact}

\theoremseparator{:}
\theoremstyle{nonumberplain}
\theoremheaderfont{\normalfont\bfseries}
\theorembodyfont{\itshape}
\theoremsymbol{}
\newtheorem{univprop}{Universal Property}

%% END : THEOREMS %%

%% BEGIN : USEFUL SETTINGS FOR THIS DOCUMENT %%

\allowdisplaybreaks %allows page change in align (but tries to avoid it)

\newcommand*{\yesnumber}{\refstepcounter{equation}\tag{\theequation}} %makes it possible to add a number in an equation

%% END : USEFUL SETTINGS FOR THIS DOCUMENT %%
